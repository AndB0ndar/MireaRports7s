\section*{ЦЕЛЬ ПРАКТИЧЕСКОЙ РАБОТЫ}
\addcontentsline{toc}{section}{ЦЕЛЬ ПРАКТИЧЕСКОЙ РАБОТЫ}

\textbf{Цель работы:}
познакомиться с возможностями, приобрести практические умения и
навыки организации домена Windows NT,
администрирования сетевых пользователей и
групп, организации доступа к файловым объектам.

Материально-техническое и программное обеспечение рабочего места:

\begin{enumerate}
	\item Персональный компьютер (настольный или ноутбук)
		под управлением ОС семейств Windows или Linux.
	\item Установленная программа VMware Player с пакетом VMware Tools.
	\item Инсталляционный пакет ОС Microsoft Windows Server
		на носителе (CD, DVD, USBфлэш) или iso-образ.
	\item Любая антивирусная программа.
\end{enumerate}

\textbf{Содержание работы:}

\begin{enumerate}
	\item Установка и настройка ОС Microsoft Windows Server.
	\item Установка узлу функциональной роли контроллера домена.
	\item Утилита <<Active Directory --- пользователи и компьютеры>>.
	\item Приёмы обслуживания сервера.
\end{enumerate}

\clearpage

\section*{ВЫПОЛНЕНИЕ ПРАКТИЧЕСКОЙ РАБОТЫ}
\addcontentsline{toc}{section}{ВЫПОЛНЕНИЕ ПРАКТИЧЕСКОЙ РАБОТЫ}

\section{Установка и настройка ОС Microsoft Windows Server}

\subsection{Создание виртуальной машины}

Создадим новую виртуальную машину.
В диалоговом окне введем название машины и тип операционной системы
(Рисунок~\ref{fig:vm:create}).

\begin{image}
	\includegrph{Screenshot from 2024-10-13 15-14-47}
	\caption{Первое диалоговое окно}
	\label{fig:vm:create}
\end{image}

В следующем окне выберем объём оперативной памяти и количество ядер.
В разделе Base Memory установим объем оперативной памяти будущего
виртуального компьютера --- не менее 2024 Мб (Рисунок~\ref{fig:vm:hardware}).

\begin{image}
	\includegrph{Screenshot from 2024-10-13 15-14-51}
	\caption{Объём оперативной памяти}
	\label{fig:vm:hardware}
\end{image}

В следующем окне зададим объём жесткого диска гостевого компьютера --- 60 Гб
(Рисунок~\ref{fig:vm:harddisk}).

\begin{image}
	\includegrph{Screenshot from 2024-10-13 15-15-10}
	\caption{Объём жесткого диска}
	\label{fig:vm:harddisk}
\end{image}

\subsection{Настройки виртуальной машины}

Перейдите в настройки виртуальной машины и установите
в качестве источника ОС ISO образ \rref{fig:vm:iso}.

\begin{image}
	\includegrph{Screenshot from 2024-10-13 15-15-52}
	\caption{Выбор iso}
	\label{fig:vm:iso}
\end{image}

Запустите новый виртуальный компьютер.\par
Проследите за ходом установки гостевой ОС Microsoft Windows Server,
укажите в отчёте основные её этапы \rdref{fig:ws:install:lang}{fig:ws:install}.
В ходе установки:

\begin{itemize}
	\item Выберите устанавливаемый язык, формат времени и денежных единиц,
		а также метод ввода (раскладку клавиатуры).
	\item Выберите Windows Server 2019 Standart с возможностями рабочего стола
		в качестве устанавливаемой операционной системы.
	\item Выберите выборочный тип установки.
	\item Задайте пароль для встроенной учетной записи администратора.
		В отчете опишите требования к паролю.
	\item Разрешите другим компьютерам и устройствам обнаруживать ваш ПК.
\end{itemize}

\begin{image}
	\includegrph{Screenshot from 2024-10-13 15-22-11}
	\caption{Выбор языка}
	\label{fig:ws:install:lang}
\end{image}

\begin{image}
	\includegrph{Screenshot from 2024-10-13 15-22-29}
	\caption{Выбор ОС}
	\label{fig:ws:install:os}
\end{image}

\begin{image}
	\includegrph{Screenshot from 2024-10-13 15-22-37}
	\caption{Условия лицензии}
	\label{fig:ws:install:license}
\end{image}

\begin{image}
	\includegrph{Screenshot from 2024-10-13 15-24-02}
	\caption{Установка}
	\label{fig:ws:install}
\end{image}

В командной строке (cmd.exe) гостевой машины
с помощью утилиты \texttt{ipconfig} (с ключом \texttt{/all})
отобразите автоназначенные IPv4-параметры \rref{fig:ipconfig}.

\begin{image}
	\includegrph{Screenshot from 2024-10-13 16-11-29}
	\caption{Вывод ipconfig}
	\label{fig:ipconfig}
\end{image}

В свойствах TCP/IPv4
(Панель управления \rarrow Центр управления сетями и общим доступом
\rarrow Изменение параметров адаптера, далее свойства сетевого подключения)
задайте вручную определённые вами ранее автоназначенные параметры
(адрес, маска, шлюз, DNS-серверы).

\begin{image}
	\includegrph{Screenshot from 2024-10-13 16-13-06}
	\caption{Установка параметров}
	\label{fig:set:adapter}
\end{image}

\section{Установка узлу функциональной роли контроллера домена}

\subsection{Установка ролей и компонентов}

Перейдите в автоматически открывшееся окно <<Диспетчер серверов>>.
Выберите <<Управление>>, затем <<Добавить или удалить роль>>.
В открывшемся мастере добавления ролей и компонентов
\rdref{fig:rule}{fig:components}:

\begin{itemize}
	\item Выберите установку ролей или компонентов.
	\item Выберите созданный вами сервер.
	\item В разделе <<Роли сервера>> поставьте флажок
		\textbf{<<Доменные службы Active Directory>>},
		добавьте необходимые компоненты, а также поставьте флажок
		\textbf{<<Диспетчер ресурсов файлового сервера>>},
		добавьте необходимые компоненты.
	\item Выберите компоненты \textbf{SMB 1.0/CIFS File Sharing Support}
		и \textbf{Система архивации данных Windows Server}.
\end{itemize}

\begin{image}
	\includegrph{Screenshot from 2024-10-13 16-15-07}
	\caption{Роли сервера}
	\label{fig:rule}
\end{image}

\begin{image}
	\includegrph{Screenshot from 2024-10-13 16-17-14}
	\caption{Компоненты}
	\label{fig:components}
\end{image}

После установки всех необходимых ролей и компонентов перезагрузите гостевой
компьютер.

\subsection{Повышение роли сервера до уровня контроллера домена}

После перезагрузки в окне Диспетчера
серверов появится предупреждение <<Конфигурация после развертывания>>,
выберите <<Повысить роль этого сервера до уровня контроллера домена>>
\rdref{fig:srule:config}{fig:srule:check}.

\begin{image}
	\includegrph{Screenshot from 2024-10-13 16-29-16}
	\caption{Конфигурация развертывания}
	\label{fig:srule:config}
\end{image}

\begin{image}
	\includegrph{Screenshot from 2024-10-13 16-29-48}
	\caption{Параметры контроллера домена}
	\label{fig:srule:domen:contr}
\end{image}

\begin{image}
	\includegrph{Screenshot from 2024-10-13 16-30-17}
	\caption{Дополнительные параметры}
	\label{fig:srule:dopparams}
\end{image}

\begin{image}
	\includegrph{Screenshot from 2024-10-13 16-30-22}
	\caption{Пути}
	\label{fig:srule:paths}
\end{image}

\begin{image}
	\includegrph{Screenshot from 2024-10-13 16-30-31}
	\caption{Просмотр параметров}
	\label{fig:srule:view}
\end{image}

\begin{image}
	\includegrph{Screenshot from 2024-10-13 16-30-50}
	\caption{Проверка предворительных требований}
	\label{fig:srule:check}
\end{image}

В окне мастера настройки доменных служб Active Directory
выберите вариант <<Добавить новый лес>>.
В качестве имени корневого домена укажите Bondar.domain.
В параметрах контроллера домена введите пароль
для режима восстановления служб каталогов.

Согласитесь с предложением перезагрузки системы.
Обратите внимание,
как изменилось содержимое окна <<Диспетчер серверов>>
\rref{fig:diff:server:dispetcher}.

\begin{image}
	\includegrph{Screenshot from 2024-10-13 16-36-39}
	\caption{Диспетчер серверов}
	\label{fig:diff:server:dispetcher}
\end{image}

\section{Утилита <<Active Directory --- пользователи и компьютеры>>}

\subsection{Открыть Active Directory}

Перейдите в раздел AD DS,
нажмите правой кнопкой мыши на созданный сервер
и выберите <<Active Directory -- домены и доверие>>,
далее в контекстном меню выберите <<Управление>>,
чтобы запустить консоль <<Active Directory --- пользователи и компьютеры>>
\rref{fig:activedir}.

\begin{image}
	\includegrph{Screenshot from 2024-10-13 17-08-45}
	\caption{Окно Active Directory}
	\label{fig:activedir}
\end{image}

\subsection{Создание пользователя и группу}

Раскройте ветку с названием домена.
В контекстном меню на значке домены выберите
\textbf{Создать \rarrow Подразделение}.
Создайте подразделение с именем вашей учебной группы \rref{fig:new:group}.

\begin{image}
	\includegrph{Screenshot from 2024-10-13 17-09-19}
	\caption{Создание подразделения}
	\label{fig:new:group}
\end{image}

В подразделении создайте нового пользователя.
В карточке укажите полные данные (ваши),
имя входа пользователя --- ваши инициалы малыми латинскими буквами
\rref{fig:new:user}.
Полное имя должно начинаться с фамилии
(в дальнейшем это упрощает поиск в списке).

\begin{image}
	\includegrph{Screenshot from 2024-10-13 17-11-07}
	\caption{Создание пользователя}
	\label{fig:new:user}
\end{image}

Установите неограниченный срок действия пароля.
Задайте пароль в соответствии с требованиями Windows Server
\rref{fig:new:user:passw}.
Пароль должен отвечать требованиям сложности
(Password must meet complexity requirements)
---~это параметр политики паролей в домене.
При включении этой политики пользователю запрещено использовать
имя своей учётной записи в пароле
(не более чем два символа подряд из username или Firstname).
Также в пароле должны использоваться 3 типа символов из следующего списка:

\begin{itemize}
	\item цифры (0–9);
	\item символы в верхнем регистре;
	\item символы в нижнем регистре;
	\item специальные символы (\$, \#, \% и т.д.).
\end{itemize}

\begin{image}
	\includegrph{Screenshot from 2024-10-13 17-11-42}
	\caption{Создание пароля пользователя}
	\label{fig:new:user:passw}
\end{image}

В окне свойств нового пользователя отметьте в отчёте,
в какие группы он включен по умолчанию \rref{fig:new:user:group}.

\begin{image}
	\includegrph{Screenshot from 2024-10-13 17-22-19}
	\caption{Группы пользователя по умолчанию}
	\label{fig:new:user:group}
\end{image}

\subsection{Сетевой каталог пользователя}

Создайте на сервере каталог \verb|C:\Home|,
в нем -- вложенную каталог, названную логином (входным именем) пользователя.
В окне свойств папки, на вкладке <<Безопасность>>,
предоставьте вновь созданному пользователю права
на изменение \rref{fig:user:prop}.

\begin{image}
	\includegrph{Screenshot from 2024-10-13 17-26-59}
	\caption{Права пользователя на изменение}
	\label{fig:user:prop}
\end{image}

Удалите группу <<Пользователи>> из списка имеющих локальные права.
Для этого в окне свойств папки,
на вкладке <<Безопасность>> перейдите в раздел <<Дополнительно>>,
выберите субъект <<Пользователи>> с особыми правами
и нажмите <<Отключение наследования>>, выберите
<<Преобразовать унаследованные разрешения в явные разрешения этого объекта>>
\rref{fig:user:dir:disabiling:inheritance}.

\begin{image}
	\includegrph{Screenshot from 2024-10-13 17-29-44}
	\caption{Отключение наследования}
	\label{fig:user:dir:disabiling:inheritance}
\end{image}

Предоставьте сетевой доступ к пользовательской папке (не к \texttt{Home}),
удалите из списка имеющих сетевой доступ группу <<Все>>,
добавьте разрешение только для созданного пользователя
\rref{fig:user:dir:net:share}.
Проверьте, что сетевое имя папки точно повторяет логин пользователя.

\begin{image}
	\includegrph{Screenshot from 2024-10-13 17-29-44}
	\caption{Отключение наследования}
	\label{fig:user:dir:net:share}
\end{image}

\subsection{Указание домашней директории}

В карточке \rref{fig:user:profile} пользователя во вкладке <<Профиль>>
укажите подключать диск \texttt{М:}
(в качестве пути укажите сетевой путь к созданной папке).

\begin{image}
	\includegrph{Screenshot from 2024-10-13 17-29-44}
	\caption{Отключение наследования}
	\label{fig:user:profile}
\end{image}

\subsection{Подключение к домену}

Запустите созданную на предыдущем занятии виртуальную станцию.
Зайдите в систему с правами администратора.
Используя окно свойств <<Мой компьютер>>, включите её в
созданный Вами домен (используя короткое имя домена).
Осуществите вход на виртуальной рабочей станции под именем пользователя,
созданным вами в домене \rdref{fig:connect:domen}{fig:connect:domen:passw}.

\begin{image}
	\includegrph{Screenshot from 2024-10-13 18-55-07}
	\caption{Выбор домена}
	\label{fig:connect:domen}
\end{image}

\begin{image}
	\includegrph{Screenshot from 2024-10-13 18-38-17}
	\caption{Ввод логина и пароля}
	\label{fig:connect:domen:passw}
\end{image}

\section{Приёмы обслуживания сервера}

\subsection{Антивирус}

Изучите классификацию вредоносного ПО
от компании <<Лаборатория Касперского>>,
в отчёте перечислите основные типы:

\begin{itemize}
	\item \textbf{Рекламные программы} или программы,
		существующие за счет демонстрации рекламы, отображают нежелательную,
		а иногда и вредоносную рекламу на экране устройств,
		перенаправляют результаты поиска на рекламные сайты
		и собирают данные пользователей,
		которые затем можно продать рекламодателям
		без согласия самих пользователей.
	\item \textbf{Шпионские программы} --- это разновидность вредоносных
		программ, скрывающихся на устройстве, отслеживающих активность
		и осуществляющих кражу конфиденциальной информации:
		финансовых данных, учетных записей, данных для входа и прочих данных. 
	\item \textbf{Программы-вымогатели} --- это вредоносные программы,
		осуществляющие блокировку или отказ доступа пользователей
		к системе или данным до момента выплаты выкупа.
		\textbf{Программы-шифровальщики} --- это тип программ-вымогателей,
		выполняющих шифрование пользовательских файлов
		и требующих оплаты в определенный срок и часто в цифровой валюте,
		например, в биткойнах.
	\item \textbf{Троянские программы} маскируются
		под легальное программное обеспечение,
		чтобы обманом заставить пользователей запустить
		вредоносные программы на компьютере.
	\item \textbf{Черви} --- это один из наиболее часто встречающихся
		типов вредоносных программ, распространяющихся по компьютерным сетям,
		используя уязвимости операционной системы.
	\item \textbf{Вирус} --- это фрагмент кода,
		который вставляется в приложение и запускается при его запуске.
	\item \textbf{Клавиатурные шпионы} --- это разновидность шпионских
		программ, отслеживающих активность пользователей. 
	\item \textbf{Бот} --- это компьютер, зараженный вредоносной программой,
		которым злоумышленники могут управлять удаленно.
		Боты, иногда называемые зомби-компьютерами,
		могут использоваться для запуска атак,
		а также стать частью \textbf{ботнета} --- набора ботов,
		объединенных в сеть.
	\item \textbf{Потенциально нежелательные программы (ПНП)}
		--- это программы, которые могут включать рекламу, панели инструментов
		и всплывающие окна, не имеющие отношения к самой загруженной программе.
	\item \textbf{Гибридные вредоносные программы} ---
		большинство современных вредоносных программ представляет
		собой комбинацию различных типов,
		часто включая элементы троянских программ, червей, а иногда и вирусов
	\item \textbf{Бесфайловые вредоносные программы} используют
		легальные программы для заражения компьютера
	\item \textbf{Логические бомбы} --- это тип вредоносных программ,
		которые активируются только при определенном условии,
		например, в определенную дату и время
		или при 20-м входе в учетную запись.
\end{itemize}

С помощью мастера добавления ролей и компонентов убедитесь,
что встроенная в ОС Windows Server
антивирусная программа Microsoft Defender установлена
\rref{fig:antivir:components}.

\begin{image}
	\includegrph{Screenshot from 2024-10-13 18-57-41}
	\caption{Проверка установки программы Microsoft Defender}
	\label{fig:antivir:components}
\end{image}

С помощью консоли управления Безопасность Windows убедитесь,
что защита включена \rref{fig:antivir:on}.

\begin{image}
	\includegrph{Screenshot from 2024-10-13 18-59-23}
	\caption{Проверка работы антивируса}
	\label{fig:antivir:on}
\end{image}

Средствами консоли управления \textbf{Безопасность Windows}
\rarrow \textbf{Защита от вирусов и угроз}
\rarrow \textbf{Параметры сканирования}
\rarrow \textbf{Настраиваемое сканирование} проверьте содержимое
диска \texttt{С:} на вирусы \rdref{fig:antivir:scan:prop}{fig:antivir:scan}.

\begin{image}
	\includegrph{Screenshot from 2024-10-13 19-00-21}
	\caption{Параметры сканирования}
	\label{fig:antivir:scan:prop}
\end{image}

\begin{image}
	\includegrph{Screenshot from 2024-10-13 19-15-56}
	\caption{Сканирование}
	\label{fig:antivir:scan}
\end{image}

\subsection{Check Disk}

Изучите возможности встроенной утилиты chkdsk (<<Check Disk>>)
на сервере для проверки жёсткого диска на наличие ошибок.
Включите проверку системного диска
при следующей перезагрузке серверного компьютера \rref{fig:chkdsk:scan}.
Найдите и отобразите результат проверки после перезагрузки системы.

\begin{image}
	\includegrph{Screenshot from 2024-10-13 19-16-10}
	\caption{Сканирование <<Check Disk>>}
	\label{fig:chkdsk:scan}
\end{image}

\subsection{Система архивации данных}

Настройте систему архивации данных на сервере таким образом,
чтобы содержимое папки Home автоматически заархивировалось через 3-4 минуты,
для этого в <<Диспетчере серверов>> в меню выберите <<Средства>>,
далее <<Система архивации данных Windows Server>>
\rdref{fig:arch:config}{fig:arch:confirmation}.

\begin{image}
	\includegrph{Screenshot from 2024-10-13 19-06-37}
	\caption{Конфигурация архивации}
	\label{fig:arch:config}
\end{image}

\begin{image}
	\includegrph{Screenshot from 2024-10-13 19-07-00}
	\caption{Объекты архивации}
	\label{fig:arch:object}
\end{image}

\begin{image}
	\includegrph{Screenshot from 2024-10-13 19-07-38}
	\caption{Время архивации}
	\label{fig:arch:time}
\end{image}

\begin{image}
	\includegrph{Screenshot from 2024-10-13 19-13-36}
	\caption{Подтверждение}
	\label{fig:arch:confirmation}
\end{image}

\clearpage

\section*{ВЫВОД}
\addcontentsline{toc}{section}{ВЫВОД}

В ходе выполнения практической работы были изучены основные возможности
операционной системы Microsoft Windows Server,
связанные с организацией домена Windows NT
и администрированием сетевых пользователей.
Были получены практические навыки установки и настройки операционной системы,
а также назначения серверу роли контроллера домена.
Изучена работа с утилитой <<Active Directory --- пользователи и компьютеры>>,
которая позволяет управлять пользователями, группами и объектами в домене.
Также были рассмотрены основные приёмы обслуживания сервера
для поддержания его работоспособности и обеспечения безопасности данных.

