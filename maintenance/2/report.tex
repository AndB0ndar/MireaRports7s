
Административные общие папки (шары, ресурсы) в Windows используются
для удаленного доступа и управление компьютером.
Windows по-умолчанию создает следующие административные шары:

\begin{itemize}
	\item \verb|Admin$| --- Remote Admin (это каталог %SystemRoot% )
		--- используется для удаленного администрирования компьютера;
	\item \verb|IPC$|
		--- Remote IPC --- используется для коммуникации программ named pipes);
	\item \verb|C$| --- Default Share. Расшаренный системный диск.
		Если на компьютере есть другие диски с назначенными буквами,
		они также автоматически публикуются
		в виде административных общих ресурсов (\verb|D$| , \verb|E$| и т.д.);
	\item \verb|Print$| --- публикуется, если вы открыли общий доступ
		к принтеру (используется, чтобы открыть доступ к каталогу
		с драйверами принтеров \texttt{C:\Windows\system32\spool\drivers});
	\item \verb|FAX$| --- используется для общего доступа к факс-серверу.
\end{itemize}

\subsection{Установка Guest Additions на гостевой машине}
Guest Additions --- это набор драйверов
и утилит для лучшей интеграции между хостом и гостевой ОС.

\begin{itemize}
	\item Запустите виртуальную машину с Windows 7.
	\item В меню VirtualBox выберите
		Devices → Insert Guest Additions CD Image....
	\item Откройте этот диск в гостевой системе (Windows)
		и установите Guest Additions. Следуйте инструкциям по установке.
	\item После завершения установки перезагрузите виртуальную машину.
\end{itemize}

