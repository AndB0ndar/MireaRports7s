\section*{ЦЕЛЬ ПРАКТИЧЕСКОЙ РАБОТЫ}
\addcontentsline{toc}{section}{ЦЕЛЬ ПРАКТИЧЕСКОЙ РАБОТЫ}

\textbf{Цель работы:} познакомиться с возможностями,
приобрести практические умения и
навыки администрирования локальных пользователей и групп, организации доступа к
файловым объектам и настройки сетевого взаимодействия
в рабочей группе Microsoft Windows.\par
Материально-техническое и программное обеспечение рабочего места:

\begin{enumerate}
	\item Персональный компьютер (настольный или ноутбук)
		под управлением ОС семейств Windows или Linux.
	\item Установленная программа VMware Player с пакетом VMware Tools.
	\item Виртуальная машина под управлением гостевой ОС Microsoft Windows.
	\item Программа AnyDesk.
\end{enumerate}

\clearpage

\section*{ВЫПОЛНЕНИЕ ПРАКТИЧЕСКОЙ РАБОТЫ}
\addcontentsline{toc}{section}{ВЫПОЛНЕНИЕ ПРАКТИЧЕСКОЙ РАБОТЫ}

\section{Настройка сетевой подсистемы рабочей станции}

\subsection{Имя компьютера и флаги доступа}

В свойствах системы (\textbf{Панель управления \rarrow Система})
гостевой машины задайте имя компьютера --- ваши инициалы латинскими буквами
\rref{fig:name}.

\begin{image}
	\includegrph{Screenshot from 2024-09-11 16-35-50}
	\caption{Имя компьютера}
	\label{fig:name}
\end{image}

Гостевой компьютер входит в рабочуюю группу \textbf{WORKGROUP}.

В Панели управления (\textbf{Панель управления
\rarrow Оформление и персонализация \rarrow Параметры папок \rarrow Вид})
проверьте, что снят флажок <<Использовать мастер общего доступа>>
\rref{fig:flag:master:share}.

\begin{image}
	\includegrph{Screenshot from 2024-09-11 16-37-52}
	\caption{Мастер общего доступа}
	\label{fig:flag:master:share}
\end{image}

В свойствах брандмауэра (\textbf{Панель управления \rarrow Система
\rarrow Брандмауэр Windows
\rarrow Разрешить запуск программы или компонента через брандмауэр Windows})
проверьте, что стоит флажок <<Общий доступ к файлам и принтерам>>.
\rref{fig:flag:share:file}

\begin{image}
	\includegrph{Screenshot from 2024-09-11 16-41-10}
	\caption{Общий доступ к файлам и принтерам}
	\label{fig:flag:share:file}
\end{image}

\section{IP адрес}

Настроим локальную сеть в VirtualBox \rref{fig:host:only:net}.

\begin{image}
	\includegrph{Screenshot from 2024-09-15 19-38-46}
	\caption{Создание локальной сети сеть}
	\label{fig:host:only:net}
\end{image}

Далее добавимь виртуальную машину в сеть \rref{fig:add:net:adapter}.

\begin{image}
	\includegrph{Screenshot from 2024-09-15 19-41-25}
	\caption{Добавление адаптера}
	\label{fig:add:net:adapter}
\end{image}

В свойствах сетевого подключения гостевой машины (в панели управления)
проверьте, что для сетевого адаптера виртуальной станции установлено
автоназначение параметров TCP/IPv4 \rref{fig:spec:net:connection}.

\begin{image}
	\includegrph{Screenshot from 2024-09-22 19-12-44}
	\caption{Свойства сетевого подключения}
	\label{fig:spec:net:connection}
\end{image}

За это отвечает протокол \textit{TCP/IPv4}.

В командной строке (\texttt{cmd.exe}) с помощью утилиты \texttt{ipconfig}
отобразите IPv4-параметры (192.168.1.101) и MAC-адрес сетевого интерфейса
виртуального компьютера \rref{fig:ipconfig}.

\begin{image}
	\includegrph{Screenshot from 2024-09-15 19-43-06}
	\caption{Вывод ipconfig}
	\label{fig:ipconfig}
\end{image}

Выведем ip адреса хост-машины \rref{fig:host:ip}.

\begin{image}
	\includegrph{Screenshot from 2024-09-15 19-43-15}
	\caption{Вывод hostname}
	\label{fig:host:ip}
\end{image}

\subsection{Эхо-запрос}

Настроим возможность отправки эхо-запроса к гостевой машине.
Откроем \textbf{Дополнительные параметры} в \textbf{Брандмауэр Windows}
\rref{fig:firewall:setting}.

\begin{image}
	\includegrph{Screenshot from 2024-09-15 20-16-06}
	\caption{Дополнительные параметры брандмауэра}
	\label{fig:firewall:setting}
\end{image}

Далее добавим правило для входящий подключений ICMP
\rdref{fig:icmp:1}{fig:icmp:9}.

\begin{image}
	\includegrph{Screenshot from 2024-09-15 20-16-17}
	\caption{Создание правила ICMP, часть 1}
	\label{fig:icmp:1}
\end{image}

\begin{image}
	\includegrph{Screenshot from 2024-09-15 20-16-22}
	\caption{Создание правила ICMP, часть 2}
	\label{fig:icmp:2}
\end{image}

\begin{image}
	\includegrph{Screenshot from 2024-09-15 20-16-48}
	\caption{Создание правила ICMP, часть 3}
	\label{fig:icmp:3}
\end{image}

\begin{image}
	\includegrph{Screenshot from 2024-09-15 20-17-39}
	\caption{Создание правила ICMP, часть 4}
	\label{fig:icmp:4}
\end{image}

\begin{image}
	\includegrph{Screenshot from 2024-09-15 20-18-05}
	\caption{Создание правила ICMP, часть 5}
	\label{fig:icmp:5}
\end{image}

\begin{image}
	\includegrph{Screenshot from 2024-09-15 20-18-22}
	\caption{Создание правила ICMP, часть 6}
	\label{fig:icmp:6}
\end{image}

\begin{image}
	\includegrph{Screenshot from 2024-09-15 20-18-34}
	\caption{Создание правила ICMP, часть 7}
	\label{fig:icmp:7}
\end{image}

\begin{image}
	\includegrph{Screenshot from 2024-09-15 20-21-15}
	\caption{Создание правила ICMP, часть 8}
	\label{fig:icmp:8}
\end{image}

\begin{image}
	\includegrph{Screenshot from 2024-09-15 20-21-43}
	\caption{Активатия правила}
	\label{fig:icmp:9}
\end{image}

С помощью утилиты ping в командной строке на хост-компьютере
проверим наличие в сети виртуального компьютера по его IPv4-адресу
\rref{fig:ping}.

\begin{image}
	\includegrph{Screenshot from 2024-09-15 20-22-38}
	\caption{Отправка эхо-запроса}
	\label{fig:ping}
\end{image}

\subsection{Видимость виртуального компьютера по его DNS-имени}

С помощью ключа \texttt{-a} утилиты \texttt{ping} проверим видимость
виртуального компьютера по его DNS-имени \rref{fig:ping:dns}.

\begin{image}
	\includegrph{Screenshot from 2024-09-22 19-37-15}
	\caption{Отправка эхо-запроса с флагом \texttt{a}}
	\label{fig:ping:dns}
\end{image}

\subsection{Схема компьютерной сети}

\begin{tikzpicture}[
    box/.style={draw
		, rounded corners
		, align=center
		, minimum width=3cm, minimum height=1cm
		},
    net/.style={align=center}
]
	% Host
	\node[box] (host) {Host Computer\\192.168.56.1\\Mask: 255.255.255.0};

	% Guest
	\node[box, right=5cm of host] (guest) {Guest Computer\\192.168.56.101\\Mask: 255.255.255.0};

	% DHCP Server
	\node[box, below=2cm of host] (dhcp) {DHCP Server\\192.168.56.100\\Mask: 255.255.255.0};

	% Arrows connecting
	\draw[-] (host) -- node[net, above] {} (guest);
	\draw[-] (host) -- node[net, left] {} (dhcp);
	\draw[-] (guest) -- node[net, right] {} (dhcp);
	\draw[-] (guest) -- node[net, below right] {} (host);
\end{tikzpicture}

\section{Администрирование локальных пользователей и групп}

\subsection{Общие ресурсы по умолчанию}

В консоли управления гостевой машины
(\textbf{Панель управления \rarrow Администрирование
\rarrow Управление компьютером})
в разделе <<Общие папки>> \rarrow <<Общие ресурсы>>
просмотрите и зафиксируйте в отчёте,
какие папки предоставлены в сетевой доступ по умолчанию.

\begin{image}
	\includegrph{Screenshot from 2024-09-16 17-11-34}
	\caption{Каталоги в сетевом доступе по умолчанию}
	\label{fig:net:share:dir}
\end{image}

Административные общие папки (шары, ресурсы) в Windows используются
для удаленного доступа и управление компьютером.
Windows по-умолчанию создает следующие административные шары:

\begin{itemize}
	\item \verb|Admin$| --- Remote Admin (это каталог %SystemRoot% )
		--- используется для удаленного администрирования компьютера;
	\item \verb|IPC$|
		--- Remote IPC --- используется для коммуникации программ named pipes);
	\item \verb|C$| --- Default Share. Расшаренный системный диск.
		Если на компьютере есть другие диски с назначенными буквами,
		они также автоматически публикуются
		в виде административных общих ресурсов (\verb|D$| , \verb|E$| и т.д.);
	\item \verb|Print$| --- публикуется, если вы открыли общий доступ
		к принтеру (используется, чтобы открыть доступ к каталогу
		с драйверами принтеров \verb|C:\Windows\system32\spool\drivers|);
	\item \verb|FAX$| --- используется для общего доступа к факс-серверу.
\end{itemize}


Чтобы отобразить на хост-компьютере содержимое какой либо
директории гостевой машины. Для начала нужно установить 
VM VirtualBox Extension Pack.\par
Guest Additions --- это набор драйверов
и утилит для лучшей интеграции между хостом и гостевой ОС.

\begin{itemize}
	\item Запустите виртуальную машину с Windows 7.
	\item В меню VirtualBox выберите
		Devices → Insert Guest Additions CD Image....
	\item Откройте этот диск в гостевой системе (Windows)
		и установите Guest Additions. Следуйте инструкциям по установке.
	\item После завершения установки перезагрузите виртуальную машину.
\end{itemize}

Перейдем в VirtualBox \textbf{<<Devices>>
\rarrow <<Insert Guest Additions CD image...>>}.
Далее в CD-дисководе гостевой машины появится инсталятор
\rref{fig:cd:installator}.

\begin{image}
	\includegrph{Screenshot from 2024-09-16 18-39-47}
	\caption{CD-дисковод}
	\label{fig:cd:installator}
\end{image}

Нажмем на него и произведем установку
\rdref{fig:cd:install:1}{fig:cd:install:6}.

\begin{image}
	\includegrph{Screenshot from 2024-09-16 18-39-55}
	\caption{Установка Guest Additions CD image, часть 1}
	\label{fig:cd:install:1}
\end{image}

\begin{image}
	\includegrph{Screenshot from 2024-09-16 18-40-06}
	\caption{Установка Guest Additions CD image, часть 2}
	\label{fig:cd:install:2}
\end{image}

\begin{image}
	\includegrph{Screenshot from 2024-09-16 18-40-11}
	\caption{Установка Guest Additions CD image, часть 3}
	\label{fig:cd:install:3}
\end{image}

\begin{image}
	\includegrph{Screenshot from 2024-09-16 18-40-15}
	\caption{Установка Guest Additions CD image, часть 4}
	\label{fig:cd:install:4}
\end{image}

\begin{image}
	\includegrph{Screenshot from 2024-09-16 18-41-06}
	\caption{Установка Guest Additions CD image, часть 5}
	\label{fig:cd:install:5}
\end{image}

\begin{image}
	\includegrph{Screenshot from 2024-09-16 18-42-24}
	\caption{Установка Guest Additions CD image, часть 6}
	\label{fig:cd:install:6}
\end{image}

Далее можем через VirtualBox создать общую директорию \rref{fig:shared:dir}.

\begin{image}
	\includegrph{Screenshot from 2024-09-16 18-47-21}
	\caption{Создание общей директории в VirtualBox}
	\label{fig:shared:dir}
\end{image}

Чтобы проверить директории, создадим в ней файл \rref{fig:shared:dir:file}.

\begin{image}
	\includegrph{Screenshot from 2024-09-16 18-56-37}
	\caption{Создание файла в общей директории}
	\label{fig:shared:dir:file}
\end{image}

Теперь откроем гостевую машину и увидим рядом с диском \texttt{C:}
подключенную общую директорию, в которой лежит созданный ранее файл
\rref{fig:shared:dir:file:win}.

\begin{image}
	\includegrph{Screenshot from 2024-09-16 18-56-58}
	\caption{Общая директория в гостевой машине}
	\label{fig:shared:dir:file:win}
\end{image}

\subsection{Общая директория}

Создадим на гостевом диске \texttt{С:} каталог \texttt{Install},
с помощью контекстного меню предоставьте общий доступ
к этому каталогу для группы <<Все>> (кнопка <<Разрешения>>)
с правом только на чтение \rref{fig:dir:inst:r}.

\begin{image}
	\includegrph{Screenshot from 2024-09-16 19-05-50}
	\caption{Устанавливаем права только на чтение}
	\label{fig:dir:inst:r}
\end{image}

Подключим этот каталог в хостовой машине. Перейдем в \textbf{Other Location}
в поле \textbf{Connect to Sertver} введем
\texttt{smb://192.168.56.101/install/}.
Далее потребуется ввести логин и пароль пользователя в гостевой машне
\rref{fig:dir:inst:open}.

\begin{image}
	\includegrph{Screenshot from 2024-09-16 19-00-46}
	\caption{Подключение каталога install}
	\label{fig:dir:inst:open}
\end{image}

В проводнике хост-компьютера попробуем создать
в общем каталоге Install каталог
\rdref{fig:dir:inst:r:create:folder}{fig:dir:inst:r:create:folder:error}.

\begin{image}
	\includegrph{Screenshot from 2024-09-16 19-09-25}
	\caption{Создание каталога в Install}
	\label{fig:dir:inst:r:create:folder}
\end{image}

\begin{image}
	\includegrph{Screenshot from 2024-09-16 19-09-36}
	\caption{Ошибка при создании каталога}
	\label{fig:dir:inst:r:create:folder:error}
\end{image}

\textit{У нас это не получится сделать,
так как ранее мы установили права только на чтение.}

Вернитесь в настройки общего доступа папки Install на гостевой машине
и добавьте группе <<Все>> право на изменение содержимого этой папки
\rref{fig:dir:inst:rw}.

\begin{image}
	\includegrph{Screenshot from 2024-09-16 19-10-17}
	\caption{Установка прав на чтение и запись}
	\label{fig:dir:inst:rw}
\end{image}

После этого каталог успешно создается \rref{fig:dir:inst:rw:create:folder}.

\begin{image}
	\includegrph{Screenshot from 2024-09-16 19-11-27}
	\caption{Успешное создание каталога в Install}
	\label{fig:dir:inst:rw:create:folder}
\end{image}

Далее в настройках сетевого доступа директории Install
установим для <<Все>> права только для чтения \rref{fig:dir:inst:r:2}.

\begin{image}
	\includegrph{Screenshot from 2024-09-16 19-21-21}
	\caption{Удаление прав на запись у Install}
	\label{fig:dir:inst:r:2}
\end{image}

Теперь при попытке удалить созданную ранее директрию в Install
возникает ошибка \rref{fig:dir:inst:r:2:error}.

\begin{image}
	\includegrph{Screenshot from 2024-09-16 19-22-46}
	\caption{Попытка удать каталог из Install без прав на запись}
	\label{fig:dir:inst:r:2:error}
\end{image}

\subsection{Безопасность каталока Install}

Откройте вкладку <<Безопасность>> в свойствах папки Install
\rref{fig:dir:satery}.

\begin{image}
	\includegrph{Screenshot from 2024-09-17 20-11-41}
	\caption{Свойство <<Безопасность>> каталога Install}
	\label{fig:dir:satery}
\end{image}

\textit{Свойства <<Безопасность>> у каталога в Windows
позволяют настроить параметры доступа к нему.}

Если одному и тому же пользователю (или группе) дать разные права доступа
к одному и тому же ресурсу во вкладках <<Доступ>> и <<Безопасность>>
приоритетнее будут настройки безопасности.\par
Для проверки приоритета зададим во вкладке <<Безопасность>>
отнимем все права доступа для пользователя \textbf{BAR},
а во вкладке <<Доступ>> додим доступ для чтения и записи
\rdref{fig:dir:prop:satery}{fig:dir:prop}.

\begin{image}
	\includegrph{Screenshot from 2024-09-17 20-13-59}
	\caption{Вкладка <<Безопасность>>}
	\label{fig:dir:prop:satery}
\end{image}

\begin{image}
	\includegrph{Screenshot from 2024-09-17 20-14-26}
	\caption{Вкладка <<Доступ>>}
	\label{fig:dir:prop}
\end{image}

В результате у нас пропал доспуп к каталогку\texttt{Install}
на хостовой машине \rref{fig:dir:install:error}.
Это подтверждает, что права безопасности приоритетнее.

\begin{image}
	\includegrph{Screenshot from 2024-09-17 20-15-07}
	\caption{Отсутствие доступа к каталогу Install}
	\label{fig:dir:install:error}
\end{image}

\section{Настройка подключения к сети Интернет}

Проверим наличие подключения к сети Интернет у гостевого компьютера,
открыв на нём во встроенном браузере Internet Explorer
ресурс \url{http://lib.ru} \rref{fig:lib}.

\begin{image}
	\includegrph{Screenshot from 2024-09-17 20-25-51}
	\caption{Провера наличия подключения к сети Интернет}
	\label{fig:lib}
\end{image}

\subsection{netstat}

Не закрывая браузер, с помощью консольной утилиты
\texttt{netstat} проанализируйте действующие сетевые подключения виртуального
компьютера \rref{fig:netstat}.

\begin{image}
	\includegrph{Screenshot from 2024-09-17 20-25-51}
	\caption{Вывод команды netstat}
	\label{fig:netstat}
\end{image}

\subsection{tracert}

С помощью консольной утилиты \texttt{tracert} определите путь
и количество хопов (15 штук) до узла \url{http://lib.ru}
\rref{fig:tracert}.

\begin{image}
	\includegrph{Screenshot from 2024-09-17 20-33-41}
	\caption{Вывод команды tracert}
	\label{fig:tracert}
\end{image}

Cреднее время перемещения IP-пакетов от нашей гостевой станции
до указанного узла составляет 7 наносекунд.


Протестируйте Интернет-подключение виртуального компьютера с помощью сайта
\url{http://speedtest.net/} \rref{fig:speedtest}.

\begin{image}
	\includegrph{Screenshot from 2024-09-17 20-35-26}
	\caption{Вывод сайте speedtest}
	\label{fig:speedtest}
\end{image}

\subsection{speedtest}

По выводу speedtest можем определить:

\begin{itemize}
	\item время прохождения ping --- 7 наносекунд;
	\item скорость входящего и исходящего трафика
		составляет 45.88 и 42.43 Мб/с соответственно;
	\item публичный IP-адрес подключения -- 109.252.46.67.
\end{itemize}

\section{Удалённое управление рабочей станцией}

\subsection{telnet}

Управление компьютером с помощью консольной утилиты \texttt{telnet}.
Проверьте на хост-компьютере наличие компонента telnet, одноимённую команду
в командной строке. При необходимости добавьте Telnet-клиент
(\textbf{Панель управления \rarrow Программы \rarrow Программы и компоненты
\rarrow <<Включение и отключение компонентов Windows>>})
\rdref{fig:telnet:client:on}{fig:telnet:server:on}.

\begin{image}
	\includegrph{Screenshot from 2024-09-17 20-39-04}
	\caption{Добавление Telnetол}
	\label{fig:telnet:client:on}
\end{image}

\begin{image}
	\includegrph{Screenshot from 2024-09-17 20-51-19}
	\caption{Добавление Telnetол}
	\label{fig:telnet:server:on}
\end{image}

На гостевом компьютере проверьте наличие и активность системной службы Telnet
(\textbf{Панель управления \rarrow Администрирование (Инструменты Windows)
\rarrow Службы}), в случае отсутствия доустановите её
(\textbf{Панель управления \rarrow Установка и удаление программ
(Программы и компоненты) \rarrow <<Установка компонентов Windows>>
(Включение или отключение компонентов Windows)})
\rdref{fig:telnet:demon:start}{fig:telnet:demon:prop}.

\begin{image}
	\includegrph{Screenshot from 2024-09-17 20-56-24}
	\caption{Запуск службы telnet}
	\label{fig:telnet:demon:start}
\end{image}

\begin{image}
	\includegrph{Screenshot from 2024-09-17 20-59-14}
	\caption{Свойства службы telnet}
	\label{fig:telnet:demon:prop}
\end{image}

Проверим наличие компонента \texttt{telnet} на гостевой машине,
добавьте пользователя в группу \texttt{TelnetClients}
(\textbf{Управление компьютером \rarrow Локальные пользователи
\rarrow Группы}) \rref{fig:telnet:group}.

\begin{image}
	\includegrph{Screenshot from 2024-09-17 21-07-44}
	\caption{Добавление пользователя в группу}
	\label{fig:telnet:group}
\end{image}


На хост-компьютере в командной строке введем команду:
\texttt{telnet open <IP-адрес\_вирт\_комп> 23}.
И по запросу консоли введем имя учетной записи
и пароль локального администратора гостевого компьютера.
\rref{fig:telnet:connection}.

\begin{image}
	\includegrph{Screenshot from 2024-09-22 18-39-32}
	\caption{Создание подключения Telnet}
	\label{fig:telnet:connection}
\end{image}

\subsection{Удаленный рабочий стол}

Разрешите на гостевом компьютере (в свойствах системы) удаленное управление
(\textbf{Панель управления \rarrow Система
\rarrow Настройка удаленного доступа})
\rref{fig:remote:desctop:on}.

\begin{image}
	\includegrph{Screenshot from 2024-09-22 18-39-32}
	\caption{Настройка удаленного доступа}
	\label{fig:remote:desctop:on}
\end{image}

Так как на Linux нет программы <<Подключение к удаленному рабочему столу>>,
а также для работы удаленного стола требуется VirtualBox External Pack,
который у меня при установке вызывал ошибку в виртуальных машинах,
данную часть у меня выполнить не удалось.

\subsection{AnyDesk}

Запустим программу удаленного управления AnyDesk на обоих компьютерах
(на виртуальной машине она была загружена в каталог \verb|C:\Install|)
\rdref{fig:anydesk}{fig:anydesk:run}.

\begin{image}
	\includegrph{Screenshot from 2024-09-22 17-59-22}
	\caption{Каталог Install с программой AnyDesk}
	\label{fig:anydesk}
\end{image}

\begin{image}
	\includegrph{Screenshot from 2024-09-22 18-00-40}
	\caption{Открытое приложение AnyDesk}
	\label{fig:anydesk:start}
\end{image}

\begin{image}
	\includegrph{Screenshot from 2024-09-22 18-05-21}
	\caption{Запущеное приложение AnyDesk}
	\label{fig:anydesk:run}
\end{image}

Программа AnyDesk может работать без подключения к интернету,
если оба устройства находятся в одной локальной сети
(например, Wi-Fi или LAN).

\clearpage

\section*{ВЫВОД}
\addcontentsline{toc}{section}{ВЫВОД}

В ходе работы была выполнена настройка имени компьютера и флагов доступа,
что позволило корректно интегрировать рабочую станцию в локальную сеть.
Также был задан IP-адрес, обеспечивающий уникальность машины в сети.\par
с помощью эхо-запросов (команда `ping`)
для проверки доступности других узлов сети.
Помимо этого, была проведена проверка видимости виртуальных машин через
их DNS-имена, что позволило убедиться в корректной работе DNS-сервиса.\par
что дало представление о топологии сети и взаимодействии устройств внутри неё.
Это помогло лучше понять физическую и логическую структуру сети.\par
Проведена настройка и управление общей директорией
с проверкой прав доступа для пользователей.
Также были изучены методы обеспечения безопасности каталогов,
в частности, каталога Install.\par
к сети Интернет использовались команды \texttt{netstat}, \texttt{tracert}
и \texttt{speedtest}. Это позволило увидеть текущие соединения,
маршруты передачи данных и оценить скорость интернет-соединения.\par
рабочей станцией, включая использование Telnet,
удаленного рабочего стола и программы AnyDesk.
Это обеспечило возможность удалённого администрирования
и управления рабочей станцией.

