\section*{\LARGE Цель практической работы}
\addcontentsline{toc}{section}{Цель практической работы}

\textbf{Цель работы}: познакомиться с возможностями систем виртуализации
на примере гипервизора VirtulaBox,
приобрести умения и навыки установки и настройки гостевых виртуальных машин.

\clearpage

\section*{\LARGE Выполнение практической работы}
\addcontentsline{toc}{section}{Выполнение практической работы}

\section{Менеджер виртуальных машин}

В качестве менеджера виртуальных машин для выполнения практической работы
было использован VirtulaBox. Так как с VMware Player возникли проблемы
с установкой на Linux.

\clearpage

\section{Создание и установка гостевого компьютера}

\subsection{Создание виртуальной машины}

Создадим новую виртуальную машину.
В диалоговом окне введем название машины
, расположение и путь к Installer disk image file (.iso)
(Рисунок~\ref{fig:vm:create}).

\begin{image}
	\includegrph{Screenshot from 2024-09-08 19-50-04}
	\caption{Первое диалоговое окно}
	\label{fig:vm:create}
\end{image}

В следующем окне введем регистрационный ключ для Windows, а также имя
первого пользователя (arbon) и произвольный (непустой) пароль
для этого пользователя (Рисунок~\ref{fig:vm:user}).

\begin{image}
	\includegrph{Screenshot from 2024-09-08 19-50-21}
	\caption{Имя первого пользователя}
	\label{fig:vm:user}
\end{image}

В следующем окне выберем объём оперативной памяти и количество ядер.
В разделе Base Memory установим объем оперативной памяти будущего
виртуального компьютера --- не менее 1024 Мб (Рисунок~\ref{fig:vm:hardware}).

\begin{image}
	\includegrph{Screenshot from 2024-09-08 19-50-33}
	\caption{Объём оперативной памяти и количество ядер}
	\label{fig:vm:hardware}
\end{image}

В следующем окне зададим объём жесткого диска гостевого компьютера --- 60 Гб
(Рисунок~\ref{fig:vm:harddisk}).

\begin{image}
	\includegrph{Screenshot from 2024-09-08 19-50-33}
	\caption{Объём оперативной памяти и количество ядер}
	\label{fig:vm:harddisk}
\end{image}

\subsection{Запуск виртуальной машины}

Запустите новый виртуальный компьютер.
Проследите за ходом установки гостевой ОС Microsoft Windows,
укажите в отчёте основные этапы (Рисунок~\ref{fig:microsoft:load}).

\begin{image}
	\includegrph{Screenshot from 2024-09-08 19-52-02}
	\caption{Установка ОС Microsoft Windows}
	\label{fig:microsoft:load}
\end{image}

\clearpage

\section{Начальная настройка гостевой операционной системы (ОС)}

\subsection{Настройка системных параметров гостевого компьютера}

Сделал правый щелчок на свободной области
\textbf{Рабочего Стола $\rightarrow$ Разрешение} экрана.
Установил разрешение 800х600 (Рисунок~\ref{fig:setting:screen}).

\begin{image}
	\includegrph{Screenshot from 2024-09-08 20-16-33}
	\caption{Разрешение экрана}
	\label{fig:setting:screen}
\end{image}

\subsection{Создание значков на Рабочем столе}

Разместил на Рабочем столе значки <<Мой компьютер>> и <<Файлы пользователя>>
(Рисунок~\ref{fig:setting:icon}).

\begin{image}
	\includegrph{Screenshot from 2024-09-08 20-19-46}
	\caption{Создание значков}
	\label{fig:setting:icon}
\end{image}

На панели задач отобразил панель быстрого запуска,
для этого нажал по панели задач правой кнопкой мыши
и выберал \textbf{Панели $\rightarrow$ Создать} панель инструментов.
При выборе папки в поиске ввел
\verb|%appdata%\Microsoft\Internet Explorer\Quick Launch|
и выберал каталог (Рисунок~\ref{fig:setting:quicklaunch}).

\begin{image}
	\includegrph{Screenshot from 2024-09-08 20-25-33}
	\caption{Отображение панель быстрого запуска}
	\label{fig:setting:quicklaunch}
\end{image}

\subsection{Настрока оформления и персонализации}

В \textbf{Панели управления} перешел в раздел
\textbf{Оформление и персонализация}, выберал
\textbf{Параметры папок $\rightarrow$ Вид},
снял флажки \textit{<<Использовать мастер общего доступа>>} и
\textit{<<Скрывать расширения для зарегистрированных типов файлов>>}.
Здесь же включил отображение скрытых и системных файлов
(Рисунок~\ref{fig:setting:personalization:flags}).

\begin{image}
	\includegrph{Screenshot from 2024-09-08 20-27-28}
	\caption{Изменение параметров папок}
	\label{fig:setting:personalization:flags}
\end{image}

В меню \textbf{Пуск} виртуального компьютера выберите
\textbf{Панель управления}, далее
\textbf{Оформление и персонализация $\rightarrow$ Персонализация}
и переключился на \textit{классическую} тему оформления
(Рисунок~\ref{fig:setting:personalization:theme}).

\begin{image}
	\includegrph{Screenshot from 2024-09-08 20-28-04}
	\caption{Выбор классической темы}
	\label{fig:setting:personalization:theme}
\end{image}

\subsection{Настрока системы и безопасности}

Далее \textbf{Система и безопасность $\rightarrow$ Система $\rightarrow$
Дополнительные параметры системы $\rightarrow$
Быстродействие $\rightarrow$ Параметры}.
Выберите \textit{Обеспечить наилучшее быстродействие}.
Внешний вид интерфейса упростился, но скорость работы программ увеличилась
(Рисунок~\ref{fig:setting:sys:speed}).

\begin{image}
	\includegrph{Screenshot from 2024-09-08 20-41-07}
	\caption{Настройка быстродействия}
	\label{fig:setting:sys:speed}
\end{image}


В свойствах гостевой ОС
(\textbf{Система и безопасность $\rightarrow$ Система
$\rightarrow$ Дополнительные параметры системы}):

\begin{itemize}
	\item включил автоматическую перезагрузку при отказе,
		отменил создание дампа памяти при перезагрузке
		(Рисунок~\ref{fig:setting:sys:reebot:damp:report});
	\item отключил отчет о программных ошибках
		(Рисунок~\ref{fig:setting:sys:reebot:damp:report});
	\item запретил отправку приглашения удаленному помощнику
		(Рисунок~\ref{fig:setting:sys:assistant});
\end{itemize}

\begin{image}
	\includegrph{Screenshot from 2024-09-08 21-01-39}
	\caption{Включение перезагрузки, отмена дампа и отключение отчета}
	\label{fig:setting:sys:reebot:damp:report}
\end{image}

\begin{image}
	\includegrph{Screenshot from 2024-09-08 20-58-26}
	\caption{Запрет отправки приглашения}
	\label{fig:setting:sys:assistant}
\end{image}

В Центре обновления Windows отключите автоматическое обновление системы
(Рисунок~\ref{fig:setting:sys:update}).

\begin{image}
	\includegrph{Screenshot from 2024-09-08 21-00-20}
	\caption{Отключение автоматического обновления системы}
	\label{fig:setting:sys:update}
\end{image}

\subsection{Изменение значения переменных}

В свойствах гостевой ОС изменил значения переменных среды
текущего пользователя \textit{TEMP}
и \textit{TMP}\footnote{
TEMP или TMP указывает путь к временным каталогам, принятым по умолчанию.
Эти каталоги используются приложениями,
доступными зарегистрированному в системе пользователю.
Некоторым приложениям требуется значение переменной TEMP,
в то время как другим --- TMP
(системная и пользовательская переменные соответственно).
}

на новое значение \verb|C:\Temp| (Рисунок~\ref{fig:setting:sys:tmp}).

\begin{image}
	\includegrph{Screenshot from 2024-09-08 21-02-44}
	\caption{Изменение значения переменных среды}
	\label{fig:setting:sys:tmp}
\end{image}

Создал соответствующий каталог \textit{Temp}
на диске \textit{С:} гостевого компьютера, после
чего перезагрузил его с помощью команды
\textbf{Пуск $\rightarrow$ Завершение работы}
(чтобы изменения вступили в силу) (Рисунок~\ref{fig:setting:sys:tmp}).

\begin{image}
	\includegrph{Screenshot from 2024-09-08 21-04-14}
	\caption{Создание каталога \textit{Temp} на \textit{C:}}
	\label{fig:setting:sys:tmp}
\end{image}

\subsection{Windows Audio и Удалённый реестр}

Открыл в \textbf{Панели управления
$\rightarrow$ Администрирование $\rightarrow$ Службы}.
Остановил и отключил службы <<Windows Audio>>\footnote{
Отключение <<Windows Audio>> приведет лишь к отключению 
на устройстве работы звука.
} (Рисунок~\ref{fig:setting:services:windowsaudio})
и <<Удалённый реестр>>\footnote{
<<Удалённый реестр>> позволяет изменять реестр Windows через сеть.
Полезен для администрирования, но увеличивает риск безопасности.
Рекомендуется отключать, если не используется.
} (Рисунок~\ref{fig:setting:services:registry}).

\begin{image}
	\includegrph{Screenshot from 2024-09-10 09-59-58}
	\caption{Отключил Windows Audio}
	\label{fig:setting:services:windowsaudio}
\end{image}

\begin{image}
	\includegrph{Screenshot from 2024-09-10 10-00-35}
	\caption{Отключил Удалённый реестр}
	\label{fig:setting:services:registry}
\end{image}

\subsection{Список оборудования гостевого компьютера}

С помощью \textbf{<<Диспетчера устройств>>}
(\textbf{Панель управления $\rightarrow$ Оборудование и звук
$\rightarrow$ Диспетчер устройств}) просмотрел список оборудования
гостевого компьютера (Рисунок~\ref{fig:setting:devices}).

\begin{image}
	\includegrph{Screenshot from 2024-09-10 10-02-15}
	\caption{Список оборудования гостевого компьютера}
	\label{fig:setting:devices}
\end{image}

\textbf{Укажите в отчёте, есть ли неопределённое системой оборудование???}

\clearpage

\section{Расширенные средства администрирования ОС Windows}
\section{Дополнительные настройки программы VMware Player}

\clearpage

\section*{\LARGE Вывод}
\addcontentsline{toc}{section}{Вывод}

