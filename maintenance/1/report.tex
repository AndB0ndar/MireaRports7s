\section*{\LARGE ЦЕЛЬ ПРАКТИЧЕСКОЙ РАБОТЫ}
\addcontentsline{toc}{section}{ЦЕЛЬ ПРАКТИЧЕСКОЙ РАБОТЫ}

\textbf{Цель работы}: познакомиться с возможностями систем виртуализации
на примере гипервизора VirtulaBox,
приобрести умения и навыки установки и настройки гостевых виртуальных машин.

\clearpage

\section*{\LARGE ВЫПОЛНЕНИЕ ПРАКТИЧЕСКОЙ РАБОТЫ}
\addcontentsline{toc}{section}{ВЫПОЛНЕНИЕ ПРАКТИЧЕСКОЙ РАБОТЫ}

\section{Менеджер виртуальных машин}

В качестве менеджера виртуальных машин для выполнения практической работы
было использован VirtulaBox. Так как с VMware Player возникли проблемы
с установкой на Linux.

\clearpage

\section{Создание и установка гостевого компьютера}

\subsection{Создание виртуальной машины}

Создадим новую виртуальную машину.
В диалоговом окне введем название машины
, расположение и путь к Installer disk image file (.iso)
(Рисунок~\ref{fig:vm:create}).

\begin{image}
	\includegrph{Screenshot from 2024-09-08 19-50-04}
	\caption{Первое диалоговое окно}
	\label{fig:vm:create}
\end{image}

В следующем окне введем регистрационный ключ для Windows, а также имя
первого пользователя (arbon) и произвольный (непустой) пароль
для этого пользователя (Рисунок~\ref{fig:vm:user}).

\begin{image}
	\includegrph{Screenshot from 2024-09-08 19-50-21}
	\caption{Имя первого пользователя}
	\label{fig:vm:user}
\end{image}

В следующем окне выберем объём оперативной памяти и количество ядер.
В разделе Base Memory установим объем оперативной памяти будущего
виртуального компьютера --- не менее 1024 Мб (Рисунок~\ref{fig:vm:hardware}).

\begin{image}
	\includegrph{Screenshot from 2024-09-08 19-50-33}
	\caption{Объём оперативной памяти и количество ядер}
	\label{fig:vm:hardware}
\end{image}

В следующем окне зададим объём жесткого диска гостевого компьютера --- 60 Гб
(Рисунок~\ref{fig:vm:harddisk}).

\begin{image}
	\includegrph{Screenshot from 2024-09-08 19-50-33}
	\caption{Объём оперативной памяти и количество ядер}
	\label{fig:vm:harddisk}
\end{image}

\clearpage

\subsection{Запуск виртуальной машины}

Запустите новый виртуальный компьютер.
Проследите за ходом установки гостевой ОС Microsoft Windows,
укажите в отчёте основные этапы (Рисунок~\ref{fig:microsoft:load}).

\begin{image}
	\includegrph{Screenshot from 2024-09-08 19-52-02}
	\caption{Установка ОС Microsoft Windows}
	\label{fig:microsoft:load}
\end{image}

\clearpage

\section{Начальная настройка гостевой операционной системы (ОС)}

\subsection{Настройка системных параметров гостевого компьютера}

Сделал правый щелчок на свободной области
\textbf{Рабочего Стола $\rightarrow$ Разрешение} экрана.
Установил разрешение 800х600 (Рисунок~\ref{fig:setting:screen}).

\begin{image}
	\includegrph{Screenshot from 2024-09-08 20-16-33}
	\caption{Разрешение экрана}
	\label{fig:setting:screen}
\end{image}

\clearpage

\subsection{Создание значков на Рабочем столе}

Разместил на Рабочем столе значки <<Мой компьютер>> и <<Файлы пользователя>>
(Рисунок~\ref{fig:setting:icon}).

\begin{image}
	\includegrph{Screenshot from 2024-09-08 20-19-46}
	\caption{Создание значков}
	\label{fig:setting:icon}
\end{image}

На панели задач отобразил панель быстрого запуска,
для этого нажал по панели задач правой кнопкой мыши
и выберал \textbf{Панели $\rightarrow$ Создать} панель инструментов.
При выборе папки в поиске ввел
\verb|%appdata%\Microsoft\Internet Explorer\Quick Launch|
и выберал каталог (Рисунок~\ref{fig:setting:quicklaunch}).

\begin{image}
	\includegrph{Screenshot from 2024-09-08 20-25-33}
	\caption{Отображение панель быстрого запуска}
	\label{fig:setting:quicklaunch}
\end{image}

\clearpage

\subsection{Настрока оформления и персонализации}

В \textbf{Панели управления} перешел в раздел
\textbf{Оформление и персонализация}, выберал
\textbf{Параметры папок $\rightarrow$ Вид},
снял флажки \textit{<<Использовать мастер общего доступа>>} и
\textit{<<Скрывать расширения для зарегистрированных типов файлов>>}.
Здесь же включил отображение скрытых и системных файлов
(Рисунок~\ref{fig:setting:personalization:flags}).

\begin{image}
	\includegrph{Screenshot from 2024-09-08 20-27-28}
	\caption{Изменение параметров папок}
	\label{fig:setting:personalization:flags}
\end{image}

В меню \textbf{Пуск} виртуального компьютера выберите
\textbf{Панель управления}, далее
\textbf{Оформление и персонализация $\rightarrow$ Персонализация}
и переключился на \textit{классическую} тему оформления
(Рисунок~\ref{fig:setting:personalization:theme}).

\begin{image}
	\includegrph{Screenshot from 2024-09-08 20-28-04}
	\caption{Выбор классической темы}
	\label{fig:setting:personalization:theme}
\end{image}

\clearpage

\subsection{Настрока системы и безопасности}

Далее \textbf{Система и безопасность $\rightarrow$ Система $\rightarrow$
Дополнительные параметры системы $\rightarrow$
Быстродействие $\rightarrow$ Параметры}.
Выберите \textit{Обеспечить наилучшее быстродействие}.
Внешний вид интерфейса упростился, но скорость работы программ увеличилась
(Рисунок~\ref{fig:setting:sys:speed}).

\begin{image}
	\includegrph{Screenshot from 2024-09-08 20-41-07}
	\caption{Настройка быстродействия}
	\label{fig:setting:sys:speed}
\end{image}


В свойствах гостевой ОС
(\textbf{Система и безопасность $\rightarrow$ Система
$\rightarrow$ Дополнительные параметры системы}):

\begin{itemize}
	\item включил автоматическую перезагрузку при отказе,
		отменил создание дампа памяти при перезагрузке
		(Рисунок~\ref{fig:setting:sys:reebot:damp:report});
	\item отключил отчет о программных ошибках
		(Рисунок~\ref{fig:setting:sys:reebot:damp:report});
	\item запретил отправку приглашения удаленному помощнику
		(Рисунок~\ref{fig:setting:sys:assistant});
\end{itemize}

\begin{image}
	\includegrph{Screenshot from 2024-09-08 21-01-39}
	\caption{Включение перезагрузки, отмена дампа и отключение отчета}
	\label{fig:setting:sys:reebot:damp:report}
\end{image}

\begin{image}
	\includegrph{Screenshot from 2024-09-08 20-58-26}
	\caption{Запрет отправки приглашения}
	\label{fig:setting:sys:assistant}
\end{image}

В Центре обновления Windows отключите автоматическое обновление системы
(Рисунок~\ref{fig:setting:sys:update}).

\begin{image}
	\includegrph{Screenshot from 2024-09-08 21-00-20}
	\caption{Отключение автоматического обновления системы}
	\label{fig:setting:sys:update}
\end{image}

\clearpage

\subsection{Изменение значения переменных}

В свойствах гостевой ОС изменил значения переменных среды
текущего пользователя \textit{TEMP}
и \textit{TMP}\footnote{
TEMP или TMP указывает путь к временным каталогам, принятым по умолчанию.
Эти каталоги используются приложениями,
доступными зарегистрированному в системе пользователю.
Некоторым приложениям требуется значение переменной TEMP,
в то время как другим --- TMP
(системная и пользовательская переменные соответственно).
}
на новое значение \verb|C:\Temp| (Рисунок~\ref{fig:setting:sys:tmp}).

\begin{image}
	\includegrph{Screenshot from 2024-09-08 21-02-44}
	\caption{Изменение значения переменных среды}
	\label{fig:setting:sys:tmp}
\end{image}

Создал соответствующий каталог \textit{Temp}
на диске \textit{С:} гостевого компьютера, после
чего перезагрузил его с помощью команды
\textbf{Пуск $\rightarrow$ Завершение работы}
(чтобы изменения вступили в силу) (Рисунок~\ref{fig:setting:sys:tmp}).

\begin{image}
	\includegrph{Screenshot from 2024-09-08 21-04-14}
	\caption{Создание каталога \textit{Temp} на \textit{C:}}
	\label{fig:setting:sys:tmp}
\end{image}

\clearpage

\subsection{Windows Audio и Удалённый реестр}

Открыл в \textbf{Панели управления
$\rightarrow$ Администрирование $\rightarrow$ Службы}.
Остановил и отключил службы <<Windows Audio>>\footnote{
Отключение <<Windows Audio>> приведет лишь к отключению 
на устройстве работы звука.
} (Рисунок~\ref{fig:setting:services:windowsaudio})
и <<Удалённый реестр>>\footnote{
<<Удалённый реестр>> позволяет изменять реестр Windows через сеть.
Полезен для администрирования, но увеличивает риск безопасности.
Рекомендуется отключать, если не используется.
} (Рисунок~\ref{fig:setting:services:registry}).

\begin{image}
	\includegrph{Screenshot from 2024-09-10 09-59-58}
	\caption{Отключил Windows Audio}
	\label{fig:setting:services:windowsaudio}
\end{image}

\begin{image}
	\includegrph{Screenshot from 2024-09-10 10-00-35}
	\caption{Отключил Удалённый реестр}
	\label{fig:setting:services:registry}
\end{image}

\clearpage

\subsection{Список оборудования гостевого компьютера}

С помощью \textbf{<<Диспетчера устройств>>}
(\textbf{Панель управления $\rightarrow$ Оборудование и звук
$\rightarrow$ Диспетчер устройств}) просмотрел список оборудования
гостевого компьютера (Рисунок~\ref{fig:setting:devices}).

\begin{image}
	\includegrph{Screenshot from 2024-09-10 10-02-15}
	\caption{Список оборудования гостевого компьютера}
	\label{fig:setting:devices}
\end{image}

В списоке оборудования есть одно неопределённое системой оборудование.

\clearpage

\section{Расширенные средства администрирования ОС Windows}

\subsection{Файл подкачки}

В свойствах системы
(\textbf{Система и безопасность $\rightarrow$ Система
$\rightarrow$ Дополнительные параметры системы $\rightarrow$ Быстродействие
$\rightarrow$ Параметры $\rightarrow$ Виртуальная память
$\rightarrow$ Изменить})
определил текущие параметры файла подкачки (расположение и размер)
(Рисунок~\ref{fig:setting:swap}).

\begin{image}
	\includegrph{Screenshot from 2024-09-10 10-04-27}
	\caption{Параметры файла подкачки}
	\label{fig:setting:swap}
\end{image}

Нашел на гостевом компьютере сам файл подкачки
(Рисунок~\ref{fig:setting:swap:locate}).

\begin{image}
	\includegrph{Screenshot from 2024-09-10 10-05-43}
	\caption{Файл подкачки}
	\label{fig:setting:swap:locate}
\end{image}

\clearpage

\subsection{Удаление содержимого файла подкачки}

В редакторе реестра Windows настроил, чтобы содержимое файла подкачки
при каждом выключении компьютера автоматически удалялось
(это позволит уменьшить размер папки с образом виртуальной машины).\par
Для запуска редактора реестра нажал \texttt{Win+R},
в поле ввода набрал Regedit (или Regedt32).
Присвоил значение 1 параметру (ключу) \texttt{ClearPageFileAtShutdown},
расположенному в разделе реестра (Рисунок~\ref{fig:setting:swap:remove}).\par
Перезагрузил гостевой компьютер.

\begin{image}
	\includegrph{Screenshot from 2024-09-10 10-11-16}
	\caption{Изменение ClearPageFileAtShutdown}
	\label{fig:setting:swap:remove}
\end{image}

\clearpage

\subsection{Знакомства с возможностями расширенной настройки}

Познакомимся с возможностями специального программного средства для
расширенной настройки ОС Windows --- консоли управления Microsoft.
Нажал \texttt{Win+R} и наберал \texttt{mmc}
откроется окно Microsoft Management Console.
Исполняемый файл этой программы лежит по пути: \verb|C:\Windows\System32|.

Выбрал \textbf{<<Файл>> $\rightarrow$ <<Добавить или удалить оснастку>>
$\rightarrow$ <<Редактор объекта групповой политики>>
$\rightarrow$ <<Добавить>>}.
Далее в ветке \textbf{<<Конфигурация пользователя>>
$\rightarrow$ <<Административные шаблоны>>
$\rightarrow$ <<Панель управления>> $\rightarrow$ <<Персонализация>>}.
Здесь активизировал запрет изменения фонового рисунка Рабочего Стола (Рисунок~\ref{fig:setting:pers:picture}).

\begin{image}
	\includegrph{Screenshot from 2024-09-10 10-16-19}
	\caption{Запрет изменения фонового рисунка}
	\label{fig:setting:pers:picture}
\end{image}

Проверил результат после перезагрузки гостевой машины
(Рисунок~\ref{fig:setting:pers:picture:change}).

\begin{image}
	\includegrph{Screenshot from 2024-09-10 10-17-14}
	\caption{Изменение фонового рисунка}
	\label{fig:setting:pers:picture:change}
\end{image}

Аналогично отключил отображение Корзины на Рабочем Столе
виртуального компьютера
(Рисунки~\ref{fig:setting:pers:garbich}\,-\,\ref{fig:setting:pers:garbich:delete}).

\begin{image}
	\includegrph{Screenshot from 2024-09-10 10-19-55}
	\caption{Отключение отображения Корзины}
	\label{fig:setting:pers:garbich}
\end{image}

\begin{image}
	\includegrph{Screenshot from 2024-09-10 10-20-24}
	\caption{Рабочий стол}
	\label{fig:setting:pers:garbich:delete}
\end{image}

\clearpage

\subsection{Консоль <<Управление компьютером>>}

С помощью консоли <<Управление компьютером>>
(\textbf{Панель управления $\rightarrow$ Администрирование
$\rightarrow$ Управление компьютером}) возможно выполнить следующие действия:

\begin{itemize}
	\item Просмотр информации о дисках и разделах;
	\item управление разделами (создание, форматирование, изменение, удаление);
	\item управление дисками (инициализация и конвертирование типа);
	\item Создание и управление зеркальными томами;
\end{itemize}

Изменил букву диска CD/DVD-привода (Рисунок~\ref{fig:setting:dvd}).

\begin{image}
	\includegrph{Screenshot from 2024-09-10 20-36-59}
	\caption{Изменение буквы диска CD/DVD-привода}
	\label{fig:setting:dvd}
\end{image}

\clearpage

\section{Дополнительные настройки программы VirtulaBox}

\subsection{Настройка Hardware}
Во вкладке Storage удалил из машины эмуляцию floppy-диска
(Рисунок~\ref{fig:vm:floppy:remove}).
В VirtulaBox не эмуляции принтера.

\begin{image}
	\includegrph{Screenshot from 2024-09-10 20-43-59}
	\caption{Удаление floppy диска}
	\label{fig:vm:floppy:remove}
\end{image}

Отключил звуковую карту (Рисунок~\ref{fig:vm:audio}).

\begin{image}
	\includegrph{Screenshot from 2024-09-10 20-45-56}
	\caption{Отключение звуковой карты}
	\label{fig:vm:audio}
\end{image}

В VirtulaBox нет настройки для добавления поддержки расширенной клавиатуры.

\clearpage

\subsection{Поддержка USB-флеш-накопителя}

Вставил в USB-порт хост-компьютера USB-флеш-накопитель.
Проверил, на какой машине он определился (появился на хостовой машине)
(Рисунок~\ref{fig:flesh:host}).

\begin{image}
	\includegrph{Screenshot from 2024-09-10 20-53-31}
	\caption{Отображение флеш-накопителя на хостовой машине}
	\label{fig:flesh:host}
\end{image}

В окне программы VirtulaBox в меню USB добавил накопитель.
Проверил доступность устройства на гостевой машине
(Рисунок~\ref{fig:flesh:guest}).

\begin{image}
	\includegrph{Screenshot from 2024-09-10 20-52-51}
	\caption{Флеш-накопителя на гостевой машине}
	\label{fig:flesh:guest}
\end{image}

С помощью USB-флеш-накопителя скопировал на гостевую машину инсталлятор
небольшой программы 7-Zip
(Рисунки~\ref{fig:flesh:app:host}\,-\,\ref{fig:flesh:app:coping}).
И существил установку этой программы.

\begin{image}
	\includegrph{Screenshot from 2024-09-10 20-54-56}
	\caption{Инсталлятор на хостовой машине}
	\label{fig:flesh:app:host}
\end{image}

\begin{image}
	\includegrph{Screenshot from 2024-09-10 20-55-56}
	\caption{Копирование на гостевую машину инсталлятора}
	\label{fig:flesh:app:coping}
\end{image}

\clearpage

\subsection{Установка <<VMware Tools>>}

Так как для практической работы использовался VirtulaBox
эту часть выполнить не получится.

\subsection{Создание резервной копии}

Создал резервную копию созданного виртуального компьютера.
Это можно сделать несколькими способами:

\begin{itemize}
	\item создвать snapshot.
	\item создвать clone.
	\item скопировать каталог с образом.
\end{itemize}

Выбрал последний способ (Рисунок~\ref{fig:backup}).

\begin{image}
	\includegrph{Screenshot from 2024-09-10 21-02-58}
	\caption{Создание резервной копии}
	\label{fig:backup}
\end{image}

\clearpage

\section*{\LARGE ОТВЕТЫ НА ВОПРОСЫ}
\addcontentsline{toc}{section}{ОТВЕТЫ НА ВОПРОСЫ}

\begin{enumerate}
	\item \textbf{Что такое программное обеспечение ЭВМ?
		Какие виды программного обеспечения можно назвать?}\par
		\textbf{Программное обеспечение ЭВМ (электронно-вычислительных машин)}
		--- это совокупность программ,
		предназначенных для управления аппаратными средствами компьютера,
		выполнения вычислений и решения различных задач.
		Существует несколько видов программного обеспечения:
		\begin{itemize}
			\item Системное (например, операционные системы, драйверы).
			\item Прикладное (офисные программы, браузеры, медиаплееры).
			\item Инструментальное (компиляторы, среды разработки).
		\end{itemize}
	\item \textbf{Что такое операционная система?
		Назовите её основную задачу и функции.}\par
		\textbf{Операционная система (ОС)} --- это базовое программное
		обеспечение, управляющее аппаратными средствами компьютера
		и обеспечивающее работу прикладных программ.
		Основная задача ОС --- управление ресурсами
		(процессорами, памятью, внешними устройствами) и выполнение программ.
		Основные функции ОС:
		\begin{itemize}
			\item Управление процессами.
			\item Управление памятью.
			\item Работа с файловой системой.
			\item Обеспечение интерфейса для пользователя и приложений.
		\end{itemize}
	\item \textbf{Какие критерии классификации операционных
			систем можно назвать?}\par
		Классификация операционных систем может осуществляться
		по следующим критериям:
		\begin{itemize}
			\item По числу одновременно работающих пользователей
				(однопользовательские и многопользовательские).
			\item По числу одновременно выполняемых задач
				(однозадачные и многозадачные).
			\item По архитектуре (32-разрядные, 64-разрядные).
			\item По типу лицензирования (проприетарные, открытые).
		\end{itemize}
	\item \textbf{Охарактеризуйте операционную систему Microsoft Windows?}\par
		\textbf{Операционная система Microsoft Windows} --- это популярная
		проприетарная ОС, ориентированная на широкую аудиторию.
		Характерные черты:
		\begin{itemize}
			\item Графический интерфейс пользователя (GUI).
			\item Поддержка большого числа программ и драйверов.
			\item Развитая экосистема приложений и инструментов.
		\end{itemize}
	\item \textbf{Что такое программа-гипервизор?
			Назовите примеры таких программ для разных хостовых
			операционных систем.}\par
		\textbf{Программа-гипервизор} --- это специальное программное
		обеспечение, которое позволяет запускать
		и управлять несколькими виртуальными машинами
		на одном физическом компьютере.
		Примеры гипервизоров:
		\begin{itemize}
			\item VMware ESXi (для серверов).
			\item Hyper-V (для Windows).
			\item KVM (для Linux).
			\item VirtualBox (кроссплатформенный).
		\end{itemize}
	\item \textbf{Как в современных вычислительных системах реализуется
		поддержка виртуальзации?}\par
		Поддержка виртуализации реализуется
		с помощью специального аппаратного и программного обеспечения.
		Основные принципы работы виртуального процессора:
		\begin{itemize}
			\item Гипервизор -- программное обеспечение,
				которое создаёт виртуальные процессоры и управляет их работой.
			\item Аппаратная поддержка --- современные процессоры оснащены
				технологиями, такими как Intel VT-x и AMD-V,
				которые значительно улучшают производительность виртуализации.
			\item Изоляция ресурсов --- каждая виртуальная машина получает
				своё собственное пространство адресов
				и набор системных ресурсов.
			\item Эмуляция и паравиртуализация --- виртуальные процессоры
				могут быть реализованы через полную эмуляцию аппаратного
				оборудования или частичную,
				где гипервизор заменяет некоторые аппаратные функции
				программными аналогами для повышения производительности.
		\end{itemize}
	\item \textbf{Какие минимальные системные требования предъявляют
			разработчики вашего гипервизора к оборудованию
			и программному обеспечению хост-компьютера?}\par
		Для корректной работы VirtualBox требуется компьютер,
		удовлетворяющий следующим условиям:
		\begin{itemize}
			\item Процессор: любой процессор Intel или AMD,
				совместимый с архитектурой x86,
				с функцией поддержки аппаратной виртуализации VT-x/AMD-V
				или без нее.
			\item Свободная оперативная память: минимум 1 Гб + RAM,
				требуемая для запуска и работы соответствующих гостевых ОС.
				Например, для Windows 7 рекомендуемый объём памяти составляет
				1024 -- 2048 Мб.
			\item Место на жестком диске:
				200 Мб для установки VirtualBox + 20 Гб для установки ВМ.
		\end{itemize}
		Данные требования являются приблизительными
		и зависят от системных требований устанавливаемых гостевых ОС.
	\item \textbf{Назовите назначение оперативной памяти в вычислительной
			системе. На что влияет объём оперативной памяти?}\par
		\textbf{Оперативная память (ОЗУ)} в вычислительной системе служит
		для временного хранения данных и команд,
		которые обрабатываются процессором.
		Объём оперативной памяти влияет на:
		\begin{itemize}
			\item Скорость работы приложений.
			\item Количество одновременно выполняемых задач.
			\item Обработку больших данных и работы с графикой или видео.
		\end{itemize}
	\item \textbf{Что такое глубина цвета? Что такое True Color
			и на что в этой палитре расходуются разряды в коде цвета?}\par
		\textbf{Глубина цвета} --- это количество бит,
		используемых для представления цвета каждого пикселя на экране.
		True Color использует 24 бита для кодирования цвета:
		8 бит для каждого из трёх цветовых каналов (красный, зелёный, синий).
		Это позволяет отображать более 16 миллионов цветов.
	\item \textbf{Что такое ярлык в ОС Windows?}\par
		\textbf{Ярлык в ОС Windows} -- это ссылка на программу или файл,
		позволяющая быстро запускать объект без поиска его в файловой системе.
	\item \textbf{Что такое реестр Windows?
			Как он организован логически и файлово?}\par
		\textbf{Реестр Windows} --- это централизованная база данных,
		содержащая информацию о настройках ОС, приложений и устройств.
		Логически реестр организован в виде дерева ключей и значений,
		а физически хранится в виде файлов на диске
		(например, SYSTEM, SOFTWARE).
	\item \textbf{Что такое форматирование дисковой памяти?
			В чём отличие низкоуровневого
			и высокоуровневого форматирования?}\par
		\textbf{Форматирование дисковой памяти} --- это процесс подготовки
		носителя к использованию.
		Различают:
		\begin{itemize}
			\item Низкоуровневое форматирование --- это физическая разметка
				диска на дорожки и секторы (обычно выполняется на заводе).
			\item Высокоуровневое форматирование --- это создание файловой
				системы на диске (NTFS, FAT32 и др.).
		\end{itemize}
	\item \textbf{Что такое дефрагментация дисковой памяти?
			Для какого типа внешних накопителей дефрагментацию
			лучше не делать и почему?}\par
		\textbf{Дефрагментация дисковой памяти} --- это процесс оптимизации
		расположения данных на жестком диске для улучшения скорости доступа.
		Для SSD-накопителей дефрагментацию лучше не делать,
		так как это может уменьшить срок их службы
		из-за ограниченного числа циклов записи.
	\item \textbf{Что такое USB? Какая самая последняя версия
			USB стандартизована на сегодняшний день?}\par
		\textbf{USB (Universal Serial Bus)} --- это универсальный
		последовательный интерфейс для подключения периферийных устройств.
		На сегодняшний день последней стандартизованной версией является
		USB 4.0, обеспечивающая скорость передачи данных до 40 Гбит/с.
\end{enumerate}

\clearpage

\section*{\LARGE ВЫВОД}
\addcontentsline{toc}{section}{ВЫВОД}

В данной практической работе мы получили навыки создания виртуальной машины
с использованием программы VirtulaBox
и настройки операционной системы Windows 7 в качестве гостевой ОС.\par
В ходе выполнения работы были рассмотрены различные аспекты установки
и конфигурации виртуальных машин,
а также их взаимодействие с физическим оборудованием
и базовой операционной системой.

\clearpage

\section*{\LARGE СПИСОК ИСПОЛЬЗОВАННЫХ ИСТОЧНИКОВ}
\addcontentsline{toc}{section}{СПИСОК ИСПОЛЬЗОВАННЫХ ИСТОЧНИКОВ}

\begin{thebibliography}{00}
	\bibitem{} Виртуальные машины с различными ОС: VMware Workstation Player
		[Электронный ресурс].
		URL: \url{https://www.vmware.com/ru/products/workstation-player.html}
		(дата обращения 31.08.2023).
	\bibitem{} VMware Tools для Windows [Электронный ресурс].
		URL: \url{https://packages.vmware.com/tools/esx/latest/windows/VMware-tools-windows-12.3.5-22544099.iso}
		(дата обращения 12.05.2024).
	\bibitem{} Каталог Центра обновления Майкрософт [Электронный ресурс].
		URL: \url{https://catalog.update.microsoft.com/home.aspx}
		(дата обращения 12.05.2024).
	\bibitem{} Колисниченко Д.Н. Самоучитель Microsoft Windows 11, 2023.
	\bibitem{} Руссинович М. и др. Внутреннее устройство Windows.
		7-е изд., 2018.
	\bibitem{} Станек У.Р. Windows 7 для продвинутых.
		Настройка, работа и администрирование, 2011.
	\bibitem{} Таненбаум Э., Бос Х. Современные операционные системы.
		4-е изд., 2015.
	\bibitem{} Чекмарев А.Н. Windows 7.
		Руководство администратора (В подлиннике), 2010.
\end{thebibliography}

