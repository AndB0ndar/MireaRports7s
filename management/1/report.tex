\begin{tabular}{|l|}
	\hline
	Управление информационно-технологическими проектами \\
	Домашнее задание \No\,1 \\ \hline
	ФИО: \textbf{Бондарь Андрей Ренатович} \\ \hline
	Группа: \textbf{ИКБО-06-21} \\ \hline
\end{tabular}

\section{Название, сфера деятельности организации}

Компания <<Камень Зевса>> --- это современный fabless-дизайн центр,
который занимается разработкой как цифровой, так и аналоговой микроэлектроники.
Она не имеет собственных производственных мощностей
и заказывает изготовление чипов на крупнейших фабриках,
таких как <<TSMC (Тайвань)>>, <<Samsung Foundry (Южная Корея)>>,
АО <<Микрон>> (Зеленоград) и АО НТЦ <<Модуль>> (Москва).
Такое сотрудничество позволяет компании использовать передовые техпроцессы,
от 180 до 7 нм, в зависимости от специфики разрабатываемых продуктов.
Компания специализируется на проектировании специализированных
программируемых вычислителей и IP-блоков,
что позволяет ей создавать энергоэффективные
и высокопроизводительные решения для различных отраслей.

\section{История развития, достижения, проблемы}

Компания была основана специалистами с глубоким научным бэкграундом,
которые имеют опыт работы с ведущими российскими вузами.
Компания с самого начала сделала ставку на качество и полный цикл разработки,
начиная с транзисторного уровня и заканчивая системным и прикладным ПО.\par
В 2013 году "Камень Зевса" выпустила свой первый процессор ZeusCore,
который получил высокие оценки благодаря высокой производительности
и низкому энергопотреблению, но иногда бился током.
Это стало прорывом для компании,
и она смогла выйти на рынок серверных решений,
обеспечив контракты с крупными дата-центрами.\par
Сегодня компания осуществляет 3-5 запусков в год,
разрабатывая как цифровые, так и аналоговые компоненты.
Одним из её ключевых достижений является умение быстро верифицировать
функционал будущих систем на ПЛИС (FPGA),
что сокращает сроки разработки и повышает гибкость решений.\par
Основные вызовы компании --- это поддержание конкурентоспособности
в высокотехнологичной отрасл, где требуются быстрые темпы инноваций
и глубокая экспертиза в различных областях.
В последние годы компания столкнулась с серьёзными проблемами,
связанными с дефицитом полупроводников и логистическими сбоями.
Также конкуренция со стороны китайских
и корейских производителей оказала давление на рынок,
снижая маржинальность продукции.\par

\section{Стратегические, долгосрочные цели}

Открытие новых исследовательских центров
для привлечения на стажировку перспективный, молодых специалистов,
потом, может быть, их даже возьмут в штат.\par
Укрепление позиций в сегментах телекоммуникаций и сетевых решений.
Компания уже успешно работает над разработкой специализированных решений
для обработки сетевых пакетов и шифрования данных.\par
Учитывая возрастающую важность энергоэффективности в микроэлектронике,
это одно из ключевых направлений для долгосрочного развития.\par
Компания планирует сохранить и усилить своё преимущество,
заключающееся в полном контроле над проектируемыми системами,
начиная от аппаратных блоков и заканчивая программным обеспечением.\par
"Камень Зевса" планирует к 2027 году разработать специализированные
процессоры для AI и машинного обучения,
конкурируя с лидерами этой индустрии.\par

\section{Примерная организационная структура, масштаб, количество сотрудников}

Организационная структура компании <<Камень Зевса>>
включает несколько ключевых отделов,
которые обеспечивают полный цикл разработки и поддержки:

\begin{itemize}
	\item \textbf{Отдел RTL-проектирования:}
		Специалисты этого отдела занимаются разработкой логики
		на уровне регистровых передач (RTL),
		работая над архитектурой цифровых схем и функциональной верификацией.
	\item \textbf{Отдел аналогового проектирования:}
		Инженеры разрабатывают аналоговые IP-блоки и схемы,
		что позволяет компании предлагать комплексные решения,
		интегрирующие как цифровую, так и аналоговую микроэлектронику.
	\item \textbf{Отдел разработки программного обеспечения:}
		Отвечает за создание системного
		и прикладного ПО для аппаратных решений,
		используя преимущественно open source-технологии.
	\item \textbf{Отдел разработки печатных плат (PCB):} Занимается
		проектированием печатных плат, интеграцией разработанных чипов
		в устройства и обеспечивает качественную реализацию схемотехники.
	\item \textbf{Отдел системных администраторов:}
		Обеспечивает бесперебойную работу ИТ-инфраструктуры,
		поддерживает серверы и сети,
		отвечая за безопасность и доступность данных.
	\item \textbf{Отдел управления проектами:}
		Курирует все этапы разработки,
		координирует работу команд,
		следит за выполнением сроков
		и соблюдением технических требований.
	\item \textbf{Отдел разработки документации:}
		Подготавливает всю необходимую техническую
		и пользовательскую документацию, включая инструкции,
		описания архитектурных решений и регламентов.
	\item \textbf{Экономический отдел:}
		Отвечает за финансы компании,
		планирование бюджета, анализ затрат и прибыли,
		а также ведение отчетности.
\end{itemize}

Эта структура, хотя и комплексная,
позволяет эффективно распределять обязанности между командами
и избегать излишнего формализма,
что поддерживает гибкость и скорость разработки.

\section{Территория (месторасположение, количество помещений, площади)}

Компания <<Камень Зевса>> расположена в удалённом
и защищённом месте на архипелаге Новая Земля, на крайнем севере России.
Это стратегическое решение позволяет минимизировать риски
кражи высоких технологий и поддерживать высокий уровень конфиденциальности
разработок. Вдали от крупных городов и промышленных центров,
компания функционирует в условиях, обеспечивающих безопасность данных
и интеллектуальной собственности.

\section{Бизнес-причина возникновения проекта}

Традиционные кремниевые процессоры сталкиваются с ограничениями
в масштабировании и дефицитом полупроводников.
Одновременно с этим растет потребность в более экологичных
и инновационных решениях. Недавние исследования показали,
что углерод, полученный из российской березы,
обладает уникальными электрическими и тепловыми свойствами,
которые превосходят обычные углеродные материалы.
Это открывает возможность для создания более энергоэффективных
и экологически чистых процессоров,
что является ключевой бизнес-причиной для запуска проекта.

\section{Общее описание целей}

Основная цель проекта --- разработать инновационный процессор,
в котором вместо кремния используются углеродные структуры,
полученные из березового угля.
Такая архитектура должна снизить затраты на производство,
повысить энергоэффективность и экологическую устойчивость процессоров.
Дополнительно компания стремится использовать этот проект
для укрепления своих позиций на рынке и создания уникального продукта,
не имеющего аналогов.

\section{Задачи, которые должен решать проект}

\begin{itemize}
	\item Исследование свойств березового угля из российской березы
		для использования в качестве основы для микросхем.
	\item Разработка новых архитектур процессоров
		с использованием углеродных структур.
	\item Оптимизация методов производства чипов на основе углерода.
	\item Создание прототипов процессоров
		для проведения тестов по энергоэффективности,
		производительности и надежности.
	\item Подготовка производственных мощностей
		для серийного выпуска новых процессоров на углеродной основе.
\end{itemize}

\section{Краткая характеристика модернизируемых бизнес-процессов}

Для реализации проекта компания должна провести модернизацию
ключевых бизнес-процессов.\par
Необходимо создание новых лабораторий для исследования уникальных свойств
березового угля, которые будут применяться в производстве чипов.\par
Также предстоит разработка новых методик и инструментов
для тестирования углеродных структур,
поскольку традиционные кремниевые методы не полностью применимы.\par
И нельзая забывать, что компании предстоит адаптировать существующие
производственные мощности к новым материалам,
а также разработать методики взаимодействия с фабриками,
готовыми работать с углеродом.

\section{Целевая аудитория (на кого направлены изменения)}

Основной фокус направлен на получение государственных контрактов и грантов,
так как разработка углеродных процессоров с уникальными свойствами может быть
использована для повышения национальной технологической независимости,
в том числе в оборонной, научной и телекоммуникационной сферах.\par
Частный сектор, особенно компании, работающие
в области высокопроизводительных вычислений и дата-центров,
где энергоэффективность и экологические стандарты играют ключевую роль.
Предложение уникальных решений может открыть доступ
к международным контрактам и партнёрствам.

\section{Требования, ограничения, допущения}

Разработка процессора должна соответствовать требованиям государственных
программ поддержки инноваций и импортозамещения.
Процессор должен показывать преимущества
по энергоэффективности и экологичности.\par
Главными ограничениями являются время на разработку,
высокая стоимость перехода на новые производственные процессы
и поиск фабрик, способных работать с углеродными материалами.\par
Предполагается, что исследование свойств березового угля
и его интеграция в производственные процессы позволит создать продукт,
отвечающий как государственным, так и международным требованиям.\par

\section{Тeкущий уровень автоматизации}

На сегодняшний день компания "Камень Зевса" использует передовые
инструменты автоматизации для разработки и тестирования чипов,
такие как Cadence и Synopsys.
Автоматизация проектирования
и тестирования позволяет оперативно выпускать прототипы
и тестировать их на ПЛИС.
Для дальнейшего развития проекта потребуется модернизация ИТ-инфраструктуры,
включая расширение вычислительных мощностей
и использование новых инструментов,
позволяющих работать с углеродными процессорами.

