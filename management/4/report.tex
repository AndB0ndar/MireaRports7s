\begin{tabular}{|l|}
	\hline
	Управление информационно-технологическими проектами \\
	Домашнее задание \No\,4 \\ \hline
	ФИО: \textbf{Бондарь Андрей Ренатович} \\ \hline
	Группа: \textbf{ИКБО-06-21} \\ \hline
	Вариант: \textbf{Григораш} \\ \hline
\end{tabular}

\section*{Реестр рисков}
\addcontentsline{toc}{section}{Реестр рисков}

\begin{small}
	\begin{longtable}{|p{0.5cm}|p{3.5cm}|p{4cm}|p{3cm}|p{2cm}|p{1.5cm}|}
		\caption{Реестр рисков} \label{table:r} \\
		\hline
		\textbf{\No}
			& \textbf{Описание риска}
			& \textbf{Потенциальное воздействие}
			& \textbf{Вероятность возникновения}
			& \textbf{Влияние на проект}
			& \textbf{Уровень риска} \\
		\hline
		\endfirsthead
		\conttable{table:r} \\
		\hline
		\textbf{\No}
			& \textbf{Описание риска}
			& \textbf{Потенциальное воздействие}
			& \textbf{Вероятность возникновения}
			& \textbf{Влияние на проект}
			& \textbf{Уровень риска} \\
		\hline
		\endhead

		1 & Невозможность интеграции с внешними API
			& Задержка сроков, перерасход бюджета
			& 0.8 & 0.9 & 0.72 \\ \hline
		2 & Утечка данных клиентов
			& Нарушение безопасности, репутационные потери
			& 0.6 & 1.0 & 0.60 \\ \hline
		3 & Ошибки в проектировании ТЗ
			& Требуется переработка, увеличение сроков и бюджета
			& 0.7 & 0.8 & 0.56 \\ \hline
		4 & Сбой серверного оборудования
			& Нарушение доступности сайта, задержка запуска
			& 0.7 & 0.7 & 0.49 \\ \hline
		5 & Сложности с обучением персонала
			& Персонал не сможет эффективно работать с новой системой
			& 0.6 & 0.8 & 0.48 \\ \hline
		6 & Недостаточный бюджет проекта
			& Прекращение работ, сокращение функций
			& 0.5 & 0.9 & 0.45 \\ \hline
		7 & Ошибки в программном коде
			& Тестирование не завершится в срок, возможны сбои после запуска
			& 0.7 & 0.6 & 0.42 \\ \hline
		8 & Недоступность ключевых специалистов
			& Задержка работ, перераспределение задач
			& 0.6 & 0.7 & 0.42 \\ \hline
		9 & Проблемы с контентом для сайта
			& Контент не будет готов к моменту завершения разработки
			& 0.6 & 0.6 & 0.36 \\ \hline
		10 & Устаревание технологий в процессе
			& Технологии не соответствуют текущим требованиям рынка
			& 0.7 & 0.5 & 0.35 \\ \hline
		11 & Задержки в согласовании решений
			& Увеличение сроков реализации проекта
			& 0.5 & 0.6 & 0.30 \\ \hline
		12 & Отсутствие доступа к тестовым API
			& Задержка тестирования и разработки
			& 0.6 & 0.5 & 0.30 \\ \hline
		13 & Конфликты внутри команды
			& Снижение эффективности работы команды
			& 0.5 & 0.5 & 0.25 \\ \hline
		14 & Неправильный выбор технологий
			& Потребуется переработка, увеличение сроков
			& 0.3 & 0.7 & 0.21 \\ \hline
		15 & Ошибки при настройке CRM
			& Система лояльности не будет работать корректно
			& 0.5 & 0.4 & 0.20 \\ \hline
		16 & Задержка со стороны подрядчиков
			& Срыв сроков завершения этапов
			& 0.6 & 0.5 & 0.30 \\ \hline
		17 & Неправильная настройка аналитики
			& Невозможность отслеживать поведение пользователей
			& 0.5 & 0.5 & 0.25 \\ \hline
		18 & Ошибки в платежной системе
			& Невозможность принимать онлайн-платежи
			& 0.8 & 0.4 & 0.32 \\ \hline
		19 & Сложности с интеграцией CRM
			& Невозможно связать данные о заказах с программой лояльности
			& 0.6 & 0.5 & 0.30 \\ \hline
		20 & Изменение требований клиента
			& Переработка проекта, увеличение сроков и бюджета
			& 0.6 & 0.6 & 0.36 \\ \hline
		21 & Низкий отклик на маркетинговую кампанию
			& Низкий уровень вовлечения пользователей
			& 0.5 & 0.5 & 0.25 \\ \hline
		22 & Несоответствие сайта ожиданиям клиента
			& Дополнительная работа по переработке дизайна и функционала
			& 0.7 & 0.4 & 0.28 \\ \hline
		23 & Неполадки в оборудовании сотрудников
			& Замедление разработки
			& 0.5 & 0.4 & 0.20 \\ \hline
		24 & Потеря данных
			& Утрата информации о клиентах и заказах
			& 0.8 & 0.3 & 0.24 \\ \hline
		25 & Перегрузка сервера
			& Сайт становится недоступным
			& 0.7 & 0.3 & 0.21 \\ \hline
		26 & Низкое качество тестирования
			& Проблемы проявляются после запуска
			& 0.4 & 0.5 & 0.20 \\ \hline
		27 & Проблемы с хостингом
			& Сайт будет недоступен
			& 0.6 & 0.4 & 0.24 \\ \hline
		28 & Неправильная настройка резервного копирования
			& Потеря критических данных
			& 0.5 & 0.4 & 0.20 \\ \hline
		29 & Сложности с настройкой мобильной версии
			& Неполная функциональность сайта
			& 0.6 & 0.4 & 0.24 \\ \hline
		30 & Замена ключевого подрядчика
			& Задержка реализации
			& 0.6 & 0.3 & 0.18 \\ \hline
	\end{longtable}
\end{small}

\clearpage

\section*{Выбор стратегии реагирования для значимых рисков}
\addcontentsline{toc}{section}{Выбор стратегии реагирования
	для значимых рисков}

\paragraph{Невозможность интеграции с внешними API}

\textit{Стратегия реагирования:} Снижение риска

\begin{enumerate}
	\item Провести предварительное тестирование API выбранных сервисов.
	\item Заключить соглашения с поставщиками API о технической поддержке.
	\item Подготовить резервные решения, включая использование альтернативных API.
	\item Заложить дополнительное время в график для отладки интеграции.
\end{enumerate}

\paragraph{Утечка данных клиентов}

\textit{Стратегия реагирования:} Уклонение от риска

\begin{enumerate}
	\item Внедрить SSL-сертификаты для шифрования данных.
	\item Проводить регулярные аудиты безопасности (пентесты).
	\item Хранить минимальный объем персональных данных.
	\item Обеспечить соответствие стандартам безопасности данных (GDPR, PCI DSS).
\end{enumerate}

\paragraph{Ошибки в проектировании ТЗ}

\textit{Стратегия реагирования:} Снижение риска

\begin{enumerate}
	\item Привлечь опытного бизнес-аналитика для формирования ТЗ.
	\item Заложить этап ревизии ТЗ перед утверждением.
	\item Организовать рабочие сессии для уточнения требований с участием клиента и команды.
	\item Установить ключевые вехи для постепенного утверждения частей ТЗ.
\end{enumerate}

\paragraph{Сбой серверного оборудования}

\textit{Стратегия реагирования:} Передача риска

\begin{enumerate}
	\item Использовать облачные серверы с резервированием (например, AWS, Azure).
	\item Заключить соглашение SLA с поставщиками серверов.
	\item Настроить регулярное резервное копирование данных.
	\item Протестировать систему отказоустойчивости перед запуском.
\end{enumerate}

\paragraph{Сложности с обучением персонала}

\textit{Стратегия реагирования:} Снижение риска

\begin{enumerate}
	\item Разработать подробные инструкции и обучающие материалы для работы с системой.
	\item Провести практические тренинги и тестирование знаний сотрудников.
	\item Включить этап технической поддержки после запуска.
	\item Назначить ответственных за обучение и поддержку в каждой рабочей смене.
\end{enumerate}

\paragraph{Недостаточный бюджет проекта}

\textit{Стратегия реагирования:} Передача риска

\begin{enumerate}
	\item Заключить договор с клиентом на предварительное утверждение бюджета с учетом резерва.
	\item Найти альтернативные источники финансирования, включая гранты или партнёров.
	\item Оптимизировать задачи, исключив низкоприоритетные функции.
	\item Разработать детальный план управления затратами для отслеживания расходов.
\end{enumerate}

\paragraph{Ошибки в программном коде}

\textit{Стратегия реагирования:} Снижение риска

\begin{enumerate}
	\item Провести код-ревью на каждом этапе разработки.
	\item Увеличить время и ресурсы на тестирование кода.
	\item Использовать автоматизированные инструменты проверки качества кода (например, SonarQube).
	\item Привлечь квалифицированных тестировщиков для работы на этапе тестирования.
\end{enumerate}

\paragraph{Недоступность ключевых специалистов}

\textit{Стратегия реагирования:} Уклонение от риска

\begin{enumerate}
	\item Заключить контракты с ключевыми специалистами до начала проекта.
	\item Включить в команду резервных специалистов.
	\item Организовать передачу знаний внутри команды для снижения зависимости от одного человека.
	\item Провести анализ доступности сотрудников на раннем этапе планирования.
\end{enumerate}

\paragraph{Проблемы с контентом для сайта}

\textit{Стратегия реагирования:} Снижение риска

\begin{enumerate}
	\item Назначить ответственного за сбор и подготовку контента на начальных этапах проекта.
	\item Заложить резервное время для подготовки контента.
	\item Использовать сторонние услуги для подготовки материалов (копирайтинг, дизайн).
	\item Составить чек-листы для контроля полноты и качества контента.
\end{enumerate}

\paragraph{Устаревание технологий}

\textit{Стратегия реагирования:} Уклонение от риска

\begin{enumerate}
	\item Использовать проверенные, но актуальные технологии с длительной поддержкой (например, React, Vue.js).
	\item Проводить регулярный мониторинг рынка IT-технологий.
	\item Заключить соглашение с разработчиками о возможной модернизации решений после запуска.
	\item Учитывать возможность замены отдельных модулей системы на новые решения.
\end{enumerate}

