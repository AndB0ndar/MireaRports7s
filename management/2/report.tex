\begin{tabular}{|l|}
	\hline
	Управление информационно-технологическими проектами \\
	Домашнее задание \No\,2 \\ \hline
	ФИО: \textbf{Бондарь Андрей Ренатович} \\ \hline
	Группа: \textbf{ИКБО-06-21} \\ \hline
\end{tabular}

\section*{Концепция для кафе <<Чай поставьте>>}
\addcontentsline{toc}{section}{Концепция для кафе <<Чай поставьте>>}

\section{Название проекта}

Разработка веб-сайта
и системы автоматизации заказов для кафе <<Чай поставьте>>.

\section{Цели и задачи проекта}

Основной целью проекта является создание веб-сайта,
который позволит клиентам кафе заказывать блюда и чай онлайн,
а также получать доставку на дом.
Это решение направлено на повышение клиентского опыта, улучшение лояльности,
расширение каналов продаж и оптимизацию ключевых бизнес-процессов заведения.

Для достижения этой цели проект предусматривает несколько задач. В первую очередь, необходимо разработать удобную и интуитивно понятную платформу для заказов, которая будет адаптирована под различные устройства, включая мобильные. Важной задачей также является автоматизация бизнес-процессов, что позволит минимизировать человеческие ошибки и ускорить обработку заказов. Помимо этого, проект требует интеграции с внешними сервисами доставки и онлайн-платежей, что значительно расширит возможности кафе по географическому охвату клиентов.

Для улучшения взаимодействия с клиентами и повышения их лояльности будет модернизирована текущая программа лояльности. Теперь клиенты смогут использовать бонусы при онлайн-заказах, а также отслеживать историю своих покупок в личных кабинетах. Не менее важным аспектом станет обеспечение безопасности данных, включая персональные данные клиентов и информацию о платежах.

\textbf{Задачи проекта:}

\begin{itemize}
	\item Разработка удобного веб-сайта для заказов и доставки.
	\item Внедрение системы автоматизации управления заказами
		и обработки платежей.
	\item Создание мобильной версии сайта или приложения.
	\item Интеграция с внешними сервисами доставки.
	\item Оптимизация внутренних процессов
		(прием заказов, распределение задач, учет финансов).
	\item Поддержка и расширение программы лояльности.
	\item Обеспечение безопасности данных клиентов.
\end{itemize}

\section{Результаты проекта, границы проекта}

\textbf{Результаты проекта:}

Главным \textbf{результатом проекта} является запущенный веб-сайт,
который позволит клиентам заказывать блюда и чай с доставкой через интернет.
Помимо этого, будет внедрена автоматизированная система управления заказами,
которая обеспечит их распределение
и обработку без участия сотрудников на этапе приема.

Для удобства клиентов будет разработана мобильная версия сайта,
позволяющая совершать заказы с любого устройства.
Сайт будет интегрирован с внешними сервисами доставки,
что ускорит процесс доставки и обеспечит более широкое географическое покрытие.
Программа лояльности будет расширена и интегрирована с онлайн-системой,
что позволит клиентам использовать бонусы и отслеживать свои покупки онлайн.

\textbf{Границы проекта} включают в себя исключительно создание
и внедрение веб-сайта и системы автоматизации.
Проект не затрагивает физическое расширение кафе
или изменения в его инфраструктуре.
Кроме того, модернизация системы учета запасов на складе
и прочие внутренние улучшения также не входят в данный проект.

\section{Предлагаемые технологии (способ реализации проекта)}

Для достижения целей проекта будут использованы современные
технологии веб-разработки и автоматизации бизнес-процессов.
Веб-сайт будет построен на базе системы управления контентом (CMS),
например, WordPress или Magento, которые обеспечат гибкость и функциональность.
Динамические элементы сайта будут разработаны
с использованием таких фреймворков, как React или Vue.js,
что обеспечит современный и удобный интерфейс.

Для приема онлайн-платежей будут подключены платежные системы,
такие как Stripe или Яндекс.Касса, которые обеспечат безопасность
и удобство транзакций. Управление заказами будет автоматизировано
с помощью платформы, например, Odoo,
что позволит автоматически распределять задачи между сотрудниками
и ускорить процесс обработки заказов.

Интеграция с внешними сервисами доставки,
такими как Яндекс.Доставка или Delivery Club,
будет реализована через API,
что обеспечит удобство для клиентов и повысит оперативность доставки.
Кроме того, для управления программой лояльности
будет использоваться CRM-система, интегрированная с сайтом.
Важное внимание будет уделено безопасности данных
--- для этого будут применяться SSL-сертификаты и другие стандарты защиты.


\section{Оценочный график проекта (структура работ)}

Проект начинается 1 ноября 2024 года.

В \textbf{первый этап} (ноябрь-декабрь 2024 год
) будет проводиться анализ требований и проектирование интерфейсов:

\begin{itemize}
	\item Подготовка технического задания (ТЗ) --- до 15 ноября.
	\item Создание прототипов сайта и приложения --- до 30 ноября.
	\item Утверждение концепции дизайна --- до 15 декабря.
\end{itemize}

\textbf{Второй этап} начнется в январе 2025 года
и будет посвящен непосредственной разработке веб-сайта:

\begin{itemize}
	\item Разработка основного функционала сайта --- до 31 января.
	\item Тестирование и отладка всех систем --- до 28 февраля.
\end{itemize}

\textbf{Третий этап} является финальным
и включает интеграцию с внешними сервисами и запукс:

\begin{itemize}
	\item Интеграция с внешними сервисами --- до 15 марта.
	\item Запуск сайта и приложение в тестовом режиме --- до 31 марта.
\end{itemize}

\section{Оценочные ресурсы проекта}

\subsection{Трудозатраты}

Проект потребует работы нескольких специалистов с различной квалификацией на каждом этапе:

\begin{itemize}
	\item Руководитель проекта (1 человек) --- будет контролировать
		и координировать все этапы разработки,
		взаимодействовать с клиентом и командой,
		следить за сроками и бюджетом проекта.
		Его участие будет необходимо на протяжении всего проекта,
		начиная с подготовки технического задания
		и заканчивая финальным запуском. Время работы: 4 месяца.
	\item Веб-разработчики (2 человека) --- будут заниматься разработкой сайта,
		включая как фронтенд (внешний интерфейс),
		так и бэкенд (серверная часть).
		Разработчики создадут сайт, интегрируют его с платежными системами
		и внешними сервисами доставки. Работают над проектом 2 месяца.
	\item Дизайнер (1 человек) --- разработает визуальные макеты сайта,
		а также создаст адаптивный дизайн для мобильных устройств.
		Дизайнер также будет работать над созданием элементов интерфейса
		для удобства пользователей. Время работы: 1 месяц.
	\item Тестировщик (1 человек) --- проверит функциональность сайта,
		протестирует систему на ошибки и уязвимости,
		обеспечит исправление всех проблем перед запуском.
		Время работы: 1 месяц.
	\item IT-специалист для интеграции (1 человек) --- будет отвечать
		за интеграцию сайта с внешними API
		(сервисы доставки, платежные системы)
		и настройку системы автоматизации внутренних процессов.
		Время работы: 1 месяц.
	\item Контент-менеджер (1 человек) --- создаст
		и загрузит весь необходимый контент на сайт
		(описание блюд, фотографии, тексты).
		Этот специалист будет следить за тем,
		чтобы контент был актуальным и обновлялся по мере необходимости.
		Время работы: 2 недели.
	\item Маркетолог (1 человек) --- займется продвижением
		нового сайта после его запуска.
		Его задачи включают разработку стратегии продвижения,
		настройку контекстной и таргетированной рекламы, работу с соцсетями.
		Время работы: 1 месяц, далее поддержка по мере необходимости.
\end{itemize}

\subsection{Программное обеспечение}

Для реализации проекта потребуется набор программных решений:

\begin{itemize}
	\item CMS (система управления контентом) --- платформа для создания и управления сайтом. Пример: WordPress, Magento. Стоимость лицензии зависит от выбранной системы и необходимых плагинов.
	\item Фреймворки для веб-разработки --- использование React, Vue.js или аналогичных технологий для разработки фронтенда сайта.
	\item Система управления заказами --- платформа для автоматизации процессов приема заказов и их распределения между сотрудниками. Пример: Odoo или 1C: Управление торговлей.
	\item API платежных систем --- интеграция с системами онлайн-платежей, такими как Stripe, PayPal, Яндекс.Касса.
	\item API сервисов доставки --- интеграция с внешними сервисами доставки, такими как Яндекс.Доставка, Delivery Club.
	\item Инструменты для тестирования --- ПО для тестирования функциональности сайта (например, Selenium, JIRA для отслеживания багов).
	\item Средства защиты данных --- шифрование данных и использование SSL-сертификатов для защиты пользовательских данных, включая стандарты безопасности для онлайн-платежей (GDPR).
\end{itemize}

\subsection{Оборудование}

Для реализации проекта необходимо:

\begin{itemize}
	\item Серверное оборудование – для хостинга сайта и базы данных, обеспечивающее стабильную работу сайта и защиту данных.
	\item Компьютеры и рабочие станции – для команды разработчиков, дизайнеров и других специалистов, задействованных в проекте.
	\item Оборудование для поддержки системы автоматизации – если потребуется улучшение внутренней ИТ-инфраструктуры кафе для интеграции с новой системой (серверы, терминалы для сотрудников).
	\item Системы для онлайн-мониторинга – ПО для мониторинга работоспособности сайта и серверов, что позволит оперативно реагировать на любые сбои.
\end{itemize}

\subsection{Дополнительные ресурсы}

Дополнительными ресурсами станут услуги хостинга
и маркетинг для продвижения сайта после его запуска.

\section{Оценочная стоимость проекта}

\textbf{Общая стоимость разработки} веб-сайта,
включая мобильную версию и автоматизацию процессов,
оценивается в 300 000 -- 400 000 рублей.\par
Интеграция с внешними сервисами доставки
и онлайн-платежей потребует дополнительных 200 000 рублей.\par
Подключение систем безопасности
и реализация защиты данных оценивается в 100 000 рублей.\par
Примерно 50 000 рублей будет потрачено на маркетинг
и продвижение после запуска.\par
Таким образом, \textbf{общая стоимость проекта}
составит около 700 000 -- 850 000 рублей.

\section{Риски проекта}

Основные риски включают технические проблемы
при интеграции с внешними сервисами,
что может привести к задержкам в реализации проекта.
Финансовые риски связаны с возможным увеличением стоимости проекта
из-за необходимости дополнительной функциональности.
Также возможны организационные риски,
связанные с обучением персонала,
который должен будет адаптироваться к новой системе работы.

\section{Критерии приемки результатов проекта}

Для успешного завершения проекта веб-сайт
должен соответствовать всем требованиям технического задания,
включая корректную работу на мобильных устройствах.\par
Онлайн-заказы должны приниматься без задержек,
а система доставки и оплата работать без сбоев.\par
Интеграция с внешними сервисами доставки и платежей должна быть завершена,
а клиенты должны иметь доступ к программе лояльности через личные кабинеты.\par
Финальная приемка проекта будет происходить после успешного тестирования
всех функций и официального запуска сайта.

