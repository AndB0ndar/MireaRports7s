\section*{\LARGE Цель практической работы}
\addcontentsline{toc}{section}{Цель практической работы}

\textbf{Цель работы:}

Целью данной работы является закрепление практических навыков
организации межсервисного взаимодействия, в частности,
работы с брокером сообщений RabbitMQ
и отправкой асинхронных сообщений между различными программами.

\textbf{Задание}:

Необходимо выполнить самостоятельно ряд заданий, имеющих варианты.
Варианты считаются в виде остатка от деления последней цифры билета
на количество вариантов.

Для выполнения практических работ можно использовать развёрнутый
учебный сервер с RabbitMQ
(хост: 51.250.26.59, порты: 5672 и 15672, логин: guest, пароль: guest123)

\clearpage

\section*{\LARGE Выполнение практической работы}
\addcontentsline{toc}{section}{Выполнение практической работы}

\section{Простое взаимодействие с сохранением при перезапуске}

Необходимо реализовать схему простейшего взаимодействия сервисов.
При этом нужно создать очередь, сохраняемую при перезапуске сервера RabbitMQ.


Для начала нужно подключиться к серверу RabbitMQ:

\begin{itemize}
	\item Хост: 51.250.26.59;
	\item Порт: 5672 (для взаимодействия с сервером по протоколу AMQP)
		или 15672 (для доступа к веб-интерфейсу);
	\item Логин и пароль: guest / guest123.
\end{itemize}
  
Затем можно создать очередь,
которая будет сохраняться после перезапуска сервера.
Для этого можно использовать Python библиотеку \texttt{pika}.

Пример на Python с использованием библиотеки pika:

\lstinputlisting[
	language=Python, caption=\leftline{Код получателя}
	]{5/src/1/receive.py}

\lstinputlisting[
	language=Python, caption=\leftline{Код отправителя}
	]{5/src/1/send.py}

\begin{image}
    \includegrph{Screenshot from 2024-10-18 20-17-07}
    \caption{Работа очереди}
    \label{fig:simply}
\end{image}

\clearpage

\section{Взаимодействия сервисов с нагрузкой}

Необходимо реализовать схему взаимодействия сервисов с нагрузкой.
Символ для обозначения времени сна: \texttt{*}. Тип обменника: \texttt{direct}.
Сообщения могут не храниться при выключении RabbitMQ.

\lstinputlisting[
	language=Python, caption=\leftline{Код отправителя}
	]{5/src/2/emit.py}

\break

\lstinputlisting[
	language=Python, caption=\leftline{Код получателя}
	]{5/src/2/receive.py}

\begin{image}
    \includegrph{Screenshot from 2024-10-18 21-15-30}
    \caption{Работа обменника direct}
    \label{fig:direct}
\end{image}

\clearpage

\section*{\LARGE Вывод}
\addcontentsline{toc}{section}{Вывод}

В ходе работы были изучены основные принципы работы с RabbitMQ:

\begin{enumerate}
	\item Создание очередей, сохраняемых на диск,
		для устойчивого обмена сообщениями между сервисами.
	\item Настройка обменников
		и очередей для распределения нагрузки между воркерами,
		которые могут временно "засыпать"
		в зависимости от содержимого сообщения.
	\item RabbitMQ позволяет эффективно управлять потоками сообщений
		в системах с микросервисной архитектурой
		и распределённой обработкой данных.
\end{enumerate}

