\section*{\LARGE Цель практической работы}
\addcontentsline{toc}{section}{Цель практической работы}

\textbf{Цель практической работы:}
Анализ и классификация выбранных стандартов информационных
технологий для оценки их актуальности, применимости и соответствия
современным требованиям стандартизации.

\textbf{Задачей} практической работы является приобретение навыков:
\begin{itemize}
	\item идентификации органа стандартизации для каждого выбранного
		стандарта и анализ их принадлежности;
	\item классификации стандартов по кодам, категориям и областям
		стандартизации;
	\item оценки актуальности стандартов и выявления элементов, требующих
		обновления;
	\item определения сфер применения каждого стандарта и их соответствие
		современным требованиям отрасли.
\end{itemize}

\clearpage

\section*{\LARGE Выполнение практической работы}
\addcontentsline{toc}{section}{Выполнение практической работы}

В практической работе необходимо выбрать
минимум по 2 стандарта из разных категорий, таких как:

\begin{enumerate}
	\item международные (Таблица \ref{table:international});
	\item национальные (Таблица \ref{table:national});
	\item межгосударственные (Таблица \ref{table:interstate});
	\item национальные,
		идентичные международным (Таблица \ref{table:national:international});
	\item национальные,
		модифицированные по отношению к международным
		(Таблица \ref{table:national:international:mod}).
\end{enumerate}

Это могут быть стандарты, касающиеся различных этапов жизненного
цикла разработки ПО, стандарты управления качеством, стандарты
информационной безопасности, стандарты по облачным вычислениям,
стандарты баз данных и т.д.

\begin{table}[h!tp]
	\centering
	\caption{\leftline{Международные}}
	\label{table:international}
	\begin{tabular}{|p{10em}|p{11em}|p{11em}|}
		\hline
		\textbf{Обозначение}
			& \textbf{ISO/IEC 27005:2022}
			& \textbf{ISO/IEC/IEEE 15288:2023} \\ \hline
		\textbf{Название стандарта}
			& Информационная безопасность, кибербезопасность
			и защита частной жизни ---
			Руководство по управлению рисками информационной безопасности
			& Системная и программная инженерия.
			Процессы жизненного цикла систем \\ \hline
		\textbf{Индекс стандарта}
			& ISO/IEC & ISO/IEC/IEEE \\ \hline
		\textbf{Регистрационный номер}
			& 27005 & 15288 \\ \hline
		\textbf{Номер комплексной системы стандартов}
			& --- & --- \\ \hline
		\textbf{Код ОКС}
			& --- & --- \\ \hline
		\textbf{Категория документа}
			& Международный & Международный \\ \hline
		\textbf{Организация по стандартизации}
			& ISO/IEC & ISO/IEC/IEEE \\ \hline
		\textbf{Область применения в ИТ}
			& Информационная безопасность, кибербезопасность
			и защита частной жизни
			& Программное обеспечение \\ \hline
		\textbf{Объект стандартизации}
			& Риски информационной безопасности
			& Жизненный цикл программного обеспечения \\ \hline
		\textbf{Аспект стандартизации}
			& Управления рисками
			& Процессы жизненного цикла \\ \hline
		\textbf{Последнее изменение} & 2022-10 & 2023-05 \\ \hline
		\textbf{Связанные стандарты}
			& ГОСТ Р ИСО/МЭК 27005-2010 (Информационная технология.
			Методы и средства обеспечения безопасности.
			Менеджмент риска информационной безопасности)
			& ГОСТ Р 57193-2016 (Системная и программная инженерия.
			Процессы жизненного цикла систем) \\ \hline
	\end{tabular}
\end{table}

\begin{table}[h!tp]
	\centering
	\caption{\leftline{Национальные}}
	\label{table:national}
	\begin{tabular}{|p{10em}|p{11em}|p{11em}|}
		\hline
		\textbf{Обозначение}
			& \textbf{ГОСТ Р 59795-2021} & \textbf{ГОСТ Р 34.10-2012} \\ \hline
		\textbf{Название стандарта}
			& Информационные технологии. Комплекс стандартов на автоматизированные системы. Автоматизированные системы. Требования к содержанию документов
			& Информационная технология. Криптографическая защита информации. Процессы формирования и проверки электронной цифровой подписи \\ \hline
		\textbf{Индекс стандарта}
			& ГОСТ Р & ГОСТ Р \\ \hline
		\textbf{Регистрационный номер}
			& 59795 & 34.10 \\ \hline
		\textbf{Номер комплексной системы стандартов}
			& & 34 \\ \hline
		\textbf{Код ОКС}
			& 35.240 ; 01.040.35 & 35.04 \\ \hline
		\textbf{Категория документа}
			& Национальный & Национальный \\ \hline
		\textbf{Организация по стандартизации}
			& Федеральное агентство по техническому регулированию и метрологии
			& Федеральное агентство по техническому регулированию и метрологии \\ \hline
		\textbf{Область применения в ИТ}
			& Автоматизированные системы
			& Криптографическая защита информации \\ \hline
		\textbf{Объект стандартизации}
			& Содержание документов & Электронная цифровая подпись \\ \hline
		\textbf{Аспект стандартизации}
			& Требования к содержанию документов
			& Формирование и проверка электронной цифровой подписи \\ \hline
		\textbf{Последнее изменение}
			& 01.01.2023 & 01.01.2021 \\ \hline
		\textbf{Связанные стандарты}
			& ГОСТ Р 2.106-2019 (Единая система конструкторской документации.
			Текстовые документы);
			ГОСТ Р 2.105-2020 (Единая система конструкторской документации.
			Общие требования к текстовым документам);
			& ГОСТ Р 34.10-2001 (Информационная технология.
			Криптографическая защита информации.
			Процессы формирования и проверки электронной цифровой подписи);
			ГОСТ Р 34.11-2012 (Криптографическая защита информации.
			Функция хэширования) \\ \hline
	\end{tabular}
\end{table}

\begin{table}[h!tp]
	\centering
	\caption{\leftline{Межгосударственные}}
	\label{table:interstate}
	\begin{tabular}{|p{10em}|p{11em}|p{11em}|}
		\hline
		\textbf{Обозначение}
			& \textbf{ГОСТ 28195-89} & \textbf{ГОСТ 28806-90} \\ \hline
		\textbf{Название стандарта}
			& Оценка качества программных средств. Общие положения
			& Качество программных средств. Термины и определения \\ \hline
		\textbf{Индекс стандарта} & ГОСТ & ГОСТ \\ \hline
		\textbf{Регистрационный номер} & 28195 & 28806 \\ \hline
		\textbf{Номер комплексной системы стандартов} & --- & --- \\ \hline
		\textbf{Код ОКС} & 35.080 & 01.040.35; 35.080 \\ \hline
		\textbf{Категория документа}
			& Межгосударственный & Межгосударственный \\ \hline
		\textbf{Организация по стандартизации}
			& Государственный комитет СССР по стандартам
			& Государственный комитет СССР по стандартам \\ \hline
		\textbf{Область применения в ИТ}
			& Оценка качества ПО & Оценка качества ПО \\ \hline
		\textbf{Объект стандартизации}
			& Программные средства & Термины и определения \\ \hline
		\textbf{Аспект стандартизации}
			& Качество программных средств
			& Термины и определения качества ПО \\ \hline
		\textbf{Последнее изменение} & 06.04.2015 & 06.04.2015 \\ \hline
		\textbf{Связанные стандарты}
			& ГОСТ 19.004-80 (Единая система программной документации. Термины и определения), ГОСТ 15467-79 (Управление качеством продукции. Основные понятия. Термины и определения)
			& ГОСТ 19781-90 (Обеспечение систем обработки информации программное. Термины и определения), ГОСТ 15971-90 (Системы обработки информации. Термины и определения) \\ \hline
	\end{tabular}
\end{table}

\begin{table}[h!tp]
	\centering
	\caption{\leftline{Национальные, идентичные международным}}
	\label{table:national:international}
	\begin{tabular}{|p{10em}|p{11em}|p{11em}|}
		\hline
		\textbf{Обозначение}
			& \textbf{ГОСТ Р ИСО/МЭК 27002-2021}
			& \textbf{ГОСТ Р ИСО 15489-1-2019} \\ \hline
		\textbf{Название стандарта}
			& Информационные технологии. Методы и средства обеспечения безопасности. Свод норм и правил применения мер обеспечения информационной безопасности
			& Система стандартов по информации, библиотечному и издательскому делу. Информация и документация. Управление документами. Часть 1. Понятия и принципы \\ \hline
		\textbf{Индекс стандарта}
			& ГОСТ Р ИСО/МЭК & ГОСТ Р ИСО \\ \hline
		\textbf{Регистрационный номер}
			& 27002 & 15489 \\ \hline
		\textbf{Номер комплексной системы стандартов}
			& --- & --- \\ \hline
		\textbf{Код ОКС}
			& 35.03 & 01.140.20 \\ \hline
		\textbf{Категория документа}
			& Нац = межд & Нац = межд \\ \hline
		\textbf{Организация по стандартизации}
			& Федеральное агентство по техническому регулированию и метрологии
			& Федеральное агентство по техническому регулированию и метрологии \\ \hline
		\textbf{Область применения в ИТ}
			& Информационная безопасность & Управление документами \\ \hline
		\textbf{Объект стандартизации}
			& Обеспечение информационной безопасности
			& Управление документами \\ \hline
		\textbf{Аспект стандартизации}
			& Свод норм и правил применения мер обеспечения информационной безопасности
			& Понятия и принципы \\ \hline
		\textbf{Последнее изменение}
			& 30.11.2021 & 01.01.2020 \\ \hline
		\textbf{Связанные стандарты}
			& ГОСТ Р ИСО/МЭК 27002-2012 (Информационная технология.
			Методы и средства обеспечения безопасности.
			Свод норм и правил менеджмента информационной безопасности)
			& ГОСТ Р ИСО 15489-1-2007 (Система стандартов по информации,
			библиотечному и издательскому делу управление документами.
			Общие требования) \\ \hline
	\end{tabular}
\end{table}

\begin{table}[h!tp]
	\centering
	\caption{\leftline{Национальные, модифицированные по отношению к международным}}
	\label{table:national:international:mod}
	\begin{tabular}{|p{10em}|p{11em}|p{11em}|}
		\hline
		\textbf{Обозначение}
			& \textbf{ГОСТ Р 58668.11-2019 (ИСО/МЭК 19794-13:2018)}
			& \textbf{ГОСТ Р 57657-2017 (ИСО 19131:2007)} \\ \hline
		\textbf{Название стандарта}
			& Информационные технологии. Биометрия.
			Форматы обмена биометрическими данными. Часть 11. Данные голоса
			& Пространственные данные.
			Спецификация информационного продукта \\ \hline
		\textbf{Индекс стандарта}
			& ГОСТ Р & ГОСТ Р \\ \hline
		\textbf{Регистрационный номер}
			& 58668.11 & 57657 \\ \hline
		\textbf{Номер комплексной системы стандартов}
			& --- & --- \\ \hline
		\textbf{Код ОКС}
			& 35.040 & 35.240.70 \\ \hline
		\textbf{Категория документа}
			& Нац, мод межд & Нац, мод межд \\ \hline
		\textbf{Организация по стандартизации}
			& Федеральное агентство по техническому регулированию и метрологии
			& Федеральное агентство по техническому регулированию и метрологии \\ \hline
		\textbf{Область применения в ИТ}
			& Биометрия & Пространственные данные \\ \hline
		\textbf{Объект стандартизации}
			& Форматы обмена биометрическими данными
			& Информационный продукт \\ \hline
		\textbf{Аспект стандартизации}
			& Данные голоса & Спецификация информационного продукта \\ \hline
		\textbf{Последнее изменение}
			& 01.06.2020 & 01.06.2018 \\ \hline
		\textbf{Связанные стандарты}
			& ИСО/МЭК 19794-13:2018 (Информационные технологии.
			Форматы обмена биометрическими данными. Часть 13. Данные голоса)
			& ИСО 19131:2007 (Географическая информация.
			Спецификация информационного продукта) \\ \hline
	\end{tabular}
\end{table}

\clearpage

\section*{\LARGE Вывод}
\addcontentsline{toc}{section}{Вывод}

В ходе выполнения практической работы были рассмотрены
и проанализированы стандарты из различных категорий.
Эти стандарты охватывают различные аспекты информационных технологий,
управления качеством и информационной безопасности,
что позволило глубже понять особенности и различия между международными,
национальными и межгосударственными стандартами,
а также их роль в разработке программного обеспечения.

\begin{itemize}
	\item Международные стандарты
		(например, ГОСТ Р ИСО/МЭК 27002-2021 и ГОСТ Р ИСО 15489-1-2019)
		обеспечивают единообразие подходов к информационной безопасности
		и управлению документами на глобальном уровне.
		Они помогают в создании совместимых систем и процессов, что важно
		для интероперабельности и сотрудничества на международной арене.
	\item Национальные стандарты (например, ГОСТ 28195-89 и ГОСТ 28806-90)
		адаптированы под специфические требования и условия конкретных стран,
		таких как Россия, что обеспечивает унификацию процессов разработки ПО
		и управление качеством внутри государства.
		Эти стандарты играют ключевую роль в обеспечении локального
		соответствия требованиям законодательства и регуляторов.
	\item Межгосударственные стандарты (например, ГОСТ 28195-89)
		упрощают координацию между странами, особенно когда они имеют
		общие технологические и экономические связи.
		Эти стандарты устанавливают согласованные требования
		к программным продуктам и процессам,
		что облегчает взаимодействие между странами.
	\item Национальные стандарты, идентичные международным
		(например, ГОСТ Р 58668.11-2019, основанный на ИСО/МЭК 19794-13:2018),
		позволяют внедрять лучшие мировые практики в национальные стандарты,
		что улучшает конкурентоспособность на международных рынках
		и поддерживает высокие стандарты качества и безопасности.
	\item Национальные стандарты, модифицированные по отношению к международным
		(например, ГОСТ Р 57657-2017, основанный на ИСО 19131:2007),
		адаптируют международные стандарты к местным условиям,
		обеспечивая при этом соответствие глобальным требованиям
		и учитывая национальные особенности и законодательство.
\end{itemize}

Таким образом, анализ вышеупомянутых стандартов
из различных категорий позволил систематизировать знания о стандартизации
в области ИТ и разработки ПО. Выбранные стандарты наглядно демонстрируют,
как различные типы стандартов взаимодействуют
и дополняют друг друга для обеспечения высокого уровня качества,
безопасности и совместимости программных решений.
