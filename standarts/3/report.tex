\section*{\LARGE Цель практической работы}
\addcontentsline{toc}{section}{Цель практической работы}

\textbf{Цель практической работы:}
Получить навыки по анализу основных характеристик и атрибутов
качества программных продуктов в соответствии с установленными
стандартами ГОСТ Р ИСО/МЭК 9126 и ГОСТ Р ИСО/МЭК 25010-2015.

\textbf{Задание на практическую работу:}

\begin{itemize}
	\item Выбрать программный продукт для анализа. В качестве программного
		продукта выступает тема дипломного проекта.
	\item Ознакомиться со стандартами
		ГОСТ Р ИСО/МЭК 9126 и ГОСТ Р ИСО/МЭК 25010-2015.
	\item Выбрав один из вышеперечисленных стандартов, оценить
		возможность соответствия программного продукта всем характеристикам
		качества согласно стандарту и сформулировать несколько требований к
		качеству программного продукта, которое должно быть удовлетворено в
		соответствии с данной характеристикой.
\end{itemize}

\clearpage

\section*{\LARGE Выполнение практической работы}
\addcontentsline{toc}{section}{Выполнение практической работы}

ISO/IEC 9126 --- международный стандарт, определяющий оценочные
характеристики качества программного обеспечения. Он состоит из 4-х частей
(модель качества, внешние метрики, внутренние метрики и применение
метрик). Российским аналогом является стандарт ГОСТ Р ИСО/МЭК 9126-93
<<Информационная технология. Оценка программной продукции.
Характеристики качества и руководства по их применению>>.
В 2011 году ISO/IEC 9126 был заменен стандартом ISO/IEC 25010:2011
(идентичный аналог ГОСТ Р ИСО/МЭК 25010-2015 <<Информационные
технологии. Системная и программная инженерия. Требования и оценка
качества систем и программного обеспечения (SQuaRE). Модели качества
систем и программных продуктов>>).

Требования к качеству программного продукта
программа-конвертер DRC правил по стандарту ГОСТ Р ИСО/МЭК 9126-93
продемонстрированны в таблице~\ref{table}.

\begin{longtable}{|p{3cm}|p{3cm}|p{5cm}|p{5cm}|}
	\caption{Требования к качеству программного продукта} \label{table} \\
	\hline
	\textbf{Характе-ристика}
	& \textbf{Атрибут}
	& \textbf{Требование 1}
	& \textbf{Требование 2} \\
	\hline
	\endfirsthead

	\conttable{table} \\
	\hline
	\textbf{Характе-ристика}
	& \textbf{Атрибут}
	& \textbf{Требование 1}
	& \textbf{Требование 2} \\
	\hline
	\endhead

	\textbf{Функцио-нальность}
	& Пригодность
	& Программа должна поддерживать
	все основные DRC команды для конвертации.
	& Программа должна предоставлять функции
	для предобработки инициализаций. \\ \hline

	& Правильность
	& Все преобразования должны соответствовать спецификациям
	и требованиям форматов.
	& Программа должна выдавать предупреждения
	при обнаружении некорректных данных. \\ \hline

	& Способность к взаимодействию
	& Программа должна корректно работать
	под управлением ОС Linux и Windows.
	& Входные и выходные данные должны быть совместимы
	с KLayout и Calibre. \\ \hline

	& Согласова-нность
	& Входные и выходные данные должны соответствовать спецификациям
	и требованиям форматов.
	& Программа должна выдавать единообразные результаты
	на разных системах. \\ \hline

	& Защищенность (безопасность)
	& Программа не должна сохранять в каком-либо виде исходный файл правил
	& Программа не должна изменять исходный файл правил \\ \hline

	\textbf{Надежность}
	& Стабильность
	& Программа должна без сбоев обрабатывать входные данные
	даже в условиях нестандартных ситуаций (например, поврежденные файлы).
	& Программа должна демонстрировать стабильную работу
	в условиях высоких нагрузок (например,
	при обработке больших наборов данных, объемом выше 5 Гб). \\ \hline

	& Устойчивость к ошибке
	& Программа должна корректно обрабатывать синтаксические ошибоки
	в исходном файле правил, не завершаясь аварийно.
	& Все ошибки должны быть отмечены в отчете о трансляции. \\ \hline

	& Восстанавли-ваемость
	& Программа не должна поддерживать функции резервного копирования,
	так как основной акцент делается на сохранности входных данных.
	& В случае возникновения ошибки программа должна выводить четкие
	сообщения о необходимости исправления данных пользователем. \\ \hline

	\textbf{Практич-ность (удобство использования)}
	& Понятность
	& Интерфейс программы должен быть интуитивно понятным
	для пользователей разного уровня.
	& Программа должна предоставлять текстовый интерфейс
	в консоли/терминале. \\ \hline

	& Обучаемость
	& Пользователи должны уметь освоить основные функции программы
	менее чем за 15 минут.
	& Программа должна предоставлять подробные руководства
	и обучающие материалы для пользователей. \\ \hline

	& Простота использования
	& Процесс запуска конвертации должен запускаться в 3-4 простых шага.
	& Программа должна поддерживать использование значений по умолчанию
	для основных настроек,
	что позволит пользователям начать работу
	без необходимости вносить изменения в конфигурацию. \\ \hline

	\textbf{Эффектив-ность (производительность)}
	& Характер изменения по времени
	& Конвертация файлов должна выполняться
	в пределах допустимого времени (не больше 5 минут на 10000 строк).
	& Время выполнения задачи должно оставаться стабильным
	при росте данных. \\ \hline

	& Характер изменения ресурсов
	& Программа должна эффективно использовать память,
	ограничивая потребление до 150 МБ при выполнении основных операций.
	& Использование процессора не должно превышать 30\%
	при выполнении тривиальных операций. \\ \hline

	\textbf{Сопровожда-емость}
	& Анализиру-емость
	& Исходный код программы должен быть задокументирован и легко читаем.
	& Программа должна создавать отчет
	о результатах конвертации операций. \\ \hline

	& Изменяемость
	& Программа должна иметь модульную архитектуру,
	что позволяет легко вносить изменения и добавлять новые функции.
	& Изменения не должны нарушать работу существующих функций. \\ \hline

	& Устойчивость
	& Программа должна оставаться стабильной после внесения изменений
	в функционал или обновления версии.
	& Все новые функции должны проходить тестирование на совместимость
	с существующими перед выпуском. \\ \hline

	& Тестируемость
	& Программа должна поддерживать возможность тестирования
	в различных средах и конфигурациях.
	& Должны быть предусмотрены автоматические тесты
	для всех модулей программы. \\ \hline

	\textbf{Мобильность}
	& Адаптиру-емость
	& Программа должна работать на различных операционных системах
	(Windows, macOS, Linux) без изменения функциональности.
	& Программа должна легко настраиваться
	для работы в различных средах. \\ \hline

	& Простота внедрения
	& Процесс установки программы должен занимать не более 10 минут
	и включать пошаговые инструкции.
	& Установка должна быть автоматизирована
	и включать проверку системы на наличие необходимых зависимостей. \\ \hline

	& Соответствие
	& Программа должна следовать современным стандартам разработки ПО.
	& Все компоненты программы должны соответствовать требованиям
	совместимости с актуальными версиями операционных систем. \\ \hline

	& Взаимозаменяемость
	& Программа должна поддерживать возможность замены одного
	модуля другим без изменения всего программного обеспечения.
	& Все модули программы должны быть совместимы друг с другом,
	что позволит легко обновлять отдельные компоненты. \\ \hline
\end{longtable}

\clearpage

\section*{\LARGE Вывод}
\addcontentsline{toc}{section}{Вывод}

В ходе выполнения практической работы был проведен анализ
программного продукта по темой дипломной работы
с опорой на стандарт ГОСТ Р ИСО/МЭК 9126,
который описывает основные характеристики качества программного обеспечения.
Продукт был оценен по таким характеристикам, как
функциональность, надежность, удобство использования, производительность,
сопровождаемость и переносимость.
В результате анализа были сформулированы требования к качеству.
Это позволит повысить общее качество программного продукта
и обеспечить его соответствие международным стандартам.

