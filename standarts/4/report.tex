\section*{\LARGE Цель практической работы}
\addcontentsline{toc}{section}{Цель практической работы}

\textbf{Цель практической работы:}
Получить навыки по определению требуемых областей
стандартизации и по поиску страндартизирующих комитетов и
существующих стандартов, созданных данными комитетами. Получить
навыки анализа стандартов предметной области для выбора наиболее
подходящих из них на основании перечня, полученного при анализе
областей стандартизации.

\textbf{Задание на практическую работу:}

\begin{enumerate}
	\item Опираясь на подпроцессы выбранной области определить
		группы стандартов, которые могут быть применены
		для выбранного дипломного проекта.
		Минимально необходимо выбрать 5 групп стандартов.
	\item Определить, какие комитеты и подкомитеты занимаются
		стандартизацией областей.
	\item Опираясь на выбранные комитеты и подкомитеты,
		определить какие стандарты могут быть использованы
		для реализуемого проекта.
		Для каждой выделенной группы стандартов необходимо
		найти минимум 2 стандарта, пригодных для проекта.
	\item На основании выбранных стандартов провести исследование,
		какие из них наиболее уместно использовать в предлагаемой разработке.
		Обоснование должно заключаться в приведении конкретных достоинств
		и недостатков выбранных стандартов по сравнению с другими.
		При приведении конкретных недостатков или достоинств стандарта
		необходимо ссылаться на пункт,
		обуславливающий это достоинство или недостаток.
\end{enumerate}

\clearpage

\section*{\LARGE Выполнение практической работы}
\addcontentsline{toc}{section}{Выполнение практической работы}

\section{Группы стандартов}

Разрабатываемый программный продукт предполагает следующие группы стандартов:

\begin{itemize}
	\item Стандарты разработки и развития ИТ систем и средств их разработки;
\item Стандарты производительности и качества ИТ продуктов и систем;
\item Стандарты безопасности ИТ систем и информации;
	\item Стандарты портативности прикладного программного обеспечения;
		\item Стандарты интероперабельности ИТ продуктов и систем;
		\item Стандарты унификации инструментов и средств разработки;
	\item Стандарты документирования ПО.
\end{itemize}

\subsection{Разработка и развитие ИТ систем и средств их разработки}

Это направление стандартизации касается создания общих стандартов
для проектирования и разработки ИТ-систем.

Оно включает в себя:

\begin{itemize}
	\item Модели архитектуры ИТ-систем, которые определяют,
		как компоненты системы взаимодействуют друг с другом.
	\item Методологии разработки программного обеспечения,
		включая Agile, Waterfall и другие,
		для обеспечения слаженной работы команд разработчиков.
	\item Инструменты разработки и программные среды
		(например, Integrated Development Environment — IDE),
		обеспечивающие совместимость между различными инструментами
		и платформами.
	\item Эти стандарты помогают разработчикам создавать более эффективные
		и надежные ИТ-системы.
\end{itemize}

\subsection{Производительность и качество ИТ продуктов и систем}

Стандарты производительности и качества ИТ-продуктов и систем
направлены на обеспечение высоких показателей работы программного обеспечения
и оборудования.

Они включают:

\begin{itemize}
	\item Метрики производительности, такие как время отклика системы,
		пропускная способность сети, скорость обработки данных и т. д.
	\item Контроль качества и тестирование продуктов на этапах разработки
		и эксплуатации для выявления ошибок и уязвимостей.
	\item Системы управления качеством, такие как ISO/IEC 25010,
		которые определяют критерии качества программного обеспечения
		(функциональность, надежность, эффективность).
	\item Эти стандарты обеспечивают, чтобы продукты
		и системы отвечали ожиданиям пользователей по качеству и стабильности.
\end{itemize}

\subsection{Безопасность ИТ систем и информации}

Стандарты в области безопасности ИТ-систем и информации (кибербезопасность)
направлены на защиту данных и предотвращение кибератак.

Они охватывают:

\begin{itemize}
	\item Криптографические методы защиты данных
		(шифрование, цифровые подписи и сертификаты)
		для защиты конфиденциальной информации.
	\item Управление доступом и аутентификацией,
		включая многофакторную аутентификацию (например, биометрия, токены)
		для обеспечения защиты от несанкционированного доступа.
	\item Политики управления рисками и инцидентами,
		которые помогают организациям реагировать на угрозы безопасности
		и минимизировать их воздействие.
\end{itemize}

\subsection{Интероперабельность ИТ продуктов и систем}

Интероперабельность означает способность различных ИТ-систем
и продуктов взаимодействовать друг с другом и обмениваться данными.

Стандарты интероперабельности охватывают:

\begin{itemize}
	\item Сетевые протоколы (например, TCP/IP, HTTP)
		для обмена данными между различными устройствами и сетями.
	\item Форматы данных (XML, JSON),
		которые обеспечивают единый способ обмена информацией
		между приложениями.
	\item API и интерфейсы для интеграции различных программных систем
		и сервисов.
	\item Эти стандарты критически важны для построения взаимосвязанных систем,
		таких как интернет вещей (IoT), облачные сервисы и корпоративные сети.
\end{itemize}

\subsection{Документирование ПО}

Стандарты в области документирования разрабатывают требования
и рекомендации для создания, оформления
и управления документацией в самых разных сферах --- от производства
и проектирования до обслуживания и безопасности.
Такие стандарты призваны обеспечить единство,
качество и удобство использования документов,
а также повысить эффективность процессов обмена информацией
и минимизировать риски,
связанные с неправильным или недостаточным документированием.

Основные цели стандартов на документирование:

\begin{itemize}
	\item Унификация и стандартизация.
		Устанавливают единые правила для создания документов,
		что облегчает их понимание, проверку и сравнение.
		Это особенно важно в компаниях и отраслях,
		где несколько отделов или организаций взаимодействуют друг с другом.
	\item Повышение качества информации.
		Стандарты требуют точности, актуальности,
		достоверности и полноты данных.
		Это помогает избежать ошибок и недопонимания,
		которые могут привести к производственным сбоям,
		потерям и несоответствиям.
	\item Соответствие требованиям законодательства.
		В ряде отраслей (например, в фармацевтике, авиации, энергетике)
		требования к документированию строго регламентированы,
		и несоответствие стандартам может привести к штрафам
		или остановке деятельности.
\end{itemize}

\clearpage

\section{Комитеты и подкомитеты}

Технический комитет (ТК) по стандартизации --- это объединение специалистов,
являющихся полномочными представителями заинтересованных предприятий
(организаций) --- членов ТК,
создаваемое на добровольной основе для разработки национальных стандартов РФ,
проведения работ в области международной (региональной)
стандартизации по закрепленным за ТК объектам стандартизации
(областям деятельности).

\begin{longtable}{|p{2cm}|p{14cm}|}
	\caption{Комитеты} \label{table:tk} \\
	\hline
	\textbf{\No\ ТК}
	& \textbf{Наименование ТК} \\
	\hline
	\endfirsthead
	\conttable{table:tk} \\
	\hline
	\textbf{\No\ ТК}
	& \textbf{Наименование ТК} \\
	\hline
	\endhead
	\textbf{022} & Информационные технологии \\ \hline
	\textbf{165}
	& Системы автоматизированного проектирования электроники \\ \hline
	\textbf{191}
	& Научно-техническая информация, библиотечное и издательское дело \\ \hline
	\textbf{362}
	& Защита информации \\ \hline
	\textbf{482}
	& Поддержка жизненного цикла продукции \\ \hline
	\textbf{700}
	& Математическое моделирование
		и высокопроизводительные вычислительные технологии \\ \hline
\end{longtable}

\subsection{Подкомитеты ТК 22 Информационные технологии}

\begin{longtable}{|p{2cm}|p{14cm}|}
	\caption{Подкомитеты ТК 22 Информационные технологии}
	\label{table:tk:22} \\
	\hline
	\textbf{\No\ ПК}
	& \textbf{Наименование ПК} \\
	\hline
	\endfirsthead
	\conttable{table:tk:22} \\
	\hline
	\textbf{\No\ ПК}
	& \textbf{Наименование ПК} \\
	\hline
	\endhead
	ПК107 (SC7) & Системная и программная инженерия \\ \hline
	ПК122 (SC22)
	& Языки программирования, их окружение
	и системы программных интерфейсов \\ \hline
	ПК125 (SC25)
	& Взаимосвязь оборудования для информационных технологий \\ \hline
	ПК127 (SC27) & Безопасность информационных технологий \\ \hline
	ПК128 (SC28) & Оборудование офисов \\ \hline
	ПК134 (SC34) & Описание документа и языки обработки \\ \hline
	ПК135 (SC35) & Пользовательские интерфейсы \\ \hline
	ПК138 (SC38) & Платформы и сервисы для распределенных приложений \\ \hline
	ПК140 (SC40)
	& Управление информационными технологиями и услугами ИТ \\ \hline
	ПК201 & Терминология в ИТ \\ \hline
	ПК206 & Интероперабельность \\ \hline
\end{longtable}

\clearpage

\section{Выбранные стандарты}

\subsection{Разработка и развитие ИТ систем и средств их разработки}

\subsubsection{Системы автоматизированного проектирования}

\paragraph{ГОСТ Р 71267 -- 2024}

\emph{\href{https://docs.cntd.ru/document/1305118385}{ГОСТ Р 71267 -- 2024}
Системы автоматизированного проектирования электроники.
Маршрут проектирования
и верификации программируемых логических интегральных схем
}

Настоящий стандарт предназначен для применения предприятиями промышленности
и организациями при использовании электронных систем проектирования
и верификации ПЛИС.
Особенности устройства и функционирования
ПЛИС требуют применения интегрированных систем,
содержащих в своем составе инструменты разработки загружаемого в ПЛИС кода,
определяющего функционирование ПЛИС в составе аппаратуры.
Такие системы проектирования и верификации ПЛИС позволяют создавать
и моделировать проект, загружаемый в ПЛИС,
с использованием принципиальных электрических схем, схем машин состояний,
а также языков описания аппаратуры на уровне регистровых передач (RTL languages).

Системы проектирования и верификации ПЛИС применяются при проектировании ЭА
и электронной компонентной базы (ЭКБ) следующего назначения:
для оборонно-промышленного комплекса, для аэрокосмической отрасли,
для судостроения, медицины, автомобильной отрасли,
для навигации и радиолокации, потребительской электроники,
для связи (телекоммуникации), для систем безопасности,
для автоматизированного транспорта и движущейся робототехники.

Моделирование ПЛИС не может полностью заменить натурные испытания,
однако может дополнить их и позволить получить данные,
которые сложно и иногда даже невозможно получить методом натурных испытаний.

Важной характеристикой при выборе САПР для проектирования
и верификации ПЛИС является возможность
и качество ее интеграции с другими применяемыми в организации САПР,
в том числе с САПР ПП.

\paragraph{ГОСТ Р 70608 -- 2022}

\emph{\href{https://docs.cntd.ru/document/1200195173}{ГОСТ Р 70608 -- 2022}
Системы автоматизированного проектирования электроники.
Состав и структура системы автоматизированного проектирования электронной компонентной базы
}

Настоящий стандарт предназначен для применения предприятиями промышленности
и организациями при использовании цифровых двойников электроники
и CALS-технологий на ранних этапах проектирования,
изготовления и испытаний ЭКБ,
а также на всех последующих этапах жизненного цикла ЭКБ.

САПР ЭКБ применяется на ранних этапах проектирования ЭКБ следующего назначения:
промышленная, для энергетики, оборонно-промышленного комплекса,
аэрокосмической отрасли, судостроения, медицинская, автомобильная,
для навигации и радиолокации, потребительская, фискального
и торгового оборудования, связи (телекоммуникации), вычислительной техники,
автоматизации и интеллектуального управления, систем безопасности,
светотехники, автоматизированного транспорта и движущейся робототехники.

ЭКБ --- это микросхемы, транзисторы, диоды, резисторы, конденсаторы и т.д.).

На ЭКБ оказывают влияние внешние дестабилизирующие факторы --- электрические,
тепловые, механические, климатические, биологические, радиационные,
электромагнитные, специальных сред и термические.
Внешние дестабилизирующие факторы могут приводить
к несоответствиям ЭКБ требованиям к их прочности и устойчивости к ВВФ.
Настоящий стандарт устанавливает состав
и структуру САПР ЭКБ на основе математического моделирования
и виртуальных испытаний ЭКБ на ВВФ при проектировании.

\subsubsection{Автоматизированные системы}

\emph{\href{https://docs.cntd.ru/document/1200181349}{ГОСТ Р 59793 -- 2021}
Информационные технологии.
Комплекс стандартов на автоматизированные системы.
Автоматизированные системы. Стадии создания
}

Настоящий стандарт распространяется на автоматизированные системы (АС),
используемые в различных видах деятельности
(исследования, управление, проектирование и т.п.),
включая их сочетания, создаваемые в организациях,
объединениях и на предприятиях (далее - организациях).

Стандарт устанавливает стадии и этапы создания АС.

\subsubsection{Жизненный цикл}

\paragraph{ГОСТ Р ИСО/МЭК 12207 -- 2010}

\emph{\href{https://docs.cntd.ru/document/1200082859}
{ГОСТ Р ИСО/МЭК 12207 -- 2010}
Информационная технология
Системная и программная инженерия
Процессы жизненного цикла программных средств
}

Настоящий стандарт, используя устоявшуюся терминологию,
устанавливает общую структуру процессов жизненного цикла программных средств,
на которую можно ориентироваться в программной индустрии.
Настоящий стандарт определяет процессы, виды деятельности и задачи,
которые используются при приобретении программного продукта или услуги,
а также при поставке, разработке, применении по назначению,
сопровождении и прекращении применения программных продуктов.
Понятие программного средства включает
в себя встроенный фирменный программный компонент.

Настоящий стандарт используется при приобретении систем,
программных продуктов и услуг, при их поставке, разработке,
применении по назначению, сопровождении
и прекращении применения программных продуктов
и программных компонентов системы как в самой организации, так и вне ее.
Эти аспекты системного определения включаются в настоящий стандарт
для обеспечения содержания понятий программных продуктов и услуг.

Настоящий стандарт устанавливает также процесс,
который может использоваться при определении, управлении
и совершенствовании процессов жизненного цикла программных средств.

Процессы, виды деятельности и задачи настоящего стандарта
--- самостоятельно либо совместно с
\href{https://docs.cntd.ru/document/1200045267}{ИСО/МЭК 15288}
--- могут также использоваться во время приобретения системы,
содержащей программные средства.

\paragraph{ГОСТ Р 57101 -- 2016/ISO/IEC/IEEE 16326:2009}

\emph{\href{https://docs.cntd.ru/document/1200139543}
{ГОСТ Р 57101 -- 2016/ISO/IEC/IEEE 16326:2009}
Разработка систем и программного обеспечения.
Процессы жизненного цикла. Управление проектом.
}

Настоящий стандарт предназначен для помощи руководителям проектов
в управлении для успешного ведения проектов,
касающихся программных средств и систем.
Настоящий стандарт определяет необходимое содержание
плана управления проектом (ПУПРП).
Этот стандарт также формулирует цели и содержание результатов процессов
проекта согласно
\href{https://docs.cntd.ru/document/1200082859}{ИСО/МЭК 12207:2008}
(IEEE 12207:2008)
и \href{https://docs.cntd.ru/document/1200045267}{ИСО/МЭК 15288:2008}
(IEEE 15288:2008)
и добавляет детальные положения для управления проектами,
в которых используются эти процессы.

\subsubsection{Упарвление}

\paragraph{ГОСТ Р 59193 -- 2020}

\emph{\href{https://docs.cntd.ru/document/1200177489}{ГОСТ Р 59193 -- 2020}
Управление конфигурацией. Общие положения
}

Настоящий стандарт устанавливает основные принципы
и задачи управления конфигурацией на всех стадиях жизненного цикла изделия,
а также объекты и субъекты управления конфигурацией.

Настоящий стандарт распространяется на изделия машиностроения
и приборостроения, в том числе на их составные части.
Применение требований стандарта к другим видам изделий определяется
по усмотрению организации разработчика.

\paragraph{ГОСТ Р 59194 -- 2020}

\emph{\href{https://docs.cntd.ru/document/573219705}{ГОСТ Р 59194 -- 2020}
Управление требованиями. Общие положения
}

Настоящий стандарт устанавливает основные цели,
задачи и принципы управления требованиями
на всех стадиях жизненного цикла изделий машиностроения,
приборостроения и их составных частей.

Применение требований стандарта к другим видам изделий определяется
по усмотрению организации-разработчика.

\subsubsection{Тестирование}

\emph{\href{https://docs.cntd.ru/document/1200134997}
{ГОСТ Р 56921 -- 2016 / ISO/IEC/IEEE 29119-2:2013}
Системная и программная инженерия.
Тестирование программ.
Часть 2 Процессы тестирования.
}

Настоящий стандарт определяет процессы тестирования,
которые могут быть использованы для руководства,
управления и реализации тестирования программного обеспечения
для любой организации, проекта или меньшего тестируемого действия.
В стандарт включены описания общих процессов тестирования,
которые определяют процессы тестирования программного обеспечения.
Кроме того, представлены вспомогательные информативные схемы,
описывающие процессы.

Настоящий стандарт применим к тестированию
для всех моделей жизненного цикла разработки программного обеспечения.

Целевая аудитория настоящего стандарта включает в себя:
тестеров, менеджеров тестирования, разработчиков и менеджеров проектов и,
особенно, ответственных за руководство, управление
и реализацию тестирования программного обеспечения,
но не ограничена этим списком.

\subsection{Производительность и качество ИТ продуктов и систем}

\subsubsection{ГОСТ Р 57700.26 -- 2020}

\emph{\href{https://docs.cntd.ru/document/573114591}{ГОСТ Р 57700.26 -- 2020}
Высокопроизводительные вычислительные системы.
Требования к приемочным испытаниям
}

Настоящий стандарт распространяется
на высокопроизводительные вычислительные системы (ВВС)
универсального назначения,
применяемые для решения широкого круга задач компьютерного моделирования
с использованием алгоритмов распараллеливания,
и устанавливает требования к видам испытаний и реализации их результатов,
порядку их проведения, а также к составу документов,
применяемых в процессе испытаний.

\subsubsection{ГОСТ Р ИСО/МЭК 9126 -- 93}

\emph{\href{https://docs.cntd.ru/document/1200009076}{ГОСТ Р ИСО/МЭК 9126 -- 93}
Информационная технология.
Оценка программной продукции.
Характеристики качества и руководства по их применению
}

Настоящий стандарт определяет шесть характеристик,
которые с минимальным дублированием описывают качество программного обеспечения.
Данные характеристики образуют основу для дальнейшего уточнения
и описания качества программного обеспечения.
Руководства описывают использование характеристик качества
для оценки качества программного обеспечения.

Настоящий стандарт не определяет подхарактеристики (комплексные показатели)
и показатели, а также методы измерения, ранжирования и оценки.
Данный стандарт придерживается определения качества по ИСО 8402.

Определения характеристик и соответствующая модель процесса оценки качества,
приведенные в настоящем стандарте, применимы тогда,
когда определены требования для программной продукции
и оценивается ее качество в процессе жизненного цикла.

Эти характеристики могут применяться к любому виду программного обеспечения,
включая программы ЭВМ и данные,
входящие в программно-технические средства (встроенные программы).

Настоящий стандарт предназначен для характеристик,
связанных с приобретением, разработкой, эксплуатацией, поддержкой,
сопровождением или проверкой программного обеспечения.

\subsubsection{ГОСТ Р ИСО/МЭК 25041 -- 2014}

\emph{\href{https://docs.cntd.ru/document/1200111122}
{ГОСТ Р ИСО/МЭК 25041 -- 2014}
Информационные технологии.
Разработка систем и программ.
Требования и оценивание качества систем и программ.
Руководство по оцениванию для разработчиков, покупателей и независимых оценщиков.
}

Настоящий стандарт содержит требования,
рекомендации и методические материалы по оценке качества продукции,
предназначенные для разработчиков, приобретателей и независимых оценщиков.
Однако он не ограничивается какой-либо конкретной прикладной областью
и может быть использован для оценки качества любого типа продукции.

Настоящий стандарт описывает процессы оценки качества продукции
и устанавливает специальные требования
для применения процесса оценки с точки зрения разработчиков,
приобретателей и независимых оценщиков.
Процесс оценки можно использовать для различных целей и подходов.
Процесс можно использовать
для оценки качества ранее разработанного программного обеспечения,
готового к использованию программного обеспечения
или разработанного по заказу как в процессе, так и по завершении разработки.

Настоящий стандарт предназначен для лиц,
ответственных за оценку программного продукта, однако,
может быть использован также разработчиками,
приобретателями и независимыми оценщиками продуктов.

Настоящий стандарт не предназначен
для оценки других аспектов программных продуктов,
таких как функциональные требования,
требования к процессу, бизнес-требования, и т.д.

%\subsubsection{}
%Численное моделирование для разработки
%и сдачи в эксплуатацию высокотехнологичных промышленных изделий.
%Применение результатов расчетов на этапах жизненного цикла изделий.

\subsection{Безопасность ИТ систем и информации}

\subsubsection{ГОСТ Р 56939 -- 2016}

\emph{\href{https://docs.cntd.ru/document/1200135525}{ГОСТ Р 56939 -- 2016}
Защита информации.
Разработка безопасного программного обеспечения.
Общие требования.
}

Настоящий стандарт устанавливает общие требования к содержанию
и порядку выполнения работ, связанных с созданием безопасного (защищенного)
программного обеспечения и формированием (поддержанием)
среды обеспечения оперативного устранения выявленных пользователями
ошибок программного обеспечения и уязвимостей программы.

Настоящий стандарт предназначен для разработчиков
и производителей программного обеспечения, а также для организаций,
выполняющих оценку соответствия процесса разработки
программного обеспечения требованиям настоящего стандарта.

Настоящий стандарт можно применять в качестве источника для формирования мер
и средств контроля и управления безопасностью программного обеспечения
в соответствии с
\href{https://docs.cntd.ru/document/1200112883}{ГОСТ Р ИСО/МЭК 27034-1}.
Настоящий стандарт можно использовать для конкретизации
или расширения компонентов доверия из
\href{https://docs.cntd.ru/document/1200105711}{ГОСТ Р ИСО/МЭК 15408-3}.

\subsubsection{ГОСТ Р 59330 -- 2021}

\emph{\href{https://docs.cntd.ru/document/1200179342}{ГОСТ Р 59330 -- 2021}
Системная инженерия.
Защита информации в процессе управления моделью жизненного цикла системы.
}

Настоящий стандарт устанавливает основные положения системного анализа
для процесса управления моделью жизненного цикла применительно
к вопросам защиты информации в системах различных областей приложения.

Требования стандарта предназначены для использования организациями,
участвующими в создании (модернизации, развитии) и эксплуатации систем
и реализующими процесс управления моделью жизненного цикла,
а также теми заинтересованными сторонами,
которые уполномочены осуществлять контроль выполнения требований
по защите информации в жизненном цикле систем.

\subsubsection{ГОСТ Р 59341 -- 2021}

\emph{\href{https://docs.cntd.ru/document/1200179545}{ГОСТ Р 59341 -- 2021}
Системная инженерия.
Защита информации в процессе управления информацией системы.
}

Настоящий стандарт устанавливает основные положения
системного анализа применительно к вопросам защиты информации
в процессе управления информацией для систем различных областей приложения.

Требования стандарта предназначены для использования организациями,
участвующими в создании (модернизации, развитии),
эксплуатации систем, выведении их из эксплуатации
и реализующими процесс управления информацией системы,
а также теми заинтересованными сторонами,
которые уполномочены осуществлять контроль выполнения требований
по защите информации в жизненном цикле систем.

\subsection{Интероперабельность ИТ продуктов и систем}

\subsubsection{ГОСТ Р 55062 -- 2021}

\emph{\href{https://docs.cntd.ru/document/1200181340}{ГОСТ Р 55062 -- 2021}
Информационные технологии. Интероперабельность. Основные положения
}

Настоящий стандарт определяет:

\begin{itemize}
	\item основные понятия, связанные с понятием <"интероперабельность">;
	\item эталонную модель интероперабельности;
	\item единый подход (методику) к обеспечению интероперабельности
		информационных систем широкого класса, представляющую метатехнологию;
	\item основные и вспомогательные этапы по достижению интероперабельности.
\end{itemize}

Настоящий стандарт предназначен для заказчиков, поставщиков, разработчиков,
потребителей, а также персонала, сопровождающего информационные системы
и осуществляющего программное обеспечение и услуги.

\subsubsection{ГОСТ Р 59797 -- 2021}

\emph{\href{https://docs.cntd.ru/document/1200181353}{ГОСТ Р 59797 -- 2021}
Информационные технологии. Сложные системы. Интероперабельность. Основные положения
}

Основная область применения настоящего стандарта
--- сложные системы различного назначения.
Настоящий стандарт предлагает набор общих правил по оценке
и обеспечению интероперабельности сложных систем.
Он может применяться на всех уровнях, включая национальный,
региональный и муниципальный,
включая государственные администрации, предприятия и организации.

Настоящий стандарт определяет:

\begin{itemize}
	\item основные термины и определения,
		связанные с понятием <<интероперабельность>> и <<сложная система>>;
	\item методику обеспечения
		и оценки интероперабельности сложных систем;
	\item описание содержания работ
		по достижению интероперабельности в ходе выполнения основных
		этапов создания (модернизации) сложных систем.
\end{itemize}

Настоящий стандарт предназначен для заказчиков,
разработчиков и потребителей,
а также персонала по сопровождению сложных систем.

\subsection{Документирование ПО}

\subsubsection{ГОСТ 34.201 -- 2020}

\emph{\href{https://docs.cntd.ru/document/1200181803}{ГОСТ 34.201 -- 2020}
Информационные технологии.
Комплекс стандартов на автоматизированные системы.
Виды, комплектность и обозначение документов
при создании автоматизированных систем.
}

Настоящий стандарт распространяется на автоматизированные системы (АС),
используемые в различных сферах деятельности
(управление, исследования, проектирование и т. п.), включая их сочетания,
и устанавливает требования к видам, наименованию,
комплектности и обозначению документов, разрабатываемых на стадиях создания АС.
В случае отсутствия выделения стадий (или деления на другие стадии)
при создании АС перечень разрабатываемой документации
и сроки ее представления определяются техническим заданием
или совместным решением заказчика и разработчика.

\subsubsection{ГОСТ Р 59988.02.1 -- 2022}

\emph{\href{https://docs.cntd.ru/document/1200192137}{ГОСТ Р 59988.02.1 -- 2022}
Системы автоматизированного проектирования электроники.
Информационное обеспечение.
Технические характеристики электронных компонентов.
Микросхемы интегральные.
Спецификации декларативных знаний по техническим характеристикам
}

Настоящий стандарт предназначен для применения при разработке
баз данных (БД), баз знаний (БЗ), технических заданий (ТЗ),
технических условий (ТУ) и прочего
и позволяет обеспечить семантическую однозначность данных
по техническим характеристикам (ТХ) электронной компонентной базы (ЭКБ).

Настоящий стандарт устанавливает правила и рекомендации
по применению в БД, БЗ и других информационных ресурсах:

\begin{itemize}
	\item предпочтительных наименований ТХ ЭКБ
		с перечнем применяемых на практике синонимов;
	\item определений ТХ ЭКБ;
	\item единиц измерения ТХ ЭКБ;
	\item квалификаторов измерения ТХ ЭКБ;
	\item типов данных ТХ ЭКБ.
\end{itemize}

Настоящий стандарт не распространяется на рассмотрение
всех проблем классификации и терминологии ТХ ЭКБ
и разработан в развитие требований государственных,
отраслевых стандартов и других руководящих документов по ЭКБ.

\subsubsection{ГОСТ Р 59988.03.1 -- 2022}

\emph{\href{https://docs.cntd.ru/document/1200195155}{ГОСТ Р 59988.03.1 -- 2022}
Системы автоматизированного проектирования электроники.
Информационное обеспечение.
Технические характеристики электронных компонентов.
Приборы и модули полупроводниковые.
Спецификации декларативных знаний по техническим характеристикам.
}

Настоящий стандарт предназначен для применения при разработке
баз данных (БД), баз знаний (БЗ), технических заданий (ТЗ),
технических условий (ТУ) и прочего,
и позволяет обеспечить семантическую однозначность данных
по техническим характеристикам (ТХ) электронной компонентной базы (ЭКБ).

Настоящий стандарт устанавливает правила и рекомендации
по применению в БД, БЗ и других информационных ресурсах:

\begin{itemize}
	\item предпочтительных наименований ТХ ЭКБ
		с перечнем применяемых на практике синонимов;
	\item определений ТХ ЭКБ;
	\item единиц измерения ТХ ЭКБ;
	\item квалификаторов измерения ТХ ЭКБ;
	\item типов данных ТХ ЭКБ.
\end{itemize}

Настоящий стандарт не распространяется на рассмотрение
всех проблем классификации и терминологии ТХ ЭКБ
и разработан в развитие требований государственных, отраслевых стандартов
и других руководящих документов по ЭКБ.

\clearpage

\section{Выбор подходящего стандарта для проекта}

Для выбора наиболее подходящих стандартов из каждой группы
для разрабатываемого проекта можно рассмотреть их достоинства и недостатки,
чтобы определить, какие из них лучше всего соответствуют специфике разработки.

\subsection{Разработка и развитие ИТ-систем и средств их разработки}

Достоинства \textbf{ГОСТ Р 59793-2021}:

\begin{itemize}
	\item Стандарт применим к широкому спектру автоматизированных систем,
		включая управление, проектирование и исследования.
		Он не ограничивается одной областью применения,
		что позволяет использовать его для различных типов АС (пункты 1 и 3).
	\item Разделение на стадии и этапы, представленные в таблице пункта 4,
		обеспечивает системный подход к планированию разработки
		и контроля выполнения задач.
\end{itemize}

Из недостатков можно выделить,
что стандарт не предоставляет конкретных рекомендаций для проектирования
и верификации электронных систем,
что усложняет его применение в задачах разработки,
связанных с САПР и DRC правилами.
Хотя стадии разработки универсальны,
их описание требует значительной адаптации для задач,
связанных с преобразованием проектных правил.

Достоинства \textbf{ГОСТ Р 70608-2022}:

\begin{itemize}
	\item Стандарт ориентирован на электронные компоненты
		и автоматизированные системы проектирования,
		что делает его идеальным для разработки системы,
		связанной с конвертацией DRC правил (пункт 1).
	\item Поддержка современных технологий позволяет обеспечить
		соответствие требованиям совместимости
		и взаимодействия с другими системами проектирования.
	\item Принципы системного единства, совместимости, типизации
		и развития предоставляют основу для построения масштабируемой
		и универсальной системы (пункты с 4.2 по 4.5).
	\item Описание проектирующих
		и обслуживающих подсистем помогает организовать
		функциональность системы, что соответствует архитектуре
		системы конвертации DRC правил (пункт 5).
	\item Наличие четких требований к компонентам видов обеспечения,
		включая математическое, информационное, программное, техническое,
		организационное и метрологическое обеспечения,
		помогает учитывать все аспекты разработки и эксплуатации системы.
		Это гарантирует соответствие системы современным требованиям
		и стандартам в области проектирования (пункт 6).
\end{itemize}

Недостаток этого стандарта --- узкая специализация
на проектировании ЭКБ может быть избыточной
для некоторых универсальных задач автоматизации.

Таким образом ГОСТ Р 70608 -- 2022 описывает ключевые принципы
и структуру построения систем автоматизированного проектирования
электронной компонентной базы (САПР ЭКБ).
Этот стандарт является наиболее подходящим
для разработки автоматизированной системы конвертации DRC правил,
так как он сочетает универсальность принципов
создания автоматизированных систем
и специализированные требования
для работы с электронной компонентной базой (ЭКБ).

ГОСТ Р 59793 -- 2021 и ГОСТ Р 71267 -- 2024 менее релевантны,
поскольку первый слишком универсален,
а второй излишне специфичен и ограничен только ПЛИС.

\subsection{Производительность и качество ИТ-продуктов и систем}

\paragraph{ИСО/МЭК 9126-93}

Достоинства:

\begin{itemize}
	\item Стандарт предоставляет шесть основных характеристик
		качества программного обеспечения (функциональность, надежность,
		удобство использования, производительность,
		сопровождаемость, переносимость),
		что позволяет структурировать анализ качества.
	\item Каждая характеристика делится на подхарактеристики,
		что детализирует анализ (например,
		в функциональности --- корректность, интероперабельность, полнота).
	\item Предлагаются количественные и качественные показатели
		для измерения характеристик,
		что делает стандарт удобным для практического применения.
	\item Подходит для оценки качества различных типов программных продуктов.
\end{itemize}

Недостатки:

\begin{itemize}
	\item Стандарт больше сосредоточен на оценке готового программного продукта
		и не охватывает полный жизненный цикл разработки.
	\item Принят в 1993 году, поэтому многие принципы
		и методики не учитывают современных технологий
		и процессов разработки.
	\item Не предоставляет методологии оценки,
		как это сделано в ИСО/МЭК 25041-2014.
\end{itemize}

\paragraph{ИСО/МЭК 25041-2014}

Достоинства:

\begin{itemize}
	\item Описывает процесс оценки, начиная с определения целей
		и заканчивая анализом результатов, что делает его универсальным
		для всех этапов жизненного цикла программного обеспечения (п. 5.1).
	\item Учитывает различия в подходах к оценке для разработчиков,
		приобретателей и независимых оценщиков,
		адаптируя процессы к их потребностям (п. 5.3).
	\item Может использоваться для оценки как статических
		(исходный код, спецификации),
		так и динамических продуктов
		(тестируемый продукт, готовое ПО) (п. 5.2).
	\item Поддерживает современную серию стандартов SQuaRE,
		включая ИСО/МЭК 25010, для оценки качества (п. 5.1).
	\item Учитывает текущие технологии и методы оценки,
		такие как использование инструментов автоматического тестирования.
\end{itemize}

Недостатки:

\begin{itemize}
	\item Процесс оценки требует значительных ресурсов,
		особенно для крупных проектов,
		где необходимо документировать каждую стадию оценки.
	\item В меньшей степени ориентирован на характеристику качества ПО,
		больше фокусируется на процессах и методах оценки.
\end{itemize}

\paragraph{Итоги сравнения}

ИСО/МЭК 9126-93 подходит для оценки качества ПО в простых проектах,
где важна структурированная модель характеристик качества.
Он полезен для базового анализа готовых продуктов,
но требует адаптации для современных проектов.

ИСО/МЭК 25041-2014 более современный и ориентирован на процессы оценки,
что делает его предпочтительным для сложных проектов.
Он предоставляет детальную методологию,
адаптированную к разным ролям и этапам жизненного цикла.

Для систем,
где требуется комплексная оценка качества на всех этапах жизненного цикла,
предпочтительнее использовать ИСО/МЭК 25041-2014.

\subsection{Безопасность ИТ-систем и информации}

\paragraph{ГОСТ Р 56939-2016}

Достоинства:

\begin{itemize}
	\item Стандарт интегрирован с процессами разработки ПО,
		установленными ГОСТ Р ИСО/МЭК 12207,
		что позволяет применять его
		на всех стадиях жизненного цикла ПО (п. 4.1).
	\item Предусматривает использование альтернативных мер
		при невозможности реализации базового набора,
		что делает стандарт гибким (п. 4.5).
	\item Учитывает требования ГОСТ Р ИСО/МЭК 15408
		для подготовки к оценке доверия и ГОСТ Р ИСО/МЭК 27034
		для обеспечения безопасности (п. 4.6).
	\item Определяет содержание руководства по разработке безопасного ПО
		и список документации,
		связанной с обеспечением безопасности (п. 4.10–4.13).
	\item Использует три уровня требований
		(обязательные, рекомендуемые, допустимые),
		что позволяет адаптировать меры безопасности к проекту (п. 4.2).
\end{itemize}

Недостатки:

\begin{itemize}
	\item Реализация полного набора мер требует значительных человеческих
		и технических ресурсов.
	\item Подразумевает проведение множества внутренних проверок,
		тестирований и экспертиз, что увеличивает время разработки.
\end{itemize}

\paragraph{ГОСТ Р 59330-2021}

Достоинства:

\begin{itemize}
	\item Фокусируется на системной инженерии,
		применяемой к процессу управления моделью жизненного цикла системы,
		включая защиту конфиденциальности,
		целостности и доступности информации (п. 4.1, 4.2).
	\item Предусматривает использование системного анализа для прогнозирования
		и управления рисками на всех стадиях жизненного цикла (п. 4.2.2).
	\item Может применяться к любым этапам жизненного цикла системы,
		что делает его универсальным для сложных систем (п. 4.3).
	\item Учитывает рекомендации и принципы ГОСТ Р ИСО/МЭК 27001,
		ГОСТ Р ИСО/МЭК 27002 и других,
		что обеспечивает его совместимость
		с международными практиками (п. 4.2.2, 4.4).
	\item Подразумевает использование элементов
		из других процессов жизненного цикла,
		что облегчает интеграцию (п. 4.3).
\end{itemize}

Недостатки:

\begin{itemize}
	\item Основное внимание уделяется защите информации на уровне системы,
		что может быть избыточным для менее сложных проектов.
	\item Для выполнения системного анализа требуется глубокое понимание
		и использование методов управления рисками, что может быть трудоемким.
\end{itemize}

\paragraph{Итоги сравнения}

ГОСТ Р 56939 -- 2016 подходит для проектов,
где требуется обеспечение безопасности на уровне программного обеспечения.
Он включает четкие требования и меры для разработки защищенного ПО,
что делает его предпочтительным для проектов,
связанных с ПО общего назначения.

ГОСТ Р 59330 -- 2021 больше ориентирован на сложные системы и процессы,
где управление моделью жизненного цикла требует прогнозирования
и защиты информации на системном уровне.
Этот стандарт предпочтителен
для крупномасштабных систем с множеством взаимосвязанных подсистем.

Для разработки системы конвертации DRC правил ГОСТ Р 56939 -- 2016
будет более подходящим, так как он ориентирован на обеспечение безопасности ПО,
включая обработку уязвимостей и документирование мер безопасности.

\subsection{Интероперабельность ИТ-продуктов и систем}

\paragraph{ГОСТ Р 55062 -- 2021}

Достоинства:

\begin{itemize}
	\item Представляет собой развитие семиуровневой эталонной модели ВОС
		для обеспечения интероперабельности.
		Это упрощает интеграцию различных
		систем благодаря проверенным решениям (п. 5).
	\item Стандарт предлагает универсальный подход
		с четким описанием технического, семантического
		и организационного уровней взаимодействия (п. 5.1–5.3),
		что помогает устранить барьеры на разных этапах интеграции.
	\item Указывает на использование признанных стандартов
		(например, TCP/IP для технического уровня
		и XML для семантического уровня),
		что обеспечивает совместимость и широкую применимость (п. 5.1, 5.2).
	\item Включает как теоретические, так и экспериментальные этапы,
		позволяя адаптировать методику под конкретные проекты (п. 6).
\end{itemize}

Недостатки:

\begin{itemize}
	\item Хотя стандарт применим к широкому классу систем,
		он не предоставляет детализированных рекомендаций
		для сложных или узкоспециализированных систем.
	\item Подходит для типовых систем,
		но не охватывает специфические характеристики многокомпонентных,
		взаимосвязанных систем.
\end{itemize}

\paragraph{ГОСТ Р 59797 -- 2021}

Достоинства:

\begin{itemize}
	\item Стандарт предназначен для оценки
		и обеспечения интероперабельности сложных систем,
		где каждая подсистема может
		быть самостоятельной системой (п. 1.1 и 4.2).
		Это делает его полезным для крупномасштабных проектов.
	\item Использует принципы системной инженерии,
		что особенно важно при проектировании взаимосвязанных
		и многоуровневых систем (п. 5.1).
	\item Предлагает трехмерную архитектурную модель,
		включающую функции, сервисы и множество подсистем,
		что помогает структурировать сложные проекты (п. 5.3.2).
	\item Уделяет внимание взаимному влиянию характеристик подсистем,
		что повышает устойчивость и совместимость сложных систем (п. 4.4).
\end{itemize}

Недостатки:

\begin{itemize}
	\item Реализация методики требует значительных ресурсов,
		что делает стандарт менее применимым для небольших проектов
		или систем с ограниченными задачами.
	\item Хотя стандарт охватывает технические и семантические аспекты,
		организационная интероперабельность
		не является его ключевым направлением.
\end{itemize}

\paragraph{Итоги сравнения}

ГОСТ Р 55062-2021 лучше подходит для типовых информационных систем
с универсальными требованиями к интероперабельности.
Он обеспечивает простую реализацию,
ориентированную на использование стандартных протоколов и методик,
что делает его предпочтительным
для проектов с ограниченным масштабом или ресурсами.

ГОСТ Р 59797-2021 предпочтителен для крупных,
многокомпонентных систем с разнородными подсистемами,
где важны детализированные подходы к обеспечению взаимодействия.
Стандарт лучше подходит для сложных систем,
таких как государственные или корпоративные платформы,
где важны интеграция и масштабируемость.

Для разработки системы конвертации DRC правил,
которая является частью экосистемы EDA (Electronic Design Automation),
более подходящим представляется ГОСТ Р 55062-2021,
так как он предлагает универсальный подход к интероперабельности,
достаточный для взаимодействия с различными инструментами проектирования.

\subsection{Документирование ПО}

\paragraph{ГОСТ 34.201 -- 2020}

Достоинства:

\begin{itemize}
	\item Применим к автоматизированным системам (АС)
		в различных сферах деятельности, включая управление,
		проектирование и исследования,
		что делает его широко применимым для большинства проектов ПО (п. 1).
	\item Устанавливает четкие требования к видам,
		наименованию и комплектности документации
		на всех стадиях разработки (п. 4.1–4.5).
		Это позволяет разработчику планировать полный объем документации
		для конкретного проекта.
	\item Допускает адаптацию структуры документов под специфику проекта,
		включая разбиение на части, расширение номенклатуры
		и использование групповых документов (п. 3).
	\item Описывает организационно-распорядительные документы,
		такие как акты приемки, планы-графики и протоколы испытаний,
		что упрощает переход к эксплуатации (п. 3).
\end{itemize}

Недостатки:

\begin{itemize}
	\item Не уделяет внимания семантической однозначности данных,
		что важно для сложных баз данных и баз знаний.
	\item Может быть избыточен для простых проектов ПО,
		не включающих элементы аппаратной части.
\end{itemize}

\paragraph{ГОСТ Р 59988.02.1 -- 2022}

Достоинства:

\begin{itemize}
	\item Предоставляет правила
		и рекомендации для обеспечения однозначности
		технических характеристик (ТХ) ЭКБ,
		что важно для баз данных и знаний (п. 1.2).
	\item Регламентирует использование предпочтительных наименований,
		синонимов, единиц измерения и квалификаторов,
		что улучшает структурирование данных (п. 4).
	\item Подробные указания по созданию спецификаций для ТХ ЭКБ,
		представленных в приложениях А и Б,
		помогают унифицировать данные (п. 5.2).
\end{itemize}

Недостатки:

\begin{itemize}
	\item Фокусируется исключительно на характеристиках
		электронной компонентной базы,
		что ограничивает применение в других сферах.
	\item Меньший акцент на общие документы и этапы разработки ПО,
		такие как акты приемки или планы-графики.
\end{itemize}

\paragraph{ГОСТ Р 59988.03.1 -- 2022}

Достоинства:

\begin{itemize}
	\item Развивает ГОСТ Р 59988.02.1-2022,
		адаптируя его к "Приборам и модулям полупроводниковым"
		и предоставляя рекомендации для этого класса компонентов (п. 4).
	\item Устанавливает стандарты классификации,
		измерений и типов данных для полупроводниковых приборов,
		что упрощает управление информацией (п. 5.1).
	\item Содержит примеры спецификаций для практического применения,
		что облегчает внедрение (п. 5.2).
\end{itemize}

Недостатки:

\begin{itemize}
	\item Применим только для систем,
		связанных с ЭКБ в области полупроводниковых приборов.
	\item Стандарт не подходит для проектов, не связанных с ЭКБ.
\end{itemize}

\paragraph{Итоги сравнения}

ГОСТ 34.201-2020 наиболее подходящий для общего документирования ПО.
Он универсален, гибок и охватывает весь жизненный цикл разработки.

ГОСТ Р 59988.02.1-2022 и ГОСТ Р 59988.03.1-2022 применимы только для систем,
связанных с разработкой баз данных и знаний по ЭКБ.
Эти стандарты лучше подходят для проектов,
связанных с техническими характеристиками компонентов,
а не для ПО общего назначения. 

ГОСТ 34.201-2020 более подходящий для использования
для разработки системы конвертации DRC правил.

\clearpage

\section*{\LARGE Вывод}
\addcontentsline{toc}{section}{Вывод}

В ходе выполнения задания были изучены различные группы стандартов,
связанные с темой проекта.
Было отмечено, что каждая группа стандартов включает
свои специфические аспекты,
такие как безопасность данных, точность конвертации правил,
взаимодействие инструментов проектирования,
разработка программного обеспечения и документирование.
Это позволило глубже понять требования и рекомендации,
которые необходимо учитывать при реализации подобных проектов.

