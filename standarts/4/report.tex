\section*{\LARGE Цель практической работы}
\addcontentsline{toc}{section}{Цель практической работы}

\textbf{Цель практической работы:}
Получить навыки по определению требуемых областей
стандартизации и по поиску страндартизирующих комитетов и
существующих стандартов, созданных данными комитетами. Получить
навыки анализа стандартов предметной области для выбора наиболее
подходящих из них на основании перечня, полученного при анализе
областей стандартизации.

\textbf{Задание на практическую работу:}

\begin{enumerate}
	\item Опираясь на подпроцессы выбранной области определить
		группы стандартов, которые могут быть применены
		для выбранного дипломного проекта.
		Минимально необходимо выбрать 5 групп стандартов.
	\item Определить, какие комитеты и подкомитеты занимаются
		стандартизацией областей.
	\item Опираясь на выбранные комитеты и подкомитеты,
		определить какие стандарты могут быть использованы
		для реализуемого проекта.
		Для каждой выделенной группы стандартов необходимо
		найти минимум 2 стандарта, пригодных для проекта.
	\item На основании выбранных стандартов провести исследование,
		какие из них наиболее уместно использовать в предлагаемой разработке.
		Обоснование должно заключаться в приведении конкретных достоинств
		и недостатков выбранных стандартов по сравнению с другими.
		При приведении конкретных недостатков или достоинств стандарта
		необходимо ссылаться на пункт,
		обуславливающий это достоинство или недостаток.
\end{enumerate}

\clearpage

\section*{\LARGE Выполнение практической работы}
\addcontentsline{toc}{section}{Выполнение практической работы}

\section{Группы стандартов}

Разрабатываемый программный продукт предполагает следующие группы стандартов:

\begin{itemize}
	\item Стандарты разработки и развития ИТ систем и средств их разработки;
\item Стандарты производительности и качества ИТ продуктов и систем;
\item Стандарты безопасности ИТ систем и информации;
	\item Стандарты портативности прикладного программного обеспечения;
		\item Стандарты интероперабельности ИТ продуктов и систем;
		\item Стандарты унификации инструментов и средств разработки;
	\item Стандарты документирования ПО.
\end{itemize}

\subsection{Разработка и развитие ИТ систем и средств их разработки}

Это направление стандартизации касается создания общих стандартов
для проектирования и разработки ИТ-систем.

Оно включает в себя:

\begin{itemize}
	\item Модели архитектуры ИТ-систем, которые определяют,
		как компоненты системы взаимодействуют друг с другом.
	\item Методологии разработки программного обеспечения,
		включая Agile, Waterfall и другие,
		для обеспечения слаженной работы команд разработчиков.
	\item Инструменты разработки и программные среды
		(например, Integrated Development Environment — IDE),
		обеспечивающие совместимость между различными инструментами
		и платформами.
	\item Эти стандарты помогают разработчикам создавать более эффективные
		и надежные ИТ-системы.
\end{itemize}

\subsection{Производительность и качество ИТ продуктов и систем}

Стандарты производительности и качества ИТ-продуктов и систем
направлены на обеспечение высоких показателей работы программного обеспечения
и оборудования.

Они включают:

\begin{itemize}
	\item Метрики производительности, такие как время отклика системы,
		пропускная способность сети, скорость обработки данных и т. д.
	\item Контроль качества и тестирование продуктов на этапах разработки
		и эксплуатации для выявления ошибок и уязвимостей.
	\item Системы управления качеством, такие как ISO/IEC 25010,
		которые определяют критерии качества программного обеспечения
		(функциональность, надежность, эффективность).
	\item Эти стандарты обеспечивают, чтобы продукты
		и системы отвечали ожиданиям пользователей по качеству и стабильности.
\end{itemize}

\subsection{Безопасность ИТ систем и информации}

Стандарты в области безопасности ИТ-систем и информации (кибербезопасность)
направлены на защиту данных и предотвращение кибератак.

Они охватывают:

\begin{itemize}
	\item Криптографические методы защиты данных
		(шифрование, цифровые подписи и сертификаты)
		для защиты конфиденциальной информации.
	\item Управление доступом и аутентификацией,
		включая многофакторную аутентификацию (например, биометрия, токены)
		для обеспечения защиты от несанкционированного доступа.
	\item Политики управления рисками и инцидентами,
		которые помогают организациям реагировать на угрозы безопасности
		и минимизировать их воздействие.
\end{itemize}

\subsection{Портативность прикладного программного обеспечения}

Портативность программного обеспечения означает способность программы
работать на различных платформах и системах без модификации исходного кода.

Стандарты в этой области включают:

\begin{itemize}
	\item Поддержку кроссплатформенных разработок,
		когда программы или приложения могут работать
		на разных операционных системах и устройствах.
	\item Использование универсальных библиотек и языков программирования
		(например, Java, Python),
		что облегчает перенос приложений на другие среды.
	\item Эти стандарты помогают разработчикам создавать программы,
		которые легко адаптируются к различным аппаратным
		и программным платформам.
\end{itemize}

\subsection{Интероперабельность ИТ продуктов и систем}

Интероперабельность означает способность различных ИТ-систем
и продуктов взаимодействовать друг с другом и обмениваться данными.

Стандарты интероперабельности охватывают:

\begin{itemize}
	\item Сетевые протоколы (например, TCP/IP, HTTP)
		для обмена данными между различными устройствами и сетями.
	\item Форматы данных (XML, JSON),
		которые обеспечивают единый способ обмена информацией
		между приложениями.
	\item API и интерфейсы для интеграции различных программных систем
		и сервисов.
	\item Эти стандарты критически важны для построения взаимосвязанных систем,
		таких как интернет вещей (IoT), облачные сервисы и корпоративные сети.
\end{itemize}

\subsection{Унификация инструментов и средств разработки}

Стандарты в этой области направлены на создание единого подхода
к использованию инструментов разработки,
что повышает совместимость между ними
и облегчает разработку программного обеспечения.

Это включает:

\begin{itemize}
	\item Единые стандарты для инструментов управления проектами
		и кодом (например, Git, Jira),
		которые позволяют эффективно управлять проектами
		и отслеживать прогресс разработки.
	\item Унификация средств автоматизированного тестирования,
		что упрощает тестирование программных решений.
	\item Эти стандарты облегчают разработку программного обеспечения
		и повышают продуктивность команд разработчиков.
\end{itemize}

\section{Комитеты и подкомитеты}

\begin{longtable}{|p{2cm}|p{14cm}|}
	\caption{Комитеты} \label{table:tk} \\
	\hline
	\textbf{\No\ ТК}
	& \textbf{Наименование ТК} \\
	\hline
	\endfirsthead
	\conttable{table:tk} \\
	\hline
	\textbf{\No\ ТК}
	& \textbf{Наименование ТК} \\
	\hline
	\endhead
	\textbf{022} & \textbf{Информационные технологии} \\ \hline
	ПК107 (SC7) & Системная и программная инженерия \\ \hline
	ПК122 (SC22)
	& Языки программирования, их окружение
	и системы программных интерфейсов \\ \hline
	ПК125 (SC25)
	& Взаимосвязь оборудования для информационных технологий \\ \hline
	ПК127 (SC27) & Безопасность информационных технологий \\ \hline
	ПК128 (SC28) & Оборудование офисов \\ \hline
	ПК134 (SC34) & Описание документа и языки обработки \\ \hline
	ПК135 (SC35) & Пользовательские интерфейсы \\ \hline
	ПК138 (SC38) & Платформы и сервисы для распределенных приложений \\ \hline
	ПК140 (SC40)
	& Управление информационными технологиями и услугами ИТ \\ \hline
	ПК201 & Терминология в ИТ \\ \hline
	ПК206 & Интероперабельность \\ \hline
	\textbf{165}
	& Системы автоматизированного проектирования электроники \\ \hline
	\textbf{191}
	& Научно-техническая информация, библиотечное и издательское дело \\ \hline
	\textbf{362}
	& Защита информации \\ \hline
	\textbf{482}
	& Поддержка жизненного цикла продукции \\ \hline
	\textbf{700}
	& Математическое моделирование
		и высокопроизводительные вычислительные технологии \\ \hline
\end{longtable}

\subsection{(22) Информационные технологии}

Информационная технология. Методы и средства обеспечения безопасности. Практические правила управления информационной безопасностью.

Информационная технология. Язык описания представления задач. Версия 1.0.

Информационная технология. Методы и средства обеспечения безопасности. Безопасность приложений. Часть 1. Общий обзор и концепции. Разработка ГОСТ Р. Прямое применение МС – IDT ISO/IEC 27034-1:2011

Информационная технология. Методы и средства обеспечения безопасности. Руководство по совместному использованию стандартов ИСО/МЭК 27001 и ИСО/МЭК 20000–1. Разработка ГОСТ Р. Прямое применение MC-IDT ISO/IEC 27013:2012

Информационные технологии. Проектирование систем и разработка программного обеспечения. Требования к качеству систем и программного обеспечения и их оценка (SQuaRE). Процесс оценки. Разработка ГОСТ Р. Прямое применение МС-IDT ISO/IEC 25040:2011

Информационные технологии. Разработка систем и программ. Требования и оценивание качества систем и программ. Руководство по оцениванию для разработчиков, покупателей и независимых оценщиков.

Информационные технологии. Разработка систем и программ. Требования к качеству систем и программ и их оценка. Элементы показателя качества. Разработка ГОСТ Р. Прямое применение МС-IDT ISO/IEC 25021:2012

Информационные технологии. Версия 1.2 3C SOAP. Второе издание. Разработка ГОСТ Р. Прямое применение МС-IDT ISO/IEC 40220:2011

Информационные технологии. Проектирование систем и разработка программного обеспечения. Требования к качеству систем и программного обеспечения и их оценка (SQuaRE). Модели качества систем и программного обеспечения. Разработка ГОСТ Р.  Прямое применение МС-IDT ISO/IEС 25010:2011

Информационные технологии. Оценка процессов. Часть 9. Профили целевого процесса. Разработка ГОСТ Р. Прямое применение MC-IDT ISO/IEC TS 15504-9:2011

Информационные технологии. Словарь. Разработка ГОСТ. Прямое применение MC-IDT ISO/IEC 2382 (DIS)

Системная и программная инженерия. Обеспечение систем и программ. Обеспечение жизненного цикла. Разработка ГОСТ Р. Прямое применение МС – IDT ISO/IEC 15026-4:2012

Системная и программная инженерия. Инструменты и методы для технического управления производственной линией. Разработка ГОСТ Р. Прямое применение МС – IDT ISO/IEC 26555:2013

Системная и программная инженерия. Испытание программ. Часть 3. Проверка документации. Разработка ГОСТ Р Прямое применение МС – IDT ISO/IEC/IEEE 29119-3:2013

Системная и программная инженерия. Тестирование программ. Часть 1. Концепции и определения. Разработка ГОСТ Р Прямое применение МС – IDT ISO/IEC/IEEE 29119-1:2013

Системная и программная инженерия. Тестирование программ. Часть 2. Процессы тестирования. Разработка ГОСТ Р Прямое применение МС – IDT ISO/IEC/IEEE 29119-2:2013

Системная и программная инженерия - Обеспечение систем и программ - Часть 1. Концепции и словарь. Разработка ГОСТ Р Прямое применение МС – IDT ISO/IEC 15026-1:2013

Информационные технологии. Нотация абстрактного синтаксиса один (АSN.1). Часть 4. Спецификация для параметризации ASN.1. Пересмотр ГОСТ Р. Прямое применение МС - IDT ISO/IEC 8824-4:2008, ГОСТ Р ИСО/МЭК 8824-4-2003

Информационные технологии. Оценка процессов. Часть 5. Образец модели оценки процессов жизненного цикла программного обеспечения. Разработка ГОСТ Р. Прямое применение МС – IDT ISO/IEC 15504-5:2012

Информационные технологии. Тематические планы. Часть 6. Компактный синтаксис. Разработка ГОСТ Р. Прямое применение МС – IDT ISO/IEC 13250-6:2010

Разработка систем и программного обеспечения. Менеджмент жизненного цикла. Руководящие указания по описанию процесса. Разработка ГОСТ Р. Прямое применение МС – IDT ISO/IEC TR 24774:2010

Разработка систем и программного обеспечения. Процессы жизненного цикла. Управление проектом. Разработка ГОСТ Р. Прямое применение МС – IDT ISO/IEC/IEEE 16326:2009

Разработка систем программного обеспечения. Менеджмент жизненного цикла. Часть 2. Руководство по применению ISO/IEC 15288. Разработка ГОСТ Р. Прямое применение МС – IDT ISO/IEC TR 24748-2:2011

Системная и программная инженерия. Требования к качеству систем и программного обеспечения и их оценка (SQuaRE). Планирование и менеджмент. Разработка ГОСТ Р. Прямое применение ISO/IEC 25001:2014 (гармонизация)

Системная и программная инженерия. Процесс измерения. Разработка ГОСТ Р Идентичен (IDT) ISO/IEC/IEEE 15939:2017

Системная и программная инженерия. Состав и содержание информационных элементов жизненного цикла (документация). Разработка ГОСТ Р Идентичен (IDT) ISO/IEC/IEEE 15289:2017

Системная и программная инженерия. Методы и инструменты реализации механизмов вариабельности для линейки программных и системных продуктов. Разработка ГОСТ Р Идентичен (IDT) ISO/IEC 26557:2016

Информационная технология. Комплекс стандартов на автоматизированные системы. Виды, комплектность и обозначение документов при создании автоматизированных систем. Пересмотр ГОСТ 34.201-89

Системная и программная инженерия. Требования и оценка качества систем и программной продукции (SQuaRE). Измерение качества системы и программной продукции. Разработка ГОСТ Р Идентичен (IDT) ISO/IEC 25023:2016

! Информационная технология. Комплекс стандартов на автоматизированные системы. Автоматизированные системы. Термины и определения. Пересмотр ГОСТ 34.003-90

! Информационная технология. Комплекс стандартов на автоматизированные системы. Техническое задание на создание автоматизированной системы. Пересмотр ГОСТ 34.602-89

! Информационные технологии. Методы и средства обеспечения безопасности. Информационная безопасность во взаимоотношениях с поставщиками. Часть 2. Требования. Разработка IDT- ISO/IEC 27036-2:2020

! Информационные технологии. Методы и средства обеспечения безопасности. Информационная безопасность во взаимоотношениях с поставщиками. Часть 3. Рекомендации по обеспечению безопасности цепи поставок информационных и коммуникационных технологий. Разработка ГОСТ Р IDT ISO/IEC 27036-3:2020

Информационные технологии. Методы и средства обеспечения безопасности. Менеджмент информационной безопасности при обмене информацией между отраслями и организациями. Разработка IDT ISO/IEC 27010:2015

Информационные технологии. Методы и средства обеспечения безопасности. Свод норм и правил применения мер обеспечения информационной безопасности. Разработка ГОСТ Р IDT ISO/IEC 27002-2013 Взамен ГОСТ Р ИСО/МЭК 27002-2012

Информационные технологии. Методы и средства обеспечения безопасности. Системы менеджмента информационной безопасности. Общий обзор и терминология. Разработка IDT ISO/IEC 27000:2018. Взамен ГОСТ Р ИОС/МЭК 27000:2012

Информационные технологии. Методы и средства обеспечения безопасности. Информационная безопасность во взаимоотношениях с поставщиками. Часть 1. Обзор и основные понятия. Разработка ГОСТ Р IDT ISO/IEC 27036-1:2014

Информационные технологии. Методы и средства обеспечения безопасности. Безопасность приложений. Часть 2. Нормативная структура организации. Разработка ГОСТ Р IDT ISO/IEC 27034-2:2015

Информационные технологии. Методы и средства обеспечения безопасности. Основы управления доступом. Разработка ГОСТ Р 59383-2021. NEQ ISO/IEC 29146:2016

Системная инженерия. Защита информации в процессе управления моделью жизненного цикла системы. Разработка ГОСТ Р 59330-2021

Системная инженерия. Защита информации в процессе управления конфигурацией системы. Разработка ГОСТ Р 59340-2021

Системная инженерия. Защита информации в процессе управления информацией системы. Разработка ГОСТ Р 59341-2021

Системная инженерия. Защита информации в процессе сопровождения системы. Разработка ГОСТ Р 59356-2021

\subsection{(165) Системы автоматизированного проектирования электроники}

ГОСТ Р 70290-2022 Системы автоматизированного проектирования электроники. Термины и определения (Утвержден 18 августа 2022 г.  Приказ № 782-ст Введен в действие с 01.10.22)

ГОСТ Р 70291-2022 Системы автоматизированного проектирования электроники. Состав и структура системы автоматизированного проектирования электронной аппаратуры (Утвержден 18 августа 2022 г.  Приказ № 783-ст Введен в действие с 01.10.22)

 ГОСТ Р 70293-2022 Системы автоматизированного проектирования электроники. Подсистема автоматизированного анализа показателей надёжности электронной аппаратуры (Утвержден 18 августа 2022 г.  Приказ № 785-ст Введен в действие с 01.10.22)

ГОСТ Р 70607-2022 Системы автоматизированного проектирования электроники. Состав и структура системы автоматизированного проектирования печатных узлов

ГОСТ Р 70608-2022 Системы автоматизированного проектирования электроники. Состав и структура системы автоматизированного проектирования электронной компонентной базы

ГОСТ Р 70911-2023 Системы автоматизированного проектирования электроники. Подсистема виртуальных испытаний электронной аппаратуры на воздействие одиночного механического удара

ГОСТ Р 70913-2023 Системы автоматизированного проектирования электроники. Подсистема виртуальных испытаний электронной аппаратуры на стационарные тепловые воздействия

ГОСТ Р 71132-2023 Системы автоматизированного проектирования электроники. Подсистема виртуальных испытаний электронной аппаратуры на воздействие статических нагрузок

ГОСТ Р 59988.00.0-2022 Системы автоматизированного проектирования электроники. Информационное обеспечение. Технические характеристики электронных компонентов. Общие положения

ГОСТ Р 59988.02.1-2022 Системы автоматизированного проектирования электроники. Информационное обеспечение. Технические характеристики электронных компонентов. Микросхемы интегральные. Спецификации декларативных знаний по техническим характеристикам

ГОСТ Р 59988.02.2-2022 Системы автоматизированного проектирования электроники. Информационное обеспечение. Технические характеристики электронных компонентов. Микросхемы интегральные. Перечень технических характеристик (Утвержден 13 июля 2022 г.  Приказ № 624-ст Введен в действие с 01.08.22)

ГОСТ Р 59988.03.1-2022 Системы автоматизированного проектирования электроники. Информационное обеспечение. Технические характеристики электронных компонентов. Приборы и модули полупроводниковые. Спецификации декларативных знаний по техническим характеристикам (Утвержден 27 декабря 2022 г.  Приказ № 1669-ст Введен в действие с 01.01.23)

ГОСТ Р 59988.03.2-2022 Системы автоматизированного проектирования электроники. Информационное обеспечение. Технические характеристики электронных компонентов. Приборы и модули полупроводниковые. Перечень технических характеристик (Утвержден 27 декабря 2022 г.  Приказ № 1670-ст Введен в действие с 01.01.23)

ГОСТ Р 71264-2024 Системы автоматизированного проектирования электроники. Технологическая подготовка производства печатных плат в системах автоматизированного проектирования (Утвержден 29 февраля 2024 г.  Приказ № 259-ст Введен в действие с 01.04.24)

ГОСТ Р 71265-2024 Системы автоматизированного проектирования электроники. Анализ целостности сигналов и питания на печатных платах. Маршрут анализа проектов и обработка результатов

ГОСТ Р 71267-2024 Системы автоматизированного проектирования электроники. Маршрут проектирования и верификации программируемых логических интегральных схем

\subsection{(191) Научно-техническая информация, библиотечное и издательское дело}

Система стандартов по информации, библиотечному и издательскому делу. Номер государственной регистрации обязательных экземпляров документов. Структура, оформление, использование

Система стандартов по информации, библиотечному и издательскому делу. Международный стандартный идентификатор библиотечного предмета учета

Система стандартов по информации, библиотечному и издательскому делу. Формирование фонда документов. Термины и определения

Система стандартов по информации, библиотечному и издательскому делу. Статьи в журналах и сборниках. Издательское оформление

Система стандартов по информации, библиотечному и издательскому делу. Библиографическая ссылка информационных электронных ресурсов и их составных частей. Общие требования и правила составления

Система стандартов по информации, библиотечному и издательскому делу. Взаимодействие тезаурусов и других словарей

\subsection{(362) Защита информации}

Защита информации. Формальное моделирование политики безопасности. Часть 1. Формальная модель управления доступом

Защита информации. Разработка безопасного программного обеспечения. Руководство по разработке безопасного программного обеспечения

Защита информации. Мониторинг информационной безопасности. Общие положения

Защита информации. Регистрация событий безопасности. Требования к регистрируемой информации

\subsection{(482) Поддержка жизненного цикла продукции}

Управление требованиями. Общие положения

Управление конфигурацией. Общие положения

Управление данными об изделии. Порядок представления результатов технологической подготовки производства и технологических данных в электронной форме

Управление данными о качестве продукции на стадиях жизненного цикла. Порядок вычисления показателей

Интегрированная логистическая поддержка экспортируемой продукции военного назначения. Общие требования к комплексным программам обеспечения эксплуатационно-технических характеристик

Управление данными о качестве продукции на стадиях жизненного цикла. Номенклатура показателей

Управление данными об изделии. Требования к составу, содержанию, оформлению, разработке и использованию нормативно-справочной документации

Управление данными о качестве продукции на стадиях жизненного цикла. Исходные данные для вычисления показателей

\subsection{(700) Математическое моделирование
	и высокопроизводительные вычислительные технологии}

\begin{itemize}
	\item 1.11.700-1.016.19
Методы математического моделирования и виртуализации испытаний электронной компонентной базы и электронной аппаратуры на механические воздействия при проектировании.
	\item 1.11.700-1.015.19
Методы математического моделирования и виртуализации испытаний электронной компонентной базы и электронной аппаратуры на тепловые воздействия при проектировании.
	\item 1.11.700-1.014.19
Методы создания карт рабочих режимов электронной компонентной базы на основе математического моделирования и виртуализации испытаний электронной компонентной базы и электронной аппаратуры на внешние воздействующие факторы при проектировании.
	\item 1.11.700-1.013.19
Технология математического моделирования и виртуализации испытаний электронной компонентной базы и электронной аппаратуры на внешние воздействующие факторы на всех этапах жизненного цикла.
	\item 1.11.700-1.008.18
Численное моделирование для разработки и сдачи в эксплуатацию высокотехнологичных промышленных изделий. Применение результатов расчетов на этапах жизненного цикла изделий.
	\item 1.11.700-1.007.18
Высокопроизводительные вычислительные системы. Термины и определения
	\item 1.11.700-1.002.18
Высокопроизводительные вычислительные системы. Требования к приемочным испытаниям
\end{itemize}

\section{Выбранные стандарты}

\subsection{Разработка и развитие ИТ систем и средств их разработки}

ГОСТ Р ИСО/МЭК 12207--2010\\
ISO/IEC 12207:2017** (Systems and software engineering — Software life cycle processes)
     - Описание процессов и методов, связанных с разработкой и сопровождением программных продуктов на протяжении всего жизненного цикла.

ISO/IEC 24744:2014** (Software Engineering — Metamodel for Development Methodologies)
     - Стандарт, описывающий модели и методологии для разработки программного обеспечения и их эволюции.

\subsection{Производительность и качество ИТ продуктов и систем}
ГОСТ Р ИСО/МЭК 25010-2015\\
%ISO/IEC 25010:2011** (Systems and software engineering — Systems and software Quality Requirements and Evaluation (SQuaRE) — System and software quality models)
ISO/IEC 25010:2023 Systems and software engineering — Systems and software Quality Requirements and Evaluation (SQuaRE) — Product quality model

     - Определяет основные качества, такие как производительность, надежность и эффективность программных систем.


ISO/IEC 25019:2023
Systems and software engineering — Systems and software Quality Requirements and Evaluation (SQuaRE) — Quality-in-use model

ISO/IEC 25002:2024
Systems and software engineering — Systems and software Quality Requirements and Evaluation (SQuaRE) — Quality model overview and usage

\subsection{Безопасность ИТ систем и информации}

ISO/IEC 27001:2013** (Information Security Management Systems — Requirements)
     - Описание системы управления информационной безопасностью и процессов для обеспечения защиты данных.

ГОСТ Р ИСО/МЭК 15408-1-2012\\
ISO/IEC 15408 (Common Criteria)** (Information Technology — Security techniques — Evaluation criteria for IT security)
     - Стандарты для оценки безопасности ИТ-систем, применимые к защите данных и конфиденциальной информации.

\subsection{Портативность прикладного программного обеспечения}

%ISO/IEC 23360-1:2006** (Linux Standard Base — Core Specification)
ISO/IEC 23360-1-2:2021
Linux Standard Base (LSB)
Part 1-2: Core specification generic part
     - Стандарт, гарантирующий переносимость программного обеспечения между различными версиями операционных систем Linux.

ISO/IEC 14598-6:2001** (Software product evaluation — Evaluation modules)
     - Стандарт, описывающий оценку программного обеспечения на предмет портативности между различными платформами.

\subsection{Интероперабельность ИТ продуктов и систем}

ГОСТ 33707-2016\\
ISO/IEC 2382-01:2015** (Information Technology — Vocabulary)
     - Определяет термины и концепции, связанные с информационными системами и их взаимодействием.

ISO/IEC 20926:2009** (Software and systems engineering — Functional size measurement)
     - Этот стандарт позволяет оценивать и измерять функциональность ПО, что способствует повышению интероперабельности между различными системами.

ГОСТ Р ИСО/МЭК 25010--2015\\
ГОСТ Р ИСО/МЭК 25040--2014\\

\subsection{Унификация инструментов и средств разработки}


\subsection{Документирование ПО}

%ISO/IEC 26514:2008** (Systems and software engineering — Requirements for designers and developers of user documentation)
ISO/IEC/IEEE 26514:2022
Systems and software engineering — Design and development of information for users
     - Стандарт, охватывающий требования к документации для конечных пользователей, разработчиков и администраторов программного обеспечения.

ГОСТ 34.201--2020\\
Информационные технологии.
Комплекс стандартов на автоматизированные системы.
Виды, комплектность и обозначение документов
при создании автоматизированных систем.


ГОСТР 58609--2019/ ISO/IEC/IEEE 15289:2017
%ISO/IEC 15289:2017 (Systems and software engineering — Content of systems and software life cycle process information products (Documentation))
ISO/IEC/IEEE 15289:2019
Systems and software engineering — Content of life-cycle information items (documentation)
     - Стандарт, описывающий требования к содержанию документации, сопровождающей систему на всех этапах её жизненного цикла.


\section{Выбор подходящего стандарта для проекта}

\break

ГОСТ Р 56938-2016 "Защита информации. Угрозы безопасности информации".

ГОСТ Р 57580.1-2017 "Безопасность финансовых (банковских) операций. Защита информации финансовых организаций. Базовый набор организационных и технических мер".

ГОСТ Р 54593-2011 "Менеджмент организации. Руководящие указания по устойчивому развитию организаций".
ГОСТ Р ИСО/МЭК 25010-2015 "Информационные технологии. Системы и программное обеспечение. Оценка характеристик системы и ее программного обеспечения".

ГОСТ Р МЭК 61968-1-2012 "Инженерная информация для зданий и автоматизация процессов строительства. Общий интерфейс обмена данными. Часть 1. Модель данных для архитектуры и интерфейсы".
ГОСТ Р 57580.2-2018 "Безопасность финансовых (банковских) операций. Защита информации финансовых организаций. Методы защиты информации".

ГОСТ Р ИСО/МЭК 12207-2010 "Информационная технология. Процессы жизненного цикла программных средств".
ГОСТ Р 58412-2019 "Системы искусственного интеллекта. Вводный курс".

ГОСТ Р ИСО/МЭК 14764-2002 "Информационная технология. Сопровождение программного обеспечения".
ГОСТ Р 56960-2016 "Системы менеджмента знаний. Основные положения и терминология".

ГОСТ Р 50380-92 "Изделия электронной техники. Методы испытаний".

ГОСТ 28907-91 "Основные положения. Единая система конструкторской документации".
ГОСТ 34.601-90 "Автоматизированные системы. Стадии создания".

ISO/IEC 27001:2013 "Информационные технологии. Методы защиты информации. Системы управления информационной безопасностью. Требования".

IEC 62304:2006 "Медицинские устройства. Программное обеспечение как медицинское устройство. Руководства по жизненному циклу разработки, верификации, валидации и управления рисками".

ГОСТ 28907-91 "Основные положения. Единая система конструкторской документации": Этот стандарт определяет общие принципы и требования к оформлению конструкторской документации, включая чертежи, спецификации и другие документы, необходимые для проектирования и производства изделий.

ГОСТ 34.601-90 "Автоматизированные системы. Стадии создания": Данный стандарт описывает этапы создания автоматизированных систем, включая анализ требований, проектирование, разработку, тестирование и внедрение. Он помогает структурировать процесс разработки и обеспечивает контроль над выполнением каждого этапа.

ГОСТ Р ИСО/МЭК 12207-2010 "Информационная технология. Процессы жизненного цикла программных средств": Этот стандарт определяет процессы жизненного цикла программных средств, включая разработку, тестирование, сопровождение и управление конфигурацией. Он также включает рекомендации по планированию, контролю и мониторингу этих процессов.

ГОСТ Р ИСО/МЭК 14764-2002 "Информационная технология. Сопровождение программного обеспечения": Этот стандарт устанавливает требования и рекомендации по организации процесса сопровождения программного обеспечения, включая мониторинг и анализ ошибок, обновление версий и поддержку пользователей.



ГОСТ 34.601-90 "Автоматизированные системы. Стадии создания". Этот стандарт определяет стадии создания автоматизированной системы, начиная от анализа требований до внедрения и сопровождения.

ГОСТ 34.602-89 "Техническое задание на создание автоматизированной системы". Этот стандарт устанавливает требования к содержанию технического задания на создание автоматизированной системы.

ГОСТ 34.603-92 "Виды испытаний автоматизированных систем". Этот стандарт определяет виды испытаний автоматизированных систем и требования к ним.

ГОСТ 34.201-89 "Виды, комплектность и обозначение документов при создании автоматизированных систем". Этот стандарт устанавливает виды, комплектность и обозначение документов, разрабатываемых при создании автоматизированных систем.

ГОСТ 34.003-90 "Информационная технология. Комплекс стандартов на автоматизированные системы. Термины и определения". Этот стандарт определяет основные понятия и термины, используемые в области автоматизированных систем.

ГОСТ 34.004-88 "Автоматизированные системы управления. Термины и определения". Этот стандарт устанавливает термины и определения, применяемые в области автоматизированных систем управления.

ГОСТ Р 56938-2016 "Защита информации. Угрозы безопасности информации".

ГОСТ Р 57580.1-2017 "Безопасность финансовых (банковских) операций. Защита информации финансовых организаций. Базовый набор организационных и технических мер".

ГОСТ Р 54593-2011 "Менеджмент организации. Руководящие указания по устойчивому развитию организаций".

ГОСТ Р ИСО/МЭК 25010-2015 "Информационные технологии. Системы и программное обеспечение. Оценка характеристик системы и ее программного обеспечения".

ГОСТ Р МЭК 61968-1-2012 "Инженерная информация для зданий и автоматизация процессов строительства. Общий интерфейс обмена данными. Часть 1. Модель данных для архитектуры и интерфейсы".

ГОСТ Р 57580.2-2018 "Безопасность финансовых (банковских) операций. Защита информации финансовых организаций. Методы защиты информации".

ГОСТ Р ИСО/МЭК 12207-2010 "Информационная технология. Процессы жизненного цикла программных средств".

ГОСТ Р 58412-2019 "Системы искусственного интеллекта. Вводный курс".

ГОСТ Р ИСО/МЭК 14764-2002 "Информационная технология. Сопровождение программного обеспечения".

ГОСТ Р 56960-2016 "Системы менеджмента знаний. Основные положения и терминология".


\break


Стандарты безопасности данных и информационной безопасности:
Национальный технический комитет по стандартизации "Информационная безопасность" (НТКСИБ).
Подкомитет "Информационная безопасность в системах электронного правительства".
Стандарты точности и качества конвертации правил:
Международный стандартный комитет по информационным технологиям (ISO/IEC JTC 1).
Подкомитет "Системы и программное обеспечение".
Стандарты взаимодействия между различными инструментами проектирования:
Международная электротехническая комиссия (МЭК).
Технический комитет "Информационные технологии для зданий и строительных сооружений" (ТК 8).
Стандарты программирования и разработки программного обеспечения:
Технический комитет "Информационные технологии" (ТК 22).
Подкомитет "Жизненный цикл программных средств".
Стандарты документирования и поддержки программного обеспечения:
Технический комитет "Информационные технологии" (ТК 22).
Подкомитет "Сопровождение программного обеспечения".


\clearpage

\section*{\LARGE Вывод}
\addcontentsline{toc}{section}{Вывод}

В ходе выполнения задания были изучены различные группы стандартов,
связанные с темой проекта.
Было отмечено, что каждая группа стандартов включает
свои специфические аспекты,
такие как безопасность данных, точность конвертации правил,
взаимодействие инструментов проектирования,
разработка программного обеспечения и документирование.
Это позволило глубже понять требования и рекомендации,
которые необходимо учитывать при реализации подобных проектов.

