\section*{\LARGE Цель практической работы}
\addcontentsline{toc}{section}{Цель практической работы}

\textbf{Цель практической работы:}
Получить навыки по определению требуемых областей
стандартизации и по поиску страндартизирующих комитетов и
существующих стандартов, созданных данными комитетами. Получить
навыки анализа стандартов предметной области для выбора наиболее
подходящих из них на основании перечня, полученного при анализе
областей стандартизации.

\textbf{Задание на практическую работу:}

\begin{enumerate}
	\item Опираясь на подпроцессы выбранной области определить
		группы стандартов, которые могут быть применены
		для выбранного дипломного проекта.
		Минимально необходимо выбрать 5 групп стандартов.
	\item Определить, какие комитеты и подкомитеты занимаются
		стандартизацией областей.
	\item Опираясь на выбранные комитеты и подкомитеты,
		определить какие стандарты могут быть использованы
		для реализуемого проекта.
		Для каждой выделенной группы стандартов необходимо
		найти минимум 2 стандарта, пригодных для проекта.
	\item На основании выбранных стандартов провести исследование,
		какие из них наиболее уместно использовать в предлагаемой разработке.
		Обоснование должно заключаться в приведении конкретных достоинств
		и недостатков выбранных стандартов по сравнению с другими.
		При приведении конкретных недостатков или достоинств стандарта
		необходимо ссылаться на пункт,
		обуславливающий это достоинство или недостаток.
\end{enumerate}

\clearpage

\section*{\LARGE Выполнение практической работы}
\addcontentsline{toc}{section}{Выполнение практической работы}

\section{Группы стандартов}

Разрабатываемый программный продукт предполагает следующие группы стандартов:

\begin{itemize}
	\item Стандарты разработки и развития ИТ систем и средств их разработки;
\item Стандарты производительности и качества ИТ продуктов и систем;
\item Стандарты безопасности ИТ систем и информации;
	\item Стандарты портативности прикладного программного обеспечения;
		\item Стандарты интероперабельности ИТ продуктов и систем;
		\item Стандарты унификации инструментов и средств разработки;
	\item Стандарты документирования ПО.
\end{itemize}

\subsection{Разработка и развитие ИТ систем и средств их разработки}
Это направление стандартизации касается создания общих стандартов
для проектирования и разработки ИТ-систем.

Оно включает в себя:

Модели архитектуры ИТ-систем, которые определяют, как компоненты системы взаимодействуют друг с другом.
Методологии разработки программного обеспечения, включая Agile, Waterfall и другие, для обеспечения слаженной работы команд разработчиков.
Инструменты разработки и программные среды (например, Integrated Development Environment — IDE), обеспечивающие совместимость между различными инструментами и платформами.
Эти стандарты помогают разработчикам создавать более эффективные и надежные ИТ-системы.

\subsection{Производительность и качество ИТ продуктов и систем}
Стандарты производительности и качества ИТ-продуктов и систем направлены на обеспечение высоких показателей работы программного обеспечения и оборудования. Они включают:

Метрики производительности, такие как время отклика системы, пропускная способность сети, скорость обработки данных и т. д.
Контроль качества и тестирование продуктов на этапах разработки и эксплуатации для выявления ошибок и уязвимостей.
Системы управления качеством, такие как ISO/IEC 25010, которые определяют критерии качества программного обеспечения (функциональность, надежность, эффективность).
Эти стандарты обеспечивают, чтобы продукты и системы отвечали ожиданиям пользователей по качеству и стабильности.

\subsection{Безопасность ИТ систем и информации}
Стандарты в области безопасности ИТ-систем и информации (кибербезопасность) направлены на защиту данных и предотвращение кибератак. Они охватывают:

Криптографические методы защиты данных (шифрование, цифровые подписи и сертификаты) для защиты конфиденциальной информации.
Управление доступом и аутентификацией, включая многофакторную аутентификацию (например, биометрия, токены) для обеспечения защиты от несанкционированного доступа.
Политики управления рисками и инцидентами, которые помогают организациям реагировать на угрозы безопасности и минимизировать их воздействие.
Пример стандартов: ISO/IEC 27001 (системы управления информационной безопасностью).

\subsection{Портативность прикладного программного обеспечения}
Портативность программного обеспечения означает способность программы работать на различных платформах и системах без модификации исходного кода. Стандарты в этой области включают:

Поддержку кроссплатформенных разработок, когда программы или приложения могут работать на разных операционных системах и устройствах.
Использование универсальных библиотек и языков программирования (например, Java, Python), что облегчает перенос приложений на другие среды.
Эти стандарты помогают разработчикам создавать программы, которые легко адаптируются к различным аппаратным и программным платформам.

\subsection{Интероперабельность ИТ продуктов и систем}
Интероперабельность означает способность различных ИТ-систем и продуктов взаимодействовать друг с другом и обмениваться данными. Стандарты интероперабельности охватывают:

Сетевые протоколы (например, TCP/IP, HTTP) для обмена данными между различными устройствами и сетями.
Форматы данных (XML, JSON), которые обеспечивают единый способ обмена информацией между приложениями.
API и интерфейсы для интеграции различных программных систем и сервисов.
Эти стандарты критически важны для построения взаимосвязанных систем, таких как интернет вещей (IoT), облачные сервисы и корпоративные сети.

\subsection{Унификация инструментов и средств разработки}
Стандарты в этой области направлены на создание единого подхода к использованию инструментов разработки, что повышает совместимость между ними и облегчает разработку программного обеспечения. Это включает:

Единые стандарты для инструментов управления проектами и кодом (например, Git, Jira), которые позволяют эффективно управлять проектами и отслеживать прогресс разработки.
Унификация средств автоматизированного тестирования, что упрощает тестирование программных решений.
Эти стандарты облегчают разработку программного обеспечения и повышают продуктивность команд разработчиков.

\section{Комитеты и подкомитеты}

\section{Выбранные стандарты}

\subsection{Разработка и развитие ИТ систем и средств их разработки}

ГОСТ Р ИСО/МЭК 12207--2010\\
ISO/IEC 12207:2017** (Systems and software engineering — Software life cycle processes)
     - Описание процессов и методов, связанных с разработкой и сопровождением программных продуктов на протяжении всего жизненного цикла.

ISO/IEC 24744:2014** (Software Engineering — Metamodel for Development Methodologies)
     - Стандарт, описывающий модели и методологии для разработки программного обеспечения и их эволюции.

\subsection{Производительность и качество ИТ продуктов и систем}
ГОСТ Р ИСО/МЭК 25010-2015\\
%ISO/IEC 25010:2011** (Systems and software engineering — Systems and software Quality Requirements and Evaluation (SQuaRE) — System and software quality models)
ISO/IEC 25010:2023 Systems and software engineering — Systems and software Quality Requirements and Evaluation (SQuaRE) — Product quality model

     - Определяет основные качества, такие как производительность, надежность и эффективность программных систем.


ISO/IEC 25019:2023
Systems and software engineering — Systems and software Quality Requirements and Evaluation (SQuaRE) — Quality-in-use model

ISO/IEC 25002:2024
Systems and software engineering — Systems and software Quality Requirements and Evaluation (SQuaRE) — Quality model overview and usage

\subsection{Безопасность ИТ систем и информации}

ISO/IEC 27001:2013** (Information Security Management Systems — Requirements)
     - Описание системы управления информационной безопасностью и процессов для обеспечения защиты данных.

ГОСТ Р ИСО/МЭК 15408-1-2012\\
ISO/IEC 15408 (Common Criteria)** (Information Technology — Security techniques — Evaluation criteria for IT security)
     - Стандарты для оценки безопасности ИТ-систем, применимые к защите данных и конфиденциальной информации.

\subsection{Портативность прикладного программного обеспечения}

%ISO/IEC 23360-1:2006** (Linux Standard Base — Core Specification)
ISO/IEC 23360-1-2:2021
Linux Standard Base (LSB)
Part 1-2: Core specification generic part
     - Стандарт, гарантирующий переносимость программного обеспечения между различными версиями операционных систем Linux.

ISO/IEC 14598-6:2001** (Software product evaluation — Evaluation modules)
     - Стандарт, описывающий оценку программного обеспечения на предмет портативности между различными платформами.

\subsection{Интероперабельность ИТ продуктов и систем}

ГОСТ 33707-2016\\
ISO/IEC 2382-01:2015** (Information Technology — Vocabulary)
     - Определяет термины и концепции, связанные с информационными системами и их взаимодействием.

ISO/IEC 20926:2009** (Software and systems engineering — Functional size measurement)
     - Этот стандарт позволяет оценивать и измерять функциональность ПО, что способствует повышению интероперабельности между различными системами.

ГОСТ Р ИСО/МЭК 25010--2015\\
ГОСТ Р ИСО/МЭК 25040--2014\\

\subsection{Унификация инструментов и средств разработки}


\subsection{Документирование ПО}

%ISO/IEC 26514:2008** (Systems and software engineering — Requirements for designers and developers of user documentation)
ISO/IEC/IEEE 26514:2022
Systems and software engineering — Design and development of information for users
     - Стандарт, охватывающий требования к документации для конечных пользователей, разработчиков и администраторов программного обеспечения.

ГОСТ 34.201--2020\\
Информационные технологии.
Комплекс стандартов на автоматизированные системы.
Виды, комплектность и обозначение документов
при создании автоматизированных систем.


ГОСТР 58609--2019/ ISO/IEC/IEEE 15289:2017
%ISO/IEC 15289:2017 (Systems and software engineering — Content of systems and software life cycle process information products (Documentation))
ISO/IEC/IEEE 15289:2019
Systems and software engineering — Content of life-cycle information items (documentation)
     - Стандарт, описывающий требования к содержанию документации, сопровождающей систему на всех этапах её жизненного цикла.


\section{Выбор подходящего стандарта для проекта}

\break

ГОСТ Р 56938-2016 "Защита информации. Угрозы безопасности информации".

ГОСТ Р 57580.1-2017 "Безопасность финансовых (банковских) операций. Защита информации финансовых организаций. Базовый набор организационных и технических мер".

ГОСТ Р 54593-2011 "Менеджмент организации. Руководящие указания по устойчивому развитию организаций".
ГОСТ Р ИСО/МЭК 25010-2015 "Информационные технологии. Системы и программное обеспечение. Оценка характеристик системы и ее программного обеспечения".

ГОСТ Р МЭК 61968-1-2012 "Инженерная информация для зданий и автоматизация процессов строительства. Общий интерфейс обмена данными. Часть 1. Модель данных для архитектуры и интерфейсы".
ГОСТ Р 57580.2-2018 "Безопасность финансовых (банковских) операций. Защита информации финансовых организаций. Методы защиты информации".

ГОСТ Р ИСО/МЭК 12207-2010 "Информационная технология. Процессы жизненного цикла программных средств".
ГОСТ Р 58412-2019 "Системы искусственного интеллекта. Вводный курс".

ГОСТ Р ИСО/МЭК 14764-2002 "Информационная технология. Сопровождение программного обеспечения".
ГОСТ Р 56960-2016 "Системы менеджмента знаний. Основные положения и терминология".

ГОСТ Р 50380-92 "Изделия электронной техники. Методы испытаний".

ГОСТ 28907-91 "Основные положения. Единая система конструкторской документации".
ГОСТ 34.601-90 "Автоматизированные системы. Стадии создания".

ISO/IEC 27001:2013 "Информационные технологии. Методы защиты информации. Системы управления информационной безопасностью. Требования".

IEC 62304:2006 "Медицинские устройства. Программное обеспечение как медицинское устройство. Руководства по жизненному циклу разработки, верификации, валидации и управления рисками".

ГОСТ 28907-91 "Основные положения. Единая система конструкторской документации": Этот стандарт определяет общие принципы и требования к оформлению конструкторской документации, включая чертежи, спецификации и другие документы, необходимые для проектирования и производства изделий.

ГОСТ 34.601-90 "Автоматизированные системы. Стадии создания": Данный стандарт описывает этапы создания автоматизированных систем, включая анализ требований, проектирование, разработку, тестирование и внедрение. Он помогает структурировать процесс разработки и обеспечивает контроль над выполнением каждого этапа.

ГОСТ Р ИСО/МЭК 12207-2010 "Информационная технология. Процессы жизненного цикла программных средств": Этот стандарт определяет процессы жизненного цикла программных средств, включая разработку, тестирование, сопровождение и управление конфигурацией. Он также включает рекомендации по планированию, контролю и мониторингу этих процессов.

ГОСТ Р ИСО/МЭК 14764-2002 "Информационная технология. Сопровождение программного обеспечения": Этот стандарт устанавливает требования и рекомендации по организации процесса сопровождения программного обеспечения, включая мониторинг и анализ ошибок, обновление версий и поддержку пользователей.



ГОСТ 34.601-90 "Автоматизированные системы. Стадии создания". Этот стандарт определяет стадии создания автоматизированной системы, начиная от анализа требований до внедрения и сопровождения.

ГОСТ 34.602-89 "Техническое задание на создание автоматизированной системы". Этот стандарт устанавливает требования к содержанию технического задания на создание автоматизированной системы.

ГОСТ 34.603-92 "Виды испытаний автоматизированных систем". Этот стандарт определяет виды испытаний автоматизированных систем и требования к ним.

ГОСТ 34.201-89 "Виды, комплектность и обозначение документов при создании автоматизированных систем". Этот стандарт устанавливает виды, комплектность и обозначение документов, разрабатываемых при создании автоматизированных систем.

ГОСТ 34.003-90 "Информационная технология. Комплекс стандартов на автоматизированные системы. Термины и определения". Этот стандарт определяет основные понятия и термины, используемые в области автоматизированных систем.

ГОСТ 34.004-88 "Автоматизированные системы управления. Термины и определения". Этот стандарт устанавливает термины и определения, применяемые в области автоматизированных систем управления.

ГОСТ Р 56938-2016 "Защита информации. Угрозы безопасности информации".

ГОСТ Р 57580.1-2017 "Безопасность финансовых (банковских) операций. Защита информации финансовых организаций. Базовый набор организационных и технических мер".

ГОСТ Р 54593-2011 "Менеджмент организации. Руководящие указания по устойчивому развитию организаций".

ГОСТ Р ИСО/МЭК 25010-2015 "Информационные технологии. Системы и программное обеспечение. Оценка характеристик системы и ее программного обеспечения".

ГОСТ Р МЭК 61968-1-2012 "Инженерная информация для зданий и автоматизация процессов строительства. Общий интерфейс обмена данными. Часть 1. Модель данных для архитектуры и интерфейсы".

ГОСТ Р 57580.2-2018 "Безопасность финансовых (банковских) операций. Защита информации финансовых организаций. Методы защиты информации".

ГОСТ Р ИСО/МЭК 12207-2010 "Информационная технология. Процессы жизненного цикла программных средств".

ГОСТ Р 58412-2019 "Системы искусственного интеллекта. Вводный курс".

ГОСТ Р ИСО/МЭК 14764-2002 "Информационная технология. Сопровождение программного обеспечения".

ГОСТ Р 56960-2016 "Системы менеджмента знаний. Основные положения и терминология".


\break


Стандарты безопасности данных и информационной безопасности:
Национальный технический комитет по стандартизации "Информационная безопасность" (НТКСИБ).
Подкомитет "Информационная безопасность в системах электронного правительства".
Стандарты точности и качества конвертации правил:
Международный стандартный комитет по информационным технологиям (ISO/IEC JTC 1).
Подкомитет "Системы и программное обеспечение".
Стандарты взаимодействия между различными инструментами проектирования:
Международная электротехническая комиссия (МЭК).
Технический комитет "Информационные технологии для зданий и строительных сооружений" (ТК 8).
Стандарты программирования и разработки программного обеспечения:
Технический комитет "Информационные технологии" (ТК 22).
Подкомитет "Жизненный цикл программных средств".
Стандарты документирования и поддержки программного обеспечения:
Технический комитет "Информационные технологии" (ТК 22).
Подкомитет "Сопровождение программного обеспечения".


\clearpage

\section*{\LARGE Вывод}
\addcontentsline{toc}{section}{Вывод}

В ходе выполнения задания были изучены различные группы стандартов,
связанные с темой проекта.
Было отмечено, что каждая группа стандартов включает
свои специфические аспекты,
такие как безопасность данных, точность конвертации правил,
взаимодействие инструментов проектирования,
разработка программного обеспечения и документирование.
Это позволило глубже понять требования и рекомендации,
которые необходимо учитывать при реализации подобных проектов.
