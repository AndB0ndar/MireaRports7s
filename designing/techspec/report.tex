\section{Функциональные требования}

Это требования, которые описывают, что система должна делать.
Они определяют конкретные функции,
которые система должна выполнять для решения задач пользователя.

Перечислим их:

\begin{itemize}
    \item Программа должна принимать на вход файлы
		формата SVRF (Standard Verification Rule Format)
		для DRC (Design Rule Check) правил,
		используемый коммерческими инструментами физической проверки
		Calibre и ICverify;
    \item Программа должна поддерживать преобразование правил из формата
		Calibre в формат поддерживаемый открытым инструментом Klayout;
    \item Программа должна мочь обрабатывать макросы перед конвертацией;
    \item Программа должна обрабатывать последовательность 
		инициализации переменных перед конвертацией.
\end{itemize}

\section{Пользовательские требования}

Эти требования формируются с точки зрения пользователя и описывают задачи,
которые пользователь должен выполнить с помощью системы.
Они часто менее технические и более ориентированы на взаимодействие.

Перечислим их:

\begin{itemize}
    \item Пользователь должен иметь возможность выбрать исходный
		и целевой форматы для конвертации правил;
    \item Интерфейс программы должен представлять собой консольное приложение;
    \item Программа должна предоставлять обратную связь о ходе конвертации;
    \item Программа должна предоставлять возможность создавать лог-файла;
\end{itemize}

\section{Нефункциональные требования}

Эти требования касаются характеристик системы, которые определяют,
как она выполняет свои функции,
но не связаны непосредственно с поведением системы.
Они включают такие аспекты, как производительность, надежность, безопасность,
удобство использования и масштабируемость.

Перечислим их:

\begin{itemize}
    \item Программа должна поддерживать высокую производительность
		и быть способной обрабатывать большие файлы правил DRC
		в разумные сроки;
    \item Программа должна быть реализован на языке Python для совместимости
		с основными операционными системами (Windows, Linux, macOS);
    \item Программа не должна обеспечивать безопасность данных
		и корректное обращение с конфиденциальной информацией в правилах DRC;
    \item Программа не должна изменять исходные файлы правил.
\end{itemize}

\section{Ограничения}

Это ограничения или условия,
которые накладываются на процесс разработки или использование системы.
Они могут касаться технологий, бюджетных ограничений, сроков или ресурсов.

Перечислим их:

\begin{itemize}
    \item Необходимо использовать только open-source библиотеки
		и инструменты;
    \item Open-source форматы могут не поддерживать
		полный набор функций, поэтому требуется поиск компромиссов
		при конвертации;
\end{itemize}

\section{Ответы на вопросы}

\subsection{Цели и задачи проекта}

\paragraph{Какие цели вы хотите достичь с помощью этого проекта?}
Реализовать конвертер DRC правил из языка SVRF в Klayout (на базе Ruby).

\paragraph{Какие конкретные задачи должны быть выполнены?}
Корректная трансляци DRC правил (совпадение результатов проверки
в коммерческом инструменте и open-source-ом).\par
Обработка инициализаций, так как в языке SVRF переменный могут использоваться
до инициализации.\par
Обработка макросов в коде, так как в Ruby они не поддерживаются.\par

\subsection{Стейкхолдеры и пользователи}

\paragraph{Кто является основными стейкхолдерами проекта?}
Проект пишется для нужд: \textbf{ООО <<Мальт Систем>>}.

\paragraph{Какие группы пользователей будут взаимодействовать
	с системой или продуктом?}
Разработчики аналоговых схем и физического уровня
и инженеры, работающие с топологией (layout).

\subsection{Функциональные требования}

\paragraph{Какие функции и возможности должны быть реализованы в системе?}
Конвертирование скрипта SVRF в Klayout и все для этого требуемое.

\paragraph{Какие бизнес-процессы должны поддерживаться?}

\begin{enumerate}
	\item Загрузка файла DRC в систему.
	\item Настройка параметров конвертации.
	\item Конвертация DRC правил.
	\item Проверка и верификация полученого скрипта.
	\item Сохранение результатов.
	\item Обратная связь и отчетность.
\end{enumerate}

\subsection{Нефункциональные требования}

\paragraph{Какие требования к производительности,
	безопасности и доступности системы у вас есть?}
Приемлимое время предобработки скрипта и конвертации.
Конкретное время зависит от объема файла,
но желательно не больше 5 минут на 10000 строк.\par
Конвертору запрещено изменять имующийся файл скрипта.

\paragraph{Есть ли особенности в области безопасности данных?}
Исходный файл со скриптом запрещено изменять. Так как данный файлы получены
от фабрики-производителя.

\subsection{Интеграция и сторонние сервисы}

\paragraph{Необходимо ли интегрировать систему с другими приложениями
	или сторонними сервисами?}
Сама система ни с чем не интегрируеся. Но результатов работы 
программы должен корректно читаться приложением Klayout.

\paragraph{Какие API или протоколы обмена данными следует использовать?}
Нет.

\subsection{Интерфейс пользователя}

\paragraph{Какие требования к пользовательскому интерфейсу (UI) у вас есть?}
Нету. У программы будет консольный интерфейс.

\paragraph{Какие элементы управления и макеты вы предпочли бы использовать?}
Консольный интерфейс.

\subsection{Требования к производительности}

\paragraph{Какие ожидания по скорости работы и времени отклика системы?}
Конкретное время зависит от объема файла,
но желательно не больше 5 минут на 10000 строк.\par

\paragraph{Какое количество пользователей системы вы ожидаете?}
Один пользователь.

\subsection{Обслуживание и поддержка}

\paragraph{Какие требования к технической поддержке
	и обновлениям после внедрения системы?}

\begin{itemize}
	\item Предоставить подробные руководства по установке,
		настройке и использованию программы.
	\item Поддержка часто задаваемых вопросов (FAQ)
		и разделов для решения типичных проблем пользователей.
	\item Выпуск плановых обновлений для улучшения функционала программы,
		повышения производительности и внедрения новых возможностей.
	\item Обеспечение обратной совместимости с предыдущими версиями,
		чтобы не нарушать рабочие процессы пользователей.
	\item Добавление поддержки новых форматов DRC правил по мере
		их появления на рынке или по запросу пользователей.
	\item Обновление программы для обеспечения совместимости
		с новыми версиями открытых инструментов, такими как Magic и Klayout.
	\item Добавление трансляции LVS правил.
\end{itemize}

\subsection{Локализация и интернационализация}

\paragraph{Есть ли требования к поддержке разных языков
и региональных настроек?}

Нет.

\subsection{Уровень безопасности}
\paragraph{Какие меры безопасности данных и управления доступом требуются?}

Необходимы права на чтение файла DRC правил.

\subsection{Требования к документации и обучению}
\paragraph{Нужна ли документация для пользователей
или администраторов системы?}

Возможно.

\paragraph{Требуется ли обучение пользователям?}

Нет.

\clearpage

\section*{\LARGE Вывод}
\addcontentsline{toc}{section}{Вывод}

В ходе выполнения практической работы были выделены
функциональные, пользовательские, нефункциональные
и ограничения приложения.\par
Также были даны ответы на интервью.

