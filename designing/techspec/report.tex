\section{Тема работы}

\textbf{Тема работы:}
Программа-конвертер DRC правил для использования в открытых инструментах
проектирования цифровых микросхем.

Проектирование цифровых микросхем требует соблюдения строгих правил
проектирования для корректного функционирования и производства чипов.
Одним из таких стандартов являются DRC (Design Rule Check) правила
--- наборы ограничений, касающихся топологии полупроводниковых схем.
Эти правила включают требования к расстояниям между элементами,
ширине дорожек и другим параметрам, важным для производственных процессов.\par
Для успешного использования открытых инструментов проектирования,
таких как KLayout, важно иметь доступ к DRC правилам, совместимым с ними.
Однако DRC правила часто поставляются в проприетарных форматах.
Программа-конвертер служит для преобразования таких правил в форматы,
поддерживаемые открытыми инструментами.

\section{Функциональные требования}

Это требования, которые описывают, что система должна делать.
Они определяют конкретные функции,
которые система должна выполнять для решения задач пользователя.

Перечислим их:

\begin{itemize}
    \item Программа должна принимать на вход файлы
		формата SVRF (Standard Verification Rule Format)
		для DRC (Design Rule Check) правил,
		используемый коммерческими инструментами физической проверки
		Calibre;
	\item Программа должна читать файл конфигурации;
    \item Программа должна поддерживать преобразование правил из формата
		Calibre в формат поддерживаемый открытым инструментом Klayout;
    \item Программа должна обрабатывать макросы перед конвертацией;
    \item Программа должна обрабатывать последовательность 
		инициализации переменных перед конвертацией.
	\item Программа должна проверять и верифицировать полученый скрипт;
	\item Программа должна создавать файл с результом конвертации;
	\item Программа должна создавать отчет о результатах конвертации;
\end{itemize}

\section{Пользовательские требования}

Эти требования формируются с точки зрения пользователя и описывают задачи,
которые пользователь должен выполнить с помощью системы.
Они часто менее технические и более ориентированы на взаимодействие.

Перечислим их:

\begin{itemize}
    \item Пользователь должен иметь возможность выбрать исходный;
    \item Пользователь должен иметь возможность выбрать
		целевой форматы для конвертации правил;
    \item Пользователь должен иметь возможность настроить ход конвертации
		через конфигурационный файл;
    \item Интерфейс программы должен представлять собой консольное приложение;
    \item Программа должна предоставлять отчет о результатах конвертации;
\end{itemize}

\section{Нефункциональные требования}

Эти требования касаются характеристик системы, которые определяют,
как она выполняет свои функции,
но не связаны непосредственно с поведением системы.
Они включают такие аспекты, как производительность, надежность, безопасность,
удобство использования и масштабируемость.

Перечислим их:

\begin{itemize}
	\item Конкретное время зависит от объема файла,
		приблизительно должно составлять не больше 5 минут на 10Мб-ый файл.
    \item Программа должна быть реализован на языке Python 3.9
		для совместимости с основными файловыми системами Windows и Linux;
    \item Программа не должна изменять исходные файлы правил;
    \item Полученные конвертацией правила должены выдавать
		такой же результат, как исходные в коммерческих приложениях.
\end{itemize}

\section{Ограничения}

Это ограничения или условия,
которые накладываются на процесс разработки или использование системы.
Они могут касаться технологий, бюджетных ограничений, сроков или ресурсов.

Перечислим их:

\begin{itemize}
    \item Необходимо использовать только open-source библиотеки
		и инструменты;
    \item Open-source форматы могут не поддерживать
		полный набор функций, поэтому требуется поиск компромиссов
		при конвертации;
    \item Версия интерпретатора Python не ниже версии 3.9.2;
    \item Версия библиотеки Lark для парсинга исходного файла
		должна иметь версию не позднее 1.1.7;
    \item Итоговый скрипт должен состаять из набора команд 
		библиотеки Klayout не позднее 0.29.6 версии;
    \item Исходный файл не должен превышать 1 Гб;
\end{itemize}

\section{Ответы на вопросы}

\subsection{Цели и задачи проекта}

\subsubsection{Какие цели вы хотите достичь с помощью этого проекта?}
Реализовать конвертер DRC правил из языка SVRF в Klayout (на базе Ruby).

\subsubsection{Какие конкретные задачи должны быть выполнены?}
Корректная трансляци DRC правил (совпадение результатов проверки
в коммерческом инструменте и open-source-ом).\par
Обработка инициализаций, так как в языке SVRF переменный могут использоваться
до инициализации.\par
Обработка макросов в коде, так как в Ruby они не поддерживаются.\par

\subsection{Стейкхолдеры и пользователи}

\subsubsection{Кто является основными стейкхолдерами проекта?}
Проект пишется для нужд: \textbf{ООО <<Мальт Систем>>}.

\subsubsection{Какие группы пользователей будут взаимодействовать
	с системой или продуктом?}
Разработчики аналоговых схем и физического уровня
и инженеры, работающие с топологией (layout).

\subsection{Функциональные требования}

\subsubsection{Какие функции и возможности должны быть реализованы в системе?}
Конвертирование скрипта SVRF в Klayout и все для этого требуемое.

\subsubsection{Какие бизнес-процессы должны поддерживаться?}

\begin{enumerate}
	\item Загрузка файла DRC в систему.
	\item Настройка параметров конвертации.
	\item Конвертация DRC правил.
	\item Проверка и верификация полученого скрипта.
	\item Сохранение результатов.
	\item Обратная связь и отчетность.
\end{enumerate}

\subsection{Нефункциональные требования}

\subsubsection{Какие требования к производительности,
	безопасности и доступности системы у вас есть?}
Приемлимое время предобработки скрипта и конвертации.
Конкретное время зависит от объема файла,
но желательно не больше 5 минут на 10000 строк.\par
Конвертору запрещено изменять имующийся файл скрипта.

\subsubsection{Есть ли особенности в области безопасности данных?}
Исходный файл со скриптом запрещено изменять. Так как данный файлы получены
от фабрики-производителя.

\subsection{Интеграция и сторонние сервисы}

\subsubsection{Необходимо ли интегрировать систему с другими приложениями
	или сторонними сервисами?}
Сама система ни с чем не интегрируеся. Но результатов работы 
программы должен корректно читаться приложением Klayout.

\subsubsection{Какие API или протоколы обмена данными следует использовать?}
Нет.

\subsection{Интерфейс пользователя}

\subsubsection{Какие требования к пользовательскому интерфейсу (UI) у вас есть?}
Нету. У программы будет консольный интерфейс.

\subsubsection{Какие элементы управления и макеты вы предпочли бы использовать?}
Консольный интерфейс.

\subsection{Требования к производительности}

\subsubsection{Какие ожидания по скорости работы и времени отклика системы?}
Конкретное время зависит от объема файла,
но желательно не больше 5 минут на 10000 строк.\par

\subsubsection{Какое количество пользователей системы вы ожидаете?}
Один пользователь.

\subsection{Обслуживание и поддержка}

\subsubsection{Какие требования к технической поддержке
	и обновлениям после внедрения системы?}

\begin{itemize}
	\item Предоставить подробные руководства по установке,
		настройке и использованию программы.
	\item Поддержка часто задаваемых вопросов (FAQ)
		и разделов для решения типичных проблем пользователей.
	\item Выпуск плановых обновлений для улучшения функционала программы,
		повышения производительности и внедрения новых возможностей.
	\item Обеспечение обратной совместимости с предыдущими версиями,
		чтобы не нарушать рабочие процессы пользователей.
	\item Добавление поддержки новых форматов DRC правил по мере
		их появления на рынке или по запросу пользователей.
	\item Обновление программы для обеспечения совместимости
		с новыми версиями открытых инструментов, такими как Magic и Klayout.
	\item Добавление конвертации LVS правил.
\end{itemize}

\subsection{Локализация и интернационализация}

\subsubsection{Есть ли требования к поддержке разных языков
и региональных настроек?}

Нет.

\subsection{Уровень безопасности}
\subsubsection{Какие меры безопасности данных и управления доступом требуются?}

Необходимы права на чтение файла DRC правил.

\subsection{Требования к документации и обучению}
\subsubsection{Нужна ли документация для пользователей
или администраторов системы?}

Возможно.

\subsubsection{Требуется ли обучение пользователям?}

Нет.

\clearpage

\section*{\LARGE Вывод}
\addcontentsline{toc}{section}{Вывод}

В ходе выполнения практической работы были выделены
функциональные, пользовательские, нефункциональные
и ограничения приложения.\par
Также были даны ответы на интервью.

