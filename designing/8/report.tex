\section{Описание используемых технологий и их обоснование}

\subsection{Язык программирования}

Python был выбран за свою простоту и читаемость,
что позволяет быстро разрабатывать приложения.
Он поддерживает множество библиотек и инструментов,
а также объектно-ориентированное программирование,
что облегчает организацию и поддержку кода.

\subsection{Библиотека Tkinter}

Tkinter является стандартной библиотекой
для создания графических интерфейсов в Python.
Она была выбрана благодаря:

\begin{itemize}
	\item Удобству использования:
		Простой интерфейс для создания окон и элементов управления.
	\item Кросс-платформенности:
		Приложения работают на Windows, macOS и Linux.
	\item Поддержке: Обширная документация и активное сообщество.
\end{itemize}

\clearpage

\section{Описание интерфейса}

Интерфейс приложения состоит из нескольких вкладок,
каждая из которых предназначена для выполнения определенных задач.

\subsection{Вкладка Input (Ввод)}

Данная вкладка содержит поля для ввода путей до исходного файла
и файла конфигурации.
Для удобства пользователя реализованы кнопки <<Browse>>,
позволяющие выбрать файлы через стандартный диалог.

Данная вкладка продемонстрированна на рисунке~\ref{fig:input}.

\begin{image}
	\includegrph{Screenshot from 2024-10-12 15-57-42}
	\caption{Вкладка Input (Ввод)}
	\label{fig:input}
\end{image}

\clearpage

\subsection{Вкладка Output (Вывод)}

В этой вкладке находятся поля для ввода путей до файлов отчёта и результата,
также с кнопками <<Browse>> для выбора файлов.
Это позволяет пользователю быстро
и удобно указывать места для сохранения результатов работы.

Данная вкладка продемонстрированна на рисунке~\ref{fig:output}.

\begin{image}
	\includegrph{Screenshot from 2024-10-12 15-57-52}
	\caption{Вкладка Output (Вывод)}
	\label{fig:output}
\end{image}

\clearpage

\subsection{Вкладка Settings (Настройки)}

Эта вкладка предоставляет текстовое поле
для ввода детальных настроек,
которые могут быть применены к различным модулям конвертора.
Кнопка <<Apply Settings>> позволяет применить введенные настройки,
что обеспечивает гибкость в конфигурации приложения.

Данная вкладка продемонстрированна на рисунке~\ref{fig:settings}.

\begin{image}
	\includegrph{Screenshot from 2024-10-12 15-57-54}
	\caption{Вкладка Settings (Настройки)}
	\label{fig:settings}
\end{image}

\clearpage

\subsection{Вкладка Logs (Логи)}

В этой вкладке реализовано текстовое поле для вывода логов работы приложения.
Пользователь может видеть сообщения об ошибках,
предупреждения и другую информацию,
необходимую для мониторинга работы программы.

Кнопка <<Clear Logs>> позволяет пользователю очищать текстовое
поле логов для упрощения восприятия информации.

Данная вкладка продемонстрированна на рисунке~\ref{fig:logs}.

\begin{image}
	\includegrph{Screenshot from 2024-10-12 15-57-57}
	\caption{Вкладка Logs (Логи)}
	\label{fig:logs}
\end{image}

\clearpage

\section*{\LARGE Вывод}
\addcontentsline{toc}{section}{Вывод}

Разработанный интерфейс приложения предоставляет пользователю удобный
и интуитивно понятный способ работы с программой-конвертором DRC правил.
Он включает все необходимые элементы для ввода данных,
выбора режимов работы, настройки параметров и вывода логов.
Такой подход к проектированию интерфейса обеспечивает
высокую степень взаимодействия с пользователем
и позволяет эффективно управлять процессами проверки проектирования.



