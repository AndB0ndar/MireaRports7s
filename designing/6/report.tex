\section{Диаграмма прецедентов}

Диаграмма прецедентов (Use case diagram) --- это диаграмма поведения,
на которой показано множество прецедентов и актёров,
а также отношения между ними.\par
Диаграммы прецедентов применяются для моделирования вида системы
с точки зрения внешнего наблюдателя.\par
Основные элементы диаграммы прецедентов:

\begin{itemize}
	\item Субъект (actor) --- любая сущность,
		взаимодействующая с системой извне.
	\item Прецеденты (use case) --- описание множества последовательностей
		действий (включая их варианты),
		которые выполняются системой для того,
		чтобы актёр получил результат,
		имеющий для него определённое значение.
\end{itemize}

Между субъектами и прецедентами могут существовать различные отношения,
которые описывают взаимодействие экземпляров одних субъектов
и прецедентов с экземплярами других субъектов и прецедентов.

\begin{image}
	\includegrph{Screenshot from 2024-10-09 21-08-03}
	\caption{Диаграмма прецедентов}
	\label{fig:use:case}
\end{image}

\section{Диаграмма последовательности}

Диаграмма последовательности (sequence diagram) --- это наглядное
представление совокупности разных элементов модели системы,
изображение того, как и в каком порядке они взаимодействуют.\par
Такие диаграммы подробно описывают, как выполняются разные операции.
При этом они показывают временной порядок или хронологию:
то, когда, как и в какой очереди передаются сообщения.\par
Диаграммы удобно использовать при проектировании или проверке архитектуры,
логики системы или интерфейса.

\begin{image}
	\includegrph{Screenshot from 2024-10-09 21-06-24}
	\caption{Диаграмма последовательности}
	\label{fig:sequence}
\end{image}

\section{Паттерн проектирования}

Для программы-конвертера DRC правил выбран паттерн
\textbf{<<Цепочка обязанностей>>} (Chain of Responsibility).\par
Паттерн <<Цепочка обязанностей>> позволяет передавать запросы
по цепочке обработчиков,
что идеально подходит для модульной структуры системы.
Каждый модуль (обработка пользовательского ввода, парсинг,
предобработка, конвертация, генерация отчета)
может обрабатывать запросы последовательно,
что упрощает реализацию и позволяет легко добавлять новые модули
или изменять порядок обработки.\par
Этот паттерн предоставляет гибкость в обработке различных этапов конвертации,
позволяя легко управлять процессом и встраивать дополнительные шаги,
если это необходимо.\par
Использование <<Цепочки обязанностей>> также упрощает управление ошибками
и статусами, так как каждый компонент может завершить выполнение
или передать управление следующему,
в зависимости от результата своей работы.\par
Таким образом, паттерн <<Цепочка обязанностей>> подходит
для разработки системы конвертации DRC правил, обеспечивая гибкость,
масштабируемость и простоту модификации.

\clearpage

\section*{\LARGE Вывод}
\addcontentsline{toc}{section}{Вывод}

В результате работы были определены ключевые аспекты проектирования системы,
включая диаграммы прецедентов и последовательностей,
выбор паттерна проектирования <<Цепочка обязанностей>>.
Этот паттерн обеспечивает гибкость, удобство использования
и эффективную обработку данных,
что делает программу-конвертер DRC правил надежной и масштабируемой.


