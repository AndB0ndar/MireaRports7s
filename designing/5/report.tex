\section{Проектирование логической модели данных}

В данной системе логическая модель данных включает в себя только объекты,
которые используются для парсинга в библиотеке \textbf{Lark}:

\subsection{Tree}

\textbf{Tree} --- это основная модель,
представляющая собой абстрактное синтаксическое дерево (AST),
которое строится в результате парсинга DRC правил.
Каждый узел дерева может представлять различные конструкции правил.

Атрибуты класса:

\begin{itemize}
	\item \textbf{data:} имя правила или псевдонима;
	\item \textbf{children:} список совпадающих подправил и терминалов.
\end{itemize}

\subsection{Token}

\textbf{Token} --- это модель, представляющая отдельные токены,
полученные в процессе лексического анализа.
Токены используются для формирования синтаксического дерева
и содержат информацию о типе и значении каждого элемента.

Атрибуты класса:

\begin{itemize}
	\item \textbf{type:} имя токена;
	\item \textbf{value:} значение токена.
\end{itemize}

\subsection{Диаграмма логической модели}

\begin{image}
	\includegrph{er.drawio}
	\caption{ER-диаграмма}
	\label{fig:er}
\end{image}

\section{Словарь данных}

На первом этапе проектирования базы данных необходимо собрать сведения о
предметной области, в том числе о назначении, способах использования и охраны
структуре данных, а по мере развития проекта осуществлять централизованное
накопление информации о концептуальной, логической, внутренней и внешних
моделях данных. Словарь данных является как раз тем средством, которое
позволяет при проектировании, эксплуатации и развитии базы данных
поддерживать и контролировать информацию о данных.

\begin{longtable}{|p{3.5cm}|p{5cm}|p{5cm}|}
	\caption{\leftline{Словарь данных Tree}} \\
	\hline
	\textbf{Наименование элемента}
	& \textbf{Определение (предназначение)}
	& \textbf{Тип} \\
	\hline
	\endhead
	\textbf{data} & Имя правила или псевдонима & Строка \\ \hline
	\textbf{children}
	& Список совпадающих подправил и терминалов
	& Список \\ \hline
\end{longtable}

\begin{longtable}{|p{3.5cm}|p{5cm}|p{5cm}|}
	\caption{\leftline{Словарь данных Token}} \\
	\hline
	\textbf{Наименование элемента}
	& \textbf{Определение (предназначение)}
	& \textbf{Тип} \\
	\hline
	\endhead
	\textbf{type} & Имя токена & Строка \\ \hline
	\textbf{value} & Значение токена & Строка \\ \hline
\end{longtable}

\clearpage

\section{Определение ролей в системе}

В системе предусмотрена одна роль: \textbf{пользователь}.
Он запускает программу, передаёт исходный файл DRC правил и файл конфигурацию,
после чего получает результат в виде преобразованного файла
и отчёта о конвертации.
Пользователь не взаимодействует напрямую
с моделями \textbf{Tree} и \textbf{Token}.

\section{Матрица доступа}

Матрица доступа для роли разработчика по отношению к объектам,
использующимся в логической модели:

\begin{longtable}{|p{3.5cm}|p{5cm}|p{5cm}|}
	\caption{\leftline{Матрица доступа}} \\
	\hline
	\textbf{Роль} & \textbf{Tree} & \textbf{Token} \\
	\hline
	\endhead
	Пользователь & Нет доступа & Нет доступа \\ \hline
\end{longtable}

\clearpage

\section*{\LARGE Вывод}
\addcontentsline{toc}{section}{Вывод}

В результате выполнения практики была спроектирована логическая модель данных,
основанная на объектах \textbf{Tree} и \textbf{Token} библиотеки Lark.
Пользователь взаимодействует с системой только через ввод исходного файла
и получение результатов,
не имея доступа к внутренним структурам данных.
Составлен словарь данных и матрица доступа,
описывающая, что пользователь не взаимодействует с моделями данных напрямую,
а система автоматически выполняет необходимые преобразования.

