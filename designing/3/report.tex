\section{Составить (нарисовать) путь пользователя в вашей разработке}

Путь пользователя --- это общий алгоритм работы с продуктом. Так
называемый User Flow или путь пользователя, это последовательный
список действий или экранов, по которым может переходить
пользователь в процессе взаимодействия с продуктом.
Как пользователь будет взаимодействовать с продуктом
продемонстрированно на рисунке~\ref{fig:user:flow}.

\begin{image}
	\includegrph{user_flow.drawio.png}
	\caption{Путь пользователя в разработке}
	\label{fig:user:flow}
\end{image}

\section{Требования к математическому, программному,
	техническому и информационному обеспечению}

Математическое обеспечение:

\begin{itemize}
	\item Алгоритмы для построения ориентированного ациклического графа.
	\item Алгоритмы для топологической сортировки ациклического орт-графа.
	\item Алгоритмы для парсинга
		(Earley Parser, абстрактное синтаксическое дерево (AST),
		токенизации или лексический анализ)
\end{itemize}

Программное обеспечение:

\begin{itemize}
	\item \textbf{ОС}: Linux, Windows, macOS;
	\item \textbf{Язык программирования}: Python (версии 3.9 и выше);
	\item \textbf{Библиотеки}: configparse, argparse, os, sys,
		Lark (версии 1.2.2) и выше, NetworkX (версии 3.3 и выше).
\end{itemize}

Техническое обеспечение:

\begin{itemize}
	\item Минимальные системные требования:
	\begin{itemize}
		\item Процессор: 2 ГГц, 2 ядра;
		\item ОЗУ: 4 ГБ;
		\item Дисковое пространство: 500 МБ для установки, 2 ГБ для данных.
	\end{itemize}
	\item Рекомендуемые системные требования:
	\begin{itemize}
		\item Процессор: 3 ГГц, 4 ядра;
		\item ОЗУ: 8 ГБ;
		\item Дисковое пространство: 5-10 ГБ для работы с большими проектами.
	\end{itemize}
\end{itemize}

Информационное обеспечение:

\begin{itemize}
	\item Входные данные в формате SVRF Calbire
		(совместимые с приложением Calibre);
	\item Выходные данные в формате Ruby (как базовый синтакс)
		и KLayout (совместимые с приложением Klayout);
	\item Конфигурационные файлы в формате Ini-файла;
	\item Отчет о конвертации в формате LaTeX и Markdown.
\end{itemize}

\section{Перечень документации на программный продукт}

Руководство пользователя:

\begin{itemize}
	\item Описание установки программы;
	\item Описание использования программы;
	\item Примеры использования с различными DRC файлами.
\end{itemize}

Руководство разработчика:

\begin{itemize}
	\item Архитектура программного продукта;
	\item Описание зависимостей;
	\item Описание кода.
\end{itemize}

Техническое задание:
Описание функциональности программы, требований к ПО
и оборудованию, критериев успешности.

Программа и методика испытаний:
План тестирования продукта,
включая тест-кейсы для различных режимов работы и данных.

\section{Требования к эксплуатации разработки}


\section{Программа и методика испытаний}


\clearpage

\section*{\LARGE Вывод}
\addcontentsline{toc}{section}{Вывод}

