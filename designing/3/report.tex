\section{Составить (нарисовать) путь пользователя в вашей разработке}

Путь пользователя --- это общий алгоритм работы с продуктом. Так
называемый User Flow или путь пользователя, это последовательный
список действий или экранов, по которым может переходить
пользователь в процессе взаимодействия с продуктом.
Как пользователь будет взаимодействовать с продуктом
продемонстрированно на рисунке~\ref{fig:user:flow}.

\begin{image}
	\includegrph{user_flow.drawio.png}
	\caption{Путь пользователя в разработке}
	\label{fig:user:flow}
\end{image}

\section{Требования к математическому, программному,
	техническому и информационному обеспечению}

\subsection{Математическое обеспечение}

\begin{itemize}
	\item Алгоритмы для построения ориентированного ациклического графа.
	\item Алгоритмы для топологической сортировки ациклического орт-графа.
	\item Алгоритмы для парсинга
		(Earley Parser, абстрактное синтаксическое дерево (AST),
		токенизации или лексический анализ)
\end{itemize}

\subsection{Программное обеспечение}

Операционные системы:

\begin{itemize}
	\item Linux;
	\item Windows;
	\item macOS.
\end{itemize}

Язык программирования: Python (версии 3.9 и выше).

Библиотеки Python:

\begin{itemize}
	\item configparse;
	\item argparse;
	\item os;
	\item sys;
	\item Lark (версии 1.2.2 и выше);
	\item NetworkX (версии 3.3 и выше).
\end{itemize}

\subsection{Техническое обеспечение}

\textbf{Минимальные системные требования:}

\begin{itemize}
	\item Процессор: 2 ГГц, 2 ядра;
	\item ОЗУ: 4 ГБ;
	\item Дисковое пространство: 500 МБ для установки, 2 ГБ для данных.
\end{itemize}

\textbf{Рекомендуемые системные требования:}

\begin{itemize}
	\item Процессор: 3 ГГц, 4 ядра;
	\item ОЗУ: 8 ГБ;
	\item Дисковое пространство: 5-10 ГБ для работы с большими проектами.
\end{itemize}

\subsection{Информационное обеспечение}

\begin{itemize}
	\item Входные данные в формате SVRF Calbire
		(совместимые с приложением Calibre);
	\item Выходные данные в форматах Ruby (как базовый синтакс)
		и KLayout (функции встроенной библиотеки Klayout);
	\item Конфигурационные файлы в формате Ini-файла;
	\item Отчет о конвертации в формате LaTeX и Markdown.
\end{itemize}

\section{Перечень документации на программный продукт}

Руководство пользователя:

\begin{itemize}
	\item Описание установки программы;
	\item Описание использования программы;
	\item Примеры использования с различными DRC файлами.
\end{itemize}

Руководство разработчика:

\begin{itemize}
	\item Архитектура программного продукта;
	\item Описание зависимостей;
	\item Описание кода.
\end{itemize}

Техническое задание:
Описание функциональности программы, требований к ПО
и оборудованию, критериев успешности.

Программа и методика испытаний:
План тестирования продукта,
включая тест-кейсы для различных режимов работы и данных.

\section{Требования к эксплуатации разработки}

1. Правильность
   Программное обеспечение должно строго соответствовать техническому заданию и обеспечивать корректное выполнение всех заложенных функций. При этом стоит отметить, что тестирование не может гарантировать нахождение всех ошибок, поэтому необходимо акцентировать внимание на исправлении самых очевидных и критичных дефектов.

2. Универсальность
   Программа должна функционировать корректно при всех допустимых входных данных и должна иметь защиту от некорректных или несанкционированных данных. Важно, чтобы универсальность поддерживалась в различных сценариях работы с данными.

3. Надежность (помехозащищенность)
   Программа должна сохранять повторяемость результатов даже при сбоях, вызванных как техническими, так и программными средствами. В случае сбоев должны быть предусмотрены механизмы восстановления данных, предотвращающие их потерю. Для этого необходимо внедрить резервные копии данных и механизмы аварийного восстановления.

4. Проверяемость
   Все полученные результаты должны быть легко проверяемыми. Необходимо сохранять исходные данные, режимы работы и другие параметры, которые могут влиять на результаты выполнения программы, чтобы они могли быть документально проверены в случае возникновения ошибок или расхождений.

5. Точность результатов
   Программа должна обеспечивать точность вычислений и результатов в пределах заданной погрешности. Точность зависит от корректности исходных данных, адекватности используемой модели

\section{Программа и методика испытаний}

\subsection{Объект испытаний}

\subsubsection{Наименование системы}

Программа-конвертер DRC правил для использования
в открытых инструментах проектирования цифровых микросхем.

\subsubsection{Область применения системы}

Программный продукт, представляющий собой конвертер DRC правил,
предназначен для преобразования правил проектирования
из проприетарных форматов (SVRF Calibre) в форматы, совместимые
с открытыми инструментами проектирования цифровых микросхем (KLayout).
Основная цель --- обеспечить удобство
и доступность работы с DRC правилами для разработчиков и инженеров,
работающих в области проектирования интегральных схем.\par
Основная функция заключается в преобразования DRC правил
из проприетарных форматах в форматы,
поддерживаемые открытыми инструментами.

Области применения:

\begin{itemize}
	\item Использование в процессе проектирования интегральных схем,
		где необходимо соблюдать DRC правила
		для корректного функционирования и производства.
	\item Интеграция в существующие инструменты проектирования
		для автоматизации работы с DRC правилами и улучшения совместимости.
	\item Применение в учебных заведениях
		и исследовательских лабораториях для обучения студентов
		и специалистов работе с правилами проектирования.
	\item Помощь компаниям, желающим перейти
		на открытые инструменты проектирования,
		в конвертации существующих данных для обеспечения совместимости.
\end{itemize}

\subsubsection{Условное обозначение системы}

Условное обозначение Системы --- ПК DRC.

\subsection{Цель испытаний}

Целью проводимых по настоящей программе и методике испытаний ПК DRC
является определение функциональной работоспособности системы
на этапе проведения испытаний. 

Программа испытаний должна удостоверить работоспособность ПК DRC
в соответствии с функциональным предназначением.

\subsection{Общие положения}

\subsubsection{Перечень руководящих документов,
	на основании которых проводятся испытания}

Приёмочные испытания ПК DRC проводятся на основании следующих документов:

\begin{itemize}
	\item Утверждённое Техническое задание на разработку ПК DRC;
	\item Настоящая Программа и методика приёмочных испытаний;
\end{itemize}

\subsubsection{Место и продолжительность испытаний}

Место проведения испытаний --- площадка Заказчика.
Продолжительность испытаний устанавливается Приказом Заказчика
о составе приёмочной комиссии и проведении приёмочных испытаний.

\subsubsection{Организации, участвующие в испытаниях}

В приёмочных испытаниях участвуют представители следующих организаций:

\begin{itemize}
	\item ООО "Мальт Систем" (Заказчик);
	\item Бондарь Андрей Ренатович (Исполнитель).
\end{itemize}

Конкретный перечень лиц, ответственных за проведение испытаний системы,
определяется Заказчиком.

\subsubsection{Перечень предъявляемых на испытания документов}

Для проведения испытаний Исполнителем предъявляются следующие документы:

\begin{itemize}
	%\item Государственный контракт №2124-05-08 от 12.11.2008;
	\item Техническое задание на создание ПК DRC;
	\item Технический проект ПК DRC.
\end{itemize}

\subsection{Объём испытаний}

\subsubsection{Перечень этапов испытаний и проверок}

В процессе проведения приёмочных испытаний должны
быть протестированы следующие подсистемы ПК DRC:

\begin{itemize}
	\item Подсистема пользовательского интерфейса;
	\item Подсистема парсинга исходного DRC файла;
	\item Подсистема предобработки;
	\item Подсистема конвертации;
	\item Подсистема генерации отчета.
\end{itemize}

Все подсистемы испытываются одновременно
на корректность взаимодействия подсистем,
влияние подсистем друг на друга, т.е. испытания проводятся комплексно.

Приемочные испытания включают проверку:

\begin{itemize}
	\item полноты и качества реализации функций, указанных в ТЗ;
	\item выполнения каждого требования, относящегося к интерфейсу Системы;
	\item работы пользователей в диалоговом режиме;
	\item полноты действий, доступных пользователю,
		и их достаточность для функционирования Системы;
	\item сложности процедур диалога,
		возможности работы пользователей без специальной подготовки;
	\item реакции системы на ошибки пользователя;
	\item практической выполнимости рекомендованных процедур.
\end{itemize}

\subsubsection{Испытания подсистемы пользовательского интерфейса}

Испытания интерфейса направлены на проверку доступности всех функций программы,
удобства использования, корректного ввода данных и обработки ошибок.
Также проверяется устойчивость интерфейса при сбоях
и его отзывчивость при работе с большими файлами.


\subsubsection{Испытания подсистемы парсинга исходного DRC файла}

Тестируется корректность извлечения данных из различных DRC форматов.
Проверяется реакция на некорректные файлы,
а также производительность при обработке больших объемов данных.
Важно убедиться в корректной работе с форматом SVRF Calibre.

\subsubsection{Испытания подсистемы предобработки}

Проверяется правильность разворачивания макросов и упорядочивания кода.
Также тестируется производительность предобработки
и корректная обработка некорректных макросов.

\subsubsection{Испытания подсистемы конвертации}

Основная цель --- убедиться, что конвертация выполняется правильно
и соответствует эталонным данным для формата KLayout.
Проверяется работоспособность выходного файла,
производительность, а также обработка ошибок.

\subsubsection{Испытания подсистемы генерации отчета}

Тестируется полнота и корректность создаваемых отчетов,
поддержка различных форматов (текст, LaTeX, Markdown),
а также корректное отражение ошибок.
Проверяется производительность генерации отчетов при больших объемах данных.

\subsection{Методика проведения испытаний}

\begin{longtable}{|c|p{7.5cm}|p{7.5cm}|}
	\caption{\leftline{Методика проведения испытаний}} \label{table:test} \\
	\hline
	\textbf{\No} & \textbf{Действие} & \textbf{Результат} \\
	\hline
	\endfirsthead
	\conttable{table:test} \\
	\hline
	\textbf{\No} & \textbf{Действие} & \textbf{Результат} \\
	\hline
	\endhead

	\textbf{1}
	& \multicolumn{2}{|l|}{\textbf{
		Сценарий <<Тестирование пользовательского интерфейса>>}} \\ \hline
	1.1
	& Ввести некорректные данные
	(например, пустой путь к несуществующему файлу),
	попытаться запустить конвертацию.
	& Отображается сообщение об ошибке,
	программа не завершает работу аварийно. \\ \hline

	1.2
	& Ввести путь к корректному файлу проверок и файлу конфигурации.
	& Конвертация успешно запущена, отображается статус выполнения. \\ \hline

	1.3
	& Открыть лог выполнения, проверить, что отображаются все этапы обработки.
	& Лог корректно отображает все этапы работы программы. \\ \hline

	\textbf{2}
	& \multicolumn{2}{|p{15cm}|}{\textbf{Сценарий
		<<Тестирование подсистемы парсинга исходного DRC файла>>}} \\ \hline
	2.1
	& Загрузить файл DRC правил формата Calibre,
	проверить, что все элементы файла корректно извлечены.
	& Элементы извлечены, парсинг завершен без ошибок. \\ \hline

	2.2
	& Загрузить файл с синтаксическими ошибками, запустить парсинг.
	& Программа сообщает об ошибке, парсинг не завершен,
	информация о проблеме отображается. \\ \hline

	2.3
	& Проверить время парсинга для большого файла DRC правил (5000+ строк).
	& Парсинг завершен, время выполнения измерено
	и соответствует заявленным параметрам. \\ \hline

	\textbf{3}
	& \multicolumn{2}{|l|}{\textbf{
		Сценарий <<Тестирование подсистемы предобработки>>}} \\ \hline
	3.1
	& Загрузить файл с макросами, запустить предобработчик.
	& Макросы успешно развернуты,
	файл подготовлен к дальнейшей обработке. \\ \hline

	3.2
	& Загрузить файл с некорректным макросом и запустить предобработку.
	& Программа сообщает об ошибке, требует корректировки данных. \\ \hline

	3.3
	& Загрузить файл с некорректной последовательностью операций,
	запустить предобработчик.
	& Последовательность восстановлена,
	файл подготовлен к дальнейшей обработке. \\ \hline

	\textbf{4}
	& \multicolumn{2}{|l|}{\textbf{
		Сценарий <<Тестирование подсистемы конвертации>>}} \\ \hline
	4.1
	& Запустить конвертацию файла DRC правил
	из формата Calibre в формат Klayout.
	& Конвертация успешно выполнена,
	целевой файл сгенерирован без ошибок. \\ \hline

	4.2
	& Загрузить файл большого объема (5000+ строк) и запустить конвертацию.
	& Конвертация завершена,
	производительность системы соответствует заявленным требованиям. \\ \hline

	4.3
	& Передать файл проверок с явной ошибкой в параметрах операции
	и запустить конвертацию.
	& Программа сообщает об ошибке и завершает конвертацию с ошибкой. \\ \hline

	\textbf{5}
	& \multicolumn{2}{|l|}{\textbf{
		Сценарий <<Тестирование подсистемы генерации отчета>>}}  \\ \hline
	5.1
	& По завершении конвертации проверить наличие
	отчета о выполненных операциях.
	& Отчет сгенерирован,
	содержит все ключевые данные (успешные шаги, ошибки). \\ \hline

	5.2
	& Проверить отчет на наличие ошибок
	и предупреждений при использовании файла правил с ошибкой.
	& Отчет включает информацию об ошибках,
	предупреждения отображены корректно. \\ \hline

	5.3
	& Сохранить отчет в разных форматах (текстовый файл, LaTeX, Markdown)
	и проверить их корректность.
	& Отчеты корректно сохранены в выбранных форматах
	и доступны для анализа. \\ \hline
\end{longtable}

\subsection{Требования по испытаниям программных средств}

Испытания программных средств ПК DRC проводятся
в процессе функционального тестирования Системы
и её нагрузочного тестирования.
Других требований по испытаниям программных средств ПК DRC не предъявляется.

\subsection{Перечень работ, проводимых после завершения испытаний}

По результатам испытаний делается заключение
о соответствии ПК DRC требованиям ТЗ на Систему
и возможности оформления акта сдачи ПК DRC в опытную эксплуатацию.
При этом производится (при необходимости) доработка программных средств
и документации.

\subsection{Условия и порядок проведения испытаний}

Испытания ПК DRC должны проводиться на целевом оборудовании Заказчика.
Оборудование должно быть предоставлено в той конфигурации,
которая запланирована для начального развёртывания системы,
и указанна в Техническом задании.

Во время испытаний проводится полное функциональное тестирование,
согласно требованиям, указанным в Техническом задании.

В ходе проведения опытной эксплуатации
для каждого участника испытаний
Системы администратор выдает доступ к исполняемому файлу Системы
для проведения полнофункционального тестирования.

Данные пользователи работают с Системой,
выполняя свои служебные обязанности,
то есть создают файлы правил для конкретных техпроцессов и выполняются
их в Klayout на дизайнах цифровых микросхем,
подвергая тем самым ПК DRC полнофункциональному тестированию
в течение установленного срока.

\subsection{Материально-техническое обеспечение испытаний}

Приёмочные испытания проводятся
на программно-аппаратном комплексе Заказчика
в следующей минимальной конфигурации:

\begin{itemize}
	\item ПК в составе АРМ пользователя
		или серверная площадка, выделенная Заказчиком на территории
		для проведения приемочных испытаний;
	\item Операционная система: Linux;
	\item Программы: Python версии 3.9 или выше
		и KLayout версии 0.29.6 или выше.
\end{itemize}

\subsection{Метрологическое обеспечение испытаний}

Программа испытаний не требует
использования специализированного измерительного оборудования.

\subsection{Отчётность}

Результаты испытаний ПК DRC, предусмотренные настоящей программой,
фиксируются в протоколах, содержащих следующие разделы:

\begin{itemize}
	\item Назначение испытаний и номер раздела требований ТЗ на ПК DRC,
		по которому проводят испытание;
	\item Состав технических и программных средств,
		используемых при испытаниях;
	\item Указание методик, в соответствии с которыми проводились испытания,
		обработка и оценка результатов;
	\item Условия проведения испытаний и характеристики исходных данных;
	\item Средства хранения и условия доступа к тестирующей программе;
	\item Обобщённые результаты испытаний;
	\item Выводы о результатах испытаний
		и соответствии созданной Системы
		определённому разделу требований ТЗ на ПК DRC.
\end{itemize}

В протоколах могут быть занесены замечания персонала
по удобству эксплуатации Системы.
Этап проведения предварительных испытаний завершается
оформлением <<Акта предварительных и приемочных испытаний ПК DRC>>.

\clearpage

\section*{\LARGE Вывод}
\addcontentsline{toc}{section}{Вывод}

