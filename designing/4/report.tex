\section{Выбрать средства разработки для вашего программного средства.
	Обосновать выбор средств разработки}

Для разработки программы-конвертера DRC правил выбраны следующие инструменты:

\textbf{Язык программирования}: Python версии 3.9 и выше.\par
\textbf{Обоснование}: Python --- это гибкий язык,
широко применяемый в области электронной разработки (EDA)
благодаря множеству библиотек
и простоте интеграции с различными форматами данных.
Поддерживает библиотеки для парсинга, работы с файлами и их преобразования.
  
\textbf{Библиотека для парсинга}: Lark.\par
\textbf{Обоснование}: Lark --- это мощная и гибкая библиотека
для работы с грамматиками.
Она поддерживает несколько алгоритмов парсинга (Earley, LALR, CYK)
и легко интегрируется с Python.
Она позволяет парсить сложные грамматики,
что необходимо для работы с DRC правилами.

\textbf{Библиотека для работы с графами}: NetworkX.\par
\textbf{Обоснование}: NetworkX --- это библиотека для работы с графами,
которая может быть полезна для построения графов зависимостей между правилами
и их дальнейшей обработки.
Использование этой библиотеки позволяет эффективно анализировать взаимосвязи
между элементами, такими как макросы или правила,
и оптимизировать порядок их применения в процессе конвертации.

\textbf{Фреймворк для тестирования}: Pytest.\par
\textbf{Обоснование}: Pytest предоставляет простой
и удобный способ написания модульных
и интеграционных тестов для Python-программ.
Он поддерживает автоматизацию тестирования,
что важно для обеспечения стабильности программы.

\textbf{Система управления версиями}: Git.\par
\textbf{Обоснование}: Git обеспечивает контроль версий кода,
что необходимо для отслеживания изменений и работы в команде.
GitHub или GitLab могут использоваться для хранения кода и организации CI/CD.

\textbf{Инструменты автоматизации}: Make.\par
\textbf{Обоснование}: Make позволяют автоматизировать процесс сборки программы,
обеспечивая удобство работы с зависимостями и процессом тестирования.

\textbf{Среда разработки (текстовые редакторы)}: Vim и Neovim (nvim).\par
\textbf{Обоснование}: Vim и Neovim (nvim) --- это лёгкие
и высокопроизводительные текстовые редакторы,
которые идеально подходят для работы с кодом в командной строке.
Они поддерживают плагины для работы с Python, автодополнение, отладку,
интеграцию с Git, а также позволяют быстро
и эффективно редактировать код.
Neovim предоставляет дополнительные возможности
для кастомизации и работы с современными плагинами.

\section{Словесно описать предлагаемую архитектуру системы}

Основной принцип архитектуры программы-конвертера DRC правил
--- \textbf{модульность}.
Программа состоит из отдельных компонентов (модулей),
которые отвечают за конкретные функции:
парсинг, конвертация, предобработка и взаимодействие с пользователем.
Это позволяет легко изменять или добавлять новые функции
без необходимости переписывать всю систему.\par
Также в программном продукте используется принцип
\textbf{разделения ответственности} 
(Separation of Concerns или SoC), где каждому модулю назначена
своя ответственность, что снижает взаимозависимость между компонентами.
Это делает систему более поддерживаемой и упрощает тестирование и отладку.

Основные модули системы:

\begin{itemize}
	\item \textbf{Модуль пользовательского интерфейса}:
	Обработывает параметры командной строки.
	Обрабатывает конфигурационный файлы,
	определяющий параметры работы программы,
	пути к исходным данным, настройкам преобразования и т.д.
	И возвращает результат преобразования в виде файла или вывода в консоль,
	в зависимости от настроек.
	\item \textbf{Модуль парсинга исходного DRC файла}:
	Отвечает за чтение исходного файла DRC Calibre
	и преобразование его с помощью библиотеки для парсинга Lark
	в предоставляемый им структуру данных для дальнейшей конвертации.
	\item \textbf{Модуль предобработки}:
	Подготавливает данные к конвертации.
	Он объединяет многострочные команды для корректного парсинга,
	разворачивает макросы, содержашиеся в коде, а также
	создает корректную последовательность команд в коде.
	\item \textbf{Модуль конвертации}:
	Основной модуль, который выполняет преобразование команд DRC Calibre
	в формат KLayout.
	\item \textbf{Модуль генерации отчета}:
	Модуль, генерирующий отчет о конвертации DRC Calibre на основе информации,
	полученной на стадии конвертации.
\end{itemize}

\section{Построить архитектурную диаграмму своей разработки
	при помощи модели C4}

Модель C4 описывает архитектуру системы на четырёх уровнях:
контекста, контейнеров, компонентов и кода.
На основе этой модели можно создать следующие диаграммы
для программы-конвертера:

1. **Диаграмма контекста системы**:
   - Включает взаимодействие пользователя с программой и внешние системы (например, исходные и целевые форматы DRC правил).
   
2. **Диаграмма контейнеров**:
   - Включает основные модули программы: загрузка данных, парсинг, предобработка, конвертация, верификация, а также систему хранения логов.
   
3. **Диаграмма компонентов**:
   - Показывает детализированную структуру каждого контейнера: взаимодействие с библиотеками (Lark, Pytest), работу с файлами, вызовы методов для конвертации и верификации.

#### Пример диаграммы контейнеров (уровень C2):
```
[Пользователь] --> [Консольная программа]
[Консольная программа] --> [Модуль загрузки данных] --> [Модуль парсинга]
[Модуль парсинга] --> [Модуль предобработки]
[Модуль предобработки] --> [Модуль конвертации]
[Модуль конвертации] --> [Модуль верификации]
[Модуль верификации] --> [Логи]
```

---

\section{Описать масштабируемость системы}

Масштабируемость системы заключается в её способности обрабатывать
всё более сложные файлы DRC правил,
увеличивать количество поддерживаемых форматов,
а также адаптироваться к различным требованиям пользователей.

Основные аспекты масштабируемости:

- **Горизонтальная масштабируемость**: Программа может быть запущена на нескольких машинах для параллельной обработки различных проектов или частей большого проекта.
  
- **Вертикальная масштабируемость**: При увеличении мощности оборудования (например, процессора и оперативной памяти) программа сможет обрабатывать более сложные и объёмные файлы DRC за меньшее время.
  
- **Масштабируемость за счёт модульности**: Каждый компонент программы (например, парсинг, предобработка, конвертация) может быть расширен или заменён, что позволит легко добавлять новые возможности без изменения существующего кода.
  
- **Интеграция с другими EDA инструментами**: Возможность добавления поддержки новых систем (например, других DRC форматов или API для работы с EDA инструментами) делает программу масштабируемой в контексте взаимодействия с внешними инструментами.

---

\section{Описать, при помощи каких инструментов будет реализован каждый компонент архитектуры}

1. **Модуль загрузки и парсинга файлов**:
   - **Lark**: Используется для парсинга исходных файлов DRC правил на основе заранее определённых грамматик.
   - **argparse**: Для обработки аргументов командной строки (например, путь к файлу, тип конвертации).

2. **Модуль предобработки**:
   - **Custom Python Code**: Самописные алгоритмы для развёртки макросов, сортировки и оптимизации правил.

3. **Модуль конвертации**:
   - **Custom Python Code**: Логика конвертации правил между различными форматами DRC (например, из Calibre в KLayout). Также включает методы для работы с геометрическими командами.

4. **Модуль верификации**:
   - **Custom Python Code**: Проверка корректности синтаксиса и семантики конвертированных файлов. Используются регулярные выражения и методы анализа абстрактного синтаксического дерева.

5. **Логирование и обработка ошибок**:
   - **logging (Python стандартная библиотека)**: Для логирования действий программы и фиксации ошибок.
  
6. **Система тестирования**:
   - **Pytest**: Для создания и выполнения тестов, проверки корректности работы программы на различных тестовых данных.

7. **Консольный интерфейс**:
   - **argparse**: Для обработки команд и аргументов из командной строки, что позволяет пользователю настраивать параметры конвертации.


\clearpage

\section*{\LARGE Вывод}
\addcontentsline{toc}{section}{Вывод}

