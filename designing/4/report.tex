\section{Выбрать средства разработки для вашего программного средства.
	Обосновать выбор средств разработки}

Для разработки программы-конвертера DRC правил выбраны следующие инструменты:

\textbf{Язык программирования}: Python версии 3.9 и выше.\par
\textbf{Обоснование}: Python --- это гибкий язык,
широко применяемый в области электронной разработки (EDA)
благодаря множеству библиотек
и простоте интеграции с различными форматами данных.
Поддерживает библиотеки для парсинга, работы с файлами и их преобразования.
  
\textbf{Библиотека для парсинга}: Lark.\par
\textbf{Обоснование}: Lark --- это мощная и гибкая библиотека
для работы с грамматиками.
Она поддерживает несколько алгоритмов парсинга (Earley, LALR, CYK)
и легко интегрируется с Python.
Она позволяет парсить сложные грамматики,
что необходимо для работы с DRC правилами.

\textbf{Библиотека для работы с графами}: NetworkX.\par
\textbf{Обоснование}: NetworkX --- это библиотека для работы с графами,
которая может быть полезна для построения графов зависимостей между правилами
и их дальнейшей обработки.
Использование этой библиотеки позволяет эффективно анализировать взаимосвязи
между элементами, такими как макросы или правила,
и оптимизировать порядок их применения в процессе конвертации.

\textbf{Фреймворк для тестирования}: Pytest.\par
\textbf{Обоснование}: Pytest предоставляет простой
и удобный способ написания модульных
и интеграционных тестов для Python-программ.
Он поддерживает автоматизацию тестирования,
что важно для обеспечения стабильности программы.

\textbf{Система управления версиями}: Git.\par
\textbf{Обоснование}: Git обеспечивает контроль версий кода,
что необходимо для отслеживания изменений и работы в команде.
GitHub или GitLab могут использоваться для хранения кода и организации CI/CD.

\textbf{Инструменты автоматизации}: Make.\par
\textbf{Обоснование}: Make позволяют автоматизировать процесс сборки программы,
обеспечивая удобство работы с зависимостями и процессом тестирования.

\textbf{Среда разработки (текстовые редакторы)}: Vim и Neovim (nvim).\par
\textbf{Обоснование}: Vim и Neovim (nvim) --- это лёгкие
и высокопроизводительные текстовые редакторы,
которые идеально подходят для работы с кодом в командной строке.
Они поддерживают плагины для работы с Python, автодополнение, отладку,
интеграцию с Git, а также позволяют быстро
и эффективно редактировать код.
Neovim предоставляет дополнительные возможности
для кастомизации и работы с современными плагинами.

\section{Словесно описать предлагаемую архитектуру системы}

Основной принцип архитектуры программы-конвертера DRC правил
--- \textbf{модульность}.
Программа состоит из отдельных компонентов (модулей),
которые отвечают за конкретные функции:
парсинг, конвертация, предобработка и взаимодействие с пользователем.
Это позволяет легко изменять или добавлять новые функции
без необходимости переписывать всю систему.\par
Также в программном продукте используется принцип
\textbf{разделения ответственности} 
(Separation of Concerns или SoC), где каждому модулю назначена
своя ответственность, что снижает взаимозависимость между компонентами.
Это делает систему более поддерживаемой и упрощает тестирование и отладку.

Основные модули системы:

\begin{itemize}
	\item \textbf{Модуль пользовательского интерфейса}:
	Обработывает параметры командной строки.
	Обрабатывает конфигурационный файлы,
	определяющий параметры работы программы,
	пути к исходным данным, настройкам преобразования и т.д.
	И возвращает результат преобразования в виде файла или вывода в консоль,
	в зависимости от настроек.
	\item \textbf{Модуль парсинга исходного DRC файла}:
	Отвечает за чтение исходного файла DRC Calibre
	и преобразование его с помощью библиотеки для парсинга Lark
	в предоставляемый им структуру данных для дальнейшей конвертации.
	\item \textbf{Модуль предобработки}:
	Подготавливает данные к конвертации.
	Он объединяет многострочные команды для корректного парсинга,
	разворачивает макросы, содержашиеся в коде, а также
	создает корректную последовательность команд в коде.
	\item \textbf{Модуль конвертации}:
	Основной модуль, который выполняет преобразование команд DRC Calibre
	в формат KLayout.
	\item \textbf{Модуль генерации отчета}:
	Модуль, генерирующий отчет о конвертации DRC Calibre на основе информации,
	полученной на стадии конвертации.
\end{itemize}

\section{Построить архитектурную диаграмму своей разработки
	при помощи модели C4}

Модель C4 описывает архитектуру системы на четырёх уровнях:
контекста, контейнеров, компонентов и кода.
На основе этой модели можно создать следующие диаграммы
для программы-конвертера:

\subsection{Диаграмма контекста системы}

В модели C4 диаграмма контекста системы (System Context Diagram)
представляет собой самый высокий уровень абстракции системы,
показывая взаимодействие системы с внешними пользователями (акторами)
и системами. Она дает общее представление о том,
кто и как взаимодействует с системой,
не углубляясь в детали внутренней реализации.
   
\begin{image}
	\includegrph[scale=0.4]{c40.drawio}
	\caption{Диаграмма контекста системы}
	\label{fig:c4:system:context}
\end{image}

Пользователь взаимодействует с системой через интерфейс командной строки (CLI)
для загрузки конфигурационных файлов и исходных DRC файлов,
а также для запуска процесса конвертации.\par
Файловая система ОС используются для чтения и записи исходных
и сконвертированных DRC правил, а также для создания отчета о конвертации.

\subsection{Диаграмма контейнеров}

Диаграмма контейнеров представляет следующий уровень детализации модели C4.
Она показывает внутреннюю структуру системы,
отображая основные контейнеры (программные компоненты) системы,
такие как приложения, базы данных, внешние системы, и их взаимодействия.
   
\begin{image}
	\includegrph[scale=0.4]{c41.drawio}
	\caption{Диаграмма компонентов}
	\label{fig:c4:container}
\end{image}

Так как приложение не разделяется на API и базу данный,
данная диаграмма отличается от предыдущей добавлением внешней системы
консоли для взаимодействия с приложением
через интерфейс командной строки (CLI).

\subsection{Диаграмма компонентов}

Диаграмма компонентов модели C4 детализирует
каждый контейнер на более глубоком уровне,
отображая его внутренние программные компоненты и взаимодействия между ними.
Это позволяет увидеть, как функционирует каждая часть контейнера
и какие компоненты обеспечивают выполнение основных функций.

\begin{image}
	\includegrph[scale=0.25]{c42.drawio}
	\caption{Диаграмма компонентов}
	\label{fig:c4:components}
\end{image}

\section{Масштабируемость системы}

Масштабируемость системы заключается в её способности обрабатывать
всё более сложные файлы DRC правил,
увеличивать количество поддерживаемых форматов,
а также адаптироваться к различным требованиям пользователей.

\subsection{Вертикальная масштабируемость}

При увеличении мощности оборудования
(например, процессора и оперативной памяти)
программа сможет обрабатывать более сложные
и объёмные файлы DRC за меньшее время.
  
\subsection{Масштабируемость за счёт модульности}

Каждый компонент программы (например, парсинг, предобработка, конвертация)
может быть расширен или заменён,
что позволит легко добавлять новые возможности
без изменения существующего кода.
  
\subsection{Интеграция с другими EDA инструментами}

Возможность добавления поддержки новых систем
(например, других DRC форматов или API для работы с EDA инструментами)
делает программу масштабируемой
в контексте взаимодействия с внешними инструментами.

\section{Инструменты, используемые для реализации компонентов архитектуры}

\subsection{Обработка пользовательского ввода}

Для обработки пользовательского ввода
через командную строку (CLI), используются следующие инструменты:

\begin{itemize}
	\item \textbf{os:} Для чтения исходных файлов;
	\item \textbf{argparse:} Для обработки команд
		и аргументов из командной строки,
		что позволяет пользователю настраивать параметры конвертации;
	\item \textbf{configparse:} Для обработки конфигурационного файла,
		что позволяет пользователю настраивать параметры конвертации.
\end{itemize}

\subsection{Парсинг}

Парсинг DRC правил является одной из ключевых функций системы,
и для реализации этого компонента применяется библиотека Lark-parser.
Это мощная библиотека для парсинга грамматик.
Она поддерживает контекстно-свободные грамматики
и позволяет создавать синтаксические деревья для последующей обработки.
\textbf{Lark} идеально подходит для разбора
сложных синтаксических конструкций DRC правил.

\subsection{Предобработка}

Предобработка включает в себя разворачивание макросов,
упорядочивание и оптимизацию данных перед конвертацией.\par
Реализация алгоритмов предобработки выполняется на Python
с использованием собственных логических структур.
Это включает работу с деревьями, построенными на этапе парсинга,
разворачивание макросов с использованием рекурсивных алгоритмов
и восстановление последовательности операций,
используя библиотеку \textbf{NetworkX}.\par
Библиотека \textbf{NetworkX} используется для работы с графами,
что полезно при работе с зависимостями между операциями.
Она помогает оптимизировать и упорядочить правила
для более эффективной конвертации.

\subsection{Компонент конвертации}

Основной процесс преобразования данных из одного формата в другой
требует специализированных инструментов
для работы с текстом и логической трансляции правил.\par
Основная часть конвертации выполняется
через специально разработанные алгоритмы.
Эти алгоритмы должны трансформировать геометрические проверки
и другие DRC правила из одного формата в другой.

\subsection{Компонент генерации отчёта}

Отчёты о результатах работы системы генерируются
с использованием пользовательского кода.
Для создания текстовых, LaTeX и Markdown видов отчетов используется
вручную написанный код.
Это обеспечивает гибкость и контроль над содержимым и форматированием отчетов.

\clearpage

\section*{\LARGE Вывод}
\addcontentsline{toc}{section}{Вывод}

В ходе работы были выбраны и обоснованы средства разработки
для создания программы-конвертора DRC правил.
Основным языком программирования был выбран Python,
так как он предоставляет широкий набор инструментов для парсинга,
работы с данными и взаимодействия с пользователем.
Библиотека Lark обеспечивает гибкий парсинг,
а argparse и configparse позволяют легко организовать
пользовательский интерфейс.\par
Архитектура системы построена по модульному принципу,
что делает её легко расширяемой и поддерживаемой.
Основные модули включают обработку пользовательского ввода, парсинг, предобработку, конвертацию и генерацию отчетов.
Каждый модуль изолирован и отвечает за свою часть работы,
что упрощает поддержку и тестирование.\par
Для реализации каждого компонента были выбраны подходящие инструменты.\par
Система получилась гибкой, масштабируемой
и легко расширяемой для решения задач конвертации DRC правил.

