\section{Системы-аналоги}

Существующие коммерческие решения, такие как Calibre и Assura,
не подходят для использования в открытых проектах ввиду их высокой стоимости
и закрытых форматов данных.
С другой стороны есть открытое решение, предоставленное приложением Klayout,
но он не поддерживат коммерческие форматы файлов проверок,
используемые большинством фабрик-производителей.
Это создает потребность в создании собственной системы,
которая позволит автоматически конвертировать правила DRC
из коммерческих форматов в открытые инструменты, такие как Klayout.


\section{Входные и выходные данные разработки}

Входные данные:

\begin{itemize}
	\item Файл правил DRC, созданный для коммерческих инструментов
		(форматы Calibre SVRF).
	\item Файл конфигурации (при необходимости)
\end{itemize}

Выходные данные:

\begin{itemize}
	\item Файл правил DRC, преобразованный для использования
		в открытых инструментах (Klayout).
	\item Лог-файл с информацией о ходе конвертации
		и предупреждениями о возможных несовместимостях.
\end{itemize}

\section{Бизнес-процессы в вариации TO-BE (нотация IDEF0)}

Использование нотации IDEF0 позволяет описать процессы на верхнем уровне
и их взаимодействие. Рассмотрим процесс конвертации DRC правил.

\begin{image}
	%\includegrph{Screenshot from 2024-09-13 20-48-30}
	%\caption{Дерево проекта}
	%\label{fig:tree:project}
\end{image}

\section{Границы автоматизируемых бизнес-процессов}

Границы автоматизируемых бизнес-процессов включают:

\begin{itemize}
	\item Автоматизация процесса конвертации правил
		из проприетарного формата в формат,
		поддерживаемый open-source инструментами;
	\item Генерация выходных данных в соответствии с параметрами,
		заданными пользователем;
	\item Автоматизация проверки успешности конвертации и предупреждений;
	\item Генерация отчета о трансляции.
\end{itemize}

Границы автоматизации не включают:

\begin{itemize}
	\item Проверку корректности конвертированных файлов
		в открытых инструментах;
	\item Валидацию данных на уровне конечных инструментов проектирования;
\end{itemize}

\section{Дерево зависимостей функций друг от друга}

Дерево зависимостей функций можно построить на основе взаимосвязанных
шагов процесса конвертации DRC правил.

\textbf{Основная функция:} Конвертация DRC правил
Она состоит из подфункций:
\begin{enumerate}
	\item Загрузка исходных файлов
	\begin{enumerate}
		\item Проверка формата файла.
	\end{enumerate}
	\item Настройка параметров конвертации
	\begin{enumerate}
		\item Выбор целевого формата.
		\item Настройка опций (игнорирование несовместимых правил и др.).
	\end{enumerate}
	\item Преобразование правил
	\begin{enumerate}
		\item Преобразование синтаксиса правил.
		\item Обработка несовместимых правил.
	\end{enumerate}
	\item Генерация выходных файлов
	\begin{enumerate}
		\item Генерация файла DRC для открытого инструмента.
		\item Создание лог-файла.
	\end{enumerate}
\end{enumerate}

Это дерево функций показывает зависимость между этапами работы программы
и связанными с ними подфункциями.

\clearpage

\section*{\LARGE Вывод}
\addcontentsline{toc}{section}{Вывод}


