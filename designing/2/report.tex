\section{Системы-аналоги}

Существующие коммерческие решения, такие как Calibre и Assura,
не подходят для использования в открытых проектах ввиду их высокой стоимости
и закрытых форматов данных.
С другой стороны есть открытое решение, предоставленное приложением Klayout,
но он не поддерживат коммерческие форматы файлов проверок,
используемые большинством фабрик-производителей.
Это создает потребность в создании собственной системы,
которая позволит автоматически конвертировать правила DRC
из коммерческих форматов в открытые инструменты, такие как Klayout.


\section{Входные и выходные данные разработки}

Входные данные:

\begin{itemize}
	\item Путь до файла правил DRC,
		созданный для коммерческих инструментов (формат SVRF Calibre).
	\item Путь до файла конфигурации
\end{itemize}

Выходные данные:

\begin{itemize}
	\item Правила DRC, преобразованный для использования
		в открытых инструментах (в формате Klayout).
	\item Отчет о результатах трансляции.
\end{itemize}

\section{Бизнес-процессы в вариации TO-BE (нотация IDEF0)}

Использование нотации IDEF0 позволяет описать процессы на верхнем уровне
и их взаимодействие
\rdref{fig:idef0}{fig:idef0:a4}.

\begin{image}
	\includegrph{2024-09-29_16-46-34}
	\caption{Контекстная диаграмма}
	\label{fig:idef0}
\end{image}

\begin{image}
	\includegrph{2024-09-29_16-46-47}
	\caption{Декомпозиция контекстной диаграммы}
	\label{fig:idef0:a0}
\end{image}

\begin{image}
	\includegrph{2024-09-29_16-16-02}
	\caption{Декомпозиция процесса предобработки}
	\label{fig:idef0:a3}
\end{image}

\begin{image}
	\includegrph{2024-09-29_16-15-22}
	\caption{Декомпозиция процесса конвертации}
	\label{fig:idef0:a4}
\end{image}

\section{Границы автоматизируемых бизнес-процессов}

Границы автоматизируемых бизнес-процессов включают:

\begin{itemize}
	\item Автоматизация процесса конвертации правил
		из проприетарного формата в формат,
		поддерживаемый open-source инструментами;
	\item Генерация отчета о трансляции;
	\item Автоматизация развертывания макросов;
	\item Автоматизация упорядочивания кода.
\end{itemize}

Границы автоматизации не включают:

\begin{itemize}
	\item Проверку корректности конвертированных файлов
		в открытых инструментах;
	\item Валидацию данных на уровне конечных инструментов проектирования;
\end{itemize}

\section{Дерево зависимостей функций друг от друга}

Дерево зависимостей функций можно построить на основе взаимосвязанных
шагов процесса конвертации DRC правил.

\begin{image}
	\includegrph{functree.drawio.png}
	\caption{Дерево зависимостей функций}
	\label{fig:tree}
\end{image}

Это дерево функций показывает зависимость между этапами работы программы
и связанными с ними подфункциями.

\clearpage

\section*{\LARGE Вывод}
\addcontentsline{toc}{section}{Вывод}

В ходе выполнения практической работы было установлена
уникальность программного продукта их-за отсутствия прямых аналогов.
Определены входные и выходные данные, необходимые для работы системы,
это позволило сформировать требования к её функциональности.
Затем смоделированы бизнес-процессы в оптимизированной вариации TO-BE
с использованием нотации IDEF0
для наглядного представления последовательности операций.
Определены границы автоматизируемых процессов,
что дало возможность очертить сферу применения системы.
Построено дерево зависимостей функций,
показывающее взаимосвязь различных компонентов системы
и обеспечивающее основу для её проектирования.

