\section{Диаграмма Ганта для разработки программного средства}

Диаграмма Ганта представляет собой визуальное отображение задач
и временных рамок, необходимых для разработки системы конвертации DRC правил.
Ниже представлена таблица
с определением задач, подзадач, времени выполнения и исполнителей.

\begin{longtable}{|p{3.5cm}|p{4cm}|p{3cm}|p{4cm}|}
	\caption{\leftline{Диаграмма Ганта}} \label{table:ganta} \\
	\hline
	\textbf{Задача}
	& \textbf{Подзадача}
	& \textbf{Время выполнения}
	& \textbf{Исполнитель} \\
	\hline
	\endfirsthead
	\conttable{table:ganta} \\
	\hline
	\textbf{Задача}
	& \textbf{Подзадача}
	& \textbf{Время выполнения}
	& \textbf{Исполнитель} \\
	\hline
	\endhead
	1. Проектирование системы
	& Определение требований
	& 7 дней
	& Бондарь А.Р. \\ \hline

	& Проектирование архитектуры
	& 7 дней
	& Бондарь А.Р. \\ \hline

	2. Разработка компонентов
	& Реализация парсинга
	& 21 дней
	& Бондарь А.Р. \\ \hline

	& Реализация конвертации
	& 60 дней
	& Бондарь А.Р. \\ \hline

	& Реализация предобработки
	& 14 дней
	& Бондарь А.Р. \\ \hline

	& Реализация работы с пользователем
	& 7 дней
	& Бондарь А.Р. \\ \hline

	& Реализация генерации отчетов
	& 7 дней
	& Бондарь А.Р. \\ \hline

	3. Тестирование
	& Разработка тестов
	& 14 дней
	& Бондарь А.Р. \\ \hline

	& Проведение тестирования
	& 3 дня
	& Бондарь А.Р. \\ \hline

	4. Документация
	& Написание пользовательской документации
	& 7 дней
	& Отдел разработки документации \\ \hline

	& Подготовка технической документации
	& 7 дней
	& Отдел разработки документации \\ \hline

	& Разработка инструкции по внедрению
	& 3 дня
	& Отдел разработки документации \\ \hline

	5. Внедрение и запуск
	& Внедрение и запуск системы
	& 3 дня
	& Бондарь А.Р. \\ \hline
\end{longtable}

Диаграмма ганта показана на рисунке~\ref{fig:gantt}.

\begin{image}
	\includegrph[scale=0.28]{Screenshot from 2024-10-12 14-02-22}
	\caption{Диаграмма Ганта}
	\label{fig:gantt}
\end{image}


\section{Таблица с перечнем основных рисков для проекта}

Таблица рисков помогает заранее определить
и подготовиться к потенциальным проблемам,
которые могут возникнуть в процессе разработки.

\begin{longtable}{|p{3.2cm}|p{3.2cm}|p{2cm}|p{3cm}|p{4cm}|}
	\hline
	\textbf{Название риска}
	& \textbf{Последствия}
	& \textbf{Кач. оценка риска}
	& \textbf{Стратегия реагирования}
	& \textbf{Мероприятия} \\ \hline
	\endfirsthead
	\hline
	\textbf{Название риска}
	& \textbf{Последствия}
	& \textbf{Кач. оценка риска}
	& \textbf{Стратегия реагирования}
	& \textbf{Мероприятия} \\ \hline
	\endhead
	Задержка в сроках разработки
	& Пропуск сроков сдачи проекта
	& Высокая
	& Смягчение 
	& Регулярные встречи команды для мониторинга статуса задач. \\ \hline
	Нехватка ресурсов
	& Увеличение времени разработки, снижение качества
	& Низкая
	& Устранение
	& Определить критические ресурсы заранее и обеспечить их наличие. \\ \hline
	Ошибки в коде
	& Необходимость доработки, потенциальные сбои системы
	& Высокая
	& Превентивные меры
	& Регулярное рецензирование кода и тестирование. \\ \hline
	Неопределен- ность требований
	& Сложности с реализацией функций
	& Низкая
	& Гибкость
	& Обсуждение требований с заказчиком на всех этапах. \\ \hline
	Технические проблемы
	& Остановка работы, снижение производитель- ности
	& Средняя
	& Резервное копирование
	& Настройка резервных копий и тестов на выявление проблем. \\ \hline
	Отсутствие аналогов для конвертируемой команды
	& Задержка в реализации и отладке из-за отсутствия готовых решений
	& Высокая
	& Исследование и разработка
	& Проводить исследование команды DRC правил
	и реализовать необходимый функционал через последовательность
	других команд. \\ \hline
\end{longtable}


\clearpage

\section*{\LARGE Вывод}
\addcontentsline{toc}{section}{Вывод}

В результате практической работы
были разработаны диаграмма Ганта и таблица рисков,
что помогает в планировании
и управлении проектом по разработке системы конвертации DRC правил.
Диаграмма Ганта визуализирует задачи, подзадачи и временные рамки,
а таблица рисков помогает заранее определить
и подготовиться к потенциальным проблемам,
которые могут возникнуть в процессе разработки.
Это способствует более эффективному управлению проектом
и повышению шансов на успешное завершение разработки в установленные сроки.


