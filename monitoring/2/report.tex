\section*{ЦЕЛЬ ПРАКТИЧЕСКОЙ РАБОТЫ}
\addcontentsline{toc}{section}{ЦЕЛЬ ПРАКТИЧЕСКОЙ РАБОТЫ}

Prometheus --- это система мониторинга и сбора метрик,
разработанная для обеспечения надежности и производительности приложений.
Она использует модель временных рядов для хранения данных
и поддерживает язык запросов PromQL для анализа метрик.

Ключевые особенности:

\begin{itemize}
	\item Сбор метрик: Регулярный пулл данных с различных источников.
	\item PromQL: Язык для создания запросов и анализа метрик.
	\item Гибкость: Масштабируемая и интегрируемая
		с инструментами визуализации, такими как Grafana.
	\item Алертинг: Настройка оповещений через Alertmanager
		для мониторинга критических состояний.
\end{itemize}

Prometheus помогает эффективно отслеживать состояние систем
и принимать меры для их оптимизации и поддержания.

\textbf{Цель работы} --- познакомится с системой мониторинга
и сбора метрик Prometheus.

\clearpage

\section*{ВЫПОЛНЕНИЕ ПРАКТИЧЕСКОЙ РАБОТЫ}
\addcontentsline{toc}{section}{ВЫПОЛНЕНИЕ ПРАКТИЧЕСКОЙ РАБОТЫ}

\section{Описание тестового стенда,
	листинги Dockerfile, docker-compose.yml, prometheus.yml,
	исходного кода web-приложения}

\subsection{Web-приложение на FastAPI с метриками Prometheus}

Создал новый Python-проект и установил нужные библиотеки:
\texttt{fastapi}, \texttt{uvicorn}, \texttt{prometheus-fastapi-instrumentator}.

\lstinputlisting[
	language=Python, caption=\leftline{Код приложения}
	]{2/src/main.py}

Это простейшее FastAPI-приложение с одним маршрутом \texttt{"/"} 
и встроенным сбором метрик Prometheus через библиотеку
\texttt{prometheus-fastapi-instrumentator}.

\subsection{Dockerfile}

Создал файл Dockerfile для сборки контейнера с приложением.

\lstinputlisting[caption=\leftline{Dockerfile}]{2/src/Dockerfile}

\subsection{prometheus.yml}

Создал файл \texttt{prometheus.yml},
в котором будут указаны адреса FastAPI-приложения и cAdvisor.

\lstinputlisting[caption=\leftline{prometheus.yml}]{2/src/prometheus.yml}

В нем указаны:

\begin{itemize}
	\item \textbf{fastapi-app}: адрес FastAPI-приложения,
		которое собирает метрики на \texttt{/metrics};
	\item \textbf{cadvisor}: адрес cAdvisor,
		который собирает метрики о контейнерах;
	\item \textbf{scrape\_interval}: интервал сбора метрик 15 секунд.
\end{itemize}

\subsection{docker-compose.yml}

Для запуска 4 контейнеров (Web-приложение, cAdvisor, Prometheus и Grafana)
с помощью Docker Compose, нужно создать\texttt{docker-compose.yml} 
с конфигурацией для всех указанных сервисов.

\lstinputlisting[
	caption=\leftline{docker-compose.yml}
	]{2/src/docker-compose.yml}

Стоит отметить следующий особенности конфигурации:

\begin{itemize}
	\item К \textbf{cAdvisor (cadvisor)} монтируются системные директории
		для сбора данных о контейнерах и хосте;
	\item В \textbf{Prometheus (prometheus)}
		используется внешний файл конфигурации prometheus.yml,
		который был описан ранее;
	\item Для \textbf{Grafana (grafana)}
		создается постоянный том \texttt{grafana-storage}
		для хранения данных Grafana;
	\item Для всех контейнеров установлено ограничение на cpu и ram.
\end{itemize}

\subsection{Запуск контейнеров}

Для запуска контейнеров используется команда \verb|docker-compose up --build|.
После запуска будут доступны следующие сервисы:

\begin{itemize}
	\item FastAPI Web-приложение: \url{http://0.0.0.0:8000/hello};
	\item cAdvisor: \url{http://0.0.0.0:8080};
	\item Prometheus: \url{http://0.0.0.0:9090};
	\item Grafana: \url{http://0.0.0.0:3000}.
\end{itemize}


\section{Скриншоты настроенного дашборда,
	параметры проведения нагрузочного тестирования}

\subsection{Grafana Dashboard}

После того, как Grafana запущена, нужно добавить Prometheus
в качестве источника данных \rdref{fig:data-source:add}{fig:prometheus:create}.

\begin{image}
    \includegrph{Screenshot from 2024-09-18 18-01-27}
    \caption{Добавление Data Source}
    \label{fig:data-source:add}
\end{image}

\begin{image}
    \includegrph{Screenshot from 2024-09-18 19-00-02}
    \caption{Настройка Prometheus}
    \label{fig:prometheus:create}
\end{image}

Теперь можно создать дашборд с нужными графиками.\par
Первый график будет отображать количество HTTP-запросов,
поступивших на веб-приложение \rref{fig:graph:requests}.
Для него используется метрика \verb|http_requests_total|.

\begin{image}
    \includegrph{Screenshot from 2024-09-18 19-02-11}
    \caption{График с количеством HTTP-запросов}
    \label{fig:graph:requests}
\end{image}

Вторым создадим график показывающий утилизацию CPU контейнера
с веб-приложением, данные будут браться из cAdvisor \rref{fig:graph:cpu}.
Для него используется метрика \verb|container_cpu_usage_seconds_total| 
для контейнера с названием \verb|2_webapp_1|
и возвращающая среднюю загрузку CPU за последнию минуту.

\begin{image}
    \includegrph{Screenshot from 2024-09-18 19-03-49}
    \caption{График с утилизацией CPU контейнера с веб-приложением}
    \label{fig:graph:cpu}
\end{image}

Последний график будет отображать утилизацию оперативной памяти контейнера
с веб-приложением, данные также будут браться из cAdvisor \rref{fig:graph:ram}.
Для него используется метрика \verb|container_memory_usage_bytes| 
для контейнера с названием \verb|2_webapp_1|.
Также значение дважды делится на 1024, чтобы перевести байты в мегабайты.

\begin{image}
    \includegrph{Screenshot from 2024-09-18 19-23-36}
    \caption{График с утилизацией RAM контейнера с веб-приложением}
    \label{fig:graph:ram}
\end{image}

После добавления всех графиков,
сохраняем и дашборд с графиками теперь доступен для мониторинга запросов,
использования CPU и RAM контейнера с веб-приложением.

\begin{image}
    \includegrph{Screenshot from 2024-09-18 19-24-07}
    \caption{Полученый дашборд}
    \label{fig:dashboard}
\end{image}

\subsection{Docker stats}

Для независимого контроля утилизации ресурсов контейнеров,
включая контейнер с web-приложением, можно использовать команду
\texttt{docker stats} или \texttt{docker stats 2\_webapp\_1}
для конкретного контейнера.
Эта команда в реальном времени выводит статистику
по каждому запущенному контейнеру \rref{fig:docker:stats}.

\begin{image}
    \includegrph{Screenshot from 2024-09-18 19-07-19}
    \caption{Полученый дашборд}
    \label{fig:docker:stats}
\end{image}

Вывод команды \texttt{docker stats} включает:

\begin{itemize}
	\item CPU --- процент использования CPU контейнером.
	\item MEM USAGE / LIMIT --- объем используемой памяти и ее лимит.
	\item MEM --- процент использования памяти от доступного лимита.
	\item NET I/O --- количество данных, переданных и полученных по сети.
	\item BLOCK I/O --- количество операций ввода-вывода с диском.
	\item PIDS --- количество процессов в контейнере.
\end{itemize}

\section{Проведение нагрузочного тестирования, анализ полученных результатов}

Для проведения нагрузочного тестирования web-приложения
и определения максимума запросов который приложение может обработать до отказа
будет используется Apache JMeter.

\subsection{Установка Apache JMeter}

Для установки приложения Apache JMeter в Ubuntu воспользуемся командой
\texttt{sudo apt install jmeter} \rref{fig:jmeter:install}.

\begin{image}
    \includegrph{Screenshot from 2024-09-18 19-13-38}
    \caption{Установка Apache JMeter}
    \label{fig:jmeter:install}
\end{image}

\subsection{Настройка теста в JMeter}

В начале нужно создать группу потоков.
В нем настраиваем следующие параметры \rref{fig:jmeter:thread:group}:

\begin{itemize}
	\item \textbf{Number of Threads (Users)}:
		количество виртуальных пользователей;
	\item \textbf{Ramp-Up Period (in seconds)}: время,
		за которое все потоки запустятся;
	\item \textbf{Loop Count}: количество итераций для каждого потока.
\end{itemize}

\begin{image}
    \includegrph{Screenshot from 2024-09-18 19-15-28}
    \caption{Настройка группы потоков}
    \label{fig:jmeter:thread:group}
\end{image}

Далее нужно добавить HTTP-запроса. В нем указываем 
URL web-приложения, порт и путь к странице \rref{fig:jmeter:http}.

\begin{image}
    \includegrph{Screenshot from 2024-09-18 19-16-24}
    \caption{Настройка HTTP-запроса}
    \label{fig:jmeter:http}
\end{image}

Также нужены:

\begin{itemize}
	\item \texttt{HTTP Header Manager} для установки заголовков запроса.
		В нем добавим заголовок Content-Type: application/json;
	\item \texttt{Listener} для визуализации результатов тестирования.
\end{itemize}

\subsection{Анализ результатов}

Максимальное количество запросов, которое приложение смогло обработать
до отказа, составило \textbf{241815} запросов.
При этом нагрузка на CPU контейнера
с веб-приложением достигла значения \textbf{0.5 CPU},
что свидетельствует о полном использовании доступных ресурсов процессора.\par
После достижения данного уровня нагрузки приложение перестало отвечать
на запросы и метрики перестали передаваться в Prometheus.

\begin{image}
    \includegrph{Screenshot from 2024-09-18 19-42-39}
    \caption{Результаты тестирования}
    \label{fig:run}
\end{image}

\clearpage

\section*{\LARGE ВЫВОД}
\addcontentsline{toc}{section}{ВЫВОД}

В ходе практической работы было проведено нагрузочное тестирование
веб-приложения, реализованного на FastAPI,
с использованием Apache JMeter для измерения производительности
и Prometheus/Grafana для мониторинга метрик системы.\par
В ходе тестирования, было достигнут предельный уровнь нагрузки на приложение,
после которого оно перестало отвечать на запросы
и метрики перестали передаваться в Prometheus.

\clearpage

\section{Ответы на контрольные вопросы}

\subsection{Возможности системы мониторинга Prometheus}

Prometheus --- система мониторинга и сбора метрик,
предоставляющая следующие возможности:

\begin{itemize}
	\item Сбор метрик с различных источников
		(приложения, экспортеры, системы);
	\item Модель данных на основе временных рядов,
		где каждая метрика хранится с меткой времени;
	\item Высокая масштабируемость для мониторинга больших инфраструктур;
	\item Язык запросов PromQL для анализа метрик и генерации графиков;
	\item Алертинг с интеграцией с системами уведомлений;
	\item Pushgateway для обработки данных из краткоживущих приложений.
\end{itemize}

\subsection{Понятие скрейпинга и его применение в системах мониторинга}

Скрейпинг --- это процесс регулярного опроса
и сбора данных с источников метрик (например, веб-приложений или экспортеров).
В Prometheus скрейпинг используется для периодического извлечения метрик
с указанных целей (например, через \texttt{/metrics} у приложений).
Это позволяет получать данные для анализа производительности
и здоровья систем в реальном времени.

\subsection{Инструментирование приложений для Prometheus}

Инструментирование --- это процесс добавления в приложение кода
для сбора метрик, таких как количество запросов, время ответа,
использование ресурсов и т.д. Для этого используются библиотеки
(например, Prometheus FastAPI Instrumentator для FastAPI).
Инструментированные приложения публикуют метрики на доступном endpoint
(например, \texttt{/metrics}), откуда Prometheus их скрейпит.

\subsection{Использование готовых экспортеров приложений для Prometheus}

Готовые экспортеры --- это специальные программы,
которые преобразуют данные сторонних приложений и сервисов в формат,
понятный Prometheus.
Например, node\_exporter собирает системные метрики (CPU, память),
а mysql\_exporter --- метрики MySQL.
Использование экспортеров облегчает мониторинг сервисов
без необходимости глубокого инструментирования.

