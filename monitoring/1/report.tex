\section*{ЦЕЛЬ ПРАКТИЧЕСКОЙ РАБОТЫ}
\addcontentsline{toc}{section}{ЦЕЛЬ ПРАКТИЧЕСКОЙ РАБОТЫ}

Zabbix --- это универсальный инструмент мониторинга, способный
отслеживать динамику работы серверов и сетевого оборудования, быстро
реагировать на внештатные ситуации и предупреждать возможные проблемы
с нагрузкой.
Система мониторинга Zabbix может собирать статистику в
указанной рабочей среде и действовать в определенных случаях заданным
образом.
Zabbix бесплатен.
Zabbix написан и распространяется под лицензией
GPL General Public License версии 2.
Это означает, что его исходный код
свободно распространяется и доступен для неограниченного круга лиц

\textbf{Цель работы} --- познакомится с системой мониторинга
инфраструктуры Zabbix.

\clearpage

\section*{ВЫПОЛНЕНИЕ ПРАКТИЧЕСКОЙ РАБОТЫ}
\addcontentsline{toc}{section}{ВЫПОЛНЕНИЕ ПРАКТИЧЕСКОЙ РАБОТЫ}

\section{Описание используемого программного обеспечения
и сетевая топология его развертывания}

Для выполнения практической работы был создан \texttt{docker-compose} файл.
В нутри него определена сеть \texttt{zabbix-net} для видимости контейнеров
между собой.

\lstinputlisting{1/src/docker-compose.yml}

Zabbix server развернут в контейнере \texttt{zabbix-server-pgsql}
Также для него потребуется создать контейнер с базой данных, она
развернута в контейнере \texttt{postgres-server}.\par
Агент развернут в контейнере \texttt{zabbix-agent}.
Но для проведения проверочного тестирования образ был модифицирован утилитами
в \texttt{Dockerfile}.

\lstinputlisting{1/src/Dockerfile}

Также создан контейнер \texttt{zabbix-web-nginx-pgsql} для создания
удобного web-интерфейса к данным Zabbix server.

\section{Настройка системы мониторинга и описание собираемых параметров.
Описание параметров мониторинга и целей, на которое оно направлено.}

\subsection{Настройка хост-машины с Zabbix agent и создание дашборд}

Настроим соединение с агентом, выбрав адрес DNS и порт \rref{fig:host-conf}.

\begin{image}
    \includegrph{Screenshot from 2024-09-11 19-16-43}
    \caption{Соединение с агентом}
    \label{fig:host-conf}
\end{image}

На главном экране видим успешное соединение \rref{fig:main-dash-board}.

\begin{image}
    \includegrph{Screenshot from 2024-09-11 19-16-57}
    \caption{Успешное соединение}
    \label{fig:main-dash-board}
\end{image}

Добавим метрики чтения с диска и записи
(Рисунки \ref{fig:read-conf}-\ref{fig:write-conf}).

\begin{image}
    \includegrph{Screenshot from 2024-09-11 19-17-31}
    \caption{Чтение с диска}
    \label{fig:read-conf}
\end{image}

\begin{image}
    \includegrph{Screenshot from 2024-09-11 19-17-37}
    \caption{Запись на диск}
    \label{fig:write-conf}
\end{image}

Далее создадим дашборд с графиком cpu.
Выбираем вид статистики и ее название,
а также хост и его метрики \rref{fig:cpu}.

\begin{image}
    \includegrph{Screenshot from 2024-09-11 18-55-20}
    \caption{Создание графика cpu}
    \label{fig:cpu}
\end{image}

Повторяем аналогичные действия для всех нужных нам параметров.
И получаем итоговый дашборд.

\begin{image}
    \includegrph{Screenshot from 2024-09-11 19-17-57}
    \caption{Итоговый дашборд}
    \label{fig:all:dash-board}
\end{image}

\subsection{Статисткика на исследуемой машине}

Подключаемся к необходимому контейнеру с помощью команды
\texttt{docker exec -ti zabbix-agent bash} и запускаем команды.
Вместо iotop был запущен iostat.

\begin{image}
    \includegrph{Screenshot from 2024-09-16 15-21-37}
    \caption{htop}
    \label{fig:htop}
\end{image}

\begin{image}
    \includegrph{Screenshot from 2024-09-16 15-22-11}
    \caption{iostat}
    \label{fig:iostat}
\end{image}

\section{Симуляция нагрузки на систему.
Анализ полученных в ходе работы результатов.}

Для теста cpu была запущенна команда \rdref{fig:test:cpu}{fig:test:cpu:zab}.

\begin{lstlisting}[language=bash]
stress-ng --cpu 16 --cpu-method matrixprod --metrics --timeout 60
\end{lstlisting}

\begin{image}
    \includegrph{test_cpu_htop}
    \caption{cpu в контейнере}
    \label{fig:test:cpu}
\end{image}

\begin{image}
    \includegrph{test_cpu}
    \caption{cpu в zabbix}
    \label{fig:test:cpu:zab}
\end{image}


Для теста ram была запущенна команда \rdref{fig:test:ram}{fig:test:ram:zab}.

\begin{lstlisting}[language=bash]
stress-ng --sequential 0 --class memory --timeout 60s --metrics-brief
\end{lstlisting}

\begin{image}
    \includegrph{test_ram_htop}
    \caption{ram в контейнере}
    \label{fig:test:ram}
\end{image}

\begin{image}
    \includegrph{test_ram}
    \caption{ram в zabbix}
    \label{fig:test:ram:zab}
\end{image}

Для теста disk была запущенна команда \rdref{fig:test:io}{fig:test:io:zab}.

\begin{lstlisting}[language=bash]
stress-ng --sequential 0 --class io --timeout 60s --metrics-brief
\end{lstlisting}

\begin{image}
    \includegrph{test_io_htop}
    \caption{disk в контейнере}
    \label{fig:test:io}
\end{image}

\begin{image}
    \includegrph{test_io}
    \caption{disk в zabbix}
    \label{fig:test:io:zab}
\end{image}

\break

Исходя из получаем результатов запущенных тестов можно заметить, что
утилита htop предоставляет данные в реальном времени,
показывая список процессов, использование CPU, памяти, свопа
и другие системные метрики на уровне конкретной машины.
Она позволяет фильтровать процессы и управлять ими,
что делает её удобной для диагностики и решения проблем "здесь и сейчас".
Этот инструмент идеально подходит для ситуаций,
когда нужно быстро оценить состояние системы
и принять меры на локальном уровне.\par
В отличие от htop, Zabbix --- это полноценная система мониторинга,
которая собирает и отображает данные за длительные периоды.
Zabbix позволяет строить графики для различных метрик,
отслеживать тренды и анализировать изменения производительности систем и сетей.
Этот инструмент особенно полезен в крупных инфраструктурах,
где необходимо контролировать множество серверов и устройств,
а также настраивать оповещения на основе предустановленных триггеров.

\clearpage

\section*{\LARGE Вывод}
\addcontentsline{toc}{section}{Вывод}
В ходе работы были настроены контейнеры с помощью docker-compose,
на одном из которых установлен Zabbix server
и подключеннный к нему контейнер с базой данных PostgreSQL,
а на другом --- Zabbix agent. Также контейнер c nginx для web-интерфейса.
Агент передавал метрики на сервер,
на котором через web-интерфейс был создан дашборд с основными показателями:
утилизация CPU, RAM, диска, и время операций чтения/записи.\par
Для локального мониторинга на исследуемой машине использовались htop и iotop.
Проведены стресс-тесты на CPU, RAM и диск с помощью stress-ng.
Сравнение локальных метрик с данными Zabbix показало их соответствие
и корректную работу системы мониторинга.

\clearpage

\section{Ответы на контрольные вопросы}
\subsection{Мониторинг SLA}
SLA (Service Level Agreement) — это соглашение о предоставлении сервиса с определённым уровнем качества между провайдером и клиентом. Мониторинг SLA включает:
- Отслеживание фактических показателей сервиса в сравнении с SLO.
- Выявление нарушений SLA (например, время простоя, превышение допустимого времени отклика).
- Генерация отчетов о соответствии SLA для предоставления клиентам или руководству.

\subsection{Развертывание системы в случае с безагентским мониторингом}
Безагентский мониторинг в Zabbix означает, что для сбора данных не требуется установка агента на устройстве. Вместо этого используется:
- SNMP (Simple Network Management Protocol) — для мониторинга сетевых устройств.
- IPMI (Intelligent Platform Management Interface) — для мониторинга оборудования серверов.
- HTTP-запросы — для проверки веб-приложений и API.
- SSH и Telnet — для подключения к серверам и выполнения команд.

Безагентский мониторинг упрощает внедрение, особенно при работе с устройствами, на которые невозможно установить агенты.

\subsection{Понятие метрики (показателя) в инфраструктуре. SLI}
Метрика или Service Level Indicator (SLI) — это количественный показатель, который используется для измерения производительности и доступности ИТ-инфраструктуры. Это конкретная метрика, отражающая уровень качества услуги (например, время отклика, процент успешных запросов, время простоя). SLI представляет собой измерение фактического состояния системы, которое помогает отслеживать и оценивать её работу.

Примеры SLI:
- Время отклика на запросы.
- Доступность системы (например, процент времени без простоев).
- Пропускная способность сети.

\subsection{Понятие критерия при мониторинге инфраструктуры. SLO}
Service Level Objective (SLO) — это целевое значение метрики, к которому стремится система. Это соглашение или внутренняя цель между бизнесом и ИТ, которая определяет минимальный приемлемый уровень сервиса для определённого времени.

Примеры SLO:
- Доступность системы на уровне 99.9% в месяц.
- Среднее время отклика веб-сервера не более 200 мс.

SLO помогает устанавливать реалистичные ожидания и определять, когда система не справляется с поддержанием необходимого качества.
