\documentclass[14pt,a4paper,oneside]{extarticle}

	\usepackage{cmap} % for pdfLaTeX
	\usepackage[T1,T2A]{fontenc} % correct encoding for pdfLaTeX
	\usepackage[utf8]{inputenc} % correct encoding source file
	\usepackage[english,russian]{babel} % correct language

	% -------------------
	% TEXT SETTINGS
	% -------------------

	\usepackage{fontspec} % use standart fonts (only for xelatex!!!)
	\setmainfont{Times New Roman}
	\setsansfont{Arial}
	\setmonofont[Scale=0.6]{Courier New}

	\usepackage[none]{hyphenat} % no word breaks

	\usepackage{setspace}
	%\singlespacing % 1.0
	\onehalfspacing % 1.5

	% -------------------
	% PAGE SETTINGS
	% -------------------

	\usepackage[left=3cm, right=1.5cm, vmargin=2cm, headheight=2cm]{geometry}
	\linespread{1.5} % line spacing 1.5
	\usepackage{indentfirst} % indent first paragraph
	\setlength{\parindent}{1.25cm}
	\sloppy

	\frenchspacing % ???
	\pagestyle{plain}
	\usepackage{fancyhdr} % for headers and footers

	\clubpenalty=10000
	\widowpenalty=10000

	% ------------------
	% ATTACHMENTS SETTINGS
	% ------------------

	\usepackage[labelsep=endash]{caption}
	%\setlength{\abovecaptionskip}{3pt}
	%\setlength{\belowcaptionskip}{3pt}

	% for non-end-to-end numbering
	%\usepackage{chngcntr}
	%\counterwithin{figure}{section}
	%\counterwithin{table}{section}

	% for long table
	\usepackage{longtable}

	% for graphics
	\usepackage{graphicx}
	\newcommand{\rref}[1]{(Рисунок~\ref{#1})}
	\newcommand{\rdref}[2]{(Рисунки~\ref{#1}\,-\,\ref{#2})}
	\newcommand{\tref}[1]{(Таблица~\ref{#1})}
	\newcommand{\tdref}[2]{(Таблици~\ref{#1}\,-\,\ref{#2})}
	\newenvironment{image}{
		\begin{figure}[h!tp]
		\centering
	}{
		\end{figure}
	}
	\newcommand\includegrph[2][width=0.8\textwidth]{\includegraphics[#1]{#2}}

	\usepackage{amsmath} % more flexibility equations (use in title!!!)
	\usepackage{pdfpages} % include pdf


	% ------------------
	% SECTION SETTINGS
	% ------------------

	\usepackage{titlesec}

	% section size
	\titleformat{\section}[block]
		{\fontsize{16pt}{20pt}\bfseries}{\thesection}{1em}{}
	\titleformat{\subsection}
		{\fontsize{14pt}{18pt}\bfseries}{\thesubsection}{1em}{}
	\titleformat{\subsubsection}
		{\fontsize{14pt}{18pt}\bfseries}{\thesubsubsection}{1em}{}
	\titlespacing*{\section}{1.25cm}{0pt}{0pt}
	\titlespacing*{\subsection}{1.25cm}{0pt}{0pt}
	\titlespacing*{\subsubsection}{1.25cm}{0pt}{0pt}
	\titlespacing*{\paragraph}{1.25cm}{0pt}{0pt}

	% ------------------
	% NEW COMMAND
	% ------------------

	\providecommand{\No}{\textnumero}

	% ------------------
	% ------------------

    \usepackage[breakable]{tcolorbox}
    

    \usepackage{graphicx}
    \usepackage{caption}

    \usepackage{xcolor} % Allow colors to be defined
    \usepackage{amsmath} % Equations
    \usepackage{amssymb} % Equations
    \usepackage{geometry} % Used to adjust the document margins

    \usepackage{fancyvrb} % verbatim replacement that allows latex

    \makeatletter % fix for old versions of grffile with XeLaTeX
    \@ifpackagelater{grffile}{2019/11/01}
    {
      % Do nothing on new versions
    }
    {
      \def\Gread@@xetex#1{%
        \IfFileExists{"\Gin@base".bb}%
        {\Gread@eps{\Gin@base.bb}}%
        {\Gread@@xetex@aux#1}%
      }
    }
    \makeatother
    \usepackage[Export]{adjustbox} % Used to constrain images to a maximum size
    \adjustboxset{max size={0.9\linewidth}{0.9\paperheight}}

    % The hyperref package gives us a pdf with properly built
    % internal navigation ('pdf bookmarks' for the table of contents,
    % internal cross-reference links, web links for URLs, etc.)
    \usepackage{hyperref}
    % The default LaTeX title has an obnoxious amount of whitespace. By default,
    % titling removes some of it. It also provides customization options.
    \usepackage{titling}
    \usepackage{longtable} % longtable support required by pandoc >1.10
    \usepackage{booktabs}  % table support for pandoc > 1.12.2
    \usepackage{array}     % table support for pandoc >= 2.11.3
    \usepackage{calc}      % table minipage width calculation for pandoc >= 2.11.1
    \usepackage[inline]{enumitem} % IRkernel/repr support (it uses the enumerate* environment)
    \usepackage[normalem]{ulem} % ulem is needed to support strikethroughs (\sout)
                                % normalem makes italics be italics, not underlines
    \usepackage{mathrsfs}
    

    
    % Colors for the hyperref package
    \definecolor{urlcolor}{rgb}{0,.145,.698}
    \definecolor{linkcolor}{rgb}{.71,0.21,0.01}
    \definecolor{citecolor}{rgb}{.12,.54,.11}

    % ANSI colors
    \definecolor{ansi-black}{HTML}{3E424D}
    \definecolor{ansi-black-intense}{HTML}{282C36}
    \definecolor{ansi-red}{HTML}{E75C58}
    \definecolor{ansi-red-intense}{HTML}{B22B31}
    \definecolor{ansi-green}{HTML}{00A250}
    \definecolor{ansi-green-intense}{HTML}{007427}
    \definecolor{ansi-yellow}{HTML}{DDB62B}
    \definecolor{ansi-yellow-intense}{HTML}{B27D12}
    \definecolor{ansi-blue}{HTML}{208FFB}
    \definecolor{ansi-blue-intense}{HTML}{0065CA}
    \definecolor{ansi-magenta}{HTML}{D160C4}
    \definecolor{ansi-magenta-intense}{HTML}{A03196}
    \definecolor{ansi-cyan}{HTML}{60C6C8}
    \definecolor{ansi-cyan-intense}{HTML}{258F8F}
    \definecolor{ansi-white}{HTML}{C5C1B4}
    \definecolor{ansi-white-intense}{HTML}{A1A6B2}
    \definecolor{ansi-default-inverse-fg}{HTML}{FFFFFF}
    \definecolor{ansi-default-inverse-bg}{HTML}{000000}

    % common color for the border for error outputs.
    \definecolor{outerrorbackground}{HTML}{FFDFDF}

    % commands and environments needed by pandoc snippets
    % extracted from the output of `pandoc -s`
    \providecommand{\tightlist}{%
      \setlength{\itemsep}{0pt}\setlength{\parskip}{0pt}}
    \DefineVerbatimEnvironment{Highlighting}{Verbatim}{commandchars=\\\{\}}
    % Add ',fontsize=\small' for more characters per line
    \newenvironment{Shaded}{}{}
    \newcommand{\KeywordTok}[1]{\textcolor[rgb]{0.00,0.44,0.13}{\textbf{{#1}}}}
    \newcommand{\DataTypeTok}[1]{\textcolor[rgb]{0.56,0.13,0.00}{{#1}}}
    \newcommand{\DecValTok}[1]{\textcolor[rgb]{0.25,0.63,0.44}{{#1}}}
    \newcommand{\BaseNTok}[1]{\textcolor[rgb]{0.25,0.63,0.44}{{#1}}}
    \newcommand{\FloatTok}[1]{\textcolor[rgb]{0.25,0.63,0.44}{{#1}}}
    \newcommand{\CharTok}[1]{\textcolor[rgb]{0.25,0.44,0.63}{{#1}}}
    \newcommand{\StringTok}[1]{\textcolor[rgb]{0.25,0.44,0.63}{{#1}}}
    \newcommand{\CommentTok}[1]{\textcolor[rgb]{0.38,0.63,0.69}{\textit{{#1}}}}
    \newcommand{\OtherTok}[1]{\textcolor[rgb]{0.00,0.44,0.13}{{#1}}}
    \newcommand{\AlertTok}[1]{\textcolor[rgb]{1.00,0.00,0.00}{\textbf{{#1}}}}
    \newcommand{\FunctionTok}[1]{\textcolor[rgb]{0.02,0.16,0.49}{{#1}}}
    \newcommand{\RegionMarkerTok}[1]{{#1}}
    \newcommand{\ErrorTok}[1]{\textcolor[rgb]{1.00,0.00,0.00}{\textbf{{#1}}}}
    \newcommand{\NormalTok}[1]{{#1}}

    % Additional commands for more recent versions of Pandoc
    \newcommand{\ConstantTok}[1]{\textcolor[rgb]{0.53,0.00,0.00}{{#1}}}
    \newcommand{\SpecialCharTok}[1]{\textcolor[rgb]{0.25,0.44,0.63}{{#1}}}
    \newcommand{\VerbatimStringTok}[1]{\textcolor[rgb]{0.25,0.44,0.63}{{#1}}}
    \newcommand{\SpecialStringTok}[1]{\textcolor[rgb]{0.73,0.40,0.53}{{#1}}}
    \newcommand{\ImportTok}[1]{{#1}}
    \newcommand{\DocumentationTok}[1]{\textcolor[rgb]{0.73,0.13,0.13}{\textit{{#1}}}}
    \newcommand{\AnnotationTok}[1]{\textcolor[rgb]{0.38,0.63,0.69}{\textbf{\textit{{#1}}}}}
    \newcommand{\CommentVarTok}[1]{\textcolor[rgb]{0.38,0.63,0.69}{\textbf{\textit{{#1}}}}}
    \newcommand{\VariableTok}[1]{\textcolor[rgb]{0.10,0.09,0.49}{{#1}}}
    \newcommand{\ControlFlowTok}[1]{\textcolor[rgb]{0.00,0.44,0.13}{\textbf{{#1}}}}
    \newcommand{\OperatorTok}[1]{\textcolor[rgb]{0.40,0.40,0.40}{{#1}}}
    \newcommand{\BuiltInTok}[1]{{#1}}
    \newcommand{\ExtensionTok}[1]{{#1}}
    \newcommand{\PreprocessorTok}[1]{\textcolor[rgb]{0.74,0.48,0.00}{{#1}}}
    \newcommand{\AttributeTok}[1]{\textcolor[rgb]{0.49,0.56,0.16}{{#1}}}
    \newcommand{\InformationTok}[1]{\textcolor[rgb]{0.38,0.63,0.69}{\textbf{\textit{{#1}}}}}
    \newcommand{\WarningTok}[1]{\textcolor[rgb]{0.38,0.63,0.69}{\textbf{\textit{{#1}}}}}

    
    
    
    
	% Pygments definitions
	\makeatletter
	\def\PY@reset{\let\PY@it=\relax \let\PY@bf=\relax%
		\let\PY@ul=\relax \let\PY@tc=\relax%
		\let\PY@bc=\relax \let\PY@ff=\relax}
	\def\PY@tok#1{\csname PY@tok@#1\endcsname}
	\def\PY@toks#1+{\ifx\relax#1\empty\else%
		\PY@tok{#1}\expandafter\PY@toks\fi}
	\def\PY@do#1{\PY@bc{\PY@tc{\PY@ul{%
		\PY@it{\PY@bf{\PY@ff{#1}}}}}}}
	\def\PY#1#2{\PY@reset\PY@toks#1+\relax+\PY@do{#2}}

	\@namedef{PY@tok@w}{\def\PY@tc##1{\textcolor[rgb]{0.73,0.73,0.73}{##1}}}
	\@namedef{PY@tok@c}{\let\PY@it=\textit\def\PY@tc##1{\textcolor[rgb]{0.24,0.48,0.48}{##1}}}
	\@namedef{PY@tok@cp}{\def\PY@tc##1{\textcolor[rgb]{0.61,0.40,0.00}{##1}}}
	\@namedef{PY@tok@k}{\let\PY@bf=\textbf\def\PY@tc##1{\textcolor[rgb]{0.00,0.50,0.00}{##1}}}
	\@namedef{PY@tok@kp}{\def\PY@tc##1{\textcolor[rgb]{0.00,0.50,0.00}{##1}}}
	\@namedef{PY@tok@kt}{\def\PY@tc##1{\textcolor[rgb]{0.69,0.00,0.25}{##1}}}
	\@namedef{PY@tok@o}{\def\PY@tc##1{\textcolor[rgb]{0.40,0.40,0.40}{##1}}}
	\@namedef{PY@tok@ow}{\let\PY@bf=\textbf\def\PY@tc##1{\textcolor[rgb]{0.67,0.13,1.00}{##1}}}
	\@namedef{PY@tok@nb}{\def\PY@tc##1{\textcolor[rgb]{0.00,0.50,0.00}{##1}}}
	\@namedef{PY@tok@nf}{\def\PY@tc##1{\textcolor[rgb]{0.00,0.00,1.00}{##1}}}
	\@namedef{PY@tok@nc}{\let\PY@bf=\textbf\def\PY@tc##1{\textcolor[rgb]{0.00,0.00,1.00}{##1}}}
	\@namedef{PY@tok@nn}{\let\PY@bf=\textbf\def\PY@tc##1{\textcolor[rgb]{0.00,0.00,1.00}{##1}}}
	\@namedef{PY@tok@ne}{\let\PY@bf=\textbf\def\PY@tc##1{\textcolor[rgb]{0.80,0.25,0.22}{##1}}}
	\@namedef{PY@tok@nv}{\def\PY@tc##1{\textcolor[rgb]{0.10,0.09,0.49}{##1}}}
	\@namedef{PY@tok@no}{\def\PY@tc##1{\textcolor[rgb]{0.53,0.00,0.00}{##1}}}
	\@namedef{PY@tok@nl}{\def\PY@tc##1{\textcolor[rgb]{0.46,0.46,0.00}{##1}}}
	\@namedef{PY@tok@ni}{\let\PY@bf=\textbf\def\PY@tc##1{\textcolor[rgb]{0.44,0.44,0.44}{##1}}}
	\@namedef{PY@tok@na}{\def\PY@tc##1{\textcolor[rgb]{0.41,0.47,0.13}{##1}}}
	\@namedef{PY@tok@nt}{\let\PY@bf=\textbf\def\PY@tc##1{\textcolor[rgb]{0.00,0.50,0.00}{##1}}}
	\@namedef{PY@tok@nd}{\def\PY@tc##1{\textcolor[rgb]{0.67,0.13,1.00}{##1}}}
	\@namedef{PY@tok@s}{\def\PY@tc##1{\textcolor[rgb]{0.73,0.13,0.13}{##1}}}
	\@namedef{PY@tok@sd}{\let\PY@it=\textit\def\PY@tc##1{\textcolor[rgb]{0.73,0.13,0.13}{##1}}}
	\@namedef{PY@tok@si}{\let\PY@bf=\textbf\def\PY@tc##1{\textcolor[rgb]{0.64,0.35,0.47}{##1}}}
	\@namedef{PY@tok@se}{\let\PY@bf=\textbf\def\PY@tc##1{\textcolor[rgb]{0.67,0.36,0.12}{##1}}}
	\@namedef{PY@tok@sr}{\def\PY@tc##1{\textcolor[rgb]{0.64,0.35,0.47}{##1}}}
	\@namedef{PY@tok@ss}{\def\PY@tc##1{\textcolor[rgb]{0.10,0.09,0.49}{##1}}}
	\@namedef{PY@tok@sx}{\def\PY@tc##1{\textcolor[rgb]{0.00,0.50,0.00}{##1}}}
	\@namedef{PY@tok@m}{\def\PY@tc##1{\textcolor[rgb]{0.40,0.40,0.40}{##1}}}
	\@namedef{PY@tok@gh}{\let\PY@bf=\textbf\def\PY@tc##1{\textcolor[rgb]{0.00,0.00,0.50}{##1}}}
	\@namedef{PY@tok@gu}{\let\PY@bf=\textbf\def\PY@tc##1{\textcolor[rgb]{0.50,0.00,0.50}{##1}}}
	\@namedef{PY@tok@gd}{\def\PY@tc##1{\textcolor[rgb]{0.63,0.00,0.00}{##1}}}
	\@namedef{PY@tok@gi}{\def\PY@tc##1{\textcolor[rgb]{0.00,0.52,0.00}{##1}}}
	\@namedef{PY@tok@gr}{\def\PY@tc##1{\textcolor[rgb]{0.89,0.00,0.00}{##1}}}
	\@namedef{PY@tok@ge}{\let\PY@it=\textit}
	\@namedef{PY@tok@gs}{\let\PY@bf=\textbf}
	\@namedef{PY@tok@gp}{\let\PY@bf=\textbf\def\PY@tc##1{\textcolor[rgb]{0.00,0.00,0.50}{##1}}}
	\@namedef{PY@tok@go}{\def\PY@tc##1{\textcolor[rgb]{0.44,0.44,0.44}{##1}}}
	\@namedef{PY@tok@gt}{\def\PY@tc##1{\textcolor[rgb]{0.00,0.27,0.87}{##1}}}
	\@namedef{PY@tok@err}{\def\PY@bc##1{{\setlength{\fboxsep}{\string -\fboxrule}\fcolorbox[rgb]{1.00,0.00,0.00}{1,1,1}{\strut ##1}}}}
	\@namedef{PY@tok@kc}{\let\PY@bf=\textbf\def\PY@tc##1{\textcolor[rgb]{0.00,0.50,0.00}{##1}}}
	\@namedef{PY@tok@kd}{\let\PY@bf=\textbf\def\PY@tc##1{\textcolor[rgb]{0.00,0.50,0.00}{##1}}}
	\@namedef{PY@tok@kn}{\let\PY@bf=\textbf\def\PY@tc##1{\textcolor[rgb]{0.00,0.50,0.00}{##1}}}
	\@namedef{PY@tok@kr}{\let\PY@bf=\textbf\def\PY@tc##1{\textcolor[rgb]{0.00,0.50,0.00}{##1}}}
	\@namedef{PY@tok@bp}{\def\PY@tc##1{\textcolor[rgb]{0.00,0.50,0.00}{##1}}}
	\@namedef{PY@tok@fm}{\def\PY@tc##1{\textcolor[rgb]{0.00,0.00,1.00}{##1}}}
	\@namedef{PY@tok@vc}{\def\PY@tc##1{\textcolor[rgb]{0.10,0.09,0.49}{##1}}}
	\@namedef{PY@tok@vg}{\def\PY@tc##1{\textcolor[rgb]{0.10,0.09,0.49}{##1}}}
	\@namedef{PY@tok@vi}{\def\PY@tc##1{\textcolor[rgb]{0.10,0.09,0.49}{##1}}}
	\@namedef{PY@tok@vm}{\def\PY@tc##1{\textcolor[rgb]{0.10,0.09,0.49}{##1}}}
	\@namedef{PY@tok@sa}{\def\PY@tc##1{\textcolor[rgb]{0.73,0.13,0.13}{##1}}}
	\@namedef{PY@tok@sb}{\def\PY@tc##1{\textcolor[rgb]{0.73,0.13,0.13}{##1}}}
	\@namedef{PY@tok@sc}{\def\PY@tc##1{\textcolor[rgb]{0.73,0.13,0.13}{##1}}}
	\@namedef{PY@tok@dl}{\def\PY@tc##1{\textcolor[rgb]{0.73,0.13,0.13}{##1}}}
	\@namedef{PY@tok@s2}{\def\PY@tc##1{\textcolor[rgb]{0.73,0.13,0.13}{##1}}}
	\@namedef{PY@tok@sh}{\def\PY@tc##1{\textcolor[rgb]{0.73,0.13,0.13}{##1}}}
	\@namedef{PY@tok@s1}{\def\PY@tc##1{\textcolor[rgb]{0.73,0.13,0.13}{##1}}}
	\@namedef{PY@tok@mb}{\def\PY@tc##1{\textcolor[rgb]{0.40,0.40,0.40}{##1}}}
	\@namedef{PY@tok@mf}{\def\PY@tc##1{\textcolor[rgb]{0.40,0.40,0.40}{##1}}}
	\@namedef{PY@tok@mh}{\def\PY@tc##1{\textcolor[rgb]{0.40,0.40,0.40}{##1}}}
	\@namedef{PY@tok@mi}{\def\PY@tc##1{\textcolor[rgb]{0.40,0.40,0.40}{##1}}}
	\@namedef{PY@tok@il}{\def\PY@tc##1{\textcolor[rgb]{0.40,0.40,0.40}{##1}}}
	\@namedef{PY@tok@mo}{\def\PY@tc##1{\textcolor[rgb]{0.40,0.40,0.40}{##1}}}
	\@namedef{PY@tok@ch}{\let\PY@it=\textit\def\PY@tc##1{\textcolor[rgb]{0.24,0.48,0.48}{##1}}}
	\@namedef{PY@tok@cm}{\let\PY@it=\textit\def\PY@tc##1{\textcolor[rgb]{0.24,0.48,0.48}{##1}}}
	\@namedef{PY@tok@cpf}{\let\PY@it=\textit\def\PY@tc##1{\textcolor[rgb]{0.24,0.48,0.48}{##1}}}
	\@namedef{PY@tok@c1}{\let\PY@it=\textit\def\PY@tc##1{\textcolor[rgb]{0.24,0.48,0.48}{##1}}}
	\@namedef{PY@tok@cs}{\let\PY@it=\textit\def\PY@tc##1{\textcolor[rgb]{0.24,0.48,0.48}{##1}}}

	\def\PYZbs{\char`\\}
	\def\PYZus{\char`\_}
	\def\PYZob{\char`\{}
	\def\PYZcb{\char`\}}
	\def\PYZca{\char`\^}
	\def\PYZam{\char`\&}
	\def\PYZlt{\char`\<}
	\def\PYZgt{\char`\>}
	\def\PYZsh{\char`\#}
	\def\PYZpc{\char`\%}
	\def\PYZdl{\char`\$}
	\def\PYZhy{\char`\-}
	\def\PYZsq{\char`\'}
	\def\PYZdq{\char`\"}
	\def\PYZti{\char`\~}
	% for compatibility with earlier versions
	\def\PYZat{@}
	\def\PYZlb{[}
	\def\PYZrb{]}
	\makeatother


    % For linebreaks inside Verbatim environment from package fancyvrb.
    \makeatletter
        \newbox\Wrappedcontinuationbox
        \newbox\Wrappedvisiblespacebox
        \newcommand*\Wrappedvisiblespace {\textcolor{red}{\textvisiblespace}}
        \newcommand*\Wrappedcontinuationsymbol {\textcolor{red}{\llap{\tiny$\m@th\hookrightarrow$}}}
        \newcommand*\Wrappedcontinuationindent {3ex }
        \newcommand*\Wrappedafterbreak {\kern\Wrappedcontinuationindent\copy\Wrappedcontinuationbox}
        % Take advantage of the already applied Pygments mark-up to insert
        % potential linebreaks for TeX processing.
        %        {, <, #, %, $, ' and ": go to next line.
        %        _, }, ^, &, >, - and ~: stay at end of broken line.
        % Use of \textquotesingle for straight quote.
        \newcommand*\Wrappedbreaksatspecials {%
            \def\PYGZus{\discretionary{\char`\_}{\Wrappedafterbreak}{\char`\_}}%
            \def\PYGZob{\discretionary{}{\Wrappedafterbreak\char`\{}{\char`\{}}%
            \def\PYGZcb{\discretionary{\char`\}}{\Wrappedafterbreak}{\char`\}}}%
            \def\PYGZca{\discretionary{\char`\^}{\Wrappedafterbreak}{\char`\^}}%
            \def\PYGZam{\discretionary{\char`\&}{\Wrappedafterbreak}{\char`\&}}%
            \def\PYGZlt{\discretionary{}{\Wrappedafterbreak\char`\<}{\char`\<}}%
            \def\PYGZgt{\discretionary{\char`\>}{\Wrappedafterbreak}{\char`\>}}%
            \def\PYGZsh{\discretionary{}{\Wrappedafterbreak\char`\#}{\char`\#}}%
            \def\PYGZpc{\discretionary{}{\Wrappedafterbreak\char`\%}{\char`\%}}%
            \def\PYGZdl{\discretionary{}{\Wrappedafterbreak\char`\$}{\char`\$}}%
            \def\PYGZhy{\discretionary{\char`\-}{\Wrappedafterbreak}{\char`\-}}%
            \def\PYGZsq{\discretionary{}{\Wrappedafterbreak\textquotesingle}{\textquotesingle}}%
            \def\PYGZdq{\discretionary{}{\Wrappedafterbreak\char`\"}{\char`\"}}%
            \def\PYGZti{\discretionary{\char`\~}{\Wrappedafterbreak}{\char`\~}}%
        }
        % Some characters . , ; ? ! / are not pygmentized.
        % This macro makes them "active" and they will insert potential linebreaks
        \newcommand*\Wrappedbreaksatpunct {%
            \lccode`\~`\.\lowercase{\def~}{\discretionary{\hbox{\char`\.}}{\Wrappedafterbreak}{\hbox{\char`\.}}}%
            \lccode`\~`\,\lowercase{\def~}{\discretionary{\hbox{\char`\,}}{\Wrappedafterbreak}{\hbox{\char`\,}}}%
            \lccode`\~`\;\lowercase{\def~}{\discretionary{\hbox{\char`\;}}{\Wrappedafterbreak}{\hbox{\char`\;}}}%
            \lccode`\~`\:\lowercase{\def~}{\discretionary{\hbox{\char`\:}}{\Wrappedafterbreak}{\hbox{\char`\:}}}%
            \lccode`\~`\?\lowercase{\def~}{\discretionary{\hbox{\char`\?}}{\Wrappedafterbreak}{\hbox{\char`\?}}}%
            \lccode`\~`\!\lowercase{\def~}{\discretionary{\hbox{\char`\!}}{\Wrappedafterbreak}{\hbox{\char`\!}}}%
            \lccode`\~`\/\lowercase{\def~}{\discretionary{\hbox{\char`\/}}{\Wrappedafterbreak}{\hbox{\char`\/}}}%
            \catcode`\.\active
            \catcode`\,\active
            \catcode`\;\active
            \catcode`\:\active
            \catcode`\?\active
            \catcode`\!\active
            \catcode`\/\active
            \lccode`\~`\~
        }
    \makeatother

    \let\OriginalVerbatim=\Verbatim
    \makeatletter
    \renewcommand{\Verbatim}[1][1]{%
        %\parskip\z@skip
        \sbox\Wrappedcontinuationbox {\Wrappedcontinuationsymbol}%
        \sbox\Wrappedvisiblespacebox {\FV@SetupFont\Wrappedvisiblespace}%
        \def\FancyVerbFormatLine ##1{\hsize\linewidth
            \vtop{\raggedright\hyphenpenalty\z@\exhyphenpenalty\z@
                \doublehyphendemerits\z@\finalhyphendemerits\z@
                \strut ##1\strut}%
        }%
        % If the linebreak is at a space, the latter will be displayed as visible
        % space at end of first line, and a continuation symbol starts next line.
        % Stretch/shrink are however usually zero for typewriter font.
        \def\FV@Space {%
            \nobreak\hskip\z@ plus\fontdimen3\font minus\fontdimen4\font
            \discretionary{\copy\Wrappedvisiblespacebox}{\Wrappedafterbreak}
            {\kern\fontdimen2\font}%
        }%

        % Allow breaks at special characters using \PYG... macros.
        \Wrappedbreaksatspecials
        % Breaks at punctuation characters . , ; ? ! and / need catcode=\active
        \OriginalVerbatim[#1,codes*=\Wrappedbreaksatpunct]%
    }
    \makeatother

    % Exact colors from NB
    \definecolor{incolor}{HTML}{303F9F}
    \definecolor{outcolor}{HTML}{D84315}
    \definecolor{cellborder}{HTML}{CFCFCF}
    \definecolor{cellbackground}{HTML}{F7F7F7}

    % prompt
    \makeatletter
    \newcommand{\boxspacing}{\kern\kvtcb@left@rule\kern\kvtcb@boxsep}
    \makeatother
    \newcommand{\prompt}[4]{
        {\ttfamily\llap{{\color{#2}[#3]:\hspace{3pt}#4}}\vspace{-\baselineskip}}
    }
    

    
    % Prevent overflowing lines due to hard-to-break entities
    \sloppy
    % Setup hyperref package
    \hypersetup{
      breaklinks=true,  % so long urls are correctly broken across lines
      colorlinks=true,
      urlcolor=urlcolor,
      linkcolor=linkcolor,
      citecolor=citecolor,
      }
    

\begin{document}
    
\singlespacing % 1.0
\begin{titlepage}
	\thispagestyle{fancy}
	\renewcommand{\headrulewidth}{0pt}
	\cfoot{Москва 2024}

	\centering
	\includegraphics[scale=0.75]{./res/logo2} \break % вставка логотипа
	{\footnotesize МИНИСТЕРСТВО НАУКИ
		И ВЫСШЕГО ОБРАЗОВАНИЯ РОССИЙСКОЙ ФЕДЕРАЦИИ}\\
	Федеральное государственное бюджетное образовательное учреждение 
		высшего образования\\
	\textbf{<<МИРЭА --- Российский технологический университет>>}\\
	\vfill
	\textbf{\large РТУ МИРЭА}\\
	\bigskip \hrule \smallskip \hrule \smallskip
	\vfill
	Институт информационных технологий (ИТ)\\
	Математического обеспечения
		и стандартизации информационных технологий (МОСИТ)\\
	\vfill
	\textbf{ОТЧЕТ ПО ПРАКТИЧЕСКОЙ РАБОТЕ \No\,1}\\
	\textbf{по дисциплине}\\
	\textbf{<<Разработка кроссплатформенных мобильных приложений>>}\\
	\vfill
	\vfill
	\vfill
	\vfill
	\begin{tabular}{p{0.7\textwidth}p{0.2\textwidth}}
		Выполнил студент группы ИКБО-06-21 & \rightline{Бондарь А.Р.} \\
		Принял старший преподаватель & \rightline{Шешуков Л.С.} \\
	\end{tabular}
	\vfill
	\vfill
	\vfill
	\vfill
\end{titlepage}
\onehalfspacing % 1.5
\setcounter{page}{2}
\clearpage


    
    
    \hypertarget{ux43eux431ux449ux438ux435-ux437ux430ux432ux438ux441ux438ux43cux43eux441ux442ux438}{%
\section{Общие
зависимости}\label{ux43eux431ux449ux438ux435-ux437ux430ux432ux438ux441ux438ux43cux43eux441ux442ux438}}

    \begin{tcolorbox}[breakable, size=fbox, boxrule=1pt, pad at break*=1mm,colback=cellbackground, colframe=cellborder]
\prompt{In}{incolor}{70}{\boxspacing}
\begin{Verbatim}[commandchars=\\\{\}]
\PY{k+kn}{import} \PY{n+nn}{pandas} \PY{k}{as} \PY{n+nn}{pd}
\PY{k+kn}{import} \PY{n+nn}{numpy} \PY{k}{as} \PY{n+nn}{np}
\PY{k+kn}{import} \PY{n+nn}{time}
\PY{k+kn}{from} \PY{n+nn}{sklearn}\PY{n+nn}{.}\PY{n+nn}{metrics} \PY{k+kn}{import} \PY{n}{accuracy\PYZus{}score}\PY{p}{,} \PY{n}{classification\PYZus{}report}
\end{Verbatim}
\end{tcolorbox}

    \hypertarget{ux43dux430ux439ux442ux438-ux434ux430ux43dux43dux44bux435-ux434ux43bux44f-ux437ux430ux434ux430ux447ux438-ux43aux43bux430ux441ux441ux438ux444ux438ux43aux430ux446ux438ux438-ux438ux43bux438-ux434ux43bux44f-ux437ux430ux434ux430ux447ux438-ux440ux435ux433ux440ux435ux441ux441ux438ux438}{%
\section{Найти данные для задачи классификации или для задачи
регрессии}\label{ux43dux430ux439ux442ux438-ux434ux430ux43dux43dux44bux435-ux434ux43bux44f-ux437ux430ux434ux430ux447ux438-ux43aux43bux430ux441ux441ux438ux444ux438ux43aux430ux446ux438ux438-ux438ux43bux438-ux434ux43bux44f-ux437ux430ux434ux430ux447ux438-ux440ux435ux433ux440ux435ux441ux441ux438ux438}}

    \begin{tcolorbox}[breakable, size=fbox, boxrule=1pt, pad at break*=1mm,colback=cellbackground, colframe=cellborder]
\prompt{In}{incolor}{71}{\boxspacing}
\begin{Verbatim}[commandchars=\\\{\}]
\PY{n}{df} \PY{o}{=} \PY{n}{pd}\PY{o}{.}\PY{n}{read\PYZus{}csv}\PY{p}{(}\PY{l+s+s2}{\PYZdq{}}\PY{l+s+s2}{/content/fish\PYZus{}data.csv}\PY{l+s+s2}{\PYZdq{}}\PY{p}{)}
\PY{n}{df}\PY{o}{.}\PY{n}{head}\PY{p}{(}\PY{p}{)}
\end{Verbatim}
\end{tcolorbox}

            \begin{tcolorbox}[breakable, size=fbox, boxrule=.5pt, pad at break*=1mm, opacityfill=0]
\prompt{Out}{outcolor}{71}{\boxspacing}
\begin{Verbatim}[commandchars=\\\{\}]
              species  length  weight  w\_l\_ratio
0  Anabas testudineus   10.66    3.45       0.32
1  Anabas testudineus    6.91    3.27       0.47
2  Anabas testudineus    8.38    3.46       0.41
3  Anabas testudineus    7.57    3.36       0.44
4  Anabas testudineus   10.83    3.38       0.31
\end{Verbatim}
\end{tcolorbox}
        
    \hypertarget{ux443ux434ux430ux43bux438ux43c-ux434ux443ux431ux43bux438ux43aux430ux442ux44b-ux438ux437-ux434ux430ux442ux430ux444ux440ux435ux439ux43cux430}{%
\subsection{Удалим дубликаты из
датафрейма}\label{ux443ux434ux430ux43bux438ux43c-ux434ux443ux431ux43bux438ux43aux430ux442ux44b-ux438ux437-ux434ux430ux442ux430ux444ux440ux435ux439ux43cux430}}

    \begin{tcolorbox}[breakable, size=fbox, boxrule=1pt, pad at break*=1mm,colback=cellbackground, colframe=cellborder]
\prompt{In}{incolor}{72}{\boxspacing}
\begin{Verbatim}[commandchars=\\\{\}]
\PY{n}{df}\PY{o}{.}\PY{n}{duplicated}\PY{p}{(}\PY{p}{)}\PY{o}{.}\PY{n}{sum}\PY{p}{(}\PY{p}{)}
\end{Verbatim}
\end{tcolorbox}

            \begin{tcolorbox}[breakable, size=fbox, boxrule=.5pt, pad at break*=1mm, opacityfill=0]
\prompt{Out}{outcolor}{72}{\boxspacing}
\begin{Verbatim}[commandchars=\\\{\}]
109
\end{Verbatim}
\end{tcolorbox}
        
    \begin{tcolorbox}[breakable, size=fbox, boxrule=1pt, pad at break*=1mm,colback=cellbackground, colframe=cellborder]
\prompt{In}{incolor}{73}{\boxspacing}
\begin{Verbatim}[commandchars=\\\{\}]
\PY{n}{df}\PY{o}{.}\PY{n}{drop\PYZus{}duplicates}\PY{p}{(}\PY{n}{inplace}\PY{o}{=}\PY{k+kc}{True}\PY{p}{)}
\end{Verbatim}
\end{tcolorbox}

    \hypertarget{ux443ux434ux430ux43bux438ux43c-ux441ux442ux43eux43bux431ux435ux446-ux432ux438ux434}{%
\subsection{Удалим столбец
Вид}\label{ux443ux434ux430ux43bux438ux43c-ux441ux442ux43eux43bux431ux435ux446-ux432ux438ux434}}

    \begin{tcolorbox}[breakable, size=fbox, boxrule=1pt, pad at break*=1mm,colback=cellbackground, colframe=cellborder]
\prompt{In}{incolor}{74}{\boxspacing}
\begin{Verbatim}[commandchars=\\\{\}]
\PY{n}{X} \PY{o}{=} \PY{n}{df}\PY{o}{.}\PY{n}{drop}\PY{p}{(}\PY{l+s+s1}{\PYZsq{}}\PY{l+s+s1}{species}\PY{l+s+s1}{\PYZsq{}}\PY{p}{,} \PY{n}{axis}\PY{o}{=}\PY{l+m+mi}{1}\PY{p}{)}
\PY{n}{y} \PY{o}{=} \PY{n}{df}\PY{o}{.}\PY{n}{species}
\end{Verbatim}
\end{tcolorbox}

    \hypertarget{ux43dux43eux440ux43cux430ux43bux438ux437ux430ux446ux438ux44f}{%
\subsection{Нормализация}\label{ux43dux43eux440ux43cux430ux43bux438ux437ux430ux446ux438ux44f}}

    \begin{tcolorbox}[breakable, size=fbox, boxrule=1pt, pad at break*=1mm,colback=cellbackground, colframe=cellborder]
\prompt{In}{incolor}{75}{\boxspacing}
\begin{Verbatim}[commandchars=\\\{\}]
\PY{k+kn}{from} \PY{n+nn}{sklearn}\PY{n+nn}{.}\PY{n+nn}{preprocessing} \PY{k+kn}{import} \PY{n}{MinMaxScaler}
\end{Verbatim}
\end{tcolorbox}

    \begin{tcolorbox}[breakable, size=fbox, boxrule=1pt, pad at break*=1mm,colback=cellbackground, colframe=cellborder]
\prompt{In}{incolor}{76}{\boxspacing}
\begin{Verbatim}[commandchars=\\\{\}]
\PY{n}{scaler} \PY{o}{=} \PY{n}{MinMaxScaler}\PY{p}{(}\PY{p}{)}
\PY{n}{X\PYZus{}scaled} \PY{o}{=} \PY{n}{scaler}\PY{o}{.}\PY{n}{fit\PYZus{}transform}\PY{p}{(}\PY{n}{X}\PY{p}{)}
\end{Verbatim}
\end{tcolorbox}

    \hypertarget{ux440ux430ux437ux434ux435ux43bux435ux43dux438ux435-ux43dux430-ux442ux440ux435ux43dux438ux440ux43eux432ux43eux447ux43dux443ux44e-ux438-ux442ux435ux441ux442ux43eux432ux443ux44e-ux432ux44bux431ux43eux440ux43aux438}{%
\section{Разделение на тренировочную и тестовую
выборки}\label{ux440ux430ux437ux434ux435ux43bux435ux43dux438ux435-ux43dux430-ux442ux440ux435ux43dux438ux440ux43eux432ux43eux447ux43dux443ux44e-ux438-ux442ux435ux441ux442ux43eux432ux443ux44e-ux432ux44bux431ux43eux440ux43aux438}}

Выделим 70\% данных для обучения и 30\% для тестирования, чтобы
проверить модели на новых данных.

    \begin{tcolorbox}[breakable, size=fbox, boxrule=1pt, pad at break*=1mm,colback=cellbackground, colframe=cellborder]
\prompt{In}{incolor}{77}{\boxspacing}
\begin{Verbatim}[commandchars=\\\{\}]
\PY{k+kn}{from} \PY{n+nn}{sklearn}\PY{n+nn}{.}\PY{n+nn}{model\PYZus{}selection} \PY{k+kn}{import} \PY{n}{train\PYZus{}test\PYZus{}split}
\end{Verbatim}
\end{tcolorbox}

    \begin{tcolorbox}[breakable, size=fbox, boxrule=1pt, pad at break*=1mm,colback=cellbackground, colframe=cellborder]
\prompt{In}{incolor}{78}{\boxspacing}
\begin{Verbatim}[commandchars=\\\{\}]
\PY{n}{X\PYZus{}train}\PY{p}{,} \PY{n}{X\PYZus{}test}\PY{p}{,} \PY{n}{y\PYZus{}train}\PY{p}{,} \PY{n}{y\PYZus{}test} \PY{o}{=} \PY{n}{train\PYZus{}test\PYZus{}split}\PY{p}{(}\PY{n}{X\PYZus{}scaled}\PY{p}{,} \PY{n}{y}\PY{p}{,} \PY{n}{test\PYZus{}size}\PY{o}{=}\PY{l+m+mf}{0.3}\PY{p}{,} \PY{n}{random\PYZus{}state}\PY{o}{=}\PY{l+m+mi}{42}\PY{p}{)}
\end{Verbatim}
\end{tcolorbox}

\clearpage

    \hypertarget{ux431ux430ux433ux433ux438ux43dux433}{%
\section{Баггинг}\label{ux431ux430ux433ux433ux438ux43dux433}}

    \hypertarget{baggingclassifier}{%
\subsection{BaggingClassifier}\label{baggingclassifier}}

    \textbf{BaggingClassifier} --- это более общий класс для алгоритмов,
использующих метод Bagging (Bootstrap Aggregating). Суть Bagging
заключается в том, чтобы создавать несколько моделей на случайных
подмножествах данных и затем объединять их прогнозы для получения более
стабильных и точных предсказаний.

    \begin{tcolorbox}[breakable, size=fbox, boxrule=1pt, pad at break*=1mm,colback=cellbackground, colframe=cellborder]
\prompt{In}{incolor}{79}{\boxspacing}
\begin{Verbatim}[commandchars=\\\{\}]
\PY{k+kn}{from} \PY{n+nn}{sklearn}\PY{n+nn}{.}\PY{n+nn}{ensemble} \PY{k+kn}{import} \PY{n}{BaggingClassifier}
\end{Verbatim}
\end{tcolorbox}

    \begin{tcolorbox}[breakable, size=fbox, boxrule=1pt, pad at break*=1mm,colback=cellbackground, colframe=cellborder]
\prompt{In}{incolor}{80}{\boxspacing}
\begin{Verbatim}[commandchars=\\\{\}]
\PY{n}{bagging\PYZus{}model} \PY{o}{=} \PY{n}{BaggingClassifier}\PY{p}{(}\PY{n}{n\PYZus{}estimators}\PY{o}{=}\PY{l+m+mi}{100}\PY{p}{,} \PY{n}{random\PYZus{}state}\PY{o}{=}\PY{l+m+mi}{42}\PY{p}{)}
\PY{n}{start\PYZus{}time} \PY{o}{=} \PY{n}{time}\PY{o}{.}\PY{n}{time}\PY{p}{(}\PY{p}{)}
\PY{n}{bagging\PYZus{}model}\PY{o}{.}\PY{n}{fit}\PY{p}{(}\PY{n}{X\PYZus{}train}\PY{p}{,} \PY{n}{y\PYZus{}train}\PY{p}{)}
\PY{n}{bagging\PYZus{}time} \PY{o}{=} \PY{n}{time}\PY{o}{.}\PY{n}{time}\PY{p}{(}\PY{p}{)} \PY{o}{\PYZhy{}} \PY{n}{start\PYZus{}time}
\end{Verbatim}
\end{tcolorbox}

    \begin{tcolorbox}[breakable, size=fbox, boxrule=1pt, pad at break*=1mm,colback=cellbackground, colframe=cellborder]
\prompt{In}{incolor}{81}{\boxspacing}
\begin{Verbatim}[commandchars=\\\{\}]
\PY{n}{y\PYZus{}pred\PYZus{}bagging} \PY{o}{=} \PY{n}{bagging\PYZus{}model}\PY{o}{.}\PY{n}{predict}\PY{p}{(}\PY{n}{X\PYZus{}test}\PY{p}{)}
\PY{n}{bagging\PYZus{}accuracy} \PY{o}{=} \PY{n}{accuracy\PYZus{}score}\PY{p}{(}\PY{n}{y\PYZus{}test}\PY{p}{,} \PY{n}{y\PYZus{}pred\PYZus{}bagging}\PY{p}{)}
\PY{n}{bagging\PYZus{}report} \PY{o}{=} \PY{n}{classification\PYZus{}report}\PY{p}{(}\PY{n}{y\PYZus{}test}\PY{p}{,} \PY{n}{y\PYZus{}pred\PYZus{}bagging}\PY{p}{)}

\PY{n+nb}{print}\PY{p}{(}\PY{l+s+sa}{f}\PY{l+s+s2}{\PYZdq{}}\PY{l+s+s2}{Точность модели: }\PY{l+s+si}{\PYZob{}}\PY{n}{bagging\PYZus{}accuracy}\PY{l+s+si}{\PYZcb{}}\PY{l+s+s2}{\PYZdq{}}\PY{p}{)}
\PY{n+nb}{print}\PY{p}{(}\PY{l+s+sa}{f}\PY{l+s+s2}{\PYZdq{}}\PY{l+s+s2}{Время обучения модели: }\PY{l+s+si}{\PYZob{}}\PY{n}{bagging\PYZus{}time}\PY{l+s+si}{\PYZcb{}}\PY{l+s+s2}{ с.}\PY{l+s+s2}{\PYZdq{}}\PY{p}{)}
\PY{n+nb}{print}\PY{p}{(}\PY{l+s+s2}{\PYZdq{}}\PY{l+s+s2}{Отчет классификации:}\PY{l+s+se}{\PYZbs{}n}\PY{l+s+s2}{\PYZdq{}}\PY{p}{,} \PY{n}{bagging\PYZus{}report}\PY{p}{)}
\end{Verbatim}
\end{tcolorbox}

    \begin{Verbatim}[commandchars=\\\{\}]
Точность модели: 0.9337248322147651
Время обучения модели: 0.43054986000061035 с.
Отчет классификации:
                         precision    recall  f1-score   support

    Anabas testudineus       1.00      1.00      1.00       157
     Coilia dussumieri       1.00      1.00      1.00       124
Otolithoides biauritus       0.72      0.70      0.71       138
     Otolithoides pama       1.00      1.00      1.00       127
     Pethia conchonius       1.00      1.00      1.00       134
  Polynemus paradiseus       1.00      1.00      1.00       142
  Puntius lateristriga       1.00      1.00      1.00       109
        Setipinna taty       0.69      0.72      0.71       132
 Sillaginopsis panijus       1.00      1.00      1.00       129

              accuracy                           0.93      1192
             macro avg       0.94      0.94      0.93      1192
          weighted avg       0.93      0.93      0.93      1192

    \end{Verbatim}

    \hypertarget{ux432ux44bux432ux43eux434}{%
\subsubsection*{\large Вывод}\label{ux432ux44bux432ux43eux434}}

\textbf{Точность}: 93.37\%, \textbf{обучение заняло} 1.19 с.

Модель идеально справляется с большинством классов, кроме Otolithoides
biauritus (F1 = 0.71) и Setipinna taty (F1 = 0.71), где есть сложности
из-за перекрытия признаков.

\clearpage

    \hypertarget{randomforestclassifier}{%
\subsection{RandomForestClassifier}\label{randomforestclassifier}}

    RandomForestClassifier --- это специализированная модель, которая также
использует метод Bagging, но с дополнительной особенностью: она строит
случайные деревья решений, что делает её более точной по сравнению с
простым BaggingClassifier с обычными деревьями.

    \begin{tcolorbox}[breakable, size=fbox, boxrule=1pt, pad at break*=1mm,colback=cellbackground, colframe=cellborder]
\prompt{In}{incolor}{82}{\boxspacing}
\begin{Verbatim}[commandchars=\\\{\}]
\PY{k+kn}{from} \PY{n+nn}{sklearn}\PY{n+nn}{.}\PY{n+nn}{ensemble} \PY{k+kn}{import} \PY{n}{RandomForestClassifier}
\end{Verbatim}
\end{tcolorbox}

    \begin{tcolorbox}[breakable, size=fbox, boxrule=1pt, pad at break*=1mm,colback=cellbackground, colframe=cellborder]
\prompt{In}{incolor}{83}{\boxspacing}
\begin{Verbatim}[commandchars=\\\{\}]
\PY{n}{bagging\PYZus{}model} \PY{o}{=} \PY{n}{RandomForestClassifier}\PY{p}{(}\PY{n}{n\PYZus{}estimators}\PY{o}{=}\PY{l+m+mi}{100}\PY{p}{,} \PY{n}{random\PYZus{}state}\PY{o}{=}\PY{l+m+mi}{42}\PY{p}{)}
\PY{n}{start\PYZus{}time} \PY{o}{=} \PY{n}{time}\PY{o}{.}\PY{n}{time}\PY{p}{(}\PY{p}{)}
\PY{n}{bagging\PYZus{}model}\PY{o}{.}\PY{n}{fit}\PY{p}{(}\PY{n}{X\PYZus{}train}\PY{p}{,} \PY{n}{y\PYZus{}train}\PY{p}{)}
\PY{n}{bagging\PYZus{}time} \PY{o}{=} \PY{n}{time}\PY{o}{.}\PY{n}{time}\PY{p}{(}\PY{p}{)} \PY{o}{\PYZhy{}} \PY{n}{start\PYZus{}time}
\end{Verbatim}
\end{tcolorbox}

    \begin{tcolorbox}[breakable, size=fbox, boxrule=1pt, pad at break*=1mm,colback=cellbackground, colframe=cellborder]
\prompt{In}{incolor}{84}{\boxspacing}
\begin{Verbatim}[commandchars=\\\{\}]
\PY{n}{y\PYZus{}pred\PYZus{}bagging} \PY{o}{=} \PY{n}{bagging\PYZus{}model}\PY{o}{.}\PY{n}{predict}\PY{p}{(}\PY{n}{X\PYZus{}test}\PY{p}{)}
\PY{n}{bagging\PYZus{}accuracy} \PY{o}{=} \PY{n}{accuracy\PYZus{}score}\PY{p}{(}\PY{n}{y\PYZus{}test}\PY{p}{,} \PY{n}{y\PYZus{}pred\PYZus{}bagging}\PY{p}{)}
\PY{n}{bagging\PYZus{}report} \PY{o}{=} \PY{n}{classification\PYZus{}report}\PY{p}{(}\PY{n}{y\PYZus{}test}\PY{p}{,} \PY{n}{y\PYZus{}pred\PYZus{}bagging}\PY{p}{)}

\PY{n+nb}{print}\PY{p}{(}\PY{l+s+sa}{f}\PY{l+s+s2}{\PYZdq{}}\PY{l+s+s2}{Точность модели: }\PY{l+s+si}{\PYZob{}}\PY{n}{bagging\PYZus{}accuracy}\PY{l+s+si}{\PYZcb{}}\PY{l+s+s2}{\PYZdq{}}\PY{p}{)}
\PY{n+nb}{print}\PY{p}{(}\PY{l+s+sa}{f}\PY{l+s+s2}{\PYZdq{}}\PY{l+s+s2}{Время обучения модели: }\PY{l+s+si}{\PYZob{}}\PY{n}{bagging\PYZus{}time}\PY{l+s+si}{\PYZcb{}}\PY{l+s+s2}{ c.}\PY{l+s+s2}{\PYZdq{}}\PY{p}{)}
\PY{n+nb}{print}\PY{p}{(}\PY{l+s+s2}{\PYZdq{}}\PY{l+s+s2}{Отчет классификации:}\PY{l+s+se}{\PYZbs{}n}\PY{l+s+s2}{\PYZdq{}}\PY{p}{,} \PY{n}{bagging\PYZus{}report}\PY{p}{)}
\end{Verbatim}
\end{tcolorbox}

    \begin{Verbatim}[commandchars=\\\{\}]
Точность модели: 0.9320469798657718
Время обучения модели: 0.24588871002197266 c.
Отчет классификации:
                         precision    recall  f1-score   support

    Anabas testudineus       1.00      1.00      1.00       157
     Coilia dussumieri       1.00      1.00      1.00       124
Otolithoides biauritus       0.71      0.69      0.70       138
     Otolithoides pama       1.00      1.00      1.00       127
     Pethia conchonius       1.00      1.00      1.00       134
  Polynemus paradiseus       1.00      1.00      1.00       142
  Puntius lateristriga       1.00      1.00      1.00       109
        Setipinna taty       0.69      0.71      0.70       132
 Sillaginopsis panijus       1.00      1.00      1.00       129

              accuracy                           0.93      1192
             macro avg       0.93      0.93      0.93      1192
          weighted avg       0.93      0.93      0.93      1192

    \end{Verbatim}

    \hypertarget{ux432ux44bux432ux43eux434}{%
\subsubsection*{\large Вывод}\label{ux432ux44bux432ux43eux434}}

\textbf{Точность}: 93.20\%, \textbf{обучение заняло} 0.41 с (значительно
быстрее, чем BaggingClassifier).

Как и у BaggingClassifier, проблемы с Otolithoides biauritus (F1 = 0.70)
и Setipinna taty (F1 = 0.70).

\clearpage

    \hypertarget{randomforestclassifier-c-gridsearchcv}{%
\subsection{RandomForestClassifier c
GridSearchCV}\label{randomforestclassifier-c-gridsearchcv}}

    Проблема заключается в том, что параметры нужно каким-то образом
выбрать. Можно провести перебор параметров и для каждой их комбинации
обучить модели, затем выбрать лучшую. Это можно сделать с помощью
GridSearchCV.

    \begin{tcolorbox}[breakable, size=fbox, boxrule=1pt, pad at break*=1mm,colback=cellbackground, colframe=cellborder]
\prompt{In}{incolor}{85}{\boxspacing}
\begin{Verbatim}[commandchars=\\\{\}]
\PY{k+kn}{from} \PY{n+nn}{sklearn}\PY{n+nn}{.}\PY{n+nn}{model\PYZus{}selection} \PY{k+kn}{import} \PY{n}{GridSearchCV}
\end{Verbatim}
\end{tcolorbox}

    \begin{tcolorbox}[breakable, size=fbox, boxrule=1pt, pad at break*=1mm,colback=cellbackground, colframe=cellborder]
\prompt{In}{incolor}{86}{\boxspacing}
\begin{Verbatim}[commandchars=\\\{\}]
\PY{n}{random\PYZus{}forest} \PY{o}{=} \PY{n}{RandomForestClassifier}\PY{p}{(}\PY{p}{)}
\PY{n}{params\PYZus{}grid} \PY{o}{=} \PY{p}{\PYZob{}}
    \PY{l+s+s2}{\PYZdq{}}\PY{l+s+s2}{max\PYZus{}depth}\PY{l+s+s2}{\PYZdq{}}\PY{p}{:} \PY{p}{[}\PY{l+m+mi}{10}\PY{p}{,} \PY{l+m+mi}{100}\PY{p}{]}\PY{p}{,}
    \PY{l+s+s2}{\PYZdq{}}\PY{l+s+s2}{min\PYZus{}samples\PYZus{}leaf}\PY{l+s+s2}{\PYZdq{}}\PY{p}{:} \PY{p}{[}\PY{l+m+mi}{3}\PY{p}{,} \PY{l+m+mi}{10}\PY{p}{]}\PY{p}{,}
    \PY{l+s+s2}{\PYZdq{}}\PY{l+s+s2}{min\PYZus{}samples\PYZus{}split}\PY{l+s+s2}{\PYZdq{}}\PY{p}{:} \PY{p}{[}\PY{l+m+mi}{6}\PY{p}{,} \PY{l+m+mi}{12}\PY{p}{]}\PY{p}{,}
\PY{p}{\PYZcb{}}
\PY{n}{grid\PYZus{}search\PYZus{}random\PYZus{}forest} \PY{o}{=} \PY{n}{GridSearchCV}\PY{p}{(}
    \PY{n}{estimator}\PY{o}{=}\PY{n}{random\PYZus{}forest}\PY{p}{,}
    \PY{n}{param\PYZus{}grid}\PY{o}{=}\PY{n}{params\PYZus{}grid}\PY{p}{,}
    \PY{n}{scoring}\PY{o}{=}\PY{l+s+s2}{\PYZdq{}}\PY{l+s+s2}{f1\PYZus{}macro}\PY{l+s+s2}{\PYZdq{}}\PY{p}{,}
    \PY{n}{cv} \PY{o}{=} \PY{l+m+mi}{4}
\PY{p}{)}
\PY{n}{start\PYZus{}time} \PY{o}{=} \PY{n}{time}\PY{o}{.}\PY{n}{time}\PY{p}{(}\PY{p}{)}
\PY{n}{grid\PYZus{}search\PYZus{}random\PYZus{}forest}\PY{o}{.}\PY{n}{fit}\PY{p}{(}\PY{n}{X\PYZus{}train}\PY{p}{,} \PY{n}{y\PYZus{}train}\PY{p}{)}
\PY{n}{bagging\PYZus{}time} \PY{o}{=} \PY{n}{time}\PY{o}{.}\PY{n}{time}\PY{p}{(}\PY{p}{)} \PY{o}{\PYZhy{}} \PY{n}{start\PYZus{}time}
\end{Verbatim}
\end{tcolorbox}

    \begin{tcolorbox}[breakable, size=fbox, boxrule=1pt, pad at break*=1mm,colback=cellbackground, colframe=cellborder]
\prompt{In}{incolor}{87}{\boxspacing}
\begin{Verbatim}[commandchars=\\\{\}]
\PY{n}{y\PYZus{}pred\PYZus{}bagging} \PY{o}{=} \PY{n}{grid\PYZus{}search\PYZus{}random\PYZus{}forest}\PY{o}{.}\PY{n}{predict}\PY{p}{(}\PY{n}{X\PYZus{}test}\PY{p}{)}
\PY{n}{bagging\PYZus{}accuracy} \PY{o}{=} \PY{n}{accuracy\PYZus{}score}\PY{p}{(}\PY{n}{y\PYZus{}test}\PY{p}{,} \PY{n}{y\PYZus{}pred\PYZus{}bagging}\PY{p}{)}
\PY{n}{bagging\PYZus{}report} \PY{o}{=} \PY{n}{classification\PYZus{}report}\PY{p}{(}\PY{n}{y\PYZus{}test}\PY{p}{,} \PY{n}{y\PYZus{}pred\PYZus{}bagging}\PY{p}{)}
\PY{n+nb}{print}\PY{p}{(}\PY{l+s+sa}{f}\PY{l+s+s2}{\PYZdq{}}\PY{l+s+s2}{Точность модели: }\PY{l+s+si}{\PYZob{}}\PY{n}{bagging\PYZus{}accuracy}\PY{l+s+si}{\PYZcb{}}\PY{l+s+s2}{\PYZdq{}}\PY{p}{)}
\PY{n+nb}{print}\PY{p}{(}\PY{l+s+sa}{f}\PY{l+s+s2}{\PYZdq{}}\PY{l+s+s2}{Время обучения модели: }\PY{l+s+si}{\PYZob{}}\PY{n}{bagging\PYZus{}time}\PY{l+s+si}{\PYZcb{}}\PY{l+s+s2}{ c.}\PY{l+s+s2}{\PYZdq{}}\PY{p}{)}
\PY{n+nb}{print}\PY{p}{(}\PY{l+s+s2}{\PYZdq{}}\PY{l+s+s2}{Отчет классификации:}\PY{l+s+se}{\PYZbs{}n}\PY{l+s+s2}{\PYZdq{}}\PY{p}{,} \PY{n}{bagging\PYZus{}report}\PY{p}{)}
\end{Verbatim}
\end{tcolorbox}

    \begin{Verbatim}[commandchars=\\\{\}]
Точность модели: 0.9429530201342282
Время обучения модели: 8.196993112564087 c.
Отчет классификации:
                         precision    recall  f1-score   support
    Anabas testudineus       1.00      1.00      1.00       157
     Coilia dussumieri       1.00      1.00      1.00       124
Otolithoides biauritus       0.74      0.79      0.76       138
     Otolithoides pama       1.00      1.00      1.00       127
     Pethia conchonius       1.00      1.00      1.00       134
  Polynemus paradiseus       1.00      1.00      1.00       142
  Puntius lateristriga       1.00      1.00      1.00       109
        Setipinna taty       0.76      0.70      0.73       132
 Sillaginopsis panijus       1.00      1.00      1.00       129

              accuracy                           0.94      1192
             macro avg       0.94      0.94      0.94      1192
          weighted avg       0.94      0.94      0.94      1192
    \end{Verbatim}

    \hypertarget{ux432ux44bux432ux43eux434}{%
\subsubsection*{\large Вывод}\label{ux432ux44bux432ux43eux434}}

\textbf{Точность}: 94.04\%, выше по сравнению с базовым RandomForest
(93.20\%) и BaggingClassifier (93.37\%).

\textbf{Время обучения}: 7.99 с --- значительно дольше из-за подбора
гиперпараметров.

Метрики Otolithoides biauritus (F1 = 0.75) и Setipinna taty (F1 = 0.72)
стали лучше, но всё ещё неидеальны.

GridSearchCV позволил повысить точность и F1-оценки для сложных классов,
хотя время обучения возросло. Большинство классов, как и ранее,
классифицируются идеально.

\clearpage

    \hypertarget{ux431ux443ux441ux442ux438ux43dux433}{%
\section{Бустинг}\label{ux431ux443ux441ux442ux438ux43dux433}}

    \hypertarget{gradientboostingclassifier}{%
\subsection{GradientBoostingClassifier}\label{gradientboostingclassifier}}

    \begin{tcolorbox}[breakable, size=fbox, boxrule=1pt, pad at break*=1mm,colback=cellbackground, colframe=cellborder]
\prompt{In}{incolor}{88}{\boxspacing}
\begin{Verbatim}[commandchars=\\\{\}]
\PY{k+kn}{from} \PY{n+nn}{sklearn}\PY{n+nn}{.}\PY{n+nn}{ensemble} \PY{k+kn}{import} \PY{n}{GradientBoostingClassifier}
\end{Verbatim}
\end{tcolorbox}

    GradientBoostingClassifier --- это класс в рамках алгоритма градиентного
бустинга (Gradient Boosting).

Gradient Boosting --- это один из наиболее мощных и гибких архитектурных
шаблонов в машинном обучении. Он позволяет строить ансамбли моделей,
комбинируя слабые ученики в сильную модель, способную решать
разнообразные задачи классификации и регрессии.

Основная идея Gradient Boosting --- последовательно обучать новые
ученики на ошибках предыдущих моделей. Это позволяет компенсировать
слабости отдельных учеников и повышать точность предсказаний.

    \begin{tcolorbox}[breakable, size=fbox, boxrule=1pt, pad at break*=1mm,colback=cellbackground, colframe=cellborder]
\prompt{In}{incolor}{89}{\boxspacing}
\begin{Verbatim}[commandchars=\\\{\}]
\PY{n}{boosting\PYZus{}model} \PY{o}{=} \PY{n}{GradientBoostingClassifier}\PY{p}{(}\PY{n}{n\PYZus{}estimators}\PY{o}{=}\PY{l+m+mi}{100}\PY{p}{,} \PY{n}{random\PYZus{}state}\PY{o}{=}\PY{l+m+mi}{42}\PY{p}{)}

\PY{n}{start\PYZus{}time} \PY{o}{=} \PY{n}{time}\PY{o}{.}\PY{n}{time}\PY{p}{(}\PY{p}{)}
\PY{n}{boosting\PYZus{}model}\PY{o}{.}\PY{n}{fit}\PY{p}{(}\PY{n}{X\PYZus{}train}\PY{p}{,} \PY{n}{y\PYZus{}train}\PY{p}{)}
\PY{n}{boosting\PYZus{}time} \PY{o}{=} \PY{n}{time}\PY{o}{.}\PY{n}{time}\PY{p}{(}\PY{p}{)} \PY{o}{\PYZhy{}} \PY{n}{start\PYZus{}time}
\end{Verbatim}
\end{tcolorbox}

    \begin{tcolorbox}[breakable, size=fbox, boxrule=1pt, pad at break*=1mm,colback=cellbackground, colframe=cellborder]
\prompt{In}{incolor}{90}{\boxspacing}
\begin{Verbatim}[commandchars=\\\{\}]
\PY{n}{y\PYZus{}pred\PYZus{}boosting} \PY{o}{=} \PY{n}{boosting\PYZus{}model}\PY{o}{.}\PY{n}{predict}\PY{p}{(}\PY{n}{X\PYZus{}test}\PY{p}{)}
\PY{n}{boosting\PYZus{}accuracy} \PY{o}{=} \PY{n}{accuracy\PYZus{}score}\PY{p}{(}\PY{n}{y\PYZus{}test}\PY{p}{,} \PY{n}{y\PYZus{}pred\PYZus{}boosting}\PY{p}{)}
\PY{n}{boosting\PYZus{}report} \PY{o}{=} \PY{n}{classification\PYZus{}report}\PY{p}{(}\PY{n}{y\PYZus{}test}\PY{p}{,} \PY{n}{y\PYZus{}pred\PYZus{}boosting}\PY{p}{)}

\PY{n+nb}{print}\PY{p}{(}\PY{l+s+sa}{f}\PY{l+s+s2}{\PYZdq{}}\PY{l+s+s2}{Точность модели: }\PY{l+s+si}{\PYZob{}}\PY{n}{boosting\PYZus{}accuracy}\PY{l+s+si}{\PYZcb{}}\PY{l+s+s2}{\PYZdq{}}\PY{p}{)}
\PY{n+nb}{print}\PY{p}{(}\PY{l+s+sa}{f}\PY{l+s+s2}{\PYZdq{}}\PY{l+s+s2}{Время обучения модели: }\PY{l+s+si}{\PYZob{}}\PY{n}{boosting\PYZus{}time}\PY{l+s+si}{\PYZcb{}}\PY{l+s+s2}{ с.}\PY{l+s+s2}{\PYZdq{}}\PY{p}{)}
\PY{n+nb}{print}\PY{p}{(}\PY{l+s+s2}{\PYZdq{}}\PY{l+s+s2}{Отчет классификации:}\PY{l+s+se}{\PYZbs{}n}\PY{l+s+s2}{\PYZdq{}}\PY{p}{,} \PY{n}{boosting\PYZus{}report}\PY{p}{)}
\end{Verbatim}
\end{tcolorbox}

    \begin{Verbatim}[commandchars=\\\{\}]
Точность модели: 0.927013422818792
Время обучения модели: 2.545854330062866 с.
Отчет классификации:
                         precision    recall  f1-score   support

    Anabas testudineus       1.00      1.00      1.00       157
     Coilia dussumieri       1.00      1.00      1.00       124
Otolithoides biauritus       0.70      0.66      0.68       138
     Otolithoides pama       1.00      0.99      1.00       127
     Pethia conchonius       1.00      1.00      1.00       134
  Polynemus paradiseus       1.00      1.00      1.00       142
  Puntius lateristriga       1.00      1.00      1.00       109
        Setipinna taty       0.66      0.70      0.68       132
 Sillaginopsis panijus       1.00      1.00      1.00       129

              accuracy                           0.93      1192
             macro avg       0.93      0.93      0.93      1192
          weighted avg       0.93      0.93      0.93      1192

    \end{Verbatim}

    \hypertarget{ux432ux44bux432ux43eux434}{%
\subsubsection*{\large Вывод}\label{ux432ux44bux432ux43eux434}}

\textbf{Точность}: 92.70\%, немного ниже, чем у RandomForest
(93.20--94.04\%) и BaggingClassifier (93.37\%).

\textbf{Время обучения}: 2.44 с --- медленнее RandomForest без
GridSearchCV (0.41 с), но быстрее GridSearchCV (7.99 с).

Otolithoides biauritus (F1 = 0.68) и Setipinna taty (F1 = 0.68) остаются
сложными для классификации.

GradientBoostingClassifier показывает чуть худшие результаты для сложных
классов, чем RandomForest с GridSearchCV (F1: 0.75/0.72 против
0.68/0.68). Время обучения быстрее, чем у GridSearchCV, но уступает
простому RandomForest.

\clearpage

    \hypertarget{catboost}{%
\subsection{CatBoost}\label{catboost}}

    

    \textbf{CatBoost} (Categorical Boosting) --- это алгоритм машинного
обучения, разработанный компанией «Яндекс». Он предназначен для работы с
табличными данными и особенно эффективен в задачах, где важную роль
играют категориальные признаки.

Основные особенности CatBoost:

\begin{itemize}
\tightlist
\item
  Работа с категориальными признаками. Алгоритм обрабатывает
  категориальные данные без необходимости в предварительном кодировании.
\item
  Обработка пропусков. Алгоритм автоматически обрабатывает пропущенные
  значения, что уменьшает вероятность возникновения ошибок, связанных с
  отсутствием данных.
\item
  Борьба с переобучением. CatBoost использует несколько методов для
  предотвращения переобучения, включая мощную регуляцию и усреднение.
\item
  Скорость и производительность. В CatBoost реализованы оптимизации,
  делающие обучение и предсказания более быстрыми по сравнению с
  аналогами.
\item
  Стабильность и воспроизводимость. Алгоритм предлагает методы,
  обеспечивающие стабильные результаты даже при изменении порядка
  входных данных.
\end{itemize}

CatBoost используется для улучшения результатов поисковой системы
«Яндекс», ранжирования персональной ленты рекомендаций, для расчёта
прогноза погоды и в других интернет-сервисах компании «Яндекс».

    \begin{tcolorbox}[breakable, size=fbox, boxrule=1pt, pad at break*=1mm,colback=cellbackground, colframe=cellborder]
\prompt{In}{incolor}{91}{\boxspacing}
\begin{Verbatim}[commandchars=\\\{\}]
\PY{o}{!}pip install catboost
\end{Verbatim}
\end{tcolorbox}

    \begin{tcolorbox}[breakable, size=fbox, boxrule=1pt, pad at break*=1mm,colback=cellbackground, colframe=cellborder]
\prompt{In}{incolor}{92}{\boxspacing}
\begin{Verbatim}[commandchars=\\\{\}]
\PY{k+kn}{import} \PY{n+nn}{catboost} \PY{k}{as} \PY{n+nn}{cb}
\end{Verbatim}
\end{tcolorbox}

    \begin{tcolorbox}[breakable, size=fbox, boxrule=1pt, pad at break*=1mm,colback=cellbackground, colframe=cellborder]
\prompt{In}{incolor}{93}{\boxspacing}
\begin{Verbatim}[commandchars=\\\{\}]
\PY{n}{model\PYZus{}catboots\PYZus{}clf} \PY{o}{=} \PY{n}{cb}\PY{o}{.}\PY{n}{CatBoostClassifier}\PY{p}{(}
    \PY{n}{iterations}\PY{o}{=}\PY{l+m+mi}{3000}\PY{p}{,} \PY{n}{task\PYZus{}type}\PY{o}{=}\PY{l+s+s2}{\PYZdq{}}\PY{l+s+s2}{GPU}\PY{l+s+s2}{\PYZdq{}}\PY{p}{,} \PY{n}{devices}\PY{o}{=}\PY{l+s+s1}{\PYZsq{}}\PY{l+s+s1}{0}\PY{l+s+s1}{\PYZsq{}}\PY{p}{)}
\PY{n}{model\PYZus{}catboots\PYZus{}clf}\PY{o}{.}\PY{n}{fit}\PY{p}{(}\PY{n}{X\PYZus{}train}\PY{p}{,} \PY{n}{y\PYZus{}train}\PY{p}{)}
\end{Verbatim}
\end{tcolorbox}

    \begin{Verbatim}[commandchars=\\\{\}]
Learning rate set to 0.032018
0:      learn: 1.9977331        total: 10.7ms   remaining: 32.1s
1:      learn: 1.8410076        total: 19.7ms   remaining: 29.5s
2:      learn: 1.7100641        total: 28.7ms   remaining: 28.6s
3:      learn: 1.5983858        total: 39.3ms   remaining: 29.4s
...
2996:   learn: 0.0647717        total: 48.4s    remaining: 48.4ms
2997:   learn: 0.0647642        total: 48.4s    remaining: 32.3ms
2998:   learn: 0.0647601        total: 48.4s    remaining: 16.1ms
2999:   learn: 0.0647578        total: 48.4s    remaining: 0us
    \end{Verbatim}

            \begin{tcolorbox}[breakable, size=fbox, boxrule=.5pt, pad at break*=1mm, opacityfill=0]
\prompt{Out}{outcolor}{93}{\boxspacing}
\begin{Verbatim}[commandchars=\\\{\}]
<catboost.core.CatBoostClassifier at 0x7d3d2622e830>
\end{Verbatim}
\end{tcolorbox}
        
    \begin{tcolorbox}[breakable, size=fbox, boxrule=1pt, pad at break*=1mm,colback=cellbackground, colframe=cellborder]
\prompt{In}{incolor}{94}{\boxspacing}
\begin{Verbatim}[commandchars=\\\{\}]
\PY{n}{y\PYZus{}pred\PYZus{}boosting} \PY{o}{=} \PY{n}{model\PYZus{}catboots\PYZus{}clf}\PY{o}{.}\PY{n}{predict}\PY{p}{(}\PY{n}{X\PYZus{}test}\PY{p}{)}
\PY{n}{boosting\PYZus{}accuracy} \PY{o}{=} \PY{n}{accuracy\PYZus{}score}\PY{p}{(}\PY{n}{y\PYZus{}test}\PY{p}{,} \PY{n}{y\PYZus{}pred\PYZus{}boosting}\PY{p}{)}
\PY{n}{boosting\PYZus{}report} \PY{o}{=} \PY{n}{classification\PYZus{}report}\PY{p}{(}\PY{n}{y\PYZus{}test}\PY{p}{,} \PY{n}{y\PYZus{}pred\PYZus{}boosting}\PY{p}{)}

\PY{n+nb}{print}\PY{p}{(}\PY{l+s+sa}{f}\PY{l+s+s2}{\PYZdq{}}\PY{l+s+s2}{Точность модели: }\PY{l+s+si}{\PYZob{}}\PY{n}{boosting\PYZus{}accuracy}\PY{l+s+si}{\PYZcb{}}\PY{l+s+s2}{\PYZdq{}}\PY{p}{)}
\PY{n+nb}{print}\PY{p}{(}\PY{l+s+sa}{f}\PY{l+s+s2}{\PYZdq{}}\PY{l+s+s2}{Время обучения модели: }\PY{l+s+si}{\PYZob{}}\PY{n}{boosting\PYZus{}time}\PY{l+s+si}{\PYZcb{}}\PY{l+s+s2}{ с.}\PY{l+s+s2}{\PYZdq{}}\PY{p}{)}
\PY{n+nb}{print}\PY{p}{(}\PY{l+s+s2}{\PYZdq{}}\PY{l+s+s2}{Отчет классификации:}\PY{l+s+se}{\PYZbs{}n}\PY{l+s+s2}{\PYZdq{}}\PY{p}{,} \PY{n}{boosting\PYZus{}report}\PY{p}{)}
\end{Verbatim}
\end{tcolorbox}

    \begin{Verbatim}[commandchars=\\\{\}]
Точность модели: 0.9303691275167785
Время обучения модели: 2.545854330062866 с.
Отчет классификации:
                         precision    recall  f1-score   support

    Anabas testudineus       1.00      1.00      1.00       157
     Coilia dussumieri       1.00      1.00      1.00       124
Otolithoides biauritus       0.72      0.65      0.68       138
     Otolithoides pama       1.00      1.00      1.00       127
     Pethia conchonius       1.00      1.00      1.00       134
  Polynemus paradiseus       1.00      1.00      1.00       142
  Puntius lateristriga       1.00      1.00      1.00       109
        Setipinna taty       0.67      0.73      0.70       132
 Sillaginopsis panijus       1.00      1.00      1.00       129

              accuracy                           0.93      1192
             macro avg       0.93      0.93      0.93      1192
          weighted avg       0.93      0.93      0.93      1192

    \end{Verbatim}

    \hypertarget{ux432ux44bux432ux43eux434}{%
\subsubsection*{\large Вывод}\label{ux432ux44bux432ux43eux434}}

\textbf{Точность}: 93.04\%, сопоставима с RandomForest (93.20\%) и
BaggingClassifier (93.37\%).

\textbf{Время обучения}: 2.44 с, аналогично GradientBoostingClassifier.

Проблемные классы: * Otolithoides biauritus: F1 = 0.68 (лучше, чем у
GradientBoosting, но хуже, чем у RandomForest с GridSearchCV). *
Setipinna taty: F1 = 0.70 (на уровне лучших моделей).

CatBoost сохраняет высокую точность для большинства классов, уступая
только RandomForest с GridSearchCV (94.04\%). Быстрее GridSearchCV, но
точность на сложных классах немного ниже.


\clearpage

    \hypertarget{ux432ux44bux99999}{%
\section*{\Large Общий вывод}\label{ux432ux44bux99999}}

Общий вывод по всем моделям классификации:

\textbf{\large Точность моделей}

Все модели показывают высокую точность (92.70–94.04\%),
что свидетельствует о хорошей предсказательной способности на данном датасете.

Лучший результат показал RandomForest с GridSearchCV --- 94.04\%,
особенно для сложных классов.

Остальные модели:

\begin{itemize}
	\item BaggingClassifier (93.37\%)
	\item RandomForest (93.20\%)
	\item CatBoost (93.04\%)
	\item GradientBoostingClassifier (92.70\%)
\end{itemize}

\textbf{\large Проблемные классы}

Классы Otolithoides biauritus и Setipinna taty
остаются проблемными для всех моделей:

Лучший F1 для Otolithoides biauritus (0.75) и Setipinna taty (0.72)
достигнут с помощью RandomForest + GridSearchCV.
Для остальных моделей F1-оценки варьируются в пределах 0.68–0.70.

\textbf{\large Время обучения}

Самая быстрая модель: RandomForest без GridSearchCV (0.41 с).
BaggingClassifier и CatBoost демонстрируют разумное время (1.19 с и 2.44 с).
GridSearchCV значительно увеличивает время обучения (7.99 с).

\textbf{\large Заключение}

RandomForest с GridSearchCV --- наиболее точный вариант.
Однако, для задач, где критична скорость обучения,
CatBoost или простой RandomForest являются отличным выбором.
    
\end{document}

