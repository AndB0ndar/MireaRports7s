\documentclass[14pt,a4paper,oneside]{extarticle}

	\usepackage{cmap} % for pdfLaTeX
	\usepackage[T1,T2A]{fontenc} % correct encoding for pdfLaTeX
	\usepackage[utf8]{inputenc} % correct encoding source file
	\usepackage[english,russian]{babel} % correct language

	% -------------------
	% TEXT SETTINGS
	% -------------------

	\usepackage{fontspec} % use standart fonts (only for xelatex!!!)
	\setmainfont{Times New Roman}
	\setsansfont{Arial}
	\setmonofont[Scale=0.6]{Courier New}

	\usepackage[none]{hyphenat} % no word breaks

	\usepackage{setspace}
	%\singlespacing % 1.0
	\onehalfspacing % 1.5

	% -------------------
	% PAGE SETTINGS
	% -------------------

	\usepackage[left=3cm, right=1.5cm, vmargin=2cm, headheight=2cm]{geometry}
	\linespread{1.5} % line spacing 1.5
	\usepackage{indentfirst} % indent first paragraph
	\setlength{\parindent}{1.25cm}
	\sloppy

	\frenchspacing % ???
	\pagestyle{plain}
	\usepackage{fancyhdr} % for headers and footers

	\clubpenalty=10000
	\widowpenalty=10000

	% ------------------
	% ATTACHMENTS SETTINGS
	% ------------------

	\usepackage[labelsep=endash]{caption}
	%\setlength{\abovecaptionskip}{3pt}
	%\setlength{\belowcaptionskip}{3pt}

	% for non-end-to-end numbering
	%\usepackage{chngcntr}
	%\counterwithin{figure}{section}
	%\counterwithin{table}{section}

	% for long table
	\usepackage{longtable}

	% for graphics
	\usepackage{graphicx}
	\newcommand{\rref}[1]{(Рисунок~\ref{#1})}
	\newcommand{\rdref}[2]{(Рисунки~\ref{#1}\,-\,\ref{#2})}
	\newcommand{\tref}[1]{(Таблица~\ref{#1})}
	\newcommand{\tdref}[2]{(Таблици~\ref{#1}\,-\,\ref{#2})}
	\newenvironment{image}{
		\begin{figure}[h!tp]
		\centering
	}{
		\end{figure}
	}
	\newcommand\includegrph[2][width=0.8\textwidth]{\includegraphics[#1]{#2}}

	\usepackage{amsmath} % more flexibility equations (use in title!!!)
	\usepackage{pdfpages} % include pdf


	% ------------------
	% SECTION SETTINGS
	% ------------------

	\usepackage{titlesec}

	% section size
	\titleformat{\chapter}[block]
	  {\fontsize{18pt}{22pt}\bfseries}
	  {\chaptertitlename\ \thechapter.}{0.5em}{}
	\titleformat{\section}
		{\fontsize{16pt}{20pt}\bfseries}{\thesection.}{1em}{}
	\titleformat{\subsection}
		{\fontsize{14pt}{18pt}\bfseries}{\thesubsection.}{1em}{}
	\titleformat{\subsubsection}
		{\fontsize{14pt}{18pt}\bfseries}{\thesubsubsection.}{1em}{}
	\titlespacing{\chapter}{1.25cm}{1pt}{1pt}
	\titlespacing{\section}{1.25cm}{1pt}{1pt}
	\titlespacing{\subsection}{1.25cm}{1pt}{1pt}
	\titlespacing{\subsubsection}{1.25cm}{1pt}{1pt}
	\titlespacing{\paragraph}{1.25cm}{0pt}{0pt}

	% new command for sections
	\newcommand\Chapter[1]{
		\refstepcounter{chapter}
		\chapter*{\textbf{ГЛАВА\;\arabic{chapter}.}
			\raggedright #1
		}
		\addcontentsline{toc}{chapter}{ГЛАВА\;\arabic{chapter}.\ #1}
	}
	\newcommand\Section[1]{
		\refstepcounter{section}
		\section*{\textbf{\arabic{chapter}.\arabic{section}.}
			\raggedright #1
		}
		\addcontentsline{toc}{section}{\arabic{chapter}.\arabic{section}.\ #1}
	}
	\newcommand\Subsection[1]{
		\refstepcounter{subsection}
		\subsection*{\textbf{\arabic{chapter}.\arabic{section}.\arabic{subsection}.}
			\raggedright #1
		}
		\addcontentsline{toc}{subsection}
			{\arabic{chapter}.\arabic{section}.\arabic{subsection}.\ #1}
	}
	\newcommand\Subsubsection[1]{
		\refstepcounter{subsubsection}
		\subsubsection*{\textbf{\arabic{chapter}.\arabic{section}.\arabic{subsection}.\arabic{subsubsection}.}
			\raggedright #1
		}
		\addcontentsline{toc}{subsubsection}
			{\arabic{chapter}.\arabic{section}.\arabic{subsection}.\arabic{subsubsection}.\ #1}
	}

	% new command for appendix
	\newcommand\AppendixChapter[1]{
		\refstepcounter{chapter}
		\chapter*{\textbf{\appendixname\;\thechapter.}
			\raggedright #1
		}
		\addcontentsline{toc}{chapter}{\appendixname~\thechapter.\ #1}
	}
	\newcommand\AppendixSection[1]{
		\refstepcounter{section}
		\section*{\textbf{\appendixname\;\thesection.}
			\raggedright #1
		}
	}
	\newcommand\AppendixSubsection[1]{
		\refstepcounter{subsection}
		\section*{\textbf{\appendixname\;\thesubsection.}
			\raggedright #1
		}
	}

	% ------------------
	% NEW COMMAND
	% ------------------

	\providecommand{\No}{\textnumero}

	% ------------------
	% ------------------

    \usepackage[breakable]{tcolorbox}
    

    \usepackage{graphicx}
    \usepackage{caption}

    \usepackage{xcolor} % Allow colors to be defined
    \usepackage{amsmath} % Equations
    \usepackage{amssymb} % Equations
    \usepackage{geometry} % Used to adjust the document margins

    \usepackage{fancyvrb} % verbatim replacement that allows latex

    \makeatletter % fix for old versions of grffile with XeLaTeX
    \@ifpackagelater{grffile}{2019/11/01}
    {
      % Do nothing on new versions
    }
    {
      \def\Gread@@xetex#1{%
        \IfFileExists{"\Gin@base".bb}%
        {\Gread@eps{\Gin@base.bb}}%
        {\Gread@@xetex@aux#1}%
      }
    }
    \makeatother
    \usepackage[Export]{adjustbox} % Used to constrain images to a maximum size
    \adjustboxset{max size={0.9\linewidth}{0.9\paperheight}}

    % The hyperref package gives us a pdf with properly built
    % internal navigation ('pdf bookmarks' for the table of contents,
    % internal cross-reference links, web links for URLs, etc.)
    \usepackage{hyperref}
    % The default LaTeX title has an obnoxious amount of whitespace. By default,
    % titling removes some of it. It also provides customization options.
    \usepackage{titling}
    \usepackage{longtable} % longtable support required by pandoc >1.10
    \usepackage{booktabs}  % table support for pandoc > 1.12.2
    \usepackage{array}     % table support for pandoc >= 2.11.3
    \usepackage{calc}      % table minipage width calculation for pandoc >= 2.11.1
    \usepackage[inline]{enumitem} % IRkernel/repr support (it uses the enumerate* environment)
    \usepackage[normalem]{ulem} % ulem is needed to support strikethroughs (\sout)
                                % normalem makes italics be italics, not underlines
    \usepackage{mathrsfs}
    

    
    % Colors for the hyperref package
    \definecolor{urlcolor}{rgb}{0,.145,.698}
    \definecolor{linkcolor}{rgb}{.71,0.21,0.01}
    \definecolor{citecolor}{rgb}{.12,.54,.11}

    % ANSI colors
    \definecolor{ansi-black}{HTML}{3E424D}
    \definecolor{ansi-black-intense}{HTML}{282C36}
    \definecolor{ansi-red}{HTML}{E75C58}
    \definecolor{ansi-red-intense}{HTML}{B22B31}
    \definecolor{ansi-green}{HTML}{00A250}
    \definecolor{ansi-green-intense}{HTML}{007427}
    \definecolor{ansi-yellow}{HTML}{DDB62B}
    \definecolor{ansi-yellow-intense}{HTML}{B27D12}
    \definecolor{ansi-blue}{HTML}{208FFB}
    \definecolor{ansi-blue-intense}{HTML}{0065CA}
    \definecolor{ansi-magenta}{HTML}{D160C4}
    \definecolor{ansi-magenta-intense}{HTML}{A03196}
    \definecolor{ansi-cyan}{HTML}{60C6C8}
    \definecolor{ansi-cyan-intense}{HTML}{258F8F}
    \definecolor{ansi-white}{HTML}{C5C1B4}
    \definecolor{ansi-white-intense}{HTML}{A1A6B2}
    \definecolor{ansi-default-inverse-fg}{HTML}{FFFFFF}
    \definecolor{ansi-default-inverse-bg}{HTML}{000000}

    % common color for the border for error outputs.
    \definecolor{outerrorbackground}{HTML}{FFDFDF}

    % commands and environments needed by pandoc snippets
    % extracted from the output of `pandoc -s`
    \providecommand{\tightlist}{%
      \setlength{\itemsep}{0pt}\setlength{\parskip}{0pt}}
    \DefineVerbatimEnvironment{Highlighting}{Verbatim}{commandchars=\\\{\}}
    % Add ',fontsize=\small' for more characters per line
    \newenvironment{Shaded}{}{}
    \newcommand{\KeywordTok}[1]{\textcolor[rgb]{0.00,0.44,0.13}{\textbf{{#1}}}}
    \newcommand{\DataTypeTok}[1]{\textcolor[rgb]{0.56,0.13,0.00}{{#1}}}
    \newcommand{\DecValTok}[1]{\textcolor[rgb]{0.25,0.63,0.44}{{#1}}}
    \newcommand{\BaseNTok}[1]{\textcolor[rgb]{0.25,0.63,0.44}{{#1}}}
    \newcommand{\FloatTok}[1]{\textcolor[rgb]{0.25,0.63,0.44}{{#1}}}
    \newcommand{\CharTok}[1]{\textcolor[rgb]{0.25,0.44,0.63}{{#1}}}
    \newcommand{\StringTok}[1]{\textcolor[rgb]{0.25,0.44,0.63}{{#1}}}
    \newcommand{\CommentTok}[1]{\textcolor[rgb]{0.38,0.63,0.69}{\textit{{#1}}}}
    \newcommand{\OtherTok}[1]{\textcolor[rgb]{0.00,0.44,0.13}{{#1}}}
    \newcommand{\AlertTok}[1]{\textcolor[rgb]{1.00,0.00,0.00}{\textbf{{#1}}}}
    \newcommand{\FunctionTok}[1]{\textcolor[rgb]{0.02,0.16,0.49}{{#1}}}
    \newcommand{\RegionMarkerTok}[1]{{#1}}
    \newcommand{\ErrorTok}[1]{\textcolor[rgb]{1.00,0.00,0.00}{\textbf{{#1}}}}
    \newcommand{\NormalTok}[1]{{#1}}

    % Additional commands for more recent versions of Pandoc
    \newcommand{\ConstantTok}[1]{\textcolor[rgb]{0.53,0.00,0.00}{{#1}}}
    \newcommand{\SpecialCharTok}[1]{\textcolor[rgb]{0.25,0.44,0.63}{{#1}}}
    \newcommand{\VerbatimStringTok}[1]{\textcolor[rgb]{0.25,0.44,0.63}{{#1}}}
    \newcommand{\SpecialStringTok}[1]{\textcolor[rgb]{0.73,0.40,0.53}{{#1}}}
    \newcommand{\ImportTok}[1]{{#1}}
    \newcommand{\DocumentationTok}[1]{\textcolor[rgb]{0.73,0.13,0.13}{\textit{{#1}}}}
    \newcommand{\AnnotationTok}[1]{\textcolor[rgb]{0.38,0.63,0.69}{\textbf{\textit{{#1}}}}}
    \newcommand{\CommentVarTok}[1]{\textcolor[rgb]{0.38,0.63,0.69}{\textbf{\textit{{#1}}}}}
    \newcommand{\VariableTok}[1]{\textcolor[rgb]{0.10,0.09,0.49}{{#1}}}
    \newcommand{\ControlFlowTok}[1]{\textcolor[rgb]{0.00,0.44,0.13}{\textbf{{#1}}}}
    \newcommand{\OperatorTok}[1]{\textcolor[rgb]{0.40,0.40,0.40}{{#1}}}
    \newcommand{\BuiltInTok}[1]{{#1}}
    \newcommand{\ExtensionTok}[1]{{#1}}
    \newcommand{\PreprocessorTok}[1]{\textcolor[rgb]{0.74,0.48,0.00}{{#1}}}
    \newcommand{\AttributeTok}[1]{\textcolor[rgb]{0.49,0.56,0.16}{{#1}}}
    \newcommand{\InformationTok}[1]{\textcolor[rgb]{0.38,0.63,0.69}{\textbf{\textit{{#1}}}}}
    \newcommand{\WarningTok}[1]{\textcolor[rgb]{0.38,0.63,0.69}{\textbf{\textit{{#1}}}}}

    
    
    
    
	% Pygments definitions
	\makeatletter
	\def\PY@reset{\let\PY@it=\relax \let\PY@bf=\relax%
		\let\PY@ul=\relax \let\PY@tc=\relax%
		\let\PY@bc=\relax \let\PY@ff=\relax}
	\def\PY@tok#1{\csname PY@tok@#1\endcsname}
	\def\PY@toks#1+{\ifx\relax#1\empty\else%
		\PY@tok{#1}\expandafter\PY@toks\fi}
	\def\PY@do#1{\PY@bc{\PY@tc{\PY@ul{%
		\PY@it{\PY@bf{\PY@ff{#1}}}}}}}
	\def\PY#1#2{\PY@reset\PY@toks#1+\relax+\PY@do{#2}}

	\@namedef{PY@tok@w}{\def\PY@tc##1{\textcolor[rgb]{0.73,0.73,0.73}{##1}}}
	\@namedef{PY@tok@c}{\let\PY@it=\textit\def\PY@tc##1{\textcolor[rgb]{0.24,0.48,0.48}{##1}}}
	\@namedef{PY@tok@cp}{\def\PY@tc##1{\textcolor[rgb]{0.61,0.40,0.00}{##1}}}
	\@namedef{PY@tok@k}{\let\PY@bf=\textbf\def\PY@tc##1{\textcolor[rgb]{0.00,0.50,0.00}{##1}}}
	\@namedef{PY@tok@kp}{\def\PY@tc##1{\textcolor[rgb]{0.00,0.50,0.00}{##1}}}
	\@namedef{PY@tok@kt}{\def\PY@tc##1{\textcolor[rgb]{0.69,0.00,0.25}{##1}}}
	\@namedef{PY@tok@o}{\def\PY@tc##1{\textcolor[rgb]{0.40,0.40,0.40}{##1}}}
	\@namedef{PY@tok@ow}{\let\PY@bf=\textbf\def\PY@tc##1{\textcolor[rgb]{0.67,0.13,1.00}{##1}}}
	\@namedef{PY@tok@nb}{\def\PY@tc##1{\textcolor[rgb]{0.00,0.50,0.00}{##1}}}
	\@namedef{PY@tok@nf}{\def\PY@tc##1{\textcolor[rgb]{0.00,0.00,1.00}{##1}}}
	\@namedef{PY@tok@nc}{\let\PY@bf=\textbf\def\PY@tc##1{\textcolor[rgb]{0.00,0.00,1.00}{##1}}}
	\@namedef{PY@tok@nn}{\let\PY@bf=\textbf\def\PY@tc##1{\textcolor[rgb]{0.00,0.00,1.00}{##1}}}
	\@namedef{PY@tok@ne}{\let\PY@bf=\textbf\def\PY@tc##1{\textcolor[rgb]{0.80,0.25,0.22}{##1}}}
	\@namedef{PY@tok@nv}{\def\PY@tc##1{\textcolor[rgb]{0.10,0.09,0.49}{##1}}}
	\@namedef{PY@tok@no}{\def\PY@tc##1{\textcolor[rgb]{0.53,0.00,0.00}{##1}}}
	\@namedef{PY@tok@nl}{\def\PY@tc##1{\textcolor[rgb]{0.46,0.46,0.00}{##1}}}
	\@namedef{PY@tok@ni}{\let\PY@bf=\textbf\def\PY@tc##1{\textcolor[rgb]{0.44,0.44,0.44}{##1}}}
	\@namedef{PY@tok@na}{\def\PY@tc##1{\textcolor[rgb]{0.41,0.47,0.13}{##1}}}
	\@namedef{PY@tok@nt}{\let\PY@bf=\textbf\def\PY@tc##1{\textcolor[rgb]{0.00,0.50,0.00}{##1}}}
	\@namedef{PY@tok@nd}{\def\PY@tc##1{\textcolor[rgb]{0.67,0.13,1.00}{##1}}}
	\@namedef{PY@tok@s}{\def\PY@tc##1{\textcolor[rgb]{0.73,0.13,0.13}{##1}}}
	\@namedef{PY@tok@sd}{\let\PY@it=\textit\def\PY@tc##1{\textcolor[rgb]{0.73,0.13,0.13}{##1}}}
	\@namedef{PY@tok@si}{\let\PY@bf=\textbf\def\PY@tc##1{\textcolor[rgb]{0.64,0.35,0.47}{##1}}}
	\@namedef{PY@tok@se}{\let\PY@bf=\textbf\def\PY@tc##1{\textcolor[rgb]{0.67,0.36,0.12}{##1}}}
	\@namedef{PY@tok@sr}{\def\PY@tc##1{\textcolor[rgb]{0.64,0.35,0.47}{##1}}}
	\@namedef{PY@tok@ss}{\def\PY@tc##1{\textcolor[rgb]{0.10,0.09,0.49}{##1}}}
	\@namedef{PY@tok@sx}{\def\PY@tc##1{\textcolor[rgb]{0.00,0.50,0.00}{##1}}}
	\@namedef{PY@tok@m}{\def\PY@tc##1{\textcolor[rgb]{0.40,0.40,0.40}{##1}}}
	\@namedef{PY@tok@gh}{\let\PY@bf=\textbf\def\PY@tc##1{\textcolor[rgb]{0.00,0.00,0.50}{##1}}}
	\@namedef{PY@tok@gu}{\let\PY@bf=\textbf\def\PY@tc##1{\textcolor[rgb]{0.50,0.00,0.50}{##1}}}
	\@namedef{PY@tok@gd}{\def\PY@tc##1{\textcolor[rgb]{0.63,0.00,0.00}{##1}}}
	\@namedef{PY@tok@gi}{\def\PY@tc##1{\textcolor[rgb]{0.00,0.52,0.00}{##1}}}
	\@namedef{PY@tok@gr}{\def\PY@tc##1{\textcolor[rgb]{0.89,0.00,0.00}{##1}}}
	\@namedef{PY@tok@ge}{\let\PY@it=\textit}
	\@namedef{PY@tok@gs}{\let\PY@bf=\textbf}
	\@namedef{PY@tok@gp}{\let\PY@bf=\textbf\def\PY@tc##1{\textcolor[rgb]{0.00,0.00,0.50}{##1}}}
	\@namedef{PY@tok@go}{\def\PY@tc##1{\textcolor[rgb]{0.44,0.44,0.44}{##1}}}
	\@namedef{PY@tok@gt}{\def\PY@tc##1{\textcolor[rgb]{0.00,0.27,0.87}{##1}}}
	\@namedef{PY@tok@err}{\def\PY@bc##1{{\setlength{\fboxsep}{\string -\fboxrule}\fcolorbox[rgb]{1.00,0.00,0.00}{1,1,1}{\strut ##1}}}}
	\@namedef{PY@tok@kc}{\let\PY@bf=\textbf\def\PY@tc##1{\textcolor[rgb]{0.00,0.50,0.00}{##1}}}
	\@namedef{PY@tok@kd}{\let\PY@bf=\textbf\def\PY@tc##1{\textcolor[rgb]{0.00,0.50,0.00}{##1}}}
	\@namedef{PY@tok@kn}{\let\PY@bf=\textbf\def\PY@tc##1{\textcolor[rgb]{0.00,0.50,0.00}{##1}}}
	\@namedef{PY@tok@kr}{\let\PY@bf=\textbf\def\PY@tc##1{\textcolor[rgb]{0.00,0.50,0.00}{##1}}}
	\@namedef{PY@tok@bp}{\def\PY@tc##1{\textcolor[rgb]{0.00,0.50,0.00}{##1}}}
	\@namedef{PY@tok@fm}{\def\PY@tc##1{\textcolor[rgb]{0.00,0.00,1.00}{##1}}}
	\@namedef{PY@tok@vc}{\def\PY@tc##1{\textcolor[rgb]{0.10,0.09,0.49}{##1}}}
	\@namedef{PY@tok@vg}{\def\PY@tc##1{\textcolor[rgb]{0.10,0.09,0.49}{##1}}}
	\@namedef{PY@tok@vi}{\def\PY@tc##1{\textcolor[rgb]{0.10,0.09,0.49}{##1}}}
	\@namedef{PY@tok@vm}{\def\PY@tc##1{\textcolor[rgb]{0.10,0.09,0.49}{##1}}}
	\@namedef{PY@tok@sa}{\def\PY@tc##1{\textcolor[rgb]{0.73,0.13,0.13}{##1}}}
	\@namedef{PY@tok@sb}{\def\PY@tc##1{\textcolor[rgb]{0.73,0.13,0.13}{##1}}}
	\@namedef{PY@tok@sc}{\def\PY@tc##1{\textcolor[rgb]{0.73,0.13,0.13}{##1}}}
	\@namedef{PY@tok@dl}{\def\PY@tc##1{\textcolor[rgb]{0.73,0.13,0.13}{##1}}}
	\@namedef{PY@tok@s2}{\def\PY@tc##1{\textcolor[rgb]{0.73,0.13,0.13}{##1}}}
	\@namedef{PY@tok@sh}{\def\PY@tc##1{\textcolor[rgb]{0.73,0.13,0.13}{##1}}}
	\@namedef{PY@tok@s1}{\def\PY@tc##1{\textcolor[rgb]{0.73,0.13,0.13}{##1}}}
	\@namedef{PY@tok@mb}{\def\PY@tc##1{\textcolor[rgb]{0.40,0.40,0.40}{##1}}}
	\@namedef{PY@tok@mf}{\def\PY@tc##1{\textcolor[rgb]{0.40,0.40,0.40}{##1}}}
	\@namedef{PY@tok@mh}{\def\PY@tc##1{\textcolor[rgb]{0.40,0.40,0.40}{##1}}}
	\@namedef{PY@tok@mi}{\def\PY@tc##1{\textcolor[rgb]{0.40,0.40,0.40}{##1}}}
	\@namedef{PY@tok@il}{\def\PY@tc##1{\textcolor[rgb]{0.40,0.40,0.40}{##1}}}
	\@namedef{PY@tok@mo}{\def\PY@tc##1{\textcolor[rgb]{0.40,0.40,0.40}{##1}}}
	\@namedef{PY@tok@ch}{\let\PY@it=\textit\def\PY@tc##1{\textcolor[rgb]{0.24,0.48,0.48}{##1}}}
	\@namedef{PY@tok@cm}{\let\PY@it=\textit\def\PY@tc##1{\textcolor[rgb]{0.24,0.48,0.48}{##1}}}
	\@namedef{PY@tok@cpf}{\let\PY@it=\textit\def\PY@tc##1{\textcolor[rgb]{0.24,0.48,0.48}{##1}}}
	\@namedef{PY@tok@c1}{\let\PY@it=\textit\def\PY@tc##1{\textcolor[rgb]{0.24,0.48,0.48}{##1}}}
	\@namedef{PY@tok@cs}{\let\PY@it=\textit\def\PY@tc##1{\textcolor[rgb]{0.24,0.48,0.48}{##1}}}

	\def\PYZbs{\char`\\}
	\def\PYZus{\char`\_}
	\def\PYZob{\char`\{}
	\def\PYZcb{\char`\}}
	\def\PYZca{\char`\^}
	\def\PYZam{\char`\&}
	\def\PYZlt{\char`\<}
	\def\PYZgt{\char`\>}
	\def\PYZsh{\char`\#}
	\def\PYZpc{\char`\%}
	\def\PYZdl{\char`\$}
	\def\PYZhy{\char`\-}
	\def\PYZsq{\char`\'}
	\def\PYZdq{\char`\"}
	\def\PYZti{\char`\~}
	% for compatibility with earlier versions
	\def\PYZat{@}
	\def\PYZlb{[}
	\def\PYZrb{]}
	\makeatother


    % For linebreaks inside Verbatim environment from package fancyvrb.
    \makeatletter
        \newbox\Wrappedcontinuationbox
        \newbox\Wrappedvisiblespacebox
        \newcommand*\Wrappedvisiblespace {\textcolor{red}{\textvisiblespace}}
        \newcommand*\Wrappedcontinuationsymbol {\textcolor{red}{\llap{\tiny$\m@th\hookrightarrow$}}}
        \newcommand*\Wrappedcontinuationindent {3ex }
        \newcommand*\Wrappedafterbreak {\kern\Wrappedcontinuationindent\copy\Wrappedcontinuationbox}
        % Take advantage of the already applied Pygments mark-up to insert
        % potential linebreaks for TeX processing.
        %        {, <, #, %, $, ' and ": go to next line.
        %        _, }, ^, &, >, - and ~: stay at end of broken line.
        % Use of \textquotesingle for straight quote.
        \newcommand*\Wrappedbreaksatspecials {%
            \def\PYGZus{\discretionary{\char`\_}{\Wrappedafterbreak}{\char`\_}}%
            \def\PYGZob{\discretionary{}{\Wrappedafterbreak\char`\{}{\char`\{}}%
            \def\PYGZcb{\discretionary{\char`\}}{\Wrappedafterbreak}{\char`\}}}%
            \def\PYGZca{\discretionary{\char`\^}{\Wrappedafterbreak}{\char`\^}}%
            \def\PYGZam{\discretionary{\char`\&}{\Wrappedafterbreak}{\char`\&}}%
            \def\PYGZlt{\discretionary{}{\Wrappedafterbreak\char`\<}{\char`\<}}%
            \def\PYGZgt{\discretionary{\char`\>}{\Wrappedafterbreak}{\char`\>}}%
            \def\PYGZsh{\discretionary{}{\Wrappedafterbreak\char`\#}{\char`\#}}%
            \def\PYGZpc{\discretionary{}{\Wrappedafterbreak\char`\%}{\char`\%}}%
            \def\PYGZdl{\discretionary{}{\Wrappedafterbreak\char`\$}{\char`\$}}%
            \def\PYGZhy{\discretionary{\char`\-}{\Wrappedafterbreak}{\char`\-}}%
            \def\PYGZsq{\discretionary{}{\Wrappedafterbreak\textquotesingle}{\textquotesingle}}%
            \def\PYGZdq{\discretionary{}{\Wrappedafterbreak\char`\"}{\char`\"}}%
            \def\PYGZti{\discretionary{\char`\~}{\Wrappedafterbreak}{\char`\~}}%
        }
        % Some characters . , ; ? ! / are not pygmentized.
        % This macro makes them "active" and they will insert potential linebreaks
        \newcommand*\Wrappedbreaksatpunct {%
            \lccode`\~`\.\lowercase{\def~}{\discretionary{\hbox{\char`\.}}{\Wrappedafterbreak}{\hbox{\char`\.}}}%
            \lccode`\~`\,\lowercase{\def~}{\discretionary{\hbox{\char`\,}}{\Wrappedafterbreak}{\hbox{\char`\,}}}%
            \lccode`\~`\;\lowercase{\def~}{\discretionary{\hbox{\char`\;}}{\Wrappedafterbreak}{\hbox{\char`\;}}}%
            \lccode`\~`\:\lowercase{\def~}{\discretionary{\hbox{\char`\:}}{\Wrappedafterbreak}{\hbox{\char`\:}}}%
            \lccode`\~`\?\lowercase{\def~}{\discretionary{\hbox{\char`\?}}{\Wrappedafterbreak}{\hbox{\char`\?}}}%
            \lccode`\~`\!\lowercase{\def~}{\discretionary{\hbox{\char`\!}}{\Wrappedafterbreak}{\hbox{\char`\!}}}%
            \lccode`\~`\/\lowercase{\def~}{\discretionary{\hbox{\char`\/}}{\Wrappedafterbreak}{\hbox{\char`\/}}}%
            \catcode`\.\active
            \catcode`\,\active
            \catcode`\;\active
            \catcode`\:\active
            \catcode`\?\active
            \catcode`\!\active
            \catcode`\/\active
            \lccode`\~`\~
        }
    \makeatother

    \let\OriginalVerbatim=\Verbatim
    \makeatletter
    \renewcommand{\Verbatim}[1][1]{%
        %\parskip\z@skip
        \sbox\Wrappedcontinuationbox {\Wrappedcontinuationsymbol}%
        \sbox\Wrappedvisiblespacebox {\FV@SetupFont\Wrappedvisiblespace}%
        \def\FancyVerbFormatLine ##1{\hsize\linewidth
            \vtop{\raggedright\hyphenpenalty\z@\exhyphenpenalty\z@
                \doublehyphendemerits\z@\finalhyphendemerits\z@
                \strut ##1\strut}%
        }%
        % If the linebreak is at a space, the latter will be displayed as visible
        % space at end of first line, and a continuation symbol starts next line.
        % Stretch/shrink are however usually zero for typewriter font.
        \def\FV@Space {%
            \nobreak\hskip\z@ plus\fontdimen3\font minus\fontdimen4\font
            \discretionary{\copy\Wrappedvisiblespacebox}{\Wrappedafterbreak}
            {\kern\fontdimen2\font}%
        }%

        % Allow breaks at special characters using \PYG... macros.
        \Wrappedbreaksatspecials
        % Breaks at punctuation characters . , ; ? ! and / need catcode=\active
        \OriginalVerbatim[#1,codes*=\Wrappedbreaksatpunct]%
    }
    \makeatother

    % Exact colors from NB
    \definecolor{incolor}{HTML}{303F9F}
    \definecolor{outcolor}{HTML}{D84315}
    \definecolor{cellborder}{HTML}{CFCFCF}
    \definecolor{cellbackground}{HTML}{F7F7F7}

    % prompt
    \makeatletter
    \newcommand{\boxspacing}{\kern\kvtcb@left@rule\kern\kvtcb@boxsep}
    \makeatother
    \newcommand{\prompt}[4]{
        {\ttfamily\llap{{\color{#2}[#3]:\hspace{3pt}#4}}\vspace{-\baselineskip}}
    }
    

    
    % Prevent overflowing lines due to hard-to-break entities
    \sloppy
    % Setup hyperref package
    \hypersetup{
      breaklinks=true,  % so long urls are correctly broken across lines
      colorlinks=true,
      urlcolor=urlcolor,
      linkcolor=linkcolor,
      citecolor=citecolor,
      }
    

\begin{document}
    
\singlespacing % 1.0
\begin{titlepage}
	\thispagestyle{fancy}
	\renewcommand{\headrulewidth}{0pt}
	\cfoot{Москва 2024}

	\centering
	\includegraphics[scale=0.75]{./res/logo2} \break % вставка логотипа
	{\footnotesize МИНИСТЕРСТВО НАУКИ
		И ВЫСШЕГО ОБРАЗОВАНИЯ РОССИЙСКОЙ ФЕДЕРАЦИИ}\\
	Федеральное государственное бюджетное образовательное учреждение 
		высшего образования\\
	\textbf{<<МИРЭА --- Российский технологический университет>>}\\
	\vfill
	\textbf{\large РТУ МИРЭА}\\
	\bigskip \hrule \smallskip \hrule \smallskip
	\vfill
	Институт информационных технологий (ИТ)\\
	Математического обеспечения
		и стандартизации информационных технологий (МОСИТ)\\
	\vfill
	\textbf{ОТЧЕТ ПО ПРАКТИЧЕСКОЙ РАБОТЕ \No\,1}\\
	\textbf{по дисциплине}\\
	\textbf{<<Разработка кроссплатформенных мобильных приложений>>}\\
	\vfill
	\vfill
	\vfill
	\vfill
	\begin{tabular}{p{0.7\textwidth}p{0.2\textwidth}}
		Выполнил студент группы ИКБО-06-21 & \rightline{Бондарь А.Р.} \\
		Принял старший преподаватель & \rightline{Шешуков Л.С.} \\
	\end{tabular}
	\vfill
	\vfill
	\vfill
	\vfill
\end{titlepage}
\onehalfspacing % 1.5
\setcounter{page}{2}
\clearpage



    
    \hypertarget{section}{%
\section{1}\label{section}}

Определить два вектора, представляющие собой число автомобилей,
припаркованных в течении 5 рабочих дней у бизнес-центра на уличной
стоянке и в подземном гараже.

    \begin{longtable}[]{@{}lll@{}}
\toprule
День & Улица & Гараж\tabularnewline
\midrule
\endhead
Понедельник & 80 & 100\tabularnewline
Вторник & 98 & 82\tabularnewline
Среда & 75 & 105\tabularnewline
Четверг & 91 & 89\tabularnewline
Пятница & 78 & 102\tabularnewline
\bottomrule
\end{longtable}

    \hypertarget{section}{%
\subsection{1.1}\label{section}}

Найти и интерпретировать корреляцию между переменными «Улица» и «Гараж»
(подсчитать корреляцию по Пирсону).

    \begin{tcolorbox}[breakable, size=fbox, boxrule=1pt, pad at break*=1mm,colback=cellbackground, colframe=cellborder]
\prompt{In}{incolor}{2}{\boxspacing}
\begin{Verbatim}[commandchars=\\\{\}]
\PY{k+kn}{import} \PY{n+nn}{numpy} \PY{k}{as} \PY{n+nn}{np}
\PY{k+kn}{import} \PY{n+nn}{matplotlib}\PY{n+nn}{.}\PY{n+nn}{pyplot} \PY{k}{as} \PY{n+nn}{plt}
\PY{k+kn}{from} \PY{n+nn}{scipy}\PY{n+nn}{.}\PY{n+nn}{stats} \PY{k+kn}{import} \PY{n}{pearsonr}
\end{Verbatim}
\end{tcolorbox}

    \begin{tcolorbox}[breakable, size=fbox, boxrule=1pt, pad at break*=1mm,colback=cellbackground, colframe=cellborder]
\prompt{In}{incolor}{3}{\boxspacing}
\begin{Verbatim}[commandchars=\\\{\}]
\PY{n}{street} \PY{o}{=} \PY{n}{np}\PY{o}{.}\PY{n}{array}\PY{p}{(}\PY{p}{[}\PY{l+m+mi}{80}\PY{p}{,} \PY{l+m+mi}{98}\PY{p}{,} \PY{l+m+mi}{75}\PY{p}{,} \PY{l+m+mi}{91}\PY{p}{,} \PY{l+m+mi}{78}\PY{p}{]}\PY{p}{)}
\PY{n}{garage} \PY{o}{=} \PY{n}{np}\PY{o}{.}\PY{n}{array}\PY{p}{(}\PY{p}{[}\PY{l+m+mi}{100}\PY{p}{,} \PY{l+m+mi}{82}\PY{p}{,} \PY{l+m+mi}{105}\PY{p}{,} \PY{l+m+mi}{89}\PY{p}{,} \PY{l+m+mi}{102}\PY{p}{]}\PY{p}{)}
\end{Verbatim}
\end{tcolorbox}

    \begin{tcolorbox}[breakable, size=fbox, boxrule=1pt, pad at break*=1mm,colback=cellbackground, colframe=cellborder]
\prompt{In}{incolor}{4}{\boxspacing}
\begin{Verbatim}[commandchars=\\\{\}]
\PY{n}{correlation}\PY{p}{,} \PY{n}{\PYZus{}} \PY{o}{=} \PY{n}{pearsonr}\PY{p}{(}\PY{n}{street}\PY{p}{,} \PY{n}{garage}\PY{p}{)}
\PY{n}{correlation}
\end{Verbatim}
\end{tcolorbox}

            \begin{tcolorbox}[breakable, size=fbox, boxrule=.5pt, pad at break*=1mm, opacityfill=0]
\prompt{Out}{outcolor}{4}{\boxspacing}
\begin{Verbatim}[commandchars=\\\{\}]
-1.0
\end{Verbatim}
\end{tcolorbox}
        
    Коэффициент корреляции Пирсона между количеством автомобилей,
припаркованных на улице и в гараже, составляет -1.0. Это означает, что
существует идеальная отрицательная корреляция между двумя переменными:
по мере увеличения числа автомобилей на улице, количество автомобилей в
гараже уменьшается и наоборот.

    \hypertarget{section}{%
\subsection{1.2}\label{section}}

Построить диаграмму рассеяния для вышеупомянутых переменных.

    \begin{tcolorbox}[breakable, size=fbox, boxrule=1pt, pad at break*=1mm,colback=cellbackground, colframe=cellborder]
\prompt{In}{incolor}{5}{\boxspacing}
\begin{Verbatim}[commandchars=\\\{\}]
\PY{n}{plt}\PY{o}{.}\PY{n}{scatter}\PY{p}{(}\PY{n}{street}\PY{p}{,} \PY{n}{garage}\PY{p}{,} \PY{n}{color}\PY{o}{=}\PY{l+s+s1}{\PYZsq{}}\PY{l+s+s1}{blue}\PY{l+s+s1}{\PYZsq{}}\PY{p}{)}
\PY{n}{plt}\PY{o}{.}\PY{n}{title}\PY{p}{(}\PY{l+s+s1}{\PYZsq{}}\PY{l+s+s1}{Scatter Plot: Street vs Garage}\PY{l+s+s1}{\PYZsq{}}\PY{p}{)}
\PY{n}{plt}\PY{o}{.}\PY{n}{xlabel}\PY{p}{(}\PY{l+s+s1}{\PYZsq{}}\PY{l+s+s1}{Number of cars (Street)}\PY{l+s+s1}{\PYZsq{}}\PY{p}{)}
\PY{n}{plt}\PY{o}{.}\PY{n}{ylabel}\PY{p}{(}\PY{l+s+s1}{\PYZsq{}}\PY{l+s+s1}{Number of cars (Garage)}\PY{l+s+s1}{\PYZsq{}}\PY{p}{)}
\PY{n}{plt}\PY{o}{.}\PY{n}{grid}\PY{p}{(}\PY{k+kc}{True}\PY{p}{)}
\PY{n}{plt}\PY{o}{.}\PY{n}{show}\PY{p}{(}\PY{p}{)}
\end{Verbatim}
\end{tcolorbox}

    \begin{center}
    \adjustimage{max size={0.9\linewidth}{0.9\paperheight}}{output_8_0.png}
    \end{center}
    { \hspace*{\fill} \\}
    
    Диаграмма рассеяния визуализирует это обратное соотношение между числом
автомобилей на улице и в гараже.

    \hypertarget{section}{%
\section{2}\label{section}}

Найти и выгрузить данные. Вывести, провести предобработку и описать
признаки.

    \hypertarget{section}{%
\subsection{2.1}\label{section}}

Построить корреляционную матрицу по одной целевой переменной. Определить
наиболее коррелирующую переменную, продолжить с ней работу в следующем
пункте.

    \begin{tcolorbox}[breakable, size=fbox, boxrule=1pt, pad at break*=1mm,colback=cellbackground, colframe=cellborder]
\prompt{In}{incolor}{18}{\boxspacing}
\begin{Verbatim}[commandchars=\\\{\}]
\PY{k+kn}{from} \PY{n+nn}{sklearn}\PY{n+nn}{.}\PY{n+nn}{metrics} \PY{k+kn}{import} \PY{n}{mean\PYZus{}squared\PYZus{}error}
\end{Verbatim}
\end{tcolorbox}

    \begin{tcolorbox}[breakable, size=fbox, boxrule=1pt, pad at break*=1mm,colback=cellbackground, colframe=cellborder]
\prompt{In}{incolor}{19}{\boxspacing}
\begin{Verbatim}[commandchars=\\\{\}]
\PY{n}{correlation\PYZus{}matrix} \PY{o}{=} \PY{n}{np}\PY{o}{.}\PY{n}{corrcoef}\PY{p}{(}\PY{n}{street}\PY{p}{,} \PY{n}{garage}\PY{p}{)}
\PY{n}{correlation\PYZus{}matrix}
\end{Verbatim}
\end{tcolorbox}

            \begin{tcolorbox}[breakable, size=fbox, boxrule=.5pt, pad at break*=1mm, opacityfill=0]
\prompt{Out}{outcolor}{19}{\boxspacing}
\begin{Verbatim}[commandchars=\\\{\}]
array([[ 1., -1.],
       [-1.,  1.]])
\end{Verbatim}
\end{tcolorbox}
        
    Корреляционная матрица показала, что между переменными ``Улица'' и
``Гараж'' существует идеальная отрицательная корреляция (-1.0). Наиболее
коррелирующей переменной является ``Гараж''.

    \hypertarget{section}{%
\subsection{2.2}\label{section}}

Реализовать регрессию вручную, отобразить наклон, сдвиг и MSE.

    \begin{tcolorbox}[breakable, size=fbox, boxrule=1pt, pad at break*=1mm,colback=cellbackground, colframe=cellborder]
\prompt{In}{incolor}{23}{\boxspacing}
\begin{Verbatim}[commandchars=\\\{\}]
\PY{c+c1}{\PYZsh{} y = slope * x + intercept}
\PY{n}{n} \PY{o}{=} \PY{n+nb}{len}\PY{p}{(}\PY{n}{street}\PY{p}{)}
\PY{n}{x\PYZus{}mean} \PY{o}{=} \PY{n}{np}\PY{o}{.}\PY{n}{mean}\PY{p}{(}\PY{n}{garage}\PY{p}{)}
\PY{n}{y\PYZus{}mean} \PY{o}{=} \PY{n}{np}\PY{o}{.}\PY{n}{mean}\PY{p}{(}\PY{n}{street}\PY{p}{)}
\PY{n}{slope} \PY{o}{=} \PY{n}{np}\PY{o}{.}\PY{n}{sum}\PY{p}{(}\PY{p}{(}\PY{n}{garage} \PY{o}{\PYZhy{}} \PY{n}{x\PYZus{}mean}\PY{p}{)} \PY{o}{*} \PY{p}{(}\PY{n}{street} \PY{o}{\PYZhy{}} \PY{n}{y\PYZus{}mean}\PY{p}{)}\PY{p}{)} \PY{o}{/} \PY{n}{np}\PY{o}{.}\PY{n}{sum}\PY{p}{(}\PY{p}{(}\PY{n}{garage} \PY{o}{\PYZhy{}} \PY{n}{x\PYZus{}mean}\PY{p}{)}\PY{o}{*}\PY{o}{*}\PY{l+m+mi}{2}\PY{p}{)}
\PY{n}{intercept} \PY{o}{=} \PY{n}{y\PYZus{}mean} \PY{o}{\PYZhy{}} \PY{n}{slope} \PY{o}{*} \PY{n}{x\PYZus{}mean}

\PY{c+c1}{\PYZsh{} Предсказание на основе регрессии}
\PY{n}{predicted\PYZus{}street} \PY{o}{=} \PY{n}{slope} \PY{o}{*} \PY{n}{garage} \PY{o}{+} \PY{n}{intercept}

\PY{c+c1}{\PYZsh{} MSE}
\PY{n}{mse} \PY{o}{=} \PY{n}{mean\PYZus{}squared\PYZus{}error}\PY{p}{(}\PY{n}{street}\PY{p}{,} \PY{n}{predicted\PYZus{}street}\PY{p}{)}

\PY{n}{slope}\PY{p}{,} \PY{n}{intercept}\PY{p}{,} \PY{n}{mse}
\end{Verbatim}
\end{tcolorbox}

            \begin{tcolorbox}[breakable, size=fbox, boxrule=.5pt, pad at break*=1mm, opacityfill=0]
\prompt{Out}{outcolor}{23}{\boxspacing}
\begin{Verbatim}[commandchars=\\\{\}]
(-1.0, 180.0, 0.0)
\end{Verbatim}
\end{tcolorbox}
        
    Реализация линейной регрессии вручную дала следующие результаты: *
Наклон (slope): -1.0 * Сдвиг (intercept): 180.0 * Среднеквадратическая
ошибка (MSE): 0.0

Это указывает на то, что модель линейной регрессии идеально подходит для
этих данных, и ошибок при предсказании нет.

    \hypertarget{section}{%
\subsection{2.3}\label{section}}

Визуализировать регрессию на графике.

    \begin{tcolorbox}[breakable, size=fbox, boxrule=1pt, pad at break*=1mm,colback=cellbackground, colframe=cellborder]
\prompt{In}{incolor}{24}{\boxspacing}
\begin{Verbatim}[commandchars=\\\{\}]
\PY{n}{plt}\PY{o}{.}\PY{n}{scatter}\PY{p}{(}\PY{n}{garage}\PY{p}{,} \PY{n}{street}\PY{p}{,} \PY{n}{color}\PY{o}{=}\PY{l+s+s1}{\PYZsq{}}\PY{l+s+s1}{blue}\PY{l+s+s1}{\PYZsq{}}\PY{p}{,} \PY{n}{label}\PY{o}{=}\PY{l+s+s1}{\PYZsq{}}\PY{l+s+s1}{Actual data}\PY{l+s+s1}{\PYZsq{}}\PY{p}{)}
\PY{n}{plt}\PY{o}{.}\PY{n}{plot}\PY{p}{(}\PY{n}{garage}\PY{p}{,} \PY{n}{predicted\PYZus{}street}\PY{p}{,} \PY{n}{color}\PY{o}{=}\PY{l+s+s1}{\PYZsq{}}\PY{l+s+s1}{red}\PY{l+s+s1}{\PYZsq{}}\PY{p}{,} \PY{n}{label}\PY{o}{=}\PY{l+s+s1}{\PYZsq{}}\PY{l+s+s1}{Regression line}\PY{l+s+s1}{\PYZsq{}}\PY{p}{)}
\PY{n}{plt}\PY{o}{.}\PY{n}{title}\PY{p}{(}\PY{l+s+s1}{\PYZsq{}}\PY{l+s+s1}{Linear Regression: Street vs Garage}\PY{l+s+s1}{\PYZsq{}}\PY{p}{)}
\PY{n}{plt}\PY{o}{.}\PY{n}{xlabel}\PY{p}{(}\PY{l+s+s1}{\PYZsq{}}\PY{l+s+s1}{Garage}\PY{l+s+s1}{\PYZsq{}}\PY{p}{)}
\PY{n}{plt}\PY{o}{.}\PY{n}{ylabel}\PY{p}{(}\PY{l+s+s1}{\PYZsq{}}\PY{l+s+s1}{Street}\PY{l+s+s1}{\PYZsq{}}\PY{p}{)}
\PY{n}{plt}\PY{o}{.}\PY{n}{legend}\PY{p}{(}\PY{p}{)}
\PY{n}{plt}\PY{o}{.}\PY{n}{grid}\PY{p}{(}\PY{k+kc}{True}\PY{p}{)}
\PY{n}{plt}\PY{o}{.}\PY{n}{show}\PY{p}{(}\PY{p}{)}
\end{Verbatim}
\end{tcolorbox}

    \begin{center}
    \adjustimage{max size={0.9\linewidth}{0.9\paperheight}}{output_19_0.png}
    \end{center}
    { \hspace*{\fill} \\}
    
    Визуализация регрессии на графике показала линию, которая точно
описывает соотношение между числом автомобилей в гараже и на улице.

    \hypertarget{section}{%
\section{3}\label{section}}

Загрузить данные: `insurance.csv'. Вывести и провести предобработку.
Вывести список уникальных регионов.

    \begin{tcolorbox}[breakable, size=fbox, boxrule=1pt, pad at break*=1mm,colback=cellbackground, colframe=cellborder]
\prompt{In}{incolor}{38}{\boxspacing}
\begin{Verbatim}[commandchars=\\\{\}]
\PY{o}{!}pip install statsmodels
\end{Verbatim}
\end{tcolorbox}

    \begin{Verbatim}[commandchars=\\\{\}]
Defaulting to user installation because normal site-packages is not writeable
Requirement already satisfied: statsmodels in
/home/arbon/.local/lib/python3.10/site-packages (0.14.4)
Requirement already satisfied: numpy<3,>=1.22.3 in
/usr/local/lib/python3.10/dist-packages (from statsmodels) (1.23.5)
Requirement already satisfied: scipy!=1.9.2,>=1.8 in
/home/arbon/.local/lib/python3.10/site-packages (from statsmodels) (1.14.1)
Requirement already satisfied: pandas!=2.1.0,>=1.4 in
/home/arbon/.local/lib/python3.10/site-packages (from statsmodels) (1.5.3)
Requirement already satisfied: patsy>=0.5.6 in
/home/arbon/.local/lib/python3.10/site-packages (from statsmodels) (0.5.6)
Requirement already satisfied: packaging>=21.3 in
/home/arbon/.local/lib/python3.10/site-packages (from statsmodels) (24.1)
Requirement already satisfied: python-dateutil>=2.8.1 in
/usr/local/lib/python3.10/dist-packages (from pandas!=2.1.0,>=1.4->statsmodels)
(2.8.2)
Requirement already satisfied: pytz>=2020.1 in /usr/lib/python3/dist-packages
(from pandas!=2.1.0,>=1.4->statsmodels) (2022.1)
Requirement already satisfied: six in /usr/lib/python3/dist-packages (from
patsy>=0.5.6->statsmodels) (1.16.0)
    \end{Verbatim}

    \begin{tcolorbox}[breakable, size=fbox, boxrule=1pt, pad at break*=1mm,colback=cellbackground, colframe=cellborder]
\prompt{In}{incolor}{39}{\boxspacing}
\begin{Verbatim}[commandchars=\\\{\}]
\PY{k+kn}{import} \PY{n+nn}{pandas} \PY{k}{as} \PY{n+nn}{pd}
\PY{k+kn}{import} \PY{n+nn}{numpy} \PY{k}{as} \PY{n+nn}{np}
\PY{k+kn}{from} \PY{n+nn}{scipy} \PY{k+kn}{import} \PY{n}{stats}
\PY{k+kn}{import} \PY{n+nn}{statsmodels}\PY{n+nn}{.}\PY{n+nn}{api} \PY{k}{as} \PY{n+nn}{sm}
\PY{k+kn}{from} \PY{n+nn}{statsmodels}\PY{n+nn}{.}\PY{n+nn}{formula}\PY{n+nn}{.}\PY{n+nn}{api} \PY{k+kn}{import} \PY{n}{ols}
\PY{k+kn}{from} \PY{n+nn}{statsmodels}\PY{n+nn}{.}\PY{n+nn}{stats}\PY{n+nn}{.}\PY{n+nn}{multicomp} \PY{k+kn}{import} \PY{n}{pairwise\PYZus{}tukeyhsd}
\PY{k+kn}{import} \PY{n+nn}{matplotlib}\PY{n+nn}{.}\PY{n+nn}{pyplot} \PY{k}{as} \PY{n+nn}{plt}
\end{Verbatim}
\end{tcolorbox}

    \begin{tcolorbox}[breakable, size=fbox, boxrule=1pt, pad at break*=1mm,colback=cellbackground, colframe=cellborder]
\prompt{In}{incolor}{40}{\boxspacing}
\begin{Verbatim}[commandchars=\\\{\}]
\PY{n}{data} \PY{o}{=} \PY{n}{pd}\PY{o}{.}\PY{n}{read\PYZus{}csv}\PY{p}{(}\PY{l+s+s1}{\PYZsq{}}\PY{l+s+s1}{4/insurance.csv}\PY{l+s+s1}{\PYZsq{}}\PY{p}{)}

\PY{n+nb}{print}\PY{p}{(}\PY{n}{data}\PY{o}{.}\PY{n}{info}\PY{p}{(}\PY{p}{)}\PY{p}{)}
\PY{n+nb}{print}\PY{p}{(}\PY{n}{data}\PY{o}{.}\PY{n}{head}\PY{p}{(}\PY{p}{)}\PY{p}{)}
\end{Verbatim}
\end{tcolorbox}

    \begin{Verbatim}[commandchars=\\\{\}]
<class 'pandas.core.frame.DataFrame'>
RangeIndex: 1338 entries, 0 to 1337
Data columns (total 7 columns):
 \#   Column    Non-Null Count  Dtype
---  ------    --------------  -----
 0   age       1338 non-null   int64
 1   sex       1338 non-null   object
 2   bmi       1338 non-null   float64
 3   children  1338 non-null   int64
 4   smoker    1338 non-null   object
 5   region    1338 non-null   object
 6   charges   1338 non-null   float64
dtypes: float64(2), int64(2), object(3)
memory usage: 73.3+ KB
None
   age     sex     bmi  children smoker     region      charges
0   19  female  27.900         0    yes  southwest  16884.92400
1   18    male  33.770         1     no  southeast   1725.55230
2   28    male  33.000         3     no  southeast   4449.46200
3   33    male  22.705         0     no  northwest  21984.47061
4   32    male  28.880         0     no  northwest   3866.85520
    \end{Verbatim}

    \begin{tcolorbox}[breakable, size=fbox, boxrule=1pt, pad at break*=1mm,colback=cellbackground, colframe=cellborder]
\prompt{In}{incolor}{41}{\boxspacing}
\begin{Verbatim}[commandchars=\\\{\}]
\PY{c+c1}{\PYZsh{} Уникальные регионы}
\PY{n}{unique\PYZus{}regions} \PY{o}{=} \PY{n}{data}\PY{p}{[}\PY{l+s+s1}{\PYZsq{}}\PY{l+s+s1}{region}\PY{l+s+s1}{\PYZsq{}}\PY{p}{]}\PY{o}{.}\PY{n}{unique}\PY{p}{(}\PY{p}{)}
\PY{n}{unique\PYZus{}regions}
\end{Verbatim}
\end{tcolorbox}

            \begin{tcolorbox}[breakable, size=fbox, boxrule=.5pt, pad at break*=1mm, opacityfill=0]
\prompt{Out}{outcolor}{41}{\boxspacing}
\begin{Verbatim}[commandchars=\\\{\}]
array(['southwest', 'southeast', 'northwest', 'northeast'], dtype=object)
\end{Verbatim}
\end{tcolorbox}
        
    \hypertarget{section}{%
\section{3.1}\label{section}}

Выполнить однофакторный ANOVA тест, чтобы проверить влияние региона на
индекс массы тела (BMI), используя первый способ, через библиотеку
Scipy.

    \begin{tcolorbox}[breakable, size=fbox, boxrule=1pt, pad at break*=1mm,colback=cellbackground, colframe=cellborder]
\prompt{In}{incolor}{43}{\boxspacing}
\begin{Verbatim}[commandchars=\\\{\}]
\PY{n}{bmi\PYZus{}by\PYZus{}region} \PY{o}{=} \PY{p}{[}\PY{n}{data}\PY{p}{[}\PY{n}{data}\PY{p}{[}\PY{l+s+s1}{\PYZsq{}}\PY{l+s+s1}{region}\PY{l+s+s1}{\PYZsq{}}\PY{p}{]} \PY{o}{==} \PY{n}{region}\PY{p}{]}\PY{p}{[}\PY{l+s+s1}{\PYZsq{}}\PY{l+s+s1}{bmi}\PY{l+s+s1}{\PYZsq{}}\PY{p}{]} \PY{k}{for} \PY{n}{region} \PY{o+ow}{in} \PY{n}{unique\PYZus{}regions}\PY{p}{]}
\PY{n}{anova\PYZus{}result\PYZus{}scipy} \PY{o}{=} \PY{n}{stats}\PY{o}{.}\PY{n}{f\PYZus{}oneway}\PY{p}{(}\PY{o}{*}\PY{n}{bmi\PYZus{}by\PYZus{}region}\PY{p}{)}
\PY{n}{anova\PYZus{}result\PYZus{}scipy}
\end{Verbatim}
\end{tcolorbox}

            \begin{tcolorbox}[breakable, size=fbox, boxrule=.5pt, pad at break*=1mm, opacityfill=0]
\prompt{Out}{outcolor}{43}{\boxspacing}
\begin{Verbatim}[commandchars=\\\{\}]
F\_onewayResult(statistic=39.49505720170283, pvalue=1.881838913929143e-24)
\end{Verbatim}
\end{tcolorbox}
        
    Результаты однофакторного ANOVA теста, выполненного с использованием
библиотеки Scipy, следующие:

\begin{itemize}
\tightlist
\item
  \textbf{Статистика F}: 39.495
\item
  \textbf{P-value}: (1.88e10-24)
\end{itemize}

\textbf{Статистика F} указывает на отношение межгрупповой и
внутригрупповой дисперсии. Высокое значение (39.495) указывает на
значительные различия между группами (регионами) по индексу массы тела
(BMI).

\textbf{P-value} очень низкое ((1.88e10-24)), что гораздо
меньше стандартного уровня значимости 0.05. Это означает, что различия
между средними значениями BMI в разных регионах статистически значимы.

Таким образом, можно заключить, что регион оказывает значительное
влияние на индекс массы тела (BMI).

    \hypertarget{section}{%
\section{3.2}\label{section}}

Выполнить однофакторный ANOVA тест, чтобы проверить влияние региона на
индекс массы тела (BMI), используя второй способ, с помощью функции
anova\_lm() из библиотеки statsmodels.

    \begin{tcolorbox}[breakable, size=fbox, boxrule=1pt, pad at break*=1mm,colback=cellbackground, colframe=cellborder]
\prompt{In}{incolor}{45}{\boxspacing}
\begin{Verbatim}[commandchars=\\\{\}]
\PY{n}{model} \PY{o}{=} \PY{n}{ols}\PY{p}{(}\PY{l+s+s1}{\PYZsq{}}\PY{l+s+s1}{bmi \PYZti{} region}\PY{l+s+s1}{\PYZsq{}}\PY{p}{,} \PY{n}{data}\PY{o}{=}\PY{n}{data}\PY{p}{)}\PY{o}{.}\PY{n}{fit}\PY{p}{(}\PY{p}{)}
\PY{n}{anova\PYZus{}result\PYZus{}statsmodels} \PY{o}{=} \PY{n}{sm}\PY{o}{.}\PY{n}{stats}\PY{o}{.}\PY{n}{anova\PYZus{}lm}\PY{p}{(}\PY{n}{model}\PY{p}{,} \PY{n}{typ}\PY{o}{=}\PY{l+m+mi}{2}\PY{p}{)}
\PY{n}{anova\PYZus{}result\PYZus{}statsmodels}
\end{Verbatim}
\end{tcolorbox}

            \begin{tcolorbox}[breakable, size=fbox, boxrule=.5pt, pad at break*=1mm, opacityfill=0]
\prompt{Out}{outcolor}{45}{\boxspacing}
\begin{Verbatim}[commandchars=\\\{\}]
                sum\_sq      df          F        PR(>F)
region     4055.880631     3.0  39.495057  1.881839e-24
Residual  45664.319755  1334.0        NaN           NaN
\end{Verbatim}
\end{tcolorbox}
        
    \textbf{sum\_sq (Сумма квадратов)}: * Для региона: 4055.88 --- это сумма
квадратов отклонений между средними значениями групп. * Для остатков:
45664.32 --- это сумма квадратов отклонений внутри групп.

\textbf{df (Степени свободы)}: * Для региона: 3.0 --- это количество
групп (регионов) минус 1. * Для остатков: 1334.0 --- это общее
количество наблюдений минус количество групп.

Значение \textbf{F} (39.495) аналогично результату, полученному из
scipy, и указывает на значительное различие между средними значениями
групп.PR(\textgreater F) (P-value):

Значение \textbf{PR(\textgreater F) (P-value)} 1.88×10−24 также
подтверждает, что различия между средними значениями BMI в разных
регионах статистически значимы.

Оба метода ANOVA (через Scipy и statsmodels) дают согласующиеся
результаты, показывающие, что различия в BMI между регионами
статистически значимы.

    \hypertarget{section}{%
\section{3.3}\label{section}}

С помощью t критерия Стьюдента перебрать все пары. Определить поправку
Бонферрони. Сделать выводы.

    \begin{tcolorbox}[breakable, size=fbox, boxrule=1pt, pad at break*=1mm,colback=cellbackground, colframe=cellborder]
\prompt{In}{incolor}{47}{\boxspacing}
\begin{Verbatim}[commandchars=\\\{\}]
\PY{k+kn}{from} \PY{n+nn}{itertools} \PY{k+kn}{import} \PY{n}{combinations}
\end{Verbatim}
\end{tcolorbox}

    \begin{tcolorbox}[breakable, size=fbox, boxrule=1pt, pad at break*=1mm,colback=cellbackground, colframe=cellborder]
\prompt{In}{incolor}{50}{\boxspacing}
\begin{Verbatim}[commandchars=\\\{\}]
\PY{n}{pairs} \PY{o}{=} \PY{n+nb}{list}\PY{p}{(}\PY{n}{combinations}\PY{p}{(}\PY{n}{unique\PYZus{}regions}\PY{p}{,} \PY{l+m+mi}{2}\PY{p}{)}\PY{p}{)}
\PY{n}{p\PYZus{}values} \PY{o}{=} \PY{p}{[}\PY{p}{]}
\PY{k}{for} \PY{n}{pair} \PY{o+ow}{in} \PY{n}{pairs}\PY{p}{:}
    \PY{n}{group1} \PY{o}{=} \PY{n}{data}\PY{p}{[}\PY{n}{data}\PY{p}{[}\PY{l+s+s1}{\PYZsq{}}\PY{l+s+s1}{region}\PY{l+s+s1}{\PYZsq{}}\PY{p}{]} \PY{o}{==} \PY{n}{pair}\PY{p}{[}\PY{l+m+mi}{0}\PY{p}{]}\PY{p}{]}\PY{p}{[}\PY{l+s+s1}{\PYZsq{}}\PY{l+s+s1}{bmi}\PY{l+s+s1}{\PYZsq{}}\PY{p}{]}
    \PY{n}{group2} \PY{o}{=} \PY{n}{data}\PY{p}{[}\PY{n}{data}\PY{p}{[}\PY{l+s+s1}{\PYZsq{}}\PY{l+s+s1}{region}\PY{l+s+s1}{\PYZsq{}}\PY{p}{]} \PY{o}{==} \PY{n}{pair}\PY{p}{[}\PY{l+m+mi}{1}\PY{p}{]}\PY{p}{]}\PY{p}{[}\PY{l+s+s1}{\PYZsq{}}\PY{l+s+s1}{bmi}\PY{l+s+s1}{\PYZsq{}}\PY{p}{]}
    \PY{n}{ttest\PYZus{}result} \PY{o}{=} \PY{n}{stats}\PY{o}{.}\PY{n}{ttest\PYZus{}ind}\PY{p}{(}\PY{n}{group1}\PY{p}{,} \PY{n}{group2}\PY{p}{)}
    \PY{n}{p\PYZus{}values}\PY{o}{.}\PY{n}{append}\PY{p}{(}\PY{n}{ttest\PYZus{}result}\PY{o}{.}\PY{n}{pvalue}\PY{p}{)}

\PY{n}{bonferroni\PYZus{}corrected\PYZus{}p\PYZus{}values} \PY{o}{=} \PY{p}{[}\PY{n}{p} \PY{o}{*} \PY{n+nb}{len}\PY{p}{(}\PY{n}{pairs}\PY{p}{)} \PY{k}{for} \PY{n}{p} \PY{o+ow}{in} \PY{n}{p\PYZus{}values}\PY{p}{]}
\PY{n}{bonferroni\PYZus{}corrected\PYZus{}p\PYZus{}values}
\end{Verbatim}
\end{tcolorbox}

            \begin{tcolorbox}[breakable, size=fbox, boxrule=.5pt, pad at break*=1mm, opacityfill=0]
\prompt{Out}{outcolor}{50}{\boxspacing}
\begin{Verbatim}[commandchars=\\\{\}]
[3.2624405783808385e-08,
 0.006461750977846171,
 0.011451697002943843,
 1.5861428431380637e-18,
 7.116089624548878e-17,
 5.711575024931184]
\end{Verbatim}
\end{tcolorbox}
        
    Поправленные \textbf{p-значения}: Поскольку у нас 6 пар, мы умножили
каждое исходное p-значение на 6 (поправка Бонферрони). Результаты
показали, что все скорректированные p-значения по-прежнему остаются
крайне малыми.

Все скорректированные p-значения значительно ниже уровня значимости
0.05, что указывает на статистически значимые различия в BMI между всеми
парами регионов.

Каждая пара регионов показывает, что их средние значения BMI
статистически различаются. Это подтверждает, что влияние региона на BMI
действительно является значимым.

    \hypertarget{section}{%
\section{3.4}\label{section}}

Выполнить пост-хок тесты Тьюки и построить график.

    \begin{tcolorbox}[breakable, size=fbox, boxrule=1pt, pad at break*=1mm,colback=cellbackground, colframe=cellborder]
\prompt{In}{incolor}{53}{\boxspacing}
\begin{Verbatim}[commandchars=\\\{\}]
\PY{n}{tukey\PYZus{}result} \PY{o}{=} \PY{n}{pairwise\PYZus{}tukeyhsd}\PY{p}{(}\PY{n}{endog}\PY{o}{=}\PY{n}{data}\PY{p}{[}\PY{l+s+s1}{\PYZsq{}}\PY{l+s+s1}{bmi}\PY{l+s+s1}{\PYZsq{}}\PY{p}{]}\PY{p}{,} \PY{n}{groups}\PY{o}{=}\PY{n}{data}\PY{p}{[}\PY{l+s+s1}{\PYZsq{}}\PY{l+s+s1}{region}\PY{l+s+s1}{\PYZsq{}}\PY{p}{]}\PY{p}{,} \PY{n}{alpha}\PY{o}{=}\PY{l+m+mf}{0.05}\PY{p}{)}
\PY{n+nb}{print}\PY{p}{(}\PY{n}{tukey\PYZus{}result}\PY{p}{)}
\end{Verbatim}
\end{tcolorbox}

    \begin{Verbatim}[commandchars=\\\{\}]
   Multiple Comparison of Means - Tukey HSD, FWER=0.05
==========================================================
  group1    group2  meandiff p-adj   lower   upper  reject
----------------------------------------------------------
northeast northwest   0.0263 0.9999 -1.1552  1.2078  False
northeast southeast   4.1825    0.0   3.033   5.332   True
northeast southwest   1.4231 0.0107  0.2416  2.6046   True
northwest southeast   4.1562    0.0  3.0077  5.3047   True
northwest southwest   1.3968 0.0127  0.2162  2.5774   True
southeast southwest  -2.7594    0.0 -3.9079 -1.6108   True
----------------------------------------------------------
    \end{Verbatim}

    \begin{tcolorbox}[breakable, size=fbox, boxrule=1pt, pad at break*=1mm,colback=cellbackground, colframe=cellborder]
\prompt{In}{incolor}{54}{\boxspacing}
\begin{Verbatim}[commandchars=\\\{\}]
\PY{n}{tukey\PYZus{}result}\PY{o}{.}\PY{n}{plot\PYZus{}simultaneous}\PY{p}{(}\PY{p}{)}
\PY{n}{plt}\PY{o}{.}\PY{n}{title}\PY{p}{(}\PY{l+s+s1}{\PYZsq{}}\PY{l+s+s1}{Tukey HSD Test: BMI by Region}\PY{l+s+s1}{\PYZsq{}}\PY{p}{)}
\PY{n}{plt}\PY{o}{.}\PY{n}{show}\PY{p}{(}\PY{p}{)}
\end{Verbatim}
\end{tcolorbox}

    \begin{center}
    \adjustimage{max size={0.9\linewidth}{0.9\paperheight}}{output_38_0.png}
    \end{center}
    { \hspace*{\fill} \\}
    
    Различия между ``northeast'' и ``southeast'' (p \textless{} 0.001) и
``southwest'' (p = 0.0107) статистически значимы.

Различия между ``northwest'' и ``southeast'' (p \textless{} 0.001) и
``southwest'' (p = 0.0127) также статистически значимы.

Различие между ``southeast'' и ``southwest'' (p \textless{} 0.001) также
значимо.

Различие между ``northeast'' и ``northwest'' не является значимым (p =
0.9999).

    \hypertarget{section}{%
\section{3.5}\label{section}}

Выполнить двухфакторный ANOVA тест, чтобы проверить влияние региона и
пола на индекс массы тела (BMI), используя функцию anova\_lm() из
библиотеки statsmodels.

    \begin{tcolorbox}[breakable, size=fbox, boxrule=1pt, pad at break*=1mm,colback=cellbackground, colframe=cellborder]
\prompt{In}{incolor}{55}{\boxspacing}
\begin{Verbatim}[commandchars=\\\{\}]
\PY{n}{model\PYZus{}two\PYZus{}way} \PY{o}{=} \PY{n}{ols}\PY{p}{(}\PY{l+s+s1}{\PYZsq{}}\PY{l+s+s1}{bmi \PYZti{} C(region) * C(sex)}\PY{l+s+s1}{\PYZsq{}}\PY{p}{,} \PY{n}{data}\PY{o}{=}\PY{n}{data}\PY{p}{)}\PY{o}{.}\PY{n}{fit}\PY{p}{(}\PY{p}{)}
\PY{n}{anova\PYZus{}two\PYZus{}way} \PY{o}{=} \PY{n}{sm}\PY{o}{.}\PY{n}{stats}\PY{o}{.}\PY{n}{anova\PYZus{}lm}\PY{p}{(}\PY{n}{model\PYZus{}two\PYZus{}way}\PY{p}{,} \PY{n}{typ}\PY{o}{=}\PY{l+m+mi}{2}\PY{p}{)}
\PY{n}{anova\PYZus{}two\PYZus{}way}
\end{Verbatim}
\end{tcolorbox}

            \begin{tcolorbox}[breakable, size=fbox, boxrule=.5pt, pad at break*=1mm, opacityfill=0]
\prompt{Out}{outcolor}{55}{\boxspacing}
\begin{Verbatim}[commandchars=\\\{\}]
                        sum\_sq      df          F        PR(>F)
C(region)          4034.975135     3.0  39.398134  2.163195e-24
C(sex)               86.007035     1.0   2.519359  1.126940e-01
C(region):C(sex)    174.157808     3.0   1.700504  1.650655e-01
Residual          45404.154911  1330.0        NaN           NaN
\end{Verbatim}
\end{tcolorbox}
        
    C(region): * sum\_sq: 4034.975 --- сумма квадратов для различий между
регионами. * F: 39.398 --- указывает на значительное различие между
средними значениями BMI по регионам. * PR(\textgreater F): 2.16×10−24
--- очень низкое значение, что говорит о статистически значимом влиянии
региона на BMI.

C(sex): * sum\_sq: 86.007 --- сумма квадратов для различий между полами.
* F: 2.519 --- не столь высокое значение по сравнению с регионом. *
PR(\textgreater F): 0.1127 --- значение выше 0.05, что указывает на
отсутствие значительного влияния пола на BMI.

C(region):C(sex): * sum\_sq: 174.158 --- сумма квадратов для
взаимодействия региона и пола. * F: 1.701 --- также незначительное
значение. * PR(\textgreater F): 0.1651 --- незначимое влияние
взаимодействия региона и пола на BMI.

Регион имеет статистически значимое влияние на индекс массы тела (BMI).
Пол не показывает статистически значимого влияния на BMI. Взаимодействие
между регионом и полом также не является значимым.

    \hypertarget{section}{%
\section{3.6}\label{section}}

Выполнить пост-хок тесты Тьюки и построить график.

    \begin{tcolorbox}[breakable, size=fbox, boxrule=1pt, pad at break*=1mm,colback=cellbackground, colframe=cellborder]
\prompt{In}{incolor}{59}{\boxspacing}
\begin{Verbatim}[commandchars=\\\{\}]
\PY{n}{data}\PY{p}{[}\PY{l+s+s1}{\PYZsq{}}\PY{l+s+s1}{combination}\PY{l+s+s1}{\PYZsq{}}\PY{p}{]} \PY{o}{=} \PY{n}{data}\PY{p}{[}\PY{l+s+s1}{\PYZsq{}}\PY{l+s+s1}{region}\PY{l+s+s1}{\PYZsq{}}\PY{p}{]} \PY{o}{+} \PY{l+s+s1}{\PYZsq{}}\PY{l+s+s1}{\PYZhy{}}\PY{l+s+s1}{\PYZsq{}} \PY{o}{+} \PY{n}{data}\PY{p}{[}\PY{l+s+s1}{\PYZsq{}}\PY{l+s+s1}{sex}\PY{l+s+s1}{\PYZsq{}}\PY{p}{]}
\PY{n}{tukey\PYZus{}two\PYZus{}way} \PY{o}{=} \PY{n}{pairwise\PYZus{}tukeyhsd}\PY{p}{(}\PY{n}{endog}\PY{o}{=}\PY{n}{data}\PY{p}{[}\PY{l+s+s1}{\PYZsq{}}\PY{l+s+s1}{bmi}\PY{l+s+s1}{\PYZsq{}}\PY{p}{]}\PY{p}{,} \PY{n}{groups}\PY{o}{=}\PY{n}{data}\PY{p}{[}\PY{l+s+s1}{\PYZsq{}}\PY{l+s+s1}{combination}\PY{l+s+s1}{\PYZsq{}}\PY{p}{]}\PY{p}{,} \PY{n}{alpha}\PY{o}{=}\PY{l+m+mf}{0.05}\PY{p}{)}
\PY{n+nb}{print}\PY{p}{(}\PY{n}{tukey\PYZus{}two\PYZus{}way}\PY{p}{)}
\end{Verbatim}
\end{tcolorbox}

    \begin{Verbatim}[commandchars=\\\{\}]
          Multiple Comparison of Means - Tukey HSD, FWER=0.05
========================================================================
     group1           group2      meandiff p-adj   lower   upper  reject
------------------------------------------------------------------------
northeast-female   northeast-male  -0.2998 0.9998 -2.2706  1.6711  False
northeast-female northwest-female  -0.0464    1.0 -2.0142  1.9215  False
northeast-female   northwest-male  -0.2042    1.0 -2.1811  1.7728  False
northeast-female southeast-female   3.3469    0.0    1.41  5.2839   True
northeast-female   southeast-male   4.6657    0.0  2.7634   6.568   True
northeast-female southwest-female   0.7362 0.9497 -1.2377    2.71  False
northeast-female   southwest-male   1.8051 0.1007 -0.1657   3.776  False
  northeast-male northwest-female   0.2534 0.9999 -1.7083  2.2152  False
  northeast-male   northwest-male   0.0956    1.0 -1.8752  2.0665  False
  northeast-male southeast-female   3.6467    0.0  1.7159  5.5775   True
  northeast-male   southeast-male   4.9655    0.0  3.0695  6.8614   True
  northeast-male southwest-female    1.036 0.7515 -0.9318  3.0037  False
  northeast-male   southwest-male   2.1049 0.0258  0.1402  4.0697   True
northwest-female   northwest-male  -0.1578    1.0 -2.1257    1.81  False
northwest-female southeast-female   3.3933    0.0  1.4656   5.321   True
northwest-female   southeast-male    4.712    0.0  2.8192  6.6049   True
northwest-female southwest-female   0.7825 0.9294 -1.1822  2.7473  False
northwest-female   southwest-male   1.8515 0.0806 -0.1103  3.8132  False
  northwest-male southeast-female   3.5511    0.0  1.6141  5.4881   True
  northwest-male   southeast-male   4.8698    0.0  2.9676  6.7721   True
  northwest-male southwest-female   0.9403 0.8354 -1.0335  2.9142  False
  northwest-male   southwest-male   2.0093  0.042  0.0385  3.9801   True
southeast-female   southeast-male   1.3187 0.3823  -0.542  3.1795  False
southeast-female southwest-female  -2.6108 0.0011 -4.5446 -0.6769   True
southeast-female   southwest-male  -1.5418 0.2304 -3.4726   0.389  False
  southeast-male southwest-female  -3.9295    0.0 -5.8286 -2.0304   True
  southeast-male   southwest-male  -2.8606 0.0001 -4.7565 -0.9646   True
southwest-female   southwest-male    1.069 0.7201 -0.8988  3.0367  False
------------------------------------------------------------------------
    \end{Verbatim}

    \begin{tcolorbox}[breakable, size=fbox, boxrule=1pt, pad at break*=1mm,colback=cellbackground, colframe=cellborder]
\prompt{In}{incolor}{60}{\boxspacing}
\begin{Verbatim}[commandchars=\\\{\}]
\PY{n}{tukey\PYZus{}two\PYZus{}way}\PY{o}{.}\PY{n}{plot\PYZus{}simultaneous}\PY{p}{(}\PY{p}{)}
\PY{n}{plt}\PY{o}{.}\PY{n}{title}\PY{p}{(}\PY{l+s+s1}{\PYZsq{}}\PY{l+s+s1}{Tukey HSD Test: BMI by Region and Sex}\PY{l+s+s1}{\PYZsq{}}\PY{p}{)}
\PY{n}{plt}\PY{o}{.}\PY{n}{show}\PY{p}{(}\PY{p}{)}
\end{Verbatim}
\end{tcolorbox}

    \begin{center}
    \adjustimage{max size={0.9\linewidth}{0.9\paperheight}}{output_45_0.png}
    \end{center}
    { \hspace*{\fill} \\}
    
    Пара southeast-female и southwest-female имеет наименьшее значение BMI
среди всех групп, в то время как northeast-female и southeast-male
показали наибольшее значение.

    \begin{tcolorbox}[breakable, size=fbox, boxrule=1pt, pad at break*=1mm,colback=cellbackground, colframe=cellborder]
\prompt{In}{incolor}{ }{\boxspacing}
\begin{Verbatim}[commandchars=\\\{\}]

\end{Verbatim}
\end{tcolorbox}


    % Add a bibliography block to the postdoc
    
    
    
\end{document}
