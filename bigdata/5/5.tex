\documentclass[14pt,a4paper,oneside]{extarticle}

	\usepackage{cmap} % for pdfLaTeX
	\usepackage[T1,T2A]{fontenc} % correct encoding for pdfLaTeX
	\usepackage[utf8]{inputenc} % correct encoding source file
	\usepackage[english,russian]{babel} % correct language

	% -------------------
	% TEXT SETTINGS
	% -------------------

	\usepackage{fontspec} % use standart fonts (only for xelatex!!!)
	\setmainfont{Times New Roman}
	\setsansfont{Arial}
	\setmonofont[Scale=0.6]{Courier New}

	\usepackage[none]{hyphenat} % no word breaks

	\usepackage{setspace}
	%\singlespacing % 1.0
	\onehalfspacing % 1.5

	% -------------------
	% PAGE SETTINGS
	% -------------------

	\usepackage[left=3cm, right=1.5cm, vmargin=2cm, headheight=2cm]{geometry}
	\linespread{1.5} % line spacing 1.5
	\usepackage{indentfirst} % indent first paragraph
	\setlength{\parindent}{1.25cm}
	\sloppy

	\frenchspacing % ???
	\pagestyle{plain}
	\usepackage{fancyhdr} % for headers and footers

	\clubpenalty=10000
	\widowpenalty=10000

	% ------------------
	% ATTACHMENTS SETTINGS
	% ------------------

	\usepackage[labelsep=endash]{caption}
	%\setlength{\abovecaptionskip}{3pt}
	%\setlength{\belowcaptionskip}{3pt}

	% for non-end-to-end numbering
	%\usepackage{chngcntr}
	%\counterwithin{figure}{section}
	%\counterwithin{table}{section}

	% for long table
	\usepackage{longtable}

	% for graphics
	\usepackage{graphicx}
	\newcommand{\rref}[1]{(Рисунок~\ref{#1})}
	\newcommand{\rdref}[2]{(Рисунки~\ref{#1}\,-\,\ref{#2})}
	\newcommand{\tref}[1]{(Таблица~\ref{#1})}
	\newcommand{\tdref}[2]{(Таблици~\ref{#1}\,-\,\ref{#2})}
	\newenvironment{image}{
		\begin{figure}[h!tp]
		\centering
	}{
		\end{figure}
	}
	\newcommand\includegrph[2][width=0.8\textwidth]{\includegraphics[#1]{#2}}

	\usepackage{amsmath} % more flexibility equations (use in title!!!)
	\usepackage{pdfpages} % include pdf


	% ------------------
	% SECTION SETTINGS
	% ------------------

	\usepackage{titlesec}

	% section size
	\titleformat{\chapter}[block]
	  {\fontsize{18pt}{22pt}\bfseries}
	  {\chaptertitlename\ \thechapter.}{0.5em}{}
	\titleformat{\section}
		{\fontsize{16pt}{20pt}\bfseries}{\thesection.}{1em}{}
	\titleformat{\subsection}
		{\fontsize{14pt}{18pt}\bfseries}{\thesubsection.}{1em}{}
	\titleformat{\subsubsection}
		{\fontsize{14pt}{18pt}\bfseries}{\thesubsubsection.}{1em}{}
	\titlespacing{\chapter}{1.25cm}{1pt}{1pt}
	\titlespacing{\section}{1.25cm}{1pt}{1pt}
	\titlespacing{\subsection}{1.25cm}{1pt}{1pt}
	\titlespacing{\subsubsection}{1.25cm}{1pt}{1pt}
	\titlespacing{\paragraph}{1.25cm}{0pt}{0pt}

	% new command for sections
	\newcommand\Chapter[1]{
		\refstepcounter{chapter}
		\chapter*{\textbf{ГЛАВА\;\arabic{chapter}.}
			\raggedright #1
		}
		\addcontentsline{toc}{chapter}{ГЛАВА\;\arabic{chapter}.\ #1}
	}
	\newcommand\Section[1]{
		\refstepcounter{section}
		\section*{\textbf{\arabic{chapter}.\arabic{section}.}
			\raggedright #1
		}
		\addcontentsline{toc}{section}{\arabic{chapter}.\arabic{section}.\ #1}
	}
	\newcommand\Subsection[1]{
		\refstepcounter{subsection}
		\subsection*{\textbf{\arabic{chapter}.\arabic{section}.\arabic{subsection}.}
			\raggedright #1
		}
		\addcontentsline{toc}{subsection}
			{\arabic{chapter}.\arabic{section}.\arabic{subsection}.\ #1}
	}
	\newcommand\Subsubsection[1]{
		\refstepcounter{subsubsection}
		\subsubsection*{\textbf{\arabic{chapter}.\arabic{section}.\arabic{subsection}.\arabic{subsubsection}.}
			\raggedright #1
		}
		\addcontentsline{toc}{subsubsection}
			{\arabic{chapter}.\arabic{section}.\arabic{subsection}.\arabic{subsubsection}.\ #1}
	}

	% new command for appendix
	\newcommand\AppendixChapter[1]{
		\refstepcounter{chapter}
		\chapter*{\textbf{\appendixname\;\thechapter.}
			\raggedright #1
		}
		\addcontentsline{toc}{chapter}{\appendixname~\thechapter.\ #1}
	}
	\newcommand\AppendixSection[1]{
		\refstepcounter{section}
		\section*{\textbf{\appendixname\;\thesection.}
			\raggedright #1
		}
	}
	\newcommand\AppendixSubsection[1]{
		\refstepcounter{subsection}
		\section*{\textbf{\appendixname\;\thesubsection.}
			\raggedright #1
		}
	}

	% ------------------
	% NEW COMMAND
	% ------------------

	\providecommand{\No}{\textnumero}

	% ------------------
	% ------------------

    \usepackage[breakable]{tcolorbox}
    

    \usepackage{graphicx}
    \usepackage{caption}

    \usepackage{xcolor} % Allow colors to be defined
    \usepackage{amsmath} % Equations
    \usepackage{amssymb} % Equations
    \usepackage{geometry} % Used to adjust the document margins

    \usepackage{fancyvrb} % verbatim replacement that allows latex

    \makeatletter % fix for old versions of grffile with XeLaTeX
    \@ifpackagelater{grffile}{2019/11/01}
    {
      % Do nothing on new versions
    }
    {
      \def\Gread@@xetex#1{%
        \IfFileExists{"\Gin@base".bb}%
        {\Gread@eps{\Gin@base.bb}}%
        {\Gread@@xetex@aux#1}%
      }
    }
    \makeatother
    \usepackage[Export]{adjustbox} % Used to constrain images to a maximum size
    \adjustboxset{max size={0.9\linewidth}{0.9\paperheight}}

    % The hyperref package gives us a pdf with properly built
    % internal navigation ('pdf bookmarks' for the table of contents,
    % internal cross-reference links, web links for URLs, etc.)
    \usepackage{hyperref}
    % The default LaTeX title has an obnoxious amount of whitespace. By default,
    % titling removes some of it. It also provides customization options.
    \usepackage{titling}
    \usepackage{longtable} % longtable support required by pandoc >1.10
    \usepackage{booktabs}  % table support for pandoc > 1.12.2
    \usepackage{array}     % table support for pandoc >= 2.11.3
    \usepackage{calc}      % table minipage width calculation for pandoc >= 2.11.1
    \usepackage[inline]{enumitem} % IRkernel/repr support (it uses the enumerate* environment)
    \usepackage[normalem]{ulem} % ulem is needed to support strikethroughs (\sout)
                                % normalem makes italics be italics, not underlines
    \usepackage{mathrsfs}
    

    
    % Colors for the hyperref package
    \definecolor{urlcolor}{rgb}{0,.145,.698}
    \definecolor{linkcolor}{rgb}{.71,0.21,0.01}
    \definecolor{citecolor}{rgb}{.12,.54,.11}

    % ANSI colors
    \definecolor{ansi-black}{HTML}{3E424D}
    \definecolor{ansi-black-intense}{HTML}{282C36}
    \definecolor{ansi-red}{HTML}{E75C58}
    \definecolor{ansi-red-intense}{HTML}{B22B31}
    \definecolor{ansi-green}{HTML}{00A250}
    \definecolor{ansi-green-intense}{HTML}{007427}
    \definecolor{ansi-yellow}{HTML}{DDB62B}
    \definecolor{ansi-yellow-intense}{HTML}{B27D12}
    \definecolor{ansi-blue}{HTML}{208FFB}
    \definecolor{ansi-blue-intense}{HTML}{0065CA}
    \definecolor{ansi-magenta}{HTML}{D160C4}
    \definecolor{ansi-magenta-intense}{HTML}{A03196}
    \definecolor{ansi-cyan}{HTML}{60C6C8}
    \definecolor{ansi-cyan-intense}{HTML}{258F8F}
    \definecolor{ansi-white}{HTML}{C5C1B4}
    \definecolor{ansi-white-intense}{HTML}{A1A6B2}
    \definecolor{ansi-default-inverse-fg}{HTML}{FFFFFF}
    \definecolor{ansi-default-inverse-bg}{HTML}{000000}

    % common color for the border for error outputs.
    \definecolor{outerrorbackground}{HTML}{FFDFDF}

    % commands and environments needed by pandoc snippets
    % extracted from the output of `pandoc -s`
    \providecommand{\tightlist}{%
      \setlength{\itemsep}{0pt}\setlength{\parskip}{0pt}}
    \DefineVerbatimEnvironment{Highlighting}{Verbatim}{commandchars=\\\{\}}
    % Add ',fontsize=\small' for more characters per line
    \newenvironment{Shaded}{}{}
    \newcommand{\KeywordTok}[1]{\textcolor[rgb]{0.00,0.44,0.13}{\textbf{{#1}}}}
    \newcommand{\DataTypeTok}[1]{\textcolor[rgb]{0.56,0.13,0.00}{{#1}}}
    \newcommand{\DecValTok}[1]{\textcolor[rgb]{0.25,0.63,0.44}{{#1}}}
    \newcommand{\BaseNTok}[1]{\textcolor[rgb]{0.25,0.63,0.44}{{#1}}}
    \newcommand{\FloatTok}[1]{\textcolor[rgb]{0.25,0.63,0.44}{{#1}}}
    \newcommand{\CharTok}[1]{\textcolor[rgb]{0.25,0.44,0.63}{{#1}}}
    \newcommand{\StringTok}[1]{\textcolor[rgb]{0.25,0.44,0.63}{{#1}}}
    \newcommand{\CommentTok}[1]{\textcolor[rgb]{0.38,0.63,0.69}{\textit{{#1}}}}
    \newcommand{\OtherTok}[1]{\textcolor[rgb]{0.00,0.44,0.13}{{#1}}}
    \newcommand{\AlertTok}[1]{\textcolor[rgb]{1.00,0.00,0.00}{\textbf{{#1}}}}
    \newcommand{\FunctionTok}[1]{\textcolor[rgb]{0.02,0.16,0.49}{{#1}}}
    \newcommand{\RegionMarkerTok}[1]{{#1}}
    \newcommand{\ErrorTok}[1]{\textcolor[rgb]{1.00,0.00,0.00}{\textbf{{#1}}}}
    \newcommand{\NormalTok}[1]{{#1}}

    % Additional commands for more recent versions of Pandoc
    \newcommand{\ConstantTok}[1]{\textcolor[rgb]{0.53,0.00,0.00}{{#1}}}
    \newcommand{\SpecialCharTok}[1]{\textcolor[rgb]{0.25,0.44,0.63}{{#1}}}
    \newcommand{\VerbatimStringTok}[1]{\textcolor[rgb]{0.25,0.44,0.63}{{#1}}}
    \newcommand{\SpecialStringTok}[1]{\textcolor[rgb]{0.73,0.40,0.53}{{#1}}}
    \newcommand{\ImportTok}[1]{{#1}}
    \newcommand{\DocumentationTok}[1]{\textcolor[rgb]{0.73,0.13,0.13}{\textit{{#1}}}}
    \newcommand{\AnnotationTok}[1]{\textcolor[rgb]{0.38,0.63,0.69}{\textbf{\textit{{#1}}}}}
    \newcommand{\CommentVarTok}[1]{\textcolor[rgb]{0.38,0.63,0.69}{\textbf{\textit{{#1}}}}}
    \newcommand{\VariableTok}[1]{\textcolor[rgb]{0.10,0.09,0.49}{{#1}}}
    \newcommand{\ControlFlowTok}[1]{\textcolor[rgb]{0.00,0.44,0.13}{\textbf{{#1}}}}
    \newcommand{\OperatorTok}[1]{\textcolor[rgb]{0.40,0.40,0.40}{{#1}}}
    \newcommand{\BuiltInTok}[1]{{#1}}
    \newcommand{\ExtensionTok}[1]{{#1}}
    \newcommand{\PreprocessorTok}[1]{\textcolor[rgb]{0.74,0.48,0.00}{{#1}}}
    \newcommand{\AttributeTok}[1]{\textcolor[rgb]{0.49,0.56,0.16}{{#1}}}
    \newcommand{\InformationTok}[1]{\textcolor[rgb]{0.38,0.63,0.69}{\textbf{\textit{{#1}}}}}
    \newcommand{\WarningTok}[1]{\textcolor[rgb]{0.38,0.63,0.69}{\textbf{\textit{{#1}}}}}

    
    
    
    
	% Pygments definitions
	\makeatletter
	\def\PY@reset{\let\PY@it=\relax \let\PY@bf=\relax%
		\let\PY@ul=\relax \let\PY@tc=\relax%
		\let\PY@bc=\relax \let\PY@ff=\relax}
	\def\PY@tok#1{\csname PY@tok@#1\endcsname}
	\def\PY@toks#1+{\ifx\relax#1\empty\else%
		\PY@tok{#1}\expandafter\PY@toks\fi}
	\def\PY@do#1{\PY@bc{\PY@tc{\PY@ul{%
		\PY@it{\PY@bf{\PY@ff{#1}}}}}}}
	\def\PY#1#2{\PY@reset\PY@toks#1+\relax+\PY@do{#2}}

	\@namedef{PY@tok@w}{\def\PY@tc##1{\textcolor[rgb]{0.73,0.73,0.73}{##1}}}
	\@namedef{PY@tok@c}{\let\PY@it=\textit\def\PY@tc##1{\textcolor[rgb]{0.24,0.48,0.48}{##1}}}
	\@namedef{PY@tok@cp}{\def\PY@tc##1{\textcolor[rgb]{0.61,0.40,0.00}{##1}}}
	\@namedef{PY@tok@k}{\let\PY@bf=\textbf\def\PY@tc##1{\textcolor[rgb]{0.00,0.50,0.00}{##1}}}
	\@namedef{PY@tok@kp}{\def\PY@tc##1{\textcolor[rgb]{0.00,0.50,0.00}{##1}}}
	\@namedef{PY@tok@kt}{\def\PY@tc##1{\textcolor[rgb]{0.69,0.00,0.25}{##1}}}
	\@namedef{PY@tok@o}{\def\PY@tc##1{\textcolor[rgb]{0.40,0.40,0.40}{##1}}}
	\@namedef{PY@tok@ow}{\let\PY@bf=\textbf\def\PY@tc##1{\textcolor[rgb]{0.67,0.13,1.00}{##1}}}
	\@namedef{PY@tok@nb}{\def\PY@tc##1{\textcolor[rgb]{0.00,0.50,0.00}{##1}}}
	\@namedef{PY@tok@nf}{\def\PY@tc##1{\textcolor[rgb]{0.00,0.00,1.00}{##1}}}
	\@namedef{PY@tok@nc}{\let\PY@bf=\textbf\def\PY@tc##1{\textcolor[rgb]{0.00,0.00,1.00}{##1}}}
	\@namedef{PY@tok@nn}{\let\PY@bf=\textbf\def\PY@tc##1{\textcolor[rgb]{0.00,0.00,1.00}{##1}}}
	\@namedef{PY@tok@ne}{\let\PY@bf=\textbf\def\PY@tc##1{\textcolor[rgb]{0.80,0.25,0.22}{##1}}}
	\@namedef{PY@tok@nv}{\def\PY@tc##1{\textcolor[rgb]{0.10,0.09,0.49}{##1}}}
	\@namedef{PY@tok@no}{\def\PY@tc##1{\textcolor[rgb]{0.53,0.00,0.00}{##1}}}
	\@namedef{PY@tok@nl}{\def\PY@tc##1{\textcolor[rgb]{0.46,0.46,0.00}{##1}}}
	\@namedef{PY@tok@ni}{\let\PY@bf=\textbf\def\PY@tc##1{\textcolor[rgb]{0.44,0.44,0.44}{##1}}}
	\@namedef{PY@tok@na}{\def\PY@tc##1{\textcolor[rgb]{0.41,0.47,0.13}{##1}}}
	\@namedef{PY@tok@nt}{\let\PY@bf=\textbf\def\PY@tc##1{\textcolor[rgb]{0.00,0.50,0.00}{##1}}}
	\@namedef{PY@tok@nd}{\def\PY@tc##1{\textcolor[rgb]{0.67,0.13,1.00}{##1}}}
	\@namedef{PY@tok@s}{\def\PY@tc##1{\textcolor[rgb]{0.73,0.13,0.13}{##1}}}
	\@namedef{PY@tok@sd}{\let\PY@it=\textit\def\PY@tc##1{\textcolor[rgb]{0.73,0.13,0.13}{##1}}}
	\@namedef{PY@tok@si}{\let\PY@bf=\textbf\def\PY@tc##1{\textcolor[rgb]{0.64,0.35,0.47}{##1}}}
	\@namedef{PY@tok@se}{\let\PY@bf=\textbf\def\PY@tc##1{\textcolor[rgb]{0.67,0.36,0.12}{##1}}}
	\@namedef{PY@tok@sr}{\def\PY@tc##1{\textcolor[rgb]{0.64,0.35,0.47}{##1}}}
	\@namedef{PY@tok@ss}{\def\PY@tc##1{\textcolor[rgb]{0.10,0.09,0.49}{##1}}}
	\@namedef{PY@tok@sx}{\def\PY@tc##1{\textcolor[rgb]{0.00,0.50,0.00}{##1}}}
	\@namedef{PY@tok@m}{\def\PY@tc##1{\textcolor[rgb]{0.40,0.40,0.40}{##1}}}
	\@namedef{PY@tok@gh}{\let\PY@bf=\textbf\def\PY@tc##1{\textcolor[rgb]{0.00,0.00,0.50}{##1}}}
	\@namedef{PY@tok@gu}{\let\PY@bf=\textbf\def\PY@tc##1{\textcolor[rgb]{0.50,0.00,0.50}{##1}}}
	\@namedef{PY@tok@gd}{\def\PY@tc##1{\textcolor[rgb]{0.63,0.00,0.00}{##1}}}
	\@namedef{PY@tok@gi}{\def\PY@tc##1{\textcolor[rgb]{0.00,0.52,0.00}{##1}}}
	\@namedef{PY@tok@gr}{\def\PY@tc##1{\textcolor[rgb]{0.89,0.00,0.00}{##1}}}
	\@namedef{PY@tok@ge}{\let\PY@it=\textit}
	\@namedef{PY@tok@gs}{\let\PY@bf=\textbf}
	\@namedef{PY@tok@gp}{\let\PY@bf=\textbf\def\PY@tc##1{\textcolor[rgb]{0.00,0.00,0.50}{##1}}}
	\@namedef{PY@tok@go}{\def\PY@tc##1{\textcolor[rgb]{0.44,0.44,0.44}{##1}}}
	\@namedef{PY@tok@gt}{\def\PY@tc##1{\textcolor[rgb]{0.00,0.27,0.87}{##1}}}
	\@namedef{PY@tok@err}{\def\PY@bc##1{{\setlength{\fboxsep}{\string -\fboxrule}\fcolorbox[rgb]{1.00,0.00,0.00}{1,1,1}{\strut ##1}}}}
	\@namedef{PY@tok@kc}{\let\PY@bf=\textbf\def\PY@tc##1{\textcolor[rgb]{0.00,0.50,0.00}{##1}}}
	\@namedef{PY@tok@kd}{\let\PY@bf=\textbf\def\PY@tc##1{\textcolor[rgb]{0.00,0.50,0.00}{##1}}}
	\@namedef{PY@tok@kn}{\let\PY@bf=\textbf\def\PY@tc##1{\textcolor[rgb]{0.00,0.50,0.00}{##1}}}
	\@namedef{PY@tok@kr}{\let\PY@bf=\textbf\def\PY@tc##1{\textcolor[rgb]{0.00,0.50,0.00}{##1}}}
	\@namedef{PY@tok@bp}{\def\PY@tc##1{\textcolor[rgb]{0.00,0.50,0.00}{##1}}}
	\@namedef{PY@tok@fm}{\def\PY@tc##1{\textcolor[rgb]{0.00,0.00,1.00}{##1}}}
	\@namedef{PY@tok@vc}{\def\PY@tc##1{\textcolor[rgb]{0.10,0.09,0.49}{##1}}}
	\@namedef{PY@tok@vg}{\def\PY@tc##1{\textcolor[rgb]{0.10,0.09,0.49}{##1}}}
	\@namedef{PY@tok@vi}{\def\PY@tc##1{\textcolor[rgb]{0.10,0.09,0.49}{##1}}}
	\@namedef{PY@tok@vm}{\def\PY@tc##1{\textcolor[rgb]{0.10,0.09,0.49}{##1}}}
	\@namedef{PY@tok@sa}{\def\PY@tc##1{\textcolor[rgb]{0.73,0.13,0.13}{##1}}}
	\@namedef{PY@tok@sb}{\def\PY@tc##1{\textcolor[rgb]{0.73,0.13,0.13}{##1}}}
	\@namedef{PY@tok@sc}{\def\PY@tc##1{\textcolor[rgb]{0.73,0.13,0.13}{##1}}}
	\@namedef{PY@tok@dl}{\def\PY@tc##1{\textcolor[rgb]{0.73,0.13,0.13}{##1}}}
	\@namedef{PY@tok@s2}{\def\PY@tc##1{\textcolor[rgb]{0.73,0.13,0.13}{##1}}}
	\@namedef{PY@tok@sh}{\def\PY@tc##1{\textcolor[rgb]{0.73,0.13,0.13}{##1}}}
	\@namedef{PY@tok@s1}{\def\PY@tc##1{\textcolor[rgb]{0.73,0.13,0.13}{##1}}}
	\@namedef{PY@tok@mb}{\def\PY@tc##1{\textcolor[rgb]{0.40,0.40,0.40}{##1}}}
	\@namedef{PY@tok@mf}{\def\PY@tc##1{\textcolor[rgb]{0.40,0.40,0.40}{##1}}}
	\@namedef{PY@tok@mh}{\def\PY@tc##1{\textcolor[rgb]{0.40,0.40,0.40}{##1}}}
	\@namedef{PY@tok@mi}{\def\PY@tc##1{\textcolor[rgb]{0.40,0.40,0.40}{##1}}}
	\@namedef{PY@tok@il}{\def\PY@tc##1{\textcolor[rgb]{0.40,0.40,0.40}{##1}}}
	\@namedef{PY@tok@mo}{\def\PY@tc##1{\textcolor[rgb]{0.40,0.40,0.40}{##1}}}
	\@namedef{PY@tok@ch}{\let\PY@it=\textit\def\PY@tc##1{\textcolor[rgb]{0.24,0.48,0.48}{##1}}}
	\@namedef{PY@tok@cm}{\let\PY@it=\textit\def\PY@tc##1{\textcolor[rgb]{0.24,0.48,0.48}{##1}}}
	\@namedef{PY@tok@cpf}{\let\PY@it=\textit\def\PY@tc##1{\textcolor[rgb]{0.24,0.48,0.48}{##1}}}
	\@namedef{PY@tok@c1}{\let\PY@it=\textit\def\PY@tc##1{\textcolor[rgb]{0.24,0.48,0.48}{##1}}}
	\@namedef{PY@tok@cs}{\let\PY@it=\textit\def\PY@tc##1{\textcolor[rgb]{0.24,0.48,0.48}{##1}}}

	\def\PYZbs{\char`\\}
	\def\PYZus{\char`\_}
	\def\PYZob{\char`\{}
	\def\PYZcb{\char`\}}
	\def\PYZca{\char`\^}
	\def\PYZam{\char`\&}
	\def\PYZlt{\char`\<}
	\def\PYZgt{\char`\>}
	\def\PYZsh{\char`\#}
	\def\PYZpc{\char`\%}
	\def\PYZdl{\char`\$}
	\def\PYZhy{\char`\-}
	\def\PYZsq{\char`\'}
	\def\PYZdq{\char`\"}
	\def\PYZti{\char`\~}
	% for compatibility with earlier versions
	\def\PYZat{@}
	\def\PYZlb{[}
	\def\PYZrb{]}
	\makeatother


    % For linebreaks inside Verbatim environment from package fancyvrb.
    \makeatletter
        \newbox\Wrappedcontinuationbox
        \newbox\Wrappedvisiblespacebox
        \newcommand*\Wrappedvisiblespace {\textcolor{red}{\textvisiblespace}}
        \newcommand*\Wrappedcontinuationsymbol {\textcolor{red}{\llap{\tiny$\m@th\hookrightarrow$}}}
        \newcommand*\Wrappedcontinuationindent {3ex }
        \newcommand*\Wrappedafterbreak {\kern\Wrappedcontinuationindent\copy\Wrappedcontinuationbox}
        % Take advantage of the already applied Pygments mark-up to insert
        % potential linebreaks for TeX processing.
        %        {, <, #, %, $, ' and ": go to next line.
        %        _, }, ^, &, >, - and ~: stay at end of broken line.
        % Use of \textquotesingle for straight quote.
        \newcommand*\Wrappedbreaksatspecials {%
            \def\PYGZus{\discretionary{\char`\_}{\Wrappedafterbreak}{\char`\_}}%
            \def\PYGZob{\discretionary{}{\Wrappedafterbreak\char`\{}{\char`\{}}%
            \def\PYGZcb{\discretionary{\char`\}}{\Wrappedafterbreak}{\char`\}}}%
            \def\PYGZca{\discretionary{\char`\^}{\Wrappedafterbreak}{\char`\^}}%
            \def\PYGZam{\discretionary{\char`\&}{\Wrappedafterbreak}{\char`\&}}%
            \def\PYGZlt{\discretionary{}{\Wrappedafterbreak\char`\<}{\char`\<}}%
            \def\PYGZgt{\discretionary{\char`\>}{\Wrappedafterbreak}{\char`\>}}%
            \def\PYGZsh{\discretionary{}{\Wrappedafterbreak\char`\#}{\char`\#}}%
            \def\PYGZpc{\discretionary{}{\Wrappedafterbreak\char`\%}{\char`\%}}%
            \def\PYGZdl{\discretionary{}{\Wrappedafterbreak\char`\$}{\char`\$}}%
            \def\PYGZhy{\discretionary{\char`\-}{\Wrappedafterbreak}{\char`\-}}%
            \def\PYGZsq{\discretionary{}{\Wrappedafterbreak\textquotesingle}{\textquotesingle}}%
            \def\PYGZdq{\discretionary{}{\Wrappedafterbreak\char`\"}{\char`\"}}%
            \def\PYGZti{\discretionary{\char`\~}{\Wrappedafterbreak}{\char`\~}}%
        }
        % Some characters . , ; ? ! / are not pygmentized.
        % This macro makes them "active" and they will insert potential linebreaks
        \newcommand*\Wrappedbreaksatpunct {%
            \lccode`\~`\.\lowercase{\def~}{\discretionary{\hbox{\char`\.}}{\Wrappedafterbreak}{\hbox{\char`\.}}}%
            \lccode`\~`\,\lowercase{\def~}{\discretionary{\hbox{\char`\,}}{\Wrappedafterbreak}{\hbox{\char`\,}}}%
            \lccode`\~`\;\lowercase{\def~}{\discretionary{\hbox{\char`\;}}{\Wrappedafterbreak}{\hbox{\char`\;}}}%
            \lccode`\~`\:\lowercase{\def~}{\discretionary{\hbox{\char`\:}}{\Wrappedafterbreak}{\hbox{\char`\:}}}%
            \lccode`\~`\?\lowercase{\def~}{\discretionary{\hbox{\char`\?}}{\Wrappedafterbreak}{\hbox{\char`\?}}}%
            \lccode`\~`\!\lowercase{\def~}{\discretionary{\hbox{\char`\!}}{\Wrappedafterbreak}{\hbox{\char`\!}}}%
            \lccode`\~`\/\lowercase{\def~}{\discretionary{\hbox{\char`\/}}{\Wrappedafterbreak}{\hbox{\char`\/}}}%
            \catcode`\.\active
            \catcode`\,\active
            \catcode`\;\active
            \catcode`\:\active
            \catcode`\?\active
            \catcode`\!\active
            \catcode`\/\active
            \lccode`\~`\~
        }
    \makeatother

    \let\OriginalVerbatim=\Verbatim
    \makeatletter
    \renewcommand{\Verbatim}[1][1]{%
        %\parskip\z@skip
        \sbox\Wrappedcontinuationbox {\Wrappedcontinuationsymbol}%
        \sbox\Wrappedvisiblespacebox {\FV@SetupFont\Wrappedvisiblespace}%
        \def\FancyVerbFormatLine ##1{\hsize\linewidth
            \vtop{\raggedright\hyphenpenalty\z@\exhyphenpenalty\z@
                \doublehyphendemerits\z@\finalhyphendemerits\z@
                \strut ##1\strut}%
        }%
        % If the linebreak is at a space, the latter will be displayed as visible
        % space at end of first line, and a continuation symbol starts next line.
        % Stretch/shrink are however usually zero for typewriter font.
        \def\FV@Space {%
            \nobreak\hskip\z@ plus\fontdimen3\font minus\fontdimen4\font
            \discretionary{\copy\Wrappedvisiblespacebox}{\Wrappedafterbreak}
            {\kern\fontdimen2\font}%
        }%

        % Allow breaks at special characters using \PYG... macros.
        \Wrappedbreaksatspecials
        % Breaks at punctuation characters . , ; ? ! and / need catcode=\active
        \OriginalVerbatim[#1,codes*=\Wrappedbreaksatpunct]%
    }
    \makeatother

    % Exact colors from NB
    \definecolor{incolor}{HTML}{303F9F}
    \definecolor{outcolor}{HTML}{D84315}
    \definecolor{cellborder}{HTML}{CFCFCF}
    \definecolor{cellbackground}{HTML}{F7F7F7}

    % prompt
    \makeatletter
    \newcommand{\boxspacing}{\kern\kvtcb@left@rule\kern\kvtcb@boxsep}
    \makeatother
    \newcommand{\prompt}[4]{
        {\ttfamily\llap{{\color{#2}[#3]:\hspace{3pt}#4}}\vspace{-\baselineskip}}
    }
    

    
    % Prevent overflowing lines due to hard-to-break entities
    \sloppy
    % Setup hyperref package
    \hypersetup{
      breaklinks=true,  % so long urls are correctly broken across lines
      colorlinks=true,
      urlcolor=urlcolor,
      linkcolor=linkcolor,
      citecolor=citecolor,
      }
    

\begin{document}
    
\singlespacing % 1.0
\begin{titlepage}
	\thispagestyle{fancy}
	\renewcommand{\headrulewidth}{0pt}
	\cfoot{Москва 2024}

	\centering
	\includegraphics[scale=0.75]{./res/logo2} \break % вставка логотипа
	{\footnotesize МИНИСТЕРСТВО НАУКИ
		И ВЫСШЕГО ОБРАЗОВАНИЯ РОССИЙСКОЙ ФЕДЕРАЦИИ}\\
	Федеральное государственное бюджетное образовательное учреждение 
		высшего образования\\
	\textbf{<<МИРЭА --- Российский технологический университет>>}\\
	\vfill
	\textbf{\large РТУ МИРЭА}\\
	\bigskip \hrule \smallskip \hrule \smallskip
	\vfill
	Институт информационных технологий (ИТ)\\
	Математического обеспечения
		и стандартизации информационных технологий (МОСИТ)\\
	\vfill
	\textbf{ОТЧЕТ ПО ПРАКТИЧЕСКОЙ РАБОТЕ \No\,1}\\
	\textbf{по дисциплине}\\
	\textbf{<<Разработка кроссплатформенных мобильных приложений>>}\\
	\vfill
	\vfill
	\vfill
	\vfill
	\begin{tabular}{p{0.7\textwidth}p{0.2\textwidth}}
		Выполнил студент группы ИКБО-06-21 & \rightline{Бондарь А.Р.} \\
		Принял старший преподаватель & \rightline{Шешуков Л.С.} \\
	\end{tabular}
	\vfill
	\vfill
	\vfill
	\vfill
\end{titlepage}
\onehalfspacing % 1.5
\setcounter{page}{2}
\clearpage


    
    \hypertarget{ux43fux43bux430ux43d-ux43fux440ux430ux43aux442ux438ux447ux435ux441ux43aux43eux439-ux440ux430ux431ux43eux442ux44b}{%
\section*{\Large{План практической работы}}\label{ux43fux43bux430ux43d-ux43fux440ux430ux43aux442ux438ux447ux435ux441ux43aux43eux439-ux440ux430ux431ux43eux442ux44b}}

    \begin{enumerate}
\def\labelenumi{\arabic{enumi}.}
\tightlist
\item
  Найти данные для классификации. Данные в группе повторяться не должны.
  Предобработать данные, если это необходимо.
\item
  Изобразить гистограмму, которая показывает баланс классов. Сделать
  выводы.
\item
  Разбить выборку на тренировочную и тестовую. Тренировочная для
  обучения модели, тестовая для проверки ее качества.
\item
  Применить алгоритмы классификации: логистическая регрессия, SVM, KNN.
  Построить матрицу ошибок по результатам работы моделей (использовать
  confusion\_matrix из sklearn.metrics).
\item
  Сравнить результаты классификации, используя accuracy, precision,
  recall и f1-меру (можно использовать classification\_report из
  sklearn.metrics). Сделать выводы.
\item
  Оформить отчет о проделанной работе.
\end{enumerate}

\clearpage

    \begin{tcolorbox}[breakable, size=fbox, boxrule=1pt, pad at break*=1mm,colback=cellbackground, colframe=cellborder]
\prompt{In}{incolor}{1}{\boxspacing}
\begin{Verbatim}[commandchars=\\\{\}]
\PY{o}{!}pip install seaborn
\end{Verbatim}
\end{tcolorbox}

    \begin{Verbatim}[commandchars=\\\{\}]
Defaulting to user installation because normal site-packages is not writeable
Requirement already satisfied: seaborn in
/home/arbon/.local/lib/python3.10/site-packages (0.13.2)
...
    \end{Verbatim}

    \hypertarget{ux434ux430ux43dux43dux44bux435-ux434ux43bux44f-ux43aux43bux430ux441ux441ux438ux444ux438ux43aux430ux446ux438ux438}{%
\section{Данные для
классификации}\label{ux434ux430ux43dux43dux44bux435-ux434ux43bux44f-ux43aux43bux430ux441ux441ux438ux444ux438ux43aux430ux446ux438ux438}}

    \begin{tcolorbox}[breakable, size=fbox, boxrule=1pt, pad at break*=1mm,colback=cellbackground, colframe=cellborder]
\prompt{In}{incolor}{2}{\boxspacing}
\begin{Verbatim}[commandchars=\\\{\}]
\PY{k+kn}{import} \PY{n+nn}{pandas} \PY{k}{as} \PY{n+nn}{pd}
\PY{k+kn}{import} \PY{n+nn}{numpy} \PY{k}{as} \PY{n+nn}{np}
\end{Verbatim}
\end{tcolorbox}

    \begin{tcolorbox}[breakable, size=fbox, boxrule=1pt, pad at break*=1mm,colback=cellbackground, colframe=cellborder]
\prompt{In}{incolor}{3}{\boxspacing}
\begin{Verbatim}[commandchars=\\\{\}]
\PY{n}{df} \PY{o}{=} \PY{n}{pd}\PY{o}{.}\PY{n}{read\PYZus{}csv}\PY{p}{(}\PY{l+s+s2}{\PYZdq{}}\PY{l+s+s2}{./5/drug200.csv}\PY{l+s+s2}{\PYZdq{}}\PY{p}{)}
\PY{n}{target\PYZus{}colomn} \PY{o}{=} \PY{l+s+s2}{\PYZdq{}}\PY{l+s+s2}{Drug}\PY{l+s+s2}{\PYZdq{}}
\PY{n}{df}
\end{Verbatim}
\end{tcolorbox}

            \begin{tcolorbox}[breakable, size=fbox, boxrule=.5pt, pad at break*=1mm, opacityfill=0]
\prompt{Out}{outcolor}{3}{\boxspacing}
\begin{Verbatim}[commandchars=\\\{\}]
     Age Sex      BP Cholesterol  Na\_to\_K   Drug
0     23   F    HIGH        HIGH   25.355  DrugY
1     47   M     LOW        HIGH   13.093  drugC
2     47   M     LOW        HIGH   10.114  drugC
3     28   F  NORMAL        HIGH    7.798  drugX
4     61   F     LOW        HIGH   18.043  DrugY
..   {\ldots}  ..     {\ldots}         {\ldots}      {\ldots}    {\ldots}
195   56   F     LOW        HIGH   11.567  drugC
196   16   M     LOW        HIGH   12.006  drugC
197   52   M  NORMAL        HIGH    9.894  drugX
198   23   M  NORMAL      NORMAL   14.020  drugX
199   40   F     LOW      NORMAL   11.349  drugX

[200 rows x 6 columns]
\end{Verbatim}
\end{tcolorbox}
        
    \hypertarget{ux43fux440ux435ux434ux43eux431ux440ux430ux431ux43eux442ux43aux430}{%
\section{Предобработка}\label{ux43fux440ux435ux434ux43eux431ux440ux430ux431ux43eux442ux43aux430}}

    \begin{tcolorbox}[breakable, size=fbox, boxrule=1pt, pad at break*=1mm,colback=cellbackground, colframe=cellborder]
\prompt{In}{incolor}{4}{\boxspacing}
\begin{Verbatim}[commandchars=\\\{\}]
\PY{n}{df}\PY{o}{.}\PY{n}{info}\PY{p}{(}\PY{p}{)}
\end{Verbatim}
\end{tcolorbox}

    \begin{Verbatim}[commandchars=\\\{\}]
<class 'pandas.core.frame.DataFrame'>
RangeIndex: 200 entries, 0 to 199
Data columns (total 6 columns):
 \#   Column       Non-Null Count  Dtype
---  ------       --------------  -----
 0   Age          200 non-null    int64
 1   Sex          200 non-null    object
 2   BP           200 non-null    object
 3   Cholesterol  200 non-null    object
 4   Na\_to\_K      200 non-null    float64
 5   Drug         200 non-null    object
dtypes: float64(1), int64(1), object(4)
memory usage: 9.5+ KB
    \end{Verbatim}

    Проверили наличие пропущенных значений в наборе данных:

    \begin{tcolorbox}[breakable, size=fbox, boxrule=1pt, pad at break*=1mm,colback=cellbackground, colframe=cellborder]
\prompt{In}{incolor}{5}{\boxspacing}
\begin{Verbatim}[commandchars=\\\{\}]
\PY{n}{df}\PY{o}{.}\PY{n}{isnull}\PY{p}{(}\PY{p}{)}\PY{o}{.}\PY{n}{any}\PY{p}{(}\PY{p}{)}
\end{Verbatim}
\end{tcolorbox}

            \begin{tcolorbox}[breakable, size=fbox, boxrule=.5pt, pad at break*=1mm, opacityfill=0]
\prompt{Out}{outcolor}{5}{\boxspacing}
\begin{Verbatim}[commandchars=\\\{\}]
Age            False
Sex            False
BP             False
Cholesterol    False
Na\_to\_K        False
Drug           False
dtype: bool
\end{Verbatim}
\end{tcolorbox}
        
    Поскольку многие алгоритмы машинного обучения работают только с
числовыми данными, необходимо преобразовать категориальные переменные
(object) в числовой формат.

Для этого используем \textbf{LabelEncoder} из библиотеки sklearn,
который позволяет быстро закодировать текстовые метки в числа:

    \begin{tcolorbox}[breakable, size=fbox, boxrule=1pt, pad at break*=1mm,colback=cellbackground, colframe=cellborder]
\prompt{In}{incolor}{6}{\boxspacing}
\begin{Verbatim}[commandchars=\\\{\}]
\PY{k+kn}{from} \PY{n+nn}{sklearn}\PY{n+nn}{.}\PY{n+nn}{preprocessing} \PY{k+kn}{import} \PY{n}{LabelEncoder}
\end{Verbatim}
\end{tcolorbox}

    \begin{tcolorbox}[breakable, size=fbox, boxrule=1pt, pad at break*=1mm,colback=cellbackground, colframe=cellborder]
\prompt{In}{incolor}{7}{\boxspacing}
\begin{Verbatim}[commandchars=\\\{\}]
\PY{n}{le}\PY{o}{=}\PY{n}{LabelEncoder}\PY{p}{(}\PY{p}{)}
\PY{k}{for} \PY{n}{col} \PY{o+ow}{in} \PY{n}{df}\PY{o}{.}\PY{n}{columns}\PY{p}{[}\PY{n}{df}\PY{o}{.}\PY{n}{dtypes}\PY{o}{==}\PY{l+s+s1}{\PYZsq{}}\PY{l+s+s1}{object}\PY{l+s+s1}{\PYZsq{}}\PY{p}{]}\PY{p}{:}
    \PY{n}{df}\PY{p}{[}\PY{n}{col}\PY{p}{]}\PY{o}{=}\PY{n}{le}\PY{o}{.}\PY{n}{fit\PYZus{}transform}\PY{p}{(}\PY{n}{df}\PY{p}{[}\PY{n}{col}\PY{p}{]}\PY{p}{)}
\end{Verbatim}
\end{tcolorbox}

    Каждая уникальная категория в столбцах, содержащих текстовые данные
(Sex, BP, Cholesterol, Drug), преобразуется в числовое представление.
Например:

\begin{itemize}
	\item Sex: ``Female'' может быть закодирован как 0,
		а ``Male'' как 1.
	\item BP: разные категории давления
		(например, ``Low'', ``Normal'', ``High'') будут преобразованы в числа.
	\item Cholesterol:
		``Normal'' и ``High'' также будут преобразованы в числа.
	\item Drug: классы лекарств будут преобразованы в числовые значения.
\end{itemize}

    После кодирования проверяем информацию о DataFrame с помощью df.info(),
чтобы убедиться, что все категориальные переменные были успешно
преобразованы в числовой формат:

    \begin{tcolorbox}[breakable, size=fbox, boxrule=1pt, pad at break*=1mm,colback=cellbackground, colframe=cellborder]
\prompt{In}{incolor}{8}{\boxspacing}
\begin{Verbatim}[commandchars=\\\{\}]
\PY{n}{df}\PY{o}{.}\PY{n}{info}\PY{p}{(}\PY{p}{)}
\end{Verbatim}
\end{tcolorbox}

    \begin{Verbatim}[commandchars=\\\{\}]
<class 'pandas.core.frame.DataFrame'>
RangeIndex: 200 entries, 0 to 199
Data columns (total 6 columns):
 \#   Column       Non-Null Count  Dtype
---  ------       --------------  -----
 0   Age          200 non-null    int64
 1   Sex          200 non-null    int64
 2   BP           200 non-null    int64
 3   Cholesterol  200 non-null    int64
 4   Na\_to\_K      200 non-null    float64
 5   Drug         200 non-null    int64
dtypes: float64(1), int64(5)
memory usage: 9.5 KB
    \end{Verbatim}

    \begin{tcolorbox}[breakable, size=fbox, boxrule=1pt, pad at break*=1mm,colback=cellbackground, colframe=cellborder]
\prompt{In}{incolor}{9}{\boxspacing}
\begin{Verbatim}[commandchars=\\\{\}]
\PY{n}{df}\PY{o}{.}\PY{n}{head}\PY{p}{(}\PY{p}{)}
\end{Verbatim}
\end{tcolorbox}

            \begin{tcolorbox}[breakable, size=fbox, boxrule=.5pt, pad at break*=1mm, opacityfill=0]
\prompt{Out}{outcolor}{9}{\boxspacing}
\begin{Verbatim}[commandchars=\\\{\}]
   Age  Sex  BP  Cholesterol  Na\_to\_K  Drug
0   23    0   0            0   25.355     0
1   47    1   1            0   13.093     3
2   47    1   1            0   10.114     3
3   28    0   2            0    7.798     4
4   61    0   1            0   18.043     0
\end{Verbatim}
\end{tcolorbox}
        
    \hypertarget{ux431ux430ux43bux430ux43dux441-ux43aux43bux430ux441ux441ux43eux432}{%
\section{Баланс
классов}\label{ux431ux430ux43bux430ux43dux441-ux43aux43bux430ux441ux441ux43eux432}}

Построим гистограмму, показывающую количество примеров каждого класса,
чтобы оценить баланс данных. Это поможет выявить дисбаланс, если он
существует, что может повлиять на качество обучения моделей.

    \begin{tcolorbox}[breakable, size=fbox, boxrule=1pt, pad at break*=1mm,colback=cellbackground, colframe=cellborder]
\prompt{In}{incolor}{10}{\boxspacing}
\begin{Verbatim}[commandchars=\\\{\}]
\PY{k+kn}{import} \PY{n+nn}{seaborn} \PY{k}{as} \PY{n+nn}{sns}
\PY{k+kn}{import} \PY{n+nn}{matplotlib}\PY{n+nn}{.}\PY{n+nn}{pyplot} \PY{k}{as} \PY{n+nn}{plt}
\end{Verbatim}
\end{tcolorbox}

    \begin{tcolorbox}[breakable, size=fbox, boxrule=1pt, pad at break*=1mm,colback=cellbackground, colframe=cellborder]
\prompt{In}{incolor}{11}{\boxspacing}
\begin{Verbatim}[commandchars=\\\{\}]
\PY{n}{plt}\PY{o}{.}\PY{n}{figure}\PY{p}{(}\PY{n}{figsize}\PY{o}{=}\PY{p}{(}\PY{l+m+mi}{15}\PY{p}{,}\PY{l+m+mi}{4}\PY{p}{)}\PY{p}{)}
\PY{n}{sns}\PY{o}{.}\PY{n}{countplot}\PY{p}{(}\PY{n}{data}\PY{o}{=}\PY{n}{df}\PY{p}{,} \PY{n}{x}\PY{o}{=}\PY{n}{target\PYZus{}colomn}\PY{p}{)}
\PY{n}{plt}\PY{o}{.}\PY{n}{title}\PY{p}{(}\PY{l+s+s1}{\PYZsq{}}\PY{l+s+s1}{Гистограмма количества элементов каждого класса}\PY{l+s+s1}{\PYZsq{}}\PY{p}{)}
\PY{n}{plt}\PY{o}{.}\PY{n}{xlabel}\PY{p}{(}\PY{l+s+s2}{\PYZdq{}}\PY{l+s+s2}{Классы лекарств}\PY{l+s+s2}{\PYZdq{}}\PY{p}{)}
\PY{n}{plt}\PY{o}{.}\PY{n}{ylabel}\PY{p}{(}\PY{l+s+s2}{\PYZdq{}}\PY{l+s+s2}{Количество}\PY{l+s+s2}{\PYZdq{}}\PY{p}{)}
\PY{n}{plt}\PY{o}{.}\PY{n}{show}\PY{p}{(}\PY{p}{)}
\end{Verbatim}
\end{tcolorbox}

    \begin{center}
    \adjustimage{max size={0.9\linewidth}{0.9\paperheight}}{output_19_0.png}
    \end{center}
    { \hspace*{\fill} \\}
    
    В целом, распределение классов можно считать нормальным для задач
классификации. Классы с меньшим числом примеров (1, 2 и 3) могут быть
менее представительными, но это не вызывает серьезного дисбаланса,
который мог бы негативно сказаться на обучении моделей. Если модели
показывают хорошие результаты на валидационных данных, то текущее
распределение классов можно считать приемлемым.

    \hypertarget{ux440ux430ux437ux434ux435ux43bux435ux43dux438ux435-ux43dux430-ux442ux440ux435ux43dux438ux440ux43eux432ux43eux447ux43dux443ux44e-ux438-ux442ux435ux441ux442ux43eux432ux443ux44e-ux432ux44bux431ux43eux440ux43aux438}{%
\section{Разделение на тренировочную и тестовую
выборки}\label{ux440ux430ux437ux434ux435ux43bux435ux43dux438ux435-ux43dux430-ux442ux440ux435ux43dux438ux440ux43eux432ux43eux447ux43dux443ux44e-ux438-ux442ux435ux441ux442ux43eux432ux443ux44e-ux432ux44bux431ux43eux440ux43aux438}}

Выделим 80\% данных для обучения и 20\% для тестирования, чтобы
проверить модели на новых данных.

    \begin{tcolorbox}[breakable, size=fbox, boxrule=1pt, pad at break*=1mm,colback=cellbackground, colframe=cellborder]
\prompt{In}{incolor}{12}{\boxspacing}
\begin{Verbatim}[commandchars=\\\{\}]
\PY{k+kn}{from} \PY{n+nn}{sklearn}\PY{n+nn}{.}\PY{n+nn}{model\PYZus{}selection} \PY{k+kn}{import} \PY{n}{train\PYZus{}test\PYZus{}split}
\end{Verbatim}
\end{tcolorbox}

    \begin{tcolorbox}[breakable, size=fbox, boxrule=1pt, pad at break*=1mm,colback=cellbackground, colframe=cellborder]
\prompt{In}{incolor}{13}{\boxspacing}
\begin{Verbatim}[commandchars=\\\{\}]
\PY{n}{X} \PY{o}{=} \PY{n}{df}\PY{o}{.}\PY{n}{drop}\PY{p}{(}\PY{n}{columns}\PY{o}{=}\PY{p}{[}\PY{n}{target\PYZus{}colomn}\PY{p}{]}\PY{p}{)}
\PY{n}{y} \PY{o}{=} \PY{n}{df}\PY{p}{[}\PY{n}{target\PYZus{}colomn}\PY{p}{]}
\end{Verbatim}
\end{tcolorbox}

    \begin{tcolorbox}[breakable, size=fbox, boxrule=1pt, pad at break*=1mm,colback=cellbackground, colframe=cellborder]
\prompt{In}{incolor}{14}{\boxspacing}
\begin{Verbatim}[commandchars=\\\{\}]
\PY{n}{X\PYZus{}train}\PY{p}{,} \PY{n}{X\PYZus{}test}\PY{p}{,} \PY{n}{y\PYZus{}train}\PY{p}{,} \PY{n}{y\PYZus{}test} \PY{o}{=} \PY{n}{train\PYZus{}test\PYZus{}split}\PY{p}{(}\PY{n}{X}\PY{p}{,} \PY{n}{y}\PY{p}{,} \PY{n}{test\PYZus{}size}\PY{o}{=}\PY{l+m+mf}{0.2}\PY{p}{,} \PY{n}{random\PYZus{}state}\PY{o}{=}\PY{l+m+mi}{20}\PY{p}{)}
\end{Verbatim}
\end{tcolorbox}

    \hypertarget{ux43aux43bux430ux441ux441ux438ux444ux438ux43aux430ux446ux438ux44f-ux441-ux438ux441ux43fux43eux43bux44cux437ux43eux432ux430ux43dux438ux435ux43c-ux43bux43eux433ux438ux441ux442ux438ux447ux435ux441ux43aux43eux439-ux440ux435ux433ux440ux435ux441ux441ux438ux438-svm-ux438-knn}{%
\section{Классификация с использованием логистической регрессии, SVM и
KNN}\label{ux43aux43bux430ux441ux441ux438ux444ux438ux43aux430ux446ux438ux44f-ux441-ux438ux441ux43fux43eux43bux44cux437ux43eux432ux430ux43dux438ux435ux43c-ux43bux43eux433ux438ux441ux442ux438ux447ux435ux441ux43aux43eux439-ux440ux435ux433ux440ux435ux441ux441ux438ux438-svm-ux438-knn}}

Теперь применим несколько алгоритмов классификации и построим матрицу
ошибок для каждой модели.

    \begin{tcolorbox}[breakable, size=fbox, boxrule=1pt, pad at break*=1mm,colback=cellbackground, colframe=cellborder]
\prompt{In}{incolor}{15}{\boxspacing}
\begin{Verbatim}[commandchars=\\\{\}]
\PY{k+kn}{from} \PY{n+nn}{sklearn}\PY{n+nn}{.}\PY{n+nn}{model\PYZus{}selection} \PY{k+kn}{import} \PY{n}{GridSearchCV}
\end{Verbatim}
\end{tcolorbox}

    \begin{tcolorbox}[breakable, size=fbox, boxrule=1pt, pad at break*=1mm,colback=cellbackground, colframe=cellborder]
\prompt{In}{incolor}{16}{\boxspacing}
\begin{Verbatim}[commandchars=\\\{\}]
\PY{k+kn}{from} \PY{n+nn}{sklearn}\PY{n+nn}{.}\PY{n+nn}{metrics} \PY{k+kn}{import} \PY{n}{classification\PYZus{}report}
\PY{k+kn}{from} \PY{n+nn}{sklearn}\PY{n+nn}{.}\PY{n+nn}{metrics} \PY{k+kn}{import} \PY{n}{confusion\PYZus{}matrix}
\end{Verbatim}
\end{tcolorbox}

    \hypertarget{ux43bux43eux433ux438ux441ux442ux438ux447ux435ux441ux43aux430ux44f-ux440ux435ux433ux440ux435ux441ux441ux438ux44f}{%
\subsection{Логистическая
регрессия}\label{ux43bux43eux433ux438ux441ux442ux438ux447ux435ux441ux43aux430ux44f-ux440ux435ux433ux440ux435ux441ux441ux438ux44f}}

    \begin{tcolorbox}[breakable, size=fbox, boxrule=1pt, pad at break*=1mm,colback=cellbackground, colframe=cellborder]
\prompt{In}{incolor}{17}{\boxspacing}
\begin{Verbatim}[commandchars=\\\{\}]
\PY{k+kn}{from} \PY{n+nn}{sklearn}\PY{n+nn}{.}\PY{n+nn}{linear\PYZus{}model} \PY{k+kn}{import} \PY{n}{LogisticRegression}
\end{Verbatim}
\end{tcolorbox}

    \begin{tcolorbox}[breakable, size=fbox, boxrule=1pt, pad at break*=1mm,colback=cellbackground, colframe=cellborder]
\prompt{In}{incolor}{18}{\boxspacing}
\begin{Verbatim}[commandchars=\\\{\}]
\PY{n}{logreg} \PY{o}{=} \PY{n}{LogisticRegression}\PY{p}{(}\PY{n}{random\PYZus{}state}\PY{o}{=}\PY{l+m+mi}{20}\PY{p}{,} \PY{n}{max\PYZus{}iter}\PY{o}{=}\PY{l+m+mi}{10000}\PY{p}{)}
\PY{n}{logreg}\PY{o}{.}\PY{n}{fit}\PY{p}{(}\PY{n}{X\PYZus{}train}\PY{p}{,} \PY{n}{y\PYZus{}train}\PY{p}{)}
\end{Verbatim}
\end{tcolorbox}

            \begin{tcolorbox}[breakable, size=fbox, boxrule=.5pt, pad at break*=1mm, opacityfill=0]
\prompt{Out}{outcolor}{18}{\boxspacing}
\begin{Verbatim}[commandchars=\\\{\}]
LogisticRegression(max\_iter=10000, random\_state=20)
\end{Verbatim}
\end{tcolorbox}
        
    \begin{tcolorbox}[breakable, size=fbox, boxrule=1pt, pad at break*=1mm,colback=cellbackground, colframe=cellborder]
\prompt{In}{incolor}{19}{\boxspacing}
\begin{Verbatim}[commandchars=\\\{\}]
\PY{n}{log\PYZus{}pred} \PY{o}{=} \PY{n}{logreg}\PY{o}{.}\PY{n}{predict}\PY{p}{(}\PY{n}{X\PYZus{}test}\PY{p}{)}
\PY{n+nb}{print}\PY{p}{(}\PY{n}{classification\PYZus{}report}\PY{p}{(}\PY{n}{y\PYZus{}test}\PY{p}{,} \PY{n}{log\PYZus{}pred}\PY{p}{)}\PY{p}{)}
\end{Verbatim}
\end{tcolorbox}

    \begin{Verbatim}[commandchars=\\\{\}]
              precision    recall  f1-score   support

           0       1.00      0.94      0.97        17
           1       0.71      1.00      0.83         5
           2       0.67      1.00      0.80         2
           3       1.00      0.67      0.80         6
           4       1.00      1.00      1.00        10

    accuracy                           0.93        40
   macro avg       0.88      0.92      0.88        40
weighted avg       0.95      0.93      0.93        40

    \end{Verbatim}

    \begin{tcolorbox}[breakable, size=fbox, boxrule=1pt, pad at break*=1mm,colback=cellbackground, colframe=cellborder]
\prompt{In}{incolor}{20}{\boxspacing}
\begin{Verbatim}[commandchars=\\\{\}]
\PY{n}{plt}\PY{o}{.}\PY{n}{figure}\PY{p}{(}\PY{n}{figsize}\PY{o}{=}\PY{p}{(}\PY{l+m+mi}{6}\PY{p}{,} \PY{l+m+mi}{4}\PY{p}{)}\PY{p}{)}
\PY{n}{cm} \PY{o}{=} \PY{n}{confusion\PYZus{}matrix}\PY{p}{(}\PY{n}{y\PYZus{}test}\PY{p}{,} \PY{n}{log\PYZus{}pred}\PY{p}{)}
\PY{n}{sns}\PY{o}{.}\PY{n}{heatmap}\PY{p}{(}\PY{n}{cm}\PY{p}{,} \PY{n}{annot}\PY{o}{=}\PY{k+kc}{True}\PY{p}{,} \PY{n}{cmap}\PY{o}{=}\PY{l+s+s1}{\PYZsq{}}\PY{l+s+s1}{Blues}\PY{l+s+s1}{\PYZsq{}}\PY{p}{)}
\PY{n}{plt}\PY{o}{.}\PY{n}{title}\PY{p}{(}\PY{l+s+s1}{\PYZsq{}}\PY{l+s+s1}{Модель: логистической регрессия}\PY{l+s+s1}{\PYZsq{}}\PY{p}{)}
\PY{n}{plt}\PY{o}{.}\PY{n}{xlabel}\PY{p}{(}\PY{l+s+s1}{\PYZsq{}}\PY{l+s+s1}{Тестовый}\PY{l+s+s1}{\PYZsq{}}\PY{p}{)}
\PY{n}{plt}\PY{o}{.}\PY{n}{ylabel}\PY{p}{(}\PY{l+s+s1}{\PYZsq{}}\PY{l+s+s1}{Прогнозируемый}\PY{l+s+s1}{\PYZsq{}}\PY{p}{)}
\PY{n}{plt}\PY{o}{.}\PY{n}{show}\PY{p}{(}\PY{p}{)}
\end{Verbatim}
\end{tcolorbox}

    \begin{center}
    \adjustimage{max size={0.9\linewidth}{0.9\paperheight}}{output_32_0.png}
    \end{center}
    { \hspace*{\fill} \\}
    
    \textbf{Точность модели (accuracy)} составляет 93\%, что свидетельствует
о высокой надежности предсказаний.

Показатели по классам: - Класс 0 показывает отличные результаты с
precision 1.00 и F1-score 0.97. - Класс 4 также демонстрирует идеальные
результаты с precision, recall и F1-score равными 1.00. - Однако для
классов 1, 2 и 3 наблюдаются проблемы: класс 1 имеет precision 0.71,
класс 2 --- 0.67, а класс 3 --- 1.00, но с recall всего 0.67.

\textbf{Средние значения} показывают macro average с precision 0.88 и
recall 0.92, а weighted average равны 0.95 и 0.93 соответственно, что
указывает на общую сбалансированность.

В целом, логистическая регрессия показала хорошие результаты, особенно
для классов 0 и 4. Тем не менее, классы 1, 2 и 3 требуют улучшения, что
можно достичь через анализ ошибок и настройку гиперпараметров.

    \hypertarget{svm-ux43cux435ux442ux43eux434-ux43eux43fux43eux440ux43dux44bux445-ux432ux435ux43aux442ux43eux440ux43eux432}{%
\subsection{SVM (метод опорных
векторов)}\label{svm-ux43cux435ux442ux43eux434-ux43eux43fux43eux440ux43dux44bux445-ux432ux435ux43aux442ux43eux440ux43eux432}}

    \begin{tcolorbox}[breakable, size=fbox, boxrule=1pt, pad at break*=1mm,colback=cellbackground, colframe=cellborder]
\prompt{In}{incolor}{21}{\boxspacing}
\begin{Verbatim}[commandchars=\\\{\}]
\PY{k+kn}{from} \PY{n+nn}{sklearn}\PY{n+nn}{.}\PY{n+nn}{svm} \PY{k+kn}{import} \PY{n}{SVC}
\end{Verbatim}
\end{tcolorbox}

    \begin{tcolorbox}[breakable, size=fbox, boxrule=1pt, pad at break*=1mm,colback=cellbackground, colframe=cellborder]
\prompt{In}{incolor}{22}{\boxspacing}
\begin{Verbatim}[commandchars=\\\{\}]
\PY{n}{parma\PYZus{}kernel} \PY{o}{=} \PY{p}{(}\PY{l+s+s1}{\PYZsq{}}\PY{l+s+s1}{linear}\PY{l+s+s1}{\PYZsq{}}\PY{p}{,} \PY{l+s+s1}{\PYZsq{}}\PY{l+s+s1}{rbf}\PY{l+s+s1}{\PYZsq{}}\PY{p}{,} \PY{l+s+s1}{\PYZsq{}}\PY{l+s+s1}{poly}\PY{l+s+s1}{\PYZsq{}}\PY{p}{,} \PY{l+s+s1}{\PYZsq{}}\PY{l+s+s1}{sigmoid}\PY{l+s+s1}{\PYZsq{}}\PY{p}{)}
\PY{n}{params} \PY{o}{=} \PY{p}{\PYZob{}}\PY{l+s+s1}{\PYZsq{}}\PY{l+s+s1}{kernel}\PY{l+s+s1}{\PYZsq{}}\PY{p}{:} \PY{n}{parma\PYZus{}kernel}\PY{p}{\PYZcb{}}
\PY{n}{svc} \PY{o}{=} \PY{n}{SVC}\PY{p}{(}\PY{p}{)}
\PY{n}{grid\PYZus{}search} \PY{o}{=} \PY{n}{GridSearchCV}\PY{p}{(}\PY{n}{estimator}\PY{o}{=}\PY{n}{svc}\PY{p}{,} \PY{n}{param\PYZus{}grid}\PY{o}{=}\PY{n}{params}\PY{p}{,} \PY{n}{cv}\PY{o}{=}\PY{l+m+mi}{10}\PY{p}{)}
\end{Verbatim}
\end{tcolorbox}

    \begin{tcolorbox}[breakable, size=fbox, boxrule=1pt, pad at break*=1mm,colback=cellbackground, colframe=cellborder]
\prompt{In}{incolor}{23}{\boxspacing}
\begin{Verbatim}[commandchars=\\\{\}]
\PY{n}{grid\PYZus{}search}\PY{o}{.}\PY{n}{fit}\PY{p}{(}\PY{n}{X\PYZus{}train}\PY{p}{,} \PY{n}{y\PYZus{}train}\PY{p}{)}
\PY{n}{best\PYZus{}model} \PY{o}{=} \PY{n}{grid\PYZus{}search}\PY{o}{.}\PY{n}{best\PYZus{}estimator\PYZus{}}
\PY{n}{grid\PYZus{}search}\PY{o}{.}\PY{n}{best\PYZus{}score\PYZus{}}\PY{p}{,} \PY{n}{grid\PYZus{}search}\PY{o}{.}\PY{n}{best\PYZus{}estimator\PYZus{}}
\end{Verbatim}
\end{tcolorbox}

            \begin{tcolorbox}[breakable, size=fbox, boxrule=.5pt, pad at break*=1mm, opacityfill=0]
\prompt{Out}{outcolor}{23}{\boxspacing}
\begin{Verbatim}[commandchars=\\\{\}]
(0.9875, SVC(kernel='linear'))
\end{Verbatim}
\end{tcolorbox}
        
    \begin{tcolorbox}[breakable, size=fbox, boxrule=1pt, pad at break*=1mm,colback=cellbackground, colframe=cellborder]
\prompt{In}{incolor}{24}{\boxspacing}
\begin{Verbatim}[commandchars=\\\{\}]
\PY{n}{best\PYZus{}model}\PY{o}{.}\PY{n}{kernel}
\end{Verbatim}
\end{tcolorbox}

            \begin{tcolorbox}[breakable, size=fbox, boxrule=.5pt, pad at break*=1mm, opacityfill=0]
\prompt{Out}{outcolor}{24}{\boxspacing}
\begin{Verbatim}[commandchars=\\\{\}]
'linear'
\end{Verbatim}
\end{tcolorbox}
        
    \begin{tcolorbox}[breakable, size=fbox, boxrule=1pt, pad at break*=1mm,colback=cellbackground, colframe=cellborder]
\prompt{In}{incolor}{25}{\boxspacing}
\begin{Verbatim}[commandchars=\\\{\}]
\PY{n}{svm\PYZus{}pred} \PY{o}{=} \PY{n}{grid\PYZus{}search}\PY{o}{.}\PY{n}{predict}\PY{p}{(}\PY{n}{X\PYZus{}test}\PY{p}{)}
\PY{n+nb}{print}\PY{p}{(}\PY{n}{classification\PYZus{}report}\PY{p}{(}\PY{n}{y\PYZus{}test}\PY{p}{,} \PY{n}{svm\PYZus{}pred}\PY{p}{,} \PY{n}{zero\PYZus{}division}\PY{o}{=}\PY{l+m+mi}{0}\PY{p}{)}\PY{p}{)}
\end{Verbatim}
\end{tcolorbox}

    \begin{Verbatim}[commandchars=\\\{\}]
              precision    recall  f1-score   support

           0       1.00      1.00      1.00        17
           1       1.00      1.00      1.00         5
           2       1.00      1.00      1.00         2
           3       1.00      1.00      1.00         6
           4       1.00      1.00      1.00        10

    accuracy                           1.00        40
   macro avg       1.00      1.00      1.00        40
weighted avg       1.00      1.00      1.00        40

    \end{Verbatim}

    \begin{tcolorbox}[breakable, size=fbox, boxrule=1pt, pad at break*=1mm,colback=cellbackground, colframe=cellborder]
\prompt{In}{incolor}{26}{\boxspacing}
\begin{Verbatim}[commandchars=\\\{\}]
\PY{n}{plt}\PY{o}{.}\PY{n}{figure}\PY{p}{(}\PY{n}{figsize}\PY{o}{=}\PY{p}{(}\PY{l+m+mi}{6}\PY{p}{,} \PY{l+m+mi}{4}\PY{p}{)}\PY{p}{)}
\PY{n}{cm} \PY{o}{=} \PY{n}{confusion\PYZus{}matrix}\PY{p}{(}\PY{n}{y\PYZus{}test}\PY{p}{,} \PY{n}{svm\PYZus{}pred}\PY{p}{)}
\PY{n}{sns}\PY{o}{.}\PY{n}{heatmap}\PY{p}{(}\PY{n}{cm}\PY{p}{,} \PY{n}{annot}\PY{o}{=}\PY{k+kc}{True}\PY{p}{,} \PY{n}{cmap}\PY{o}{=}\PY{l+s+s1}{\PYZsq{}}\PY{l+s+s1}{Blues}\PY{l+s+s1}{\PYZsq{}}\PY{p}{)}
\PY{n}{plt}\PY{o}{.}\PY{n}{title}\PY{p}{(}\PY{l+s+s1}{\PYZsq{}}\PY{l+s+s1}{Модель: SVC}\PY{l+s+s1}{\PYZsq{}}\PY{p}{)}
\PY{n}{plt}\PY{o}{.}\PY{n}{xlabel}\PY{p}{(}\PY{l+s+s1}{\PYZsq{}}\PY{l+s+s1}{Тестовый}\PY{l+s+s1}{\PYZsq{}}\PY{p}{)}
\PY{n}{plt}\PY{o}{.}\PY{n}{ylabel}\PY{p}{(}\PY{l+s+s1}{\PYZsq{}}\PY{l+s+s1}{Прогнозируемый}\PY{l+s+s1}{\PYZsq{}}\PY{p}{)}
\PY{n}{plt}\PY{o}{.}\PY{n}{show}\PY{p}{(}\PY{p}{)}
\end{Verbatim}
\end{tcolorbox}

    \begin{center}
    \adjustimage{max size={0.9\linewidth}{0.9\paperheight}}{output_40_0.png}
    \end{center}
    { \hspace*{\fill} \\}
    
    В ходе настройки гиперпараметров с помощью Grid Search был выбран лучший
параметр \textbf{ядра} --- linear, что привело к высокой оценке модели.
Лучший результат \textbf{кросс-валидации} составил 98.75\%.

Все классы (0, 1, 2, 3 и 4) показали отличные результаты с
\textbf{precision}, \textbf{recall} и \textbf{F1-score}, равными 1.00.
Это свидетельствует о том, что модель идеально классифицировала все
примеры в тестовом наборе.

Общая точность модели составила 100\%, что указывает на полное
отсутствие ошибок в классификации тестовых данных. Также \textbf{macro
average} и \textbf{weighted average} показывают показатели 1.00 для всех
метрик, что подтверждает сбалансированность и высокую производительность
модели.

Модель SVM с линейным ядром показала выдающиеся результаты в
классификации. Все классы были предсказаны без ошибок, что делает данную
модель надежным инструментом для данной задачи. При дальнейшем
использовании стоит сохранить параметры модели и провести дополнительные
тесты на других выборках для проверки обобщающей способности.

    \hypertarget{knn-ux43a-ux431ux43bux438ux436ux430ux439ux448ux438ux445-ux441ux43eux441ux435ux434ux435ux439}{%
\subsection{KNN (к ближайших
соседей)}\label{knn-ux43a-ux431ux43bux438ux436ux430ux439ux448ux438ux445-ux441ux43eux441ux435ux434ux435ux439}}

    \begin{tcolorbox}[breakable, size=fbox, boxrule=1pt, pad at break*=1mm,colback=cellbackground, colframe=cellborder]
\prompt{In}{incolor}{27}{\boxspacing}
\begin{Verbatim}[commandchars=\\\{\}]
\PY{k+kn}{from} \PY{n+nn}{sklearn}\PY{n+nn}{.}\PY{n+nn}{neighbors} \PY{k+kn}{import} \PY{n}{KNeighborsClassifier}
\end{Verbatim}
\end{tcolorbox}

    \begin{tcolorbox}[breakable, size=fbox, boxrule=1pt, pad at break*=1mm,colback=cellbackground, colframe=cellborder]
\prompt{In}{incolor}{28}{\boxspacing}
\begin{Verbatim}[commandchars=\\\{\}]
\PY{n}{parmas} \PY{o}{=} \PY{p}{\PYZob{}}\PY{l+s+s1}{\PYZsq{}}\PY{l+s+s1}{n\PYZus{}neighbors}\PY{l+s+s1}{\PYZsq{}}\PY{p}{:} \PY{n+nb}{list}\PY{p}{(}\PY{n}{np}\PY{o}{.}\PY{n}{arange}\PY{p}{(}\PY{l+m+mi}{2}\PY{p}{,}\PY{l+m+mi}{20}\PY{p}{)}\PY{p}{)}\PY{p}{\PYZcb{}}
\PY{n}{knn} \PY{o}{=} \PY{n}{KNeighborsClassifier}\PY{p}{(}\PY{p}{)}
\PY{n}{grid\PYZus{}search} \PY{o}{=} \PY{n}{GridSearchCV}\PY{p}{(}\PY{n}{estimator}\PY{o}{=}\PY{n}{knn}\PY{p}{,} \PY{n}{param\PYZus{}grid}\PY{o}{=}\PY{n}{parmas}\PY{p}{,} \PY{n}{cv}\PY{o}{=}\PY{l+m+mi}{10}\PY{p}{)}
\end{Verbatim}
\end{tcolorbox}

    \begin{tcolorbox}[breakable, size=fbox, boxrule=1pt, pad at break*=1mm,colback=cellbackground, colframe=cellborder]
\prompt{In}{incolor}{29}{\boxspacing}
\begin{Verbatim}[commandchars=\\\{\}]
\PY{n}{grid\PYZus{}search}\PY{o}{.}\PY{n}{fit}\PY{p}{(}\PY{n}{X\PYZus{}train}\PY{p}{,} \PY{n}{y\PYZus{}train}\PY{p}{)}
\PY{n}{grid\PYZus{}search}\PY{o}{.}\PY{n}{best\PYZus{}score\PYZus{}}\PY{p}{,} \PY{n}{grid\PYZus{}search}\PY{o}{.}\PY{n}{best\PYZus{}estimator\PYZus{}}
\end{Verbatim}
\end{tcolorbox}

            \begin{tcolorbox}[breakable, size=fbox, boxrule=.5pt, pad at break*=1mm, opacityfill=0]
\prompt{Out}{outcolor}{29}{\boxspacing}
\begin{Verbatim}[commandchars=\\\{\}]
(0.71875, KNeighborsClassifier(n\_neighbors=3))
\end{Verbatim}
\end{tcolorbox}
        
    \begin{tcolorbox}[breakable, size=fbox, boxrule=1pt, pad at break*=1mm,colback=cellbackground, colframe=cellborder]
\prompt{In}{incolor}{30}{\boxspacing}
\begin{Verbatim}[commandchars=\\\{\}]
\PY{n}{knn\PYZus{}pred} \PY{o}{=} \PY{n}{grid\PYZus{}search}\PY{o}{.}\PY{n}{predict}\PY{p}{(}\PY{n}{X\PYZus{}test}\PY{p}{)}
\PY{n+nb}{print}\PY{p}{(}\PY{n}{classification\PYZus{}report}\PY{p}{(}\PY{n}{y\PYZus{}test}\PY{p}{,} \PY{n}{knn\PYZus{}pred}\PY{p}{)}\PY{p}{)}
\end{Verbatim}
\end{tcolorbox}

    \begin{Verbatim}[commandchars=\\\{\}]
              precision    recall  f1-score   support

           0       0.84      0.94      0.89        17
           1       0.50      0.40      0.44         5
           2       0.40      1.00      0.57         2
           3       0.25      0.17      0.20         6
           4       0.62      0.50      0.56        10

    accuracy                           0.65        40
   macro avg       0.52      0.60      0.53        40
weighted avg       0.63      0.65      0.63        40

    \end{Verbatim}

    \begin{tcolorbox}[breakable, size=fbox, boxrule=1pt, pad at break*=1mm,colback=cellbackground, colframe=cellborder]
\prompt{In}{incolor}{31}{\boxspacing}
\begin{Verbatim}[commandchars=\\\{\}]
\PY{n}{plt}\PY{o}{.}\PY{n}{figure}\PY{p}{(}\PY{n}{figsize}\PY{o}{=}\PY{p}{(}\PY{l+m+mi}{6}\PY{p}{,} \PY{l+m+mi}{4}\PY{p}{)}\PY{p}{)}
\PY{n}{cm} \PY{o}{=} \PY{n}{confusion\PYZus{}matrix}\PY{p}{(}\PY{n}{y\PYZus{}test}\PY{p}{,} \PY{n}{knn\PYZus{}pred}\PY{p}{)}
\PY{n}{sns}\PY{o}{.}\PY{n}{heatmap}\PY{p}{(}\PY{n}{cm}\PY{p}{,} \PY{n}{annot}\PY{o}{=}\PY{k+kc}{True}\PY{p}{,} \PY{n}{cmap}\PY{o}{=}\PY{l+s+s1}{\PYZsq{}}\PY{l+s+s1}{Blues}\PY{l+s+s1}{\PYZsq{}}\PY{p}{)}
\PY{n}{plt}\PY{o}{.}\PY{n}{title}\PY{p}{(}\PY{l+s+s1}{\PYZsq{}}\PY{l+s+s1}{Модель: KNN}\PY{l+s+s1}{\PYZsq{}}\PY{p}{)}
\PY{n}{plt}\PY{o}{.}\PY{n}{xlabel}\PY{p}{(}\PY{l+s+s1}{\PYZsq{}}\PY{l+s+s1}{Тестовый}\PY{l+s+s1}{\PYZsq{}}\PY{p}{)}
\PY{n}{plt}\PY{o}{.}\PY{n}{ylabel}\PY{p}{(}\PY{l+s+s1}{\PYZsq{}}\PY{l+s+s1}{Прогнозируемый}\PY{l+s+s1}{\PYZsq{}}\PY{p}{)}
\PY{n}{plt}\PY{o}{.}\PY{n}{show}\PY{p}{(}\PY{p}{)}
\end{Verbatim}
\end{tcolorbox}

    \begin{center}
    \adjustimage{max size={0.9\linewidth}{0.9\paperheight}}{output_47_0.png}
    \end{center}
    { \hspace*{\fill} \\}
    
    В ходе настройки гиперпараметров с помощью Grid Search был выбран
оптимальный параметр \textbf{n\_neighbors}, равный \textbf{3}. Однако
результат кросс-валидации составил только \textbf{71.88\%}, что
указывает на умеренную производительность модели.

Показатели по классам: - Класс 0 показал приемлемые результаты с
\textbf{precision} 0.84 и \textbf{recall} 0.94, что указывает на высокую
точность предсказаний для этого класса. - Класс 1 продемонстрировал
низкие результаты, с \textbf{precision} 0.50 и \textbf{recall} 0.40, что
говорит о проблемах в классификации этого класса. - Класс 2 имел высокую
полноту (\textbf{recall} 1.00), но низкую точность (\textbf{precision}
0.40), что указывает на возможные ложные срабатывания. - Класс 3 показал
крайне низкие результаты с \textbf{precision} 0.25 и \textbf{recall}
0.17, что указывает на серьезные проблемы с классификацией. - Класс 4
имел \textbf{precision} 0.62 и \textbf{recall} 0.50, что также говорит о
недостаточной надежности предсказаний.

Общая точность модели составила \textbf{65\%}, что является относительно
низким результатом. \textbf{Macro average} показывает \textbf{precision}
0.52 и \textbf{recall} 0.60, что подтверждает дисбаланс в классификации
по классам. \textbf{Weighted average} составил 0.63, что указывает на
более высокую производительность при учете частоты классов, но все еще
далек от идеала.

Модель KNN с оптимальным количеством соседей 3 продемонстрировала
умеренные результаты, однако показатели для классов 1, 2 и 3 требуют
значительного улучшения. Это может быть достигнуто путем настройки
гиперпараметров, увеличения объема данных для обучения или применения
методов обработки данных, таких как балансировка классов. В целом, KNN
оказалась менее эффективной по сравнению с другими моделями, такими как
логистическая регрессия и SVM.

\clearpage

    \hypertarget{ux432ux44bux432ux43eux434}{%
\section*{\Large{Вывод}}\label{ux432ux44bux432ux43eux434}}

    В ходе практической работы по классификации были исследованы три
алгоритма: логистическая регрессия, метод опорных векторов (SVM) и метод
k-ближайших соседей (KNN).

\textbf{Логистическая регрессия} продемонстрировала хорошие результаты с
общей точностью 93\%, хотя классы 1, 2 и 3 требуют улучшения.

\textbf{Метод опорных векторов (SVM)} показал идеальную точность 100\%,
классифицировав все классы без ошибок, что указывает на его высокую
надежность.

\textbf{Метод k-ближайших соседей (KNN)} оказался менее эффективным, с
общей точностью всего 65\%, и значительными проблемами в классификации
классов 1, 2 и 3.

В целом, работа продемонстрировала важность выбора алгоритма и его
настройки для достижения высоких показателей классификации.
Рекомендуется продолжить эксперименты с другими методами и техниками
предобработки.

\end{document}
