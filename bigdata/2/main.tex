\documentclass[14pt,a4paper,oneside]{extarticle}

	\usepackage{cmap} % for pdfLaTeX
	\usepackage[T1,T2A]{fontenc} % correct encoding for pdfLaTeX
	\usepackage[utf8]{inputenc} % correct encoding source file
	\usepackage[english,russian]{babel} % correct language

	% -------------------
	% TEXT SETTINGS
	% -------------------

	\usepackage{fontspec} % use standart fonts (only for xelatex!!!)
	\setmainfont{Times New Roman}
	\setsansfont{Arial}
	\setmonofont[Scale=0.6]{Courier New}

	\usepackage[none]{hyphenat} % no word breaks

	\usepackage{setspace}
	%\singlespacing % 1.0
	\onehalfspacing % 1.5

	% -------------------
	% PAGE SETTINGS
	% -------------------

	\usepackage[left=3cm, right=1.5cm, vmargin=2cm, headheight=2cm]{geometry}
	\linespread{1.5} % line spacing 1.5
	\usepackage{indentfirst} % indent first paragraph
	\setlength{\parindent}{1.25cm}
	\sloppy

	\frenchspacing % ???
	\pagestyle{plain}
	\usepackage{fancyhdr} % for headers and footers

	\clubpenalty=10000
	\widowpenalty=10000

	% ------------------
	% ATTACHMENTS SETTINGS
	% ------------------

	\usepackage[labelsep=endash]{caption}
	%\setlength{\abovecaptionskip}{3pt}
	%\setlength{\belowcaptionskip}{3pt}

	% for non-end-to-end numbering
	%\usepackage{chngcntr}
	%\counterwithin{figure}{section}
	%\counterwithin{table}{section}

	% for long table
	\usepackage{longtable}

	% for graphics
	\usepackage{graphicx}
	\newcommand{\rref}[1]{(Рисунок~\ref{#1})}
	\newcommand{\rdref}[2]{(Рисунки~\ref{#1}\,-\,\ref{#2})}
	\newcommand{\tref}[1]{(Таблица~\ref{#1})}
	\newcommand{\tdref}[2]{(Таблици~\ref{#1}\,-\,\ref{#2})}
	\newenvironment{image}{
		\begin{figure}[h!tp]
		\centering
	}{
		\end{figure}
	}
	\newcommand\includegrph[2][width=0.8\textwidth]{\includegraphics[#1]{#2}}

	\usepackage{amsmath} % more flexibility equations (use in title!!!)
	\usepackage{pdfpages} % include pdf


	% ------------------
	% SECTION SETTINGS
	% ------------------

	\usepackage{titlesec}

	% section size
	\titleformat{\chapter}[block]
	  {\fontsize{18pt}{22pt}\bfseries}
	  {\chaptertitlename\ \thechapter.}{0.5em}{}
	\titleformat{\section}
		{\fontsize{16pt}{20pt}\bfseries}{\thesection.}{1em}{}
	\titleformat{\subsection}
		{\fontsize{14pt}{18pt}\bfseries}{\thesubsection.}{1em}{}
	\titleformat{\subsubsection}
		{\fontsize{14pt}{18pt}\bfseries}{\thesubsubsection.}{1em}{}
	\titlespacing{\chapter}{1.25cm}{1pt}{1pt}
	\titlespacing{\section}{1.25cm}{1pt}{1pt}
	\titlespacing{\subsection}{1.25cm}{1pt}{1pt}
	\titlespacing{\subsubsection}{1.25cm}{1pt}{1pt}
	\titlespacing{\paragraph}{1.25cm}{0pt}{0pt}

	% new command for sections
	\newcommand\Chapter[1]{
		\refstepcounter{chapter}
		\chapter*{\textbf{ГЛАВА\;\arabic{chapter}.}
			\raggedright #1
		}
		\addcontentsline{toc}{chapter}{ГЛАВА\;\arabic{chapter}.\ #1}
	}
	\newcommand\Section[1]{
		\refstepcounter{section}
		\section*{\textbf{\arabic{chapter}.\arabic{section}.}
			\raggedright #1
		}
		\addcontentsline{toc}{section}{\arabic{chapter}.\arabic{section}.\ #1}
	}
	\newcommand\Subsection[1]{
		\refstepcounter{subsection}
		\subsection*{\textbf{\arabic{chapter}.\arabic{section}.\arabic{subsection}.}
			\raggedright #1
		}
		\addcontentsline{toc}{subsection}
			{\arabic{chapter}.\arabic{section}.\arabic{subsection}.\ #1}
	}
	\newcommand\Subsubsection[1]{
		\refstepcounter{subsubsection}
		\subsubsection*{\textbf{\arabic{chapter}.\arabic{section}.\arabic{subsection}.\arabic{subsubsection}.}
			\raggedright #1
		}
		\addcontentsline{toc}{subsubsection}
			{\arabic{chapter}.\arabic{section}.\arabic{subsection}.\arabic{subsubsection}.\ #1}
	}

	% new command for appendix
	\newcommand\AppendixChapter[1]{
		\refstepcounter{chapter}
		\chapter*{\textbf{\appendixname\;\thechapter.}
			\raggedright #1
		}
		\addcontentsline{toc}{chapter}{\appendixname~\thechapter.\ #1}
	}
	\newcommand\AppendixSection[1]{
		\refstepcounter{section}
		\section*{\textbf{\appendixname\;\thesection.}
			\raggedright #1
		}
	}
	\newcommand\AppendixSubsection[1]{
		\refstepcounter{subsection}
		\section*{\textbf{\appendixname\;\thesubsection.}
			\raggedright #1
		}
	}

	% ------------------
	% NEW COMMAND
	% ------------------

	\providecommand{\No}{\textnumero}

	% ------------------
	% ------------------

    \usepackage[breakable]{tcolorbox}
    

    \usepackage{graphicx}
    \usepackage{caption}

    \usepackage{xcolor} % Allow colors to be defined
    \usepackage{geometry} % Used to adjust the document margins

    \usepackage{fancyvrb} % verbatim replacement that allows latex

    \makeatletter % fix for old versions of grffile with XeLaTeX
    \@ifpackagelater{grffile}{2019/11/01}
    {
      % Do nothing on new versions
    }
    {
      \def\Gread@@xetex#1{%
        \IfFileExists{"\Gin@base".bb}%
        {\Gread@eps{\Gin@base.bb}}%
        {\Gread@@xetex@aux#1}%
      }
    }
    \makeatother
    \usepackage[Export]{adjustbox} % Used to constrain images to a maximum size
    \adjustboxset{max size={0.9\linewidth}{0.9\paperheight}}

    % The hyperref package gives us a pdf with properly built
    % internal navigation ('pdf bookmarks' for the table of contents,
    % internal cross-reference links, web links for URLs, etc.)
    \usepackage{hyperref}
    % The default LaTeX title has an obnoxious amount of whitespace. By default,
    % titling removes some of it. It also provides customization options.
    \usepackage{titling}
    \usepackage{longtable} % longtable support required by pandoc >1.10
    \usepackage{booktabs}  % table support for pandoc > 1.12.2
    \usepackage{array}     % table support for pandoc >= 2.11.3
    \usepackage{calc}      % table minipage width calculation for pandoc >= 2.11.1
    \usepackage[inline]{enumitem} % IRkernel/repr support (it uses the enumerate* environment)
    \usepackage[normalem]{ulem} % ulem is needed to support strikethroughs (\sout)
                                % normalem makes italics be italics, not underlines
    \usepackage{mathrsfs}
    

    
    % Colors for the hyperref package
    \definecolor{urlcolor}{rgb}{0,.145,.698}
    \definecolor{linkcolor}{rgb}{.71,0.21,0.01}
    \definecolor{citecolor}{rgb}{.12,.54,.11}

    % ANSI colors
    \definecolor{ansi-black}{HTML}{3E424D}
    \definecolor{ansi-black-intense}{HTML}{282C36}
    \definecolor{ansi-red}{HTML}{E75C58}
    \definecolor{ansi-red-intense}{HTML}{B22B31}
    \definecolor{ansi-green}{HTML}{00A250}
    \definecolor{ansi-green-intense}{HTML}{007427}
    \definecolor{ansi-yellow}{HTML}{DDB62B}
    \definecolor{ansi-yellow-intense}{HTML}{B27D12}
    \definecolor{ansi-blue}{HTML}{208FFB}
    \definecolor{ansi-blue-intense}{HTML}{0065CA}
    \definecolor{ansi-magenta}{HTML}{D160C4}
    \definecolor{ansi-magenta-intense}{HTML}{A03196}
    \definecolor{ansi-cyan}{HTML}{60C6C8}
    \definecolor{ansi-cyan-intense}{HTML}{258F8F}
    \definecolor{ansi-white}{HTML}{C5C1B4}
    \definecolor{ansi-white-intense}{HTML}{A1A6B2}
    \definecolor{ansi-default-inverse-fg}{HTML}{FFFFFF}
    \definecolor{ansi-default-inverse-bg}{HTML}{000000}

    % common color for the border for error outputs.
    \definecolor{outerrorbackground}{HTML}{FFDFDF}

    % commands and environments needed by pandoc snippets
    % extracted from the output of `pandoc -s`
    \providecommand{\tightlist}{%
      \setlength{\itemsep}{0pt}\setlength{\parskip}{0pt}}
    \DefineVerbatimEnvironment{Highlighting}{Verbatim}{commandchars=\\\{\}}
    % Add ',fontsize=\small' for more characters per line
    \newenvironment{Shaded}{}{}
    \newcommand{\KeywordTok}[1]{\textcolor[rgb]{0.00,0.44,0.13}{\textbf{{#1}}}}
    \newcommand{\DataTypeTok}[1]{\textcolor[rgb]{0.56,0.13,0.00}{{#1}}}
    \newcommand{\DecValTok}[1]{\textcolor[rgb]{0.25,0.63,0.44}{{#1}}}
    \newcommand{\BaseNTok}[1]{\textcolor[rgb]{0.25,0.63,0.44}{{#1}}}
    \newcommand{\FloatTok}[1]{\textcolor[rgb]{0.25,0.63,0.44}{{#1}}}
    \newcommand{\CharTok}[1]{\textcolor[rgb]{0.25,0.44,0.63}{{#1}}}
    \newcommand{\StringTok}[1]{\textcolor[rgb]{0.25,0.44,0.63}{{#1}}}
    \newcommand{\CommentTok}[1]{\textcolor[rgb]{0.38,0.63,0.69}{\textit{{#1}}}}
    \newcommand{\OtherTok}[1]{\textcolor[rgb]{0.00,0.44,0.13}{{#1}}}
    \newcommand{\AlertTok}[1]{\textcolor[rgb]{1.00,0.00,0.00}{\textbf{{#1}}}}
    \newcommand{\FunctionTok}[1]{\textcolor[rgb]{0.02,0.16,0.49}{{#1}}}
    \newcommand{\RegionMarkerTok}[1]{{#1}}
    \newcommand{\ErrorTok}[1]{\textcolor[rgb]{1.00,0.00,0.00}{\textbf{{#1}}}}
    \newcommand{\NormalTok}[1]{{#1}}

    % Additional commands for more recent versions of Pandoc
    \newcommand{\ConstantTok}[1]{\textcolor[rgb]{0.53,0.00,0.00}{{#1}}}
    \newcommand{\SpecialCharTok}[1]{\textcolor[rgb]{0.25,0.44,0.63}{{#1}}}
    \newcommand{\VerbatimStringTok}[1]{\textcolor[rgb]{0.25,0.44,0.63}{{#1}}}
    \newcommand{\SpecialStringTok}[1]{\textcolor[rgb]{0.73,0.40,0.53}{{#1}}}
    \newcommand{\ImportTok}[1]{{#1}}
    \newcommand{\DocumentationTok}[1]{\textcolor[rgb]{0.73,0.13,0.13}{\textit{{#1}}}}
    \newcommand{\AnnotationTok}[1]{\textcolor[rgb]{0.38,0.63,0.69}{\textbf{\textit{{#1}}}}}
    \newcommand{\CommentVarTok}[1]{\textcolor[rgb]{0.38,0.63,0.69}{\textbf{\textit{{#1}}}}}
    \newcommand{\VariableTok}[1]{\textcolor[rgb]{0.10,0.09,0.49}{{#1}}}
    \newcommand{\ControlFlowTok}[1]{\textcolor[rgb]{0.00,0.44,0.13}{\textbf{{#1}}}}
    \newcommand{\OperatorTok}[1]{\textcolor[rgb]{0.40,0.40,0.40}{{#1}}}
    \newcommand{\BuiltInTok}[1]{{#1}}
    \newcommand{\ExtensionTok}[1]{{#1}}
    \newcommand{\PreprocessorTok}[1]{\textcolor[rgb]{0.74,0.48,0.00}{{#1}}}
    \newcommand{\AttributeTok}[1]{\textcolor[rgb]{0.49,0.56,0.16}{{#1}}}
    \newcommand{\InformationTok}[1]{\textcolor[rgb]{0.38,0.63,0.69}{\textbf{\textit{{#1}}}}}
    \newcommand{\WarningTok}[1]{\textcolor[rgb]{0.38,0.63,0.69}{\textbf{\textit{{#1}}}}}

    
    
    
    
	% Pygments definitions
	\makeatletter
	\def\PY@reset{\let\PY@it=\relax \let\PY@bf=\relax%
		\let\PY@ul=\relax \let\PY@tc=\relax%
		\let\PY@bc=\relax \let\PY@ff=\relax}
	\def\PY@tok#1{\csname PY@tok@#1\endcsname}
	\def\PY@toks#1+{\ifx\relax#1\empty\else%
		\PY@tok{#1}\expandafter\PY@toks\fi}
	\def\PY@do#1{\PY@bc{\PY@tc{\PY@ul{%
		\PY@it{\PY@bf{\PY@ff{#1}}}}}}}
	\def\PY#1#2{\PY@reset\PY@toks#1+\relax+\PY@do{#2}}

	\@namedef{PY@tok@w}{\def\PY@tc##1{\textcolor[rgb]{0.73,0.73,0.73}{##1}}}
	\@namedef{PY@tok@c}{\let\PY@it=\textit\def\PY@tc##1{\textcolor[rgb]{0.24,0.48,0.48}{##1}}}
	\@namedef{PY@tok@cp}{\def\PY@tc##1{\textcolor[rgb]{0.61,0.40,0.00}{##1}}}
	\@namedef{PY@tok@k}{\let\PY@bf=\textbf\def\PY@tc##1{\textcolor[rgb]{0.00,0.50,0.00}{##1}}}
	\@namedef{PY@tok@kp}{\def\PY@tc##1{\textcolor[rgb]{0.00,0.50,0.00}{##1}}}
	\@namedef{PY@tok@kt}{\def\PY@tc##1{\textcolor[rgb]{0.69,0.00,0.25}{##1}}}
	\@namedef{PY@tok@o}{\def\PY@tc##1{\textcolor[rgb]{0.40,0.40,0.40}{##1}}}
	\@namedef{PY@tok@ow}{\let\PY@bf=\textbf\def\PY@tc##1{\textcolor[rgb]{0.67,0.13,1.00}{##1}}}
	\@namedef{PY@tok@nb}{\def\PY@tc##1{\textcolor[rgb]{0.00,0.50,0.00}{##1}}}
	\@namedef{PY@tok@nf}{\def\PY@tc##1{\textcolor[rgb]{0.00,0.00,1.00}{##1}}}
	\@namedef{PY@tok@nc}{\let\PY@bf=\textbf\def\PY@tc##1{\textcolor[rgb]{0.00,0.00,1.00}{##1}}}
	\@namedef{PY@tok@nn}{\let\PY@bf=\textbf\def\PY@tc##1{\textcolor[rgb]{0.00,0.00,1.00}{##1}}}
	\@namedef{PY@tok@ne}{\let\PY@bf=\textbf\def\PY@tc##1{\textcolor[rgb]{0.80,0.25,0.22}{##1}}}
	\@namedef{PY@tok@nv}{\def\PY@tc##1{\textcolor[rgb]{0.10,0.09,0.49}{##1}}}
	\@namedef{PY@tok@no}{\def\PY@tc##1{\textcolor[rgb]{0.53,0.00,0.00}{##1}}}
	\@namedef{PY@tok@nl}{\def\PY@tc##1{\textcolor[rgb]{0.46,0.46,0.00}{##1}}}
	\@namedef{PY@tok@ni}{\let\PY@bf=\textbf\def\PY@tc##1{\textcolor[rgb]{0.44,0.44,0.44}{##1}}}
	\@namedef{PY@tok@na}{\def\PY@tc##1{\textcolor[rgb]{0.41,0.47,0.13}{##1}}}
	\@namedef{PY@tok@nt}{\let\PY@bf=\textbf\def\PY@tc##1{\textcolor[rgb]{0.00,0.50,0.00}{##1}}}
	\@namedef{PY@tok@nd}{\def\PY@tc##1{\textcolor[rgb]{0.67,0.13,1.00}{##1}}}
	\@namedef{PY@tok@s}{\def\PY@tc##1{\textcolor[rgb]{0.73,0.13,0.13}{##1}}}
	\@namedef{PY@tok@sd}{\let\PY@it=\textit\def\PY@tc##1{\textcolor[rgb]{0.73,0.13,0.13}{##1}}}
	\@namedef{PY@tok@si}{\let\PY@bf=\textbf\def\PY@tc##1{\textcolor[rgb]{0.64,0.35,0.47}{##1}}}
	\@namedef{PY@tok@se}{\let\PY@bf=\textbf\def\PY@tc##1{\textcolor[rgb]{0.67,0.36,0.12}{##1}}}
	\@namedef{PY@tok@sr}{\def\PY@tc##1{\textcolor[rgb]{0.64,0.35,0.47}{##1}}}
	\@namedef{PY@tok@ss}{\def\PY@tc##1{\textcolor[rgb]{0.10,0.09,0.49}{##1}}}
	\@namedef{PY@tok@sx}{\def\PY@tc##1{\textcolor[rgb]{0.00,0.50,0.00}{##1}}}
	\@namedef{PY@tok@m}{\def\PY@tc##1{\textcolor[rgb]{0.40,0.40,0.40}{##1}}}
	\@namedef{PY@tok@gh}{\let\PY@bf=\textbf\def\PY@tc##1{\textcolor[rgb]{0.00,0.00,0.50}{##1}}}
	\@namedef{PY@tok@gu}{\let\PY@bf=\textbf\def\PY@tc##1{\textcolor[rgb]{0.50,0.00,0.50}{##1}}}
	\@namedef{PY@tok@gd}{\def\PY@tc##1{\textcolor[rgb]{0.63,0.00,0.00}{##1}}}
	\@namedef{PY@tok@gi}{\def\PY@tc##1{\textcolor[rgb]{0.00,0.52,0.00}{##1}}}
	\@namedef{PY@tok@gr}{\def\PY@tc##1{\textcolor[rgb]{0.89,0.00,0.00}{##1}}}
	\@namedef{PY@tok@ge}{\let\PY@it=\textit}
	\@namedef{PY@tok@gs}{\let\PY@bf=\textbf}
	\@namedef{PY@tok@gp}{\let\PY@bf=\textbf\def\PY@tc##1{\textcolor[rgb]{0.00,0.00,0.50}{##1}}}
	\@namedef{PY@tok@go}{\def\PY@tc##1{\textcolor[rgb]{0.44,0.44,0.44}{##1}}}
	\@namedef{PY@tok@gt}{\def\PY@tc##1{\textcolor[rgb]{0.00,0.27,0.87}{##1}}}
	\@namedef{PY@tok@err}{\def\PY@bc##1{{\setlength{\fboxsep}{\string -\fboxrule}\fcolorbox[rgb]{1.00,0.00,0.00}{1,1,1}{\strut ##1}}}}
	\@namedef{PY@tok@kc}{\let\PY@bf=\textbf\def\PY@tc##1{\textcolor[rgb]{0.00,0.50,0.00}{##1}}}
	\@namedef{PY@tok@kd}{\let\PY@bf=\textbf\def\PY@tc##1{\textcolor[rgb]{0.00,0.50,0.00}{##1}}}
	\@namedef{PY@tok@kn}{\let\PY@bf=\textbf\def\PY@tc##1{\textcolor[rgb]{0.00,0.50,0.00}{##1}}}
	\@namedef{PY@tok@kr}{\let\PY@bf=\textbf\def\PY@tc##1{\textcolor[rgb]{0.00,0.50,0.00}{##1}}}
	\@namedef{PY@tok@bp}{\def\PY@tc##1{\textcolor[rgb]{0.00,0.50,0.00}{##1}}}
	\@namedef{PY@tok@fm}{\def\PY@tc##1{\textcolor[rgb]{0.00,0.00,1.00}{##1}}}
	\@namedef{PY@tok@vc}{\def\PY@tc##1{\textcolor[rgb]{0.10,0.09,0.49}{##1}}}
	\@namedef{PY@tok@vg}{\def\PY@tc##1{\textcolor[rgb]{0.10,0.09,0.49}{##1}}}
	\@namedef{PY@tok@vi}{\def\PY@tc##1{\textcolor[rgb]{0.10,0.09,0.49}{##1}}}
	\@namedef{PY@tok@vm}{\def\PY@tc##1{\textcolor[rgb]{0.10,0.09,0.49}{##1}}}
	\@namedef{PY@tok@sa}{\def\PY@tc##1{\textcolor[rgb]{0.73,0.13,0.13}{##1}}}
	\@namedef{PY@tok@sb}{\def\PY@tc##1{\textcolor[rgb]{0.73,0.13,0.13}{##1}}}
	\@namedef{PY@tok@sc}{\def\PY@tc##1{\textcolor[rgb]{0.73,0.13,0.13}{##1}}}
	\@namedef{PY@tok@dl}{\def\PY@tc##1{\textcolor[rgb]{0.73,0.13,0.13}{##1}}}
	\@namedef{PY@tok@s2}{\def\PY@tc##1{\textcolor[rgb]{0.73,0.13,0.13}{##1}}}
	\@namedef{PY@tok@sh}{\def\PY@tc##1{\textcolor[rgb]{0.73,0.13,0.13}{##1}}}
	\@namedef{PY@tok@s1}{\def\PY@tc##1{\textcolor[rgb]{0.73,0.13,0.13}{##1}}}
	\@namedef{PY@tok@mb}{\def\PY@tc##1{\textcolor[rgb]{0.40,0.40,0.40}{##1}}}
	\@namedef{PY@tok@mf}{\def\PY@tc##1{\textcolor[rgb]{0.40,0.40,0.40}{##1}}}
	\@namedef{PY@tok@mh}{\def\PY@tc##1{\textcolor[rgb]{0.40,0.40,0.40}{##1}}}
	\@namedef{PY@tok@mi}{\def\PY@tc##1{\textcolor[rgb]{0.40,0.40,0.40}{##1}}}
	\@namedef{PY@tok@il}{\def\PY@tc##1{\textcolor[rgb]{0.40,0.40,0.40}{##1}}}
	\@namedef{PY@tok@mo}{\def\PY@tc##1{\textcolor[rgb]{0.40,0.40,0.40}{##1}}}
	\@namedef{PY@tok@ch}{\let\PY@it=\textit\def\PY@tc##1{\textcolor[rgb]{0.24,0.48,0.48}{##1}}}
	\@namedef{PY@tok@cm}{\let\PY@it=\textit\def\PY@tc##1{\textcolor[rgb]{0.24,0.48,0.48}{##1}}}
	\@namedef{PY@tok@cpf}{\let\PY@it=\textit\def\PY@tc##1{\textcolor[rgb]{0.24,0.48,0.48}{##1}}}
	\@namedef{PY@tok@c1}{\let\PY@it=\textit\def\PY@tc##1{\textcolor[rgb]{0.24,0.48,0.48}{##1}}}
	\@namedef{PY@tok@cs}{\let\PY@it=\textit\def\PY@tc##1{\textcolor[rgb]{0.24,0.48,0.48}{##1}}}

	\def\PYZbs{\char`\\}
	\def\PYZus{\char`\_}
	\def\PYZob{\char`\{}
	\def\PYZcb{\char`\}}
	\def\PYZca{\char`\^}
	\def\PYZam{\char`\&}
	\def\PYZlt{\char`\<}
	\def\PYZgt{\char`\>}
	\def\PYZsh{\char`\#}
	\def\PYZpc{\char`\%}
	\def\PYZdl{\char`\$}
	\def\PYZhy{\char`\-}
	\def\PYZsq{\char`\'}
	\def\PYZdq{\char`\"}
	\def\PYZti{\char`\~}
	% for compatibility with earlier versions
	\def\PYZat{@}
	\def\PYZlb{[}
	\def\PYZrb{]}
	\makeatother


    % For linebreaks inside Verbatim environment from package fancyvrb.
    \makeatletter
        \newbox\Wrappedcontinuationbox
        \newbox\Wrappedvisiblespacebox
        \newcommand*\Wrappedvisiblespace {\textcolor{red}{\textvisiblespace}}
        \newcommand*\Wrappedcontinuationsymbol {\textcolor{red}{\llap{\tiny$\m@th\hookrightarrow$}}}
        \newcommand*\Wrappedcontinuationindent {3ex }
        \newcommand*\Wrappedafterbreak {\kern\Wrappedcontinuationindent\copy\Wrappedcontinuationbox}
        % Take advantage of the already applied Pygments mark-up to insert
        % potential linebreaks for TeX processing.
        %        {, <, #, %, $, ' and ": go to next line.
        %        _, }, ^, &, >, - and ~: stay at end of broken line.
        % Use of \textquotesingle for straight quote.
        \newcommand*\Wrappedbreaksatspecials {%
            \def\PYGZus{\discretionary{\char`\_}{\Wrappedafterbreak}{\char`\_}}%
            \def\PYGZob{\discretionary{}{\Wrappedafterbreak\char`\{}{\char`\{}}%
            \def\PYGZcb{\discretionary{\char`\}}{\Wrappedafterbreak}{\char`\}}}%
            \def\PYGZca{\discretionary{\char`\^}{\Wrappedafterbreak}{\char`\^}}%
            \def\PYGZam{\discretionary{\char`\&}{\Wrappedafterbreak}{\char`\&}}%
            \def\PYGZlt{\discretionary{}{\Wrappedafterbreak\char`\<}{\char`\<}}%
            \def\PYGZgt{\discretionary{\char`\>}{\Wrappedafterbreak}{\char`\>}}%
            \def\PYGZsh{\discretionary{}{\Wrappedafterbreak\char`\#}{\char`\#}}%
            \def\PYGZpc{\discretionary{}{\Wrappedafterbreak\char`\%}{\char`\%}}%
            \def\PYGZdl{\discretionary{}{\Wrappedafterbreak\char`\$}{\char`\$}}%
            \def\PYGZhy{\discretionary{\char`\-}{\Wrappedafterbreak}{\char`\-}}%
            \def\PYGZsq{\discretionary{}{\Wrappedafterbreak\textquotesingle}{\textquotesingle}}%
            \def\PYGZdq{\discretionary{}{\Wrappedafterbreak\char`\"}{\char`\"}}%
            \def\PYGZti{\discretionary{\char`\~}{\Wrappedafterbreak}{\char`\~}}%
        }
        % Some characters . , ; ? ! / are not pygmentized.
        % This macro makes them "active" and they will insert potential linebreaks
        \newcommand*\Wrappedbreaksatpunct {%
            \lccode`\~`\.\lowercase{\def~}{\discretionary{\hbox{\char`\.}}{\Wrappedafterbreak}{\hbox{\char`\.}}}%
            \lccode`\~`\,\lowercase{\def~}{\discretionary{\hbox{\char`\,}}{\Wrappedafterbreak}{\hbox{\char`\,}}}%
            \lccode`\~`\;\lowercase{\def~}{\discretionary{\hbox{\char`\;}}{\Wrappedafterbreak}{\hbox{\char`\;}}}%
            \lccode`\~`\:\lowercase{\def~}{\discretionary{\hbox{\char`\:}}{\Wrappedafterbreak}{\hbox{\char`\:}}}%
            \lccode`\~`\?\lowercase{\def~}{\discretionary{\hbox{\char`\?}}{\Wrappedafterbreak}{\hbox{\char`\?}}}%
            \lccode`\~`\!\lowercase{\def~}{\discretionary{\hbox{\char`\!}}{\Wrappedafterbreak}{\hbox{\char`\!}}}%
            \lccode`\~`\/\lowercase{\def~}{\discretionary{\hbox{\char`\/}}{\Wrappedafterbreak}{\hbox{\char`\/}}}%
            \catcode`\.\active
            \catcode`\,\active
            \catcode`\;\active
            \catcode`\:\active
            \catcode`\?\active
            \catcode`\!\active
            \catcode`\/\active
            \lccode`\~`\~
        }
    \makeatother

    \let\OriginalVerbatim=\Verbatim
    \makeatletter
    \renewcommand{\Verbatim}[1][1]{%
        %\parskip\z@skip
        \sbox\Wrappedcontinuationbox {\Wrappedcontinuationsymbol}%
        \sbox\Wrappedvisiblespacebox {\FV@SetupFont\Wrappedvisiblespace}%
        \def\FancyVerbFormatLine ##1{\hsize\linewidth
            \vtop{\raggedright\hyphenpenalty\z@\exhyphenpenalty\z@
                \doublehyphendemerits\z@\finalhyphendemerits\z@
                \strut ##1\strut}%
        }%
        % If the linebreak is at a space, the latter will be displayed as visible
        % space at end of first line, and a continuation symbol starts next line.
        % Stretch/shrink are however usually zero for typewriter font.
        \def\FV@Space {%
            \nobreak\hskip\z@ plus\fontdimen3\font minus\fontdimen4\font
            \discretionary{\copy\Wrappedvisiblespacebox}{\Wrappedafterbreak}
            {\kern\fontdimen2\font}%
        }%

        % Allow breaks at special characters using \PYG... macros.
        \Wrappedbreaksatspecials
        % Breaks at punctuation characters . , ; ? ! and / need catcode=\active
        \OriginalVerbatim[#1,codes*=\Wrappedbreaksatpunct]%
    }
    \makeatother

    % Exact colors from NB
    \definecolor{incolor}{HTML}{303F9F}
    \definecolor{outcolor}{HTML}{D84315}
    \definecolor{cellborder}{HTML}{CFCFCF}
    \definecolor{cellbackground}{HTML}{F7F7F7}

    % prompt
    \makeatletter
    \newcommand{\boxspacing}{\kern\kvtcb@left@rule\kern\kvtcb@boxsep}
    \makeatother
    \newcommand{\prompt}[4]{
        {\ttfamily\llap{{\color{#2}[#3]:\hspace{3pt}#4}}\vspace{-\baselineskip}}
    }
    

    
    % Prevent overflowing lines due to hard-to-break entities
    \sloppy
    % Setup hyperref package
    \hypersetup{
      breaklinks=true,  % so long urls are correctly broken across lines
      colorlinks=true,
      urlcolor=urlcolor,
      linkcolor=linkcolor,
      citecolor=citecolor,
      }
    

\begin{document}

\singlespacing % 1.0
\begin{titlepage}
	\thispagestyle{fancy}
	\renewcommand{\headrulewidth}{0pt}
	\cfoot{Москва 2024}

	\centering
	\includegraphics[scale=0.75]{./res/logo2} \break % вставка логотипа
	{\footnotesize МИНИСТЕРСТВО НАУКИ
		И ВЫСШЕГО ОБРАЗОВАНИЯ РОССИЙСКОЙ ФЕДЕРАЦИИ}\\
	Федеральное государственное бюджетное образовательное учреждение 
		высшего образования\\
	\textbf{<<МИРЭА --- Российский технологический университет>>}\\
	\vfill
	\textbf{\large РТУ МИРЭА}\\
	\bigskip \hrule \smallskip \hrule \smallskip
	\vfill
	Институт информационных технологий (ИТ)\\
	Математического обеспечения
		и стандартизации информационных технологий (МОСИТ)\\
	\vfill
	\textbf{ОТЧЕТ ПО ПРАКТИЧЕСКОЙ РАБОТЕ \No\,1}\\
	\textbf{по дисциплине}\\
	\textbf{<<Разработка кроссплатформенных мобильных приложений>>}\\
	\vfill
	\vfill
	\vfill
	\vfill
	\begin{tabular}{p{0.7\textwidth}p{0.2\textwidth}}
		Выполнил студент группы ИКБО-06-21 & \rightline{Бондарь А.Р.} \\
		Принял старший преподаватель & \rightline{Шешуков Л.С.} \\
	\end{tabular}
	\vfill
	\vfill
	\vfill
	\vfill
\end{titlepage}
\onehalfspacing % 1.5
\setcounter{page}{2}
\clearpage



\section{1}

Найти и выгрузить многомерные данные (с большим количеством признаков --
столбцов) с использованием библиотеки pandas. В отчёте описать найденные
данные.

\begin{tcolorbox}[breakable, size=fbox, boxrule=1pt,
	pad at break*=1mm,colback=cellbackground, colframe=cellborder]
\prompt{In}{incolor}{1}{\boxspacing}
\begin{Verbatim}[commandchars=\\\{\}]
\PY{k+kn}{import} \PY{n+nn}{numpy} \PY{k}{as} \PY{n+nn}{np}
\PY{k+kn}{import} \PY{n+nn}{pandas} \PY{k}{as} \PY{n+nn}{pd}
\end{Verbatim}
\end{tcolorbox}

\begin{tcolorbox}[breakable, size=fbox, boxrule=1pt,
	pad at break*=1mm,colback=cellbackground, colframe=cellborder]
\prompt{In}{incolor}{52}{\boxspacing}
\begin{Verbatim}[commandchars=\\\{\}]
\PY{n}{input\PYZus{}file} \PY{o}{=} \PY{l+s+s2}{\PYZdq{}}\PY{l+s+s2}{food\PYZus{}recipes.csv}\PY{l+s+s2}{\PYZdq{}}
\PY{n}{df} \PY{o}{=} \PY{n}{pd}\PY{o}{.}\PY{n}{read\PYZus{}csv}\PY{p}{(}\PY{n}{input\PYZus{}file}\PY{p}{,} \PY{n}{header} \PY{o}{=} \PY{l+m+mi}{0}\PY{p}{)}
\PY{n}{df}\PY{p}{[}\PY{l+s+s1}{\PYZsq{}}\PY{l+s+s1}{prep\PYZus{}time}\PY{l+s+s1}{\PYZsq{}}\PY{p}{]} \PY{o}{=} \PY{n}{df}\PY{p}{[}\PY{l+s+s1}{\PYZsq{}}\PY{l+s+s1}{prep\PYZus{}time}\PY{l+s+s1}{\PYZsq{}}\PY{p}{]}\PY{o}{.}\PY{n}{str}\PY{o}{.}\PY{n}{extract}\PY{p}{(}\PY{l+s+s1}{\PYZsq{}}\PY{l+s+s1}{(}\PY{l+s+s1}{\PYZbs{}}\PY{l+s+s1}{d+)}\PY{l+s+s1}{\PYZsq{}}\PY{p}{)}\PY{o}{.}\PY{n}{astype}\PY{p}{(}\PY{n+nb}{float}\PY{p}{)}
\PY{n}{df}\PY{p}{[}\PY{l+s+s1}{\PYZsq{}}\PY{l+s+s1}{cook\PYZus{}time}\PY{l+s+s1}{\PYZsq{}}\PY{p}{]} \PY{o}{=} \PY{n}{df}\PY{p}{[}\PY{l+s+s1}{\PYZsq{}}\PY{l+s+s1}{cook\PYZus{}time}\PY{l+s+s1}{\PYZsq{}}\PY{p}{]}\PY{o}{.}\PY{n}{str}\PY{o}{.}\PY{n}{extract}\PY{p}{(}\PY{l+s+s1}{\PYZsq{}}\PY{l+s+s1}{(}\PY{l+s+s1}{\PYZbs{}}\PY{l+s+s1}{d+)}\PY{l+s+s1}{\PYZsq{}}\PY{p}{)}\PY{o}{.}\PY{n}{astype}\PY{p}{(}\PY{n+nb}{float}\PY{p}{)}
\end{Verbatim}
\end{tcolorbox}

\section{2}

Вывести информацию о данных при помощи методов .info(), .head().
Проверить данные на наличие пустых значений. В случае их наличия удалить
данные строки или интерполировать пропущенные значения. При
необходимости дополнительно предобработать данные для дальнейшей работы
с ними.

\begin{tcolorbox}[breakable, size=fbox, boxrule=1pt,
	pad at break*=1mm,colback=cellbackground, colframe=cellborder]
\prompt{In}{incolor}{53}{\boxspacing}
\begin{Verbatim}[commandchars=\\\{\}]
\PY{n}{df}\PY{o}{.}\PY{n}{info}\PY{p}{(}\PY{p}{)}
\end{Verbatim}
\end{tcolorbox}

\begin{Verbatim}[commandchars=\\\{\}]
<class 'pandas.core.frame.DataFrame'>
RangeIndex: 8009 entries, 0 to 8008
Data columns (total 16 columns):
\#   Column         Non-Null Count  Dtype
---  ------         --------------  -----
0   recipe\_title   8009 non-null   object
1   url            8009 non-null   object
2   record\_health  8009 non-null   object
3   vote\_count     8009 non-null   int64
4   rating         8009 non-null   float64
5   description    7994 non-null   object
6   cuisine        7943 non-null   object
7   course         7854 non-null   object
8   diet           7858 non-null   object
9   prep\_time      7979 non-null   float64
10  cook\_time      7979 non-null   float64
11  ingredients    7997 non-null   object
12  instructions   8009 non-null   object
13  author         8009 non-null   object
14  tags           7930 non-null   object
15  category       8009 non-null   object
dtypes: float64(3), int64(1), object(12)
memory usage: 1001.2+ KB
\end{Verbatim}

\begin{tcolorbox}[breakable, size=fbox, boxrule=1pt,
	pad at break*=1mm,colback=cellbackground, colframe=cellborder]
\prompt{In}{incolor}{54}{\boxspacing}
\begin{Verbatim}[commandchars=\\\{\}]
\PY{n}{df}\PY{o}{.}\PY{n}{head}\PY{p}{(}\PY{p}{)}
\end{Verbatim}
\end{tcolorbox}

\begin{tcolorbox}[breakable, size=fbox, boxrule=.5pt,
	pad at break*=1mm, opacityfill=0]
\prompt{Out}{outcolor}{54}{\boxspacing}
\begin{Verbatim}[commandchars=\\\{\}]
	recipe\_title  \textbackslash{}
0  Roasted Peppers And Mushroom Tortilla Pizza Re{\ldots}
1  Thakkali Gotsu Recipe | Thakkali Curry | Spicy{\ldots}
2               Spicy Grilled Pineapple Salsa Recipe
3     Karwar Style Dali Thoy Recipe - Toor dal Curry
4    Rajma Kofta In Milk And Poppy Seed Gravy Recipe

			 url record\_health  \textbackslash{}
0  https://www.archanaskitchen.com/roasted-pepper{\ldots}          good
1  https://www.archanaskitchen.com/tomato-gotsu-r{\ldots}          good
2  https://www.archanaskitchen.com/spicy-grilled-{\ldots}          good
3  https://www.archanaskitchen.com/dali-thoy-reci{\ldots}          good
4  https://www.archanaskitchen.com/rajma-kofta-in{\ldots}          good

vote\_count    rating                                        description  \textbackslash{}
0         434  4.958525   is a quicker version pizza to satisfy your cr{\ldots}
1        3423  4.932223   also known as the   is a quick and easy to ma{\ldots}
2        2091  4.945959  Spicy Grilled Pineapple Salsa is a simple reci{\ldots}
3         990  4.888889  The   is a quintessential of Konkani dish whic{\ldots}
4         345  4.828986  Koftas are traditional Indian recipes mostly w{\ldots}

cuisine     course                     diet  prep\_time  \textbackslash{}
0               Mexican     Dinner               Vegetarian       15.0
1  South Indian Recipes      Lunch               Vegetarian       10.0
2               Mexican  Side Dish               Vegetarian       10.0
3     Coastal Karnataka  Side Dish  High Protein Vegetarian        5.0
4  North Indian Recipes  Side Dish  High Protein Vegetarian       20.0

cook\_time                                        ingredients  \textbackslash{}
0       15.0  Tortillas|Extra Virgin Olive Oil|Garlic|Mozzar{\ldots}
1       20.0  Sesame (Gingelly) Oil|Mustard seeds (Rai/ Kadu{\ldots}
2        0.0  Extra Virgin Olive Oil|Pineapple|White onion|R{\ldots}
3       20.0  Arhar dal (Split Toor Dal)|Turmeric powder (Ha{\ldots}
4       30.0  Rajma (Large Kidney Beans)|Cashew nuts|Sultana{\ldots}

	instructions             author  \textbackslash{}
0  To begin making the Roasted Peppers And Mushro{\ldots}  Divya Shivaraman
1  To begin making Tomato Gotsu Recipe/ Thakkali {\ldots}      Archana Doshi
2  To begin making the Spicy Grilled Pineapple Sa{\ldots}  Archana's Kitchen
3  To prepare Karwar Style Dali Thoy Recipe (Toor{\ldots}      Jyothi Rajesh
4  To begin making Rajma Kofta In Milk And Poppy {\ldots}        RUBY PATHAK

			tags              category
0  Party Food Recipes|Tea Party Recipes|Mushroom {\ldots}         Pizza Recipes
1  Vegetarian Recipes|Tomato Recipes|South Indian{\ldots}  Indian Curry Recipes
2  Party Starter \& Appetizer Recipes|Pineapple Re{\ldots}       Mexican Recipes
3  Side Dish Recipes|South Indian Recipes|Indian {\ldots}  Indian Curry Recipes
4  Side Dish Recipes|Indian Lunch Recipes|Office {\ldots}         Kofta Recipes
\end{Verbatim}
\end{tcolorbox}

\section{3}

Построить столбчатую диаграмму (.bar) с использованием модуля
graph\_objs из библиотеки Plotly со следующими параметрами:

\begin{enumerate}
	\item По оси Х указать дату или название, по оси У указать количественный показатель.
	\item Сделать так, чтобы столбец принимал цвет в зависимости от значения
показателя (marker=dict(color=признак, coloraxis=``coloraxis'')).
	\item Сделать так, чтобы границы каждого столбца были выделены чёрной линией с толщиной равной 2.
	\item Отобразить заголовок диаграммы, разместив его по центру сверху, с 20 размером текста.
	\item Добавить подписи для осей X и Y с размером текста, равным 16. Для оси абсцисс развернуть метки так, чтобы они читались под углом, равным 315.
	\item Размер текста меток осей сделать равным 14.
	\item Расположить график во всю ширину рабочей области и присвоить высоту, равную 700 пикселей.
	\item Добавить сетку на график, сделать её цвет `ivory' и толщину равную 2. (Можно сделать это при настройке осей с помощью gridwidth=2, gridcolor=`ivory')
	\item Убрать лишние отступы по краям.
\end{enumerate}


\begin{tcolorbox}[breakable, size=fbox, boxrule=1pt,
	pad at break*=1mm,colback=cellbackground, colframe=cellborder]
\prompt{In}{incolor}{55}{\boxspacing}
\begin{Verbatim}[commandchars=\\\{\}]
\PY{o}{!}pip install plotly
\end{Verbatim}
\end{tcolorbox}

\begin{tcolorbox}[breakable, size=fbox, boxrule=1pt,
	pad at break*=1mm,colback=cellbackground, colframe=cellborder]
\prompt{In}{incolor}{56}{\boxspacing}
\begin{Verbatim}[commandchars=\\\{\}]
\PY{k+kn}{import} \PY{n+nn}{plotly}\PY{n+nn}{.}\PY{n+nn}{graph\PYZus{}objs} \PY{k}{as} \PY{n+nn}{go}
\PY{k+kn}{import} \PY{n+nn}{plotly}\PY{n+nn}{.}\PY{n+nn}{express} \PY{k}{as} \PY{n+nn}{px}
\end{Verbatim}
\end{tcolorbox}

\begin{tcolorbox}[breakable, size=fbox, boxrule=1pt,
	pad at break*=1mm,colback=cellbackground, colframe=cellborder]
\prompt{In}{incolor}{57}{\boxspacing}
\begin{Verbatim}[commandchars=\\\{\}]
\PY{n}{recipes} \PY{o}{=} \PY{n}{df}\PY{p}{[}\PY{l+s+s1}{\PYZsq{}}\PY{l+s+s1}{recipe\PYZus{}title}\PY{l+s+s1}{\PYZsq{}}\PY{p}{]}\PY{p}{[}\PY{p}{:}\PY{l+m+mi}{10}\PY{p}{]}
\PY{n}{votes} \PY{o}{=} \PY{n}{df}\PY{p}{[}\PY{l+s+s1}{\PYZsq{}}\PY{l+s+s1}{vote\PYZus{}count}\PY{l+s+s1}{\PYZsq{}}\PY{p}{]}\PY{p}{[}\PY{p}{:}\PY{l+m+mi}{10}\PY{p}{]}
\end{Verbatim}
\end{tcolorbox}

\begin{tcolorbox}[breakable, size=fbox, boxrule=1pt,
	pad at break*=1mm,colback=cellbackground, colframe=cellborder]
\prompt{In}{incolor}{58}{\boxspacing}
\begin{Verbatim}[commandchars=\\\{\}]
\PY{k}{def} \PY{n+nf}{fig\PYZus{}update}\PY{p}{(}\PY{n}{fig}\PY{p}{,} \PY{n}{title}\PY{p}{,} \PY{n}{xtitle}\PY{o}{=}\PY{l+s+s1}{\PYZsq{}}\PY{l+s+s1}{\PYZsq{}}\PY{p}{,} \PY{n}{ytitle}\PY{o}{=}\PY{l+s+s1}{\PYZsq{}}\PY{l+s+s1}{\PYZsq{}}\PY{p}{)}\PY{p}{:}
\PY{n}{fig}\PY{o}{.}\PY{n}{update\PYZus{}layout}\PY{p}{(}
\PY{n}{title}\PY{o}{=}\PY{n+nb}{dict}\PY{p}{(}
\PY{n}{text}\PY{o}{=}\PY{l+s+s1}{\PYZsq{}}\PY{l+s+s1}{Заголовок диаграммы}\PY{l+s+s1}{\PYZsq{}}\PY{p}{,}
\PY{n}{x}\PY{o}{=}\PY{l+m+mf}{0.5}\PY{p}{,}  \PY{c+c1}{\PYZsh{} По центру}
\PY{n}{font}\PY{o}{=}\PY{n+nb}{dict}\PY{p}{(}\PY{n}{size}\PY{o}{=}\PY{l+m+mi}{20}\PY{p}{)}  \PY{c+c1}{\PYZsh{} Размер текста заголовка}
\PY{p}{)}\PY{p}{,}
\PY{n}{xaxis}\PY{o}{=}\PY{n+nb}{dict}\PY{p}{(}
\PY{n}{title}\PY{o}{=}\PY{n+nb}{dict}\PY{p}{(}\PY{n}{text}\PY{o}{=}\PY{l+s+s1}{\PYZsq{}}\PY{l+s+s1}{Ось X}\PY{l+s+s1}{\PYZsq{}}\PY{p}{,} \PY{n}{font}\PY{o}{=}\PY{n+nb}{dict}\PY{p}{(}\PY{n}{size}\PY{o}{=}\PY{l+m+mi}{16}\PY{p}{)}\PY{p}{)}\PY{p}{,}  \PY{c+c1}{\PYZsh{} Подпись оси X}
\PY{n}{tickangle}\PY{o}{=}\PY{l+m+mi}{315}\PY{p}{,}  \PY{c+c1}{\PYZsh{} Угол наклона меток оси X}
\PY{n}{tickfont}\PY{o}{=}\PY{n+nb}{dict}\PY{p}{(}\PY{n}{size}\PY{o}{=}\PY{l+m+mi}{14}\PY{p}{)}\PY{p}{,}  \PY{c+c1}{\PYZsh{} Размер текста меток оси X}
\PY{n}{showgrid}\PY{o}{=}\PY{k+kc}{True}\PY{p}{,}
\PY{n}{gridwidth}\PY{o}{=}\PY{l+m+mi}{2}\PY{p}{,}
\PY{n}{gridcolor}\PY{o}{=}\PY{l+s+s1}{\PYZsq{}}\PY{l+s+s1}{ivory}\PY{l+s+s1}{\PYZsq{}}
\PY{p}{)}\PY{p}{,}
\PY{n}{yaxis}\PY{o}{=}\PY{n+nb}{dict}\PY{p}{(}
\PY{n}{title}\PY{o}{=}\PY{n+nb}{dict}\PY{p}{(}\PY{n}{text}\PY{o}{=}\PY{l+s+s1}{\PYZsq{}}\PY{l+s+s1}{Ось Y}\PY{l+s+s1}{\PYZsq{}}\PY{p}{,} \PY{n}{font}\PY{o}{=}\PY{n+nb}{dict}\PY{p}{(}\PY{n}{size}\PY{o}{=}\PY{l+m+mi}{16}\PY{p}{)}\PY{p}{)}\PY{p}{,}  \PY{c+c1}{\PYZsh{} Подпись оси Y}
\PY{n}{tickfont}\PY{o}{=}\PY{n+nb}{dict}\PY{p}{(}\PY{n}{size}\PY{o}{=}\PY{l+m+mi}{14}\PY{p}{)}\PY{p}{,}  \PY{c+c1}{\PYZsh{} Размер текста меток оси Y}
\PY{n}{showgrid}\PY{o}{=}\PY{k+kc}{True}\PY{p}{,}
\PY{n}{gridwidth}\PY{o}{=}\PY{l+m+mi}{2}\PY{p}{,}
\PY{n}{gridcolor}\PY{o}{=}\PY{l+s+s1}{\PYZsq{}}\PY{l+s+s1}{ivory}\PY{l+s+s1}{\PYZsq{}}
\PY{p}{)}\PY{p}{,}
\PY{n}{margin}\PY{o}{=}\PY{n+nb}{dict}\PY{p}{(}\PY{n}{l}\PY{o}{=}\PY{l+m+mi}{0}\PY{p}{,} \PY{n}{r}\PY{o}{=}\PY{l+m+mi}{0}\PY{p}{,} \PY{n}{t}\PY{o}{=}\PY{l+m+mi}{50}\PY{p}{,} \PY{n}{b}\PY{o}{=}\PY{l+m+mi}{0}\PY{p}{)}\PY{p}{,}  \PY{c+c1}{\PYZsh{} Убираем лишние отступы}
\PY{n}{width}\PY{o}{=}\PY{l+m+mi}{1200}\PY{p}{,}  \PY{c+c1}{\PYZsh{} Ширина графика}
\PY{n}{height}\PY{o}{=}\PY{l+m+mi}{700}\PY{p}{,}  \PY{c+c1}{\PYZsh{} Высота графика}
\PY{p}{)}
\end{Verbatim}
\end{tcolorbox}

\begin{tcolorbox}[breakable, size=fbox, boxrule=1pt,
	pad at break*=1mm,colback=cellbackground, colframe=cellborder]
\prompt{In}{incolor}{59}{\boxspacing}
\begin{Verbatim}[commandchars=\\\{\}]
\PY{n}{fig} \PY{o}{=} \PY{n}{go}\PY{o}{.}\PY{n}{Figure}\PY{p}{(}\PY{p}{)}
\PY{n}{fig}\PY{o}{.}\PY{n}{add\PYZus{}trace}\PY{p}{(}\PY{n}{go}\PY{o}{.}\PY{n}{Bar}\PY{p}{(}
\PY{n}{x}\PY{o}{=}\PY{n}{recipes}\PY{p}{,}
\PY{n}{y}\PY{o}{=}\PY{n}{votes}\PY{p}{,}
\PY{n}{marker}\PY{o}{=}\PY{n+nb}{dict}\PY{p}{(}
\PY{n}{color}\PY{o}{=}\PY{n}{votes}\PY{p}{,}
\PY{n}{coloraxis}\PY{o}{=}\PY{l+s+s2}{\PYZdq{}}\PY{l+s+s2}{coloraxis}\PY{l+s+s2}{\PYZdq{}}\PY{p}{,}
\PY{n}{line}\PY{o}{=}\PY{n+nb}{dict}\PY{p}{(}\PY{n}{color}\PY{o}{=}\PY{l+s+s1}{\PYZsq{}}\PY{l+s+s1}{black}\PY{l+s+s1}{\PYZsq{}}\PY{p}{,} \PY{n}{width}\PY{o}{=}\PY{l+m+mi}{2}\PY{p}{)}
\PY{p}{)}
\PY{p}{)}\PY{p}{)}
\PY{n}{fig}\PY{o}{.}\PY{n}{update\PYZus{}layout}\PY{p}{(}\PY{n}{coloraxis}\PY{o}{=}\PY{n+nb}{dict}\PY{p}{(}\PY{n}{colorscale}\PY{o}{=}\PY{l+s+s1}{\PYZsq{}}\PY{l+s+s1}{Viridis}\PY{l+s+s1}{\PYZsq{}}\PY{p}{)}\PY{p}{)}
\PY{n}{fig\PYZus{}update}\PY{p}{(}\PY{n}{fig}\PY{p}{,} \PY{l+s+s1}{\PYZsq{}}\PY{l+s+s1}{Голосование за рецепты}\PY{l+s+s1}{\PYZsq{}}\PY{p}{,} \PY{l+s+s1}{\PYZsq{}}\PY{l+s+s1}{Название рецепта}\PY{l+s+s1}{\PYZsq{}}\PY{p}{,} \PY{l+s+s1}{\PYZsq{}}\PY{l+s+s1}{Количество голосов}\PY{l+s+s1}{\PYZsq{}}\PY{p}{)}
\PY{n}{fig}\PY{o}{.}\PY{n}{show}\PY{p}{(}\PY{p}{)}
\end{Verbatim}
\end{tcolorbox}

\begin{center}
\adjustimage{max size={0.9\linewidth}{0.9\paperheight}}{output_11_0.png}
\end{center}

\section{4}

Построить круговую диаграмму (go.Pie), использовав данные и стиль
оформления из предыдущего графика. Сделать так, чтобы границы каждой
доли были выделены чёрной линией с толщиной, равной 2 и категории
круговой диаграммы были читаемы (к примеру, объединить часть объектов)

\begin{tcolorbox}[breakable, size=fbox, boxrule=1pt,
	pad at break*=1mm,colback=cellbackground, colframe=cellborder]
\prompt{In}{incolor}{60}{\boxspacing}
\begin{Verbatim}[commandchars=\\\{\}]
\PY{n}{fig} \PY{o}{=} \PY{n}{go}\PY{o}{.}\PY{n}{Figure}\PY{p}{(}\PY{p}{)}
\PY{n}{fig}\PY{o}{.}\PY{n}{add\PYZus{}trace}\PY{p}{(}\PY{n}{go}\PY{o}{.}\PY{n}{Pie}\PY{p}{(}
\PY{n}{labels}\PY{o}{=}\PY{n}{recipes}\PY{p}{,}
\PY{n}{values}\PY{o}{=}\PY{n}{votes}\PY{p}{,}
\PY{n}{marker}\PY{o}{=}\PY{n+nb}{dict}\PY{p}{(}
\PY{n}{line}\PY{o}{=}\PY{n+nb}{dict}\PY{p}{(}\PY{n}{color}\PY{o}{=}\PY{l+s+s1}{\PYZsq{}}\PY{l+s+s1}{black}\PY{l+s+s1}{\PYZsq{}}\PY{p}{,} \PY{n}{width}\PY{o}{=}\PY{l+m+mi}{2}\PY{p}{)}
\PY{p}{)}\PY{p}{,}
\PY{n}{textinfo}\PY{o}{=}\PY{l+s+s1}{\PYZsq{}}\PY{l+s+s1}{label+percent}\PY{l+s+s1}{\PYZsq{}}\PY{p}{,}
\PY{n}{insidetextorientation}\PY{o}{=}\PY{l+s+s1}{\PYZsq{}}\PY{l+s+s1}{radial}\PY{l+s+s1}{\PYZsq{}}
\PY{p}{)}\PY{p}{)}
\PY{n}{fig\PYZus{}update}\PY{p}{(}\PY{n}{fig}\PY{p}{,} \PY{l+s+s1}{\PYZsq{}}\PY{l+s+s1}{Голосование за рецепты}\PY{l+s+s1}{\PYZsq{}}\PY{p}{,} \PY{l+s+s1}{\PYZsq{}}\PY{l+s+s1}{Название рецепта}\PY{l+s+s1}{\PYZsq{}}\PY{p}{,} \PY{l+s+s1}{\PYZsq{}}\PY{l+s+s1}{Количество голосов}\PY{l+s+s1}{\PYZsq{}}\PY{p}{)}
\PY{n}{fig}\PY{o}{.}\PY{n}{show}\PY{p}{(}\PY{p}{)}
\end{Verbatim}
\end{tcolorbox}

\begin{center}
\adjustimage{max size={0.9\linewidth}{0.9\paperheight}}{output_13_0.png}
\end{center}

\section{5}

Построить линейные графики, взять один из параметров и определить
зависимость между другими несколькими (от 2 до 5) показателями с
использованием библиотеки matplotlib. Сделать вывод. 1. Сделать график с
линиями и маркерами, цвет линии `crimson', цвет точек `white', цвет
границ точек `black', толщина границ точек равна 2. 2. Добавить сетку на
график, сделать её цвет `mistyrose' и толщину равную 2. (Можно сделать
это при настройке осей с помощью linewidth=2, color=`mistyrose').

Для построения линейного графика и анализа зависимости одного параметра
от нескольких других с использованием библиотеки matplotlib, можно взять
пример с вашего DataFrame. В этом примере мы будем строить график
зависимости параметра (rating) от нескольких других параметров
(vote\_count, prep\_time и cook\_time).

\begin{tcolorbox}[breakable, size=fbox, boxrule=1pt,
	pad at break*=1mm,colback=cellbackground, colframe=cellborder]
\prompt{In}{incolor}{61}{\boxspacing}
\begin{Verbatim}[commandchars=\\\{\}]
\PY{o}{!}pip install matplotlib
\end{Verbatim}
\end{tcolorbox}

\begin{tcolorbox}[breakable, size=fbox, boxrule=1pt,
	pad at break*=1mm,colback=cellbackground, colframe=cellborder]
\prompt{In}{incolor}{62}{\boxspacing}
\begin{Verbatim}[commandchars=\\\{\}]
\PY{k+kn}{import} \PY{n+nn}{matplotlib}\PY{n+nn}{.}\PY{n+nn}{pyplot} \PY{k}{as} \PY{n+nn}{plt}
\end{Verbatim}
\end{tcolorbox}

\begin{tcolorbox}[breakable, size=fbox, boxrule=1pt,
	pad at break*=1mm,colback=cellbackground, colframe=cellborder]
\prompt{In}{incolor}{65}{\boxspacing}
\begin{Verbatim}[commandchars=\\\{\}]
\PY{n}{count} \PY{o}{=} \PY{l+m+mi}{10}
\PY{n}{x} \PY{o}{=} \PY{n}{df}\PY{p}{[}\PY{l+s+s1}{\PYZsq{}}\PY{l+s+s1}{rating}\PY{l+s+s1}{\PYZsq{}}\PY{p}{]}\PY{p}{[}\PY{p}{:}\PY{n}{count}\PY{p}{]}
\PY{n}{y1} \PY{o}{=} \PY{n}{df}\PY{p}{[}\PY{l+s+s1}{\PYZsq{}}\PY{l+s+s1}{prep\PYZus{}time}\PY{l+s+s1}{\PYZsq{}}\PY{p}{]}\PY{p}{[}\PY{p}{:}\PY{n}{count}\PY{p}{]}
\PY{n}{y2} \PY{o}{=} \PY{n}{df}\PY{p}{[}\PY{l+s+s1}{\PYZsq{}}\PY{l+s+s1}{cook\PYZus{}time}\PY{l+s+s1}{\PYZsq{}}\PY{p}{]}\PY{p}{[}\PY{p}{:}\PY{n}{count}\PY{p}{]}
\end{Verbatim}
\end{tcolorbox}

\begin{tcolorbox}[breakable, size=fbox, boxrule=1pt,
	pad at break*=1mm,colback=cellbackground, colframe=cellborder]
\prompt{In}{incolor}{66}{\boxspacing}
\begin{Verbatim}[commandchars=\\\{\}]
\PY{n}{plt}\PY{o}{.}\PY{n}{figure}\PY{p}{(}\PY{n}{figsize}\PY{o}{=}\PY{p}{(}\PY{l+m+mi}{10}\PY{p}{,} \PY{l+m+mi}{6}\PY{p}{)}\PY{p}{)}
\PY{n}{plt}\PY{o}{.}\PY{n}{plot}\PY{p}{(}\PY{n}{x}\PY{p}{,} \PY{n}{y1}\PY{p}{,} \PY{n}{label}\PY{o}{=}\PY{l+s+s1}{\PYZsq{}}\PY{l+s+s1}{Prep Time}\PY{l+s+s1}{\PYZsq{}}\PY{p}{,} \PY{n}{color}\PY{o}{=}\PY{l+s+s1}{\PYZsq{}}\PY{l+s+s1}{crimson}\PY{l+s+s1}{\PYZsq{}}\PY{p}{,} \PY{n}{marker}\PY{o}{=}\PY{l+s+s1}{\PYZsq{}}\PY{l+s+s1}{s}\PY{l+s+s1}{\PYZsq{}}\PY{p}{,} \PY{n}{markersize}\PY{o}{=}\PY{l+m+mi}{8}\PY{p}{,}
\PY{n}{markerfacecolor}\PY{o}{=}\PY{l+s+s1}{\PYZsq{}}\PY{l+s+s1}{white}\PY{l+s+s1}{\PYZsq{}}\PY{p}{,} \PY{n}{markeredgecolor}\PY{o}{=}\PY{l+s+s1}{\PYZsq{}}\PY{l+s+s1}{black}\PY{l+s+s1}{\PYZsq{}}\PY{p}{,} \PY{n}{markeredgewidth}\PY{o}{=}\PY{l+m+mi}{2}\PY{p}{)}
\PY{n}{plt}\PY{o}{.}\PY{n}{plot}\PY{p}{(}\PY{n}{x}\PY{p}{,} \PY{n}{y2}\PY{p}{,} \PY{n}{label}\PY{o}{=}\PY{l+s+s1}{\PYZsq{}}\PY{l+s+s1}{Cook Time}\PY{l+s+s1}{\PYZsq{}}\PY{p}{,} \PY{n}{color}\PY{o}{=}\PY{l+s+s1}{\PYZsq{}}\PY{l+s+s1}{crimson}\PY{l+s+s1}{\PYZsq{}}\PY{p}{,} \PY{n}{marker}\PY{o}{=}\PY{l+s+s1}{\PYZsq{}}\PY{l+s+s1}{\PYZca{}}\PY{l+s+s1}{\PYZsq{}}\PY{p}{,} \PY{n}{markersize}\PY{o}{=}\PY{l+m+mi}{8}\PY{p}{,}
\PY{n}{markerfacecolor}\PY{o}{=}\PY{l+s+s1}{\PYZsq{}}\PY{l+s+s1}{white}\PY{l+s+s1}{\PYZsq{}}\PY{p}{,} \PY{n}{markeredgecolor}\PY{o}{=}\PY{l+s+s1}{\PYZsq{}}\PY{l+s+s1}{black}\PY{l+s+s1}{\PYZsq{}}\PY{p}{,} \PY{n}{markeredgewidth}\PY{o}{=}\PY{l+m+mi}{2}\PY{p}{)}
\PY{n}{plt}\PY{o}{.}\PY{n}{grid}\PY{p}{(}\PY{k+kc}{True}\PY{p}{,} \PY{n}{which}\PY{o}{=}\PY{l+s+s1}{\PYZsq{}}\PY{l+s+s1}{both}\PY{l+s+s1}{\PYZsq{}}\PY{p}{,} \PY{n}{linewidth}\PY{o}{=}\PY{l+m+mi}{2}\PY{p}{,} \PY{n}{color}\PY{o}{=}\PY{l+s+s1}{\PYZsq{}}\PY{l+s+s1}{mistyrose}\PY{l+s+s1}{\PYZsq{}}\PY{p}{)}
\PY{n}{plt}\PY{o}{.}\PY{n}{title}\PY{p}{(}\PY{l+s+s1}{\PYZsq{}}\PY{l+s+s1}{Зависимость параметров от количества голосов}\PY{l+s+s1}{\PYZsq{}}\PY{p}{,} \PY{n}{fontsize}\PY{o}{=}\PY{l+m+mi}{16}\PY{p}{)}
\PY{n}{plt}\PY{o}{.}\PY{n}{xlabel}\PY{p}{(}\PY{l+s+s1}{\PYZsq{}}\PY{l+s+s1}{Рейтинг}\PY{l+s+s1}{\PYZsq{}}\PY{p}{,} \PY{n}{fontsize}\PY{o}{=}\PY{l+m+mi}{14}\PY{p}{)}
\PY{n}{plt}\PY{o}{.}\PY{n}{ylabel}\PY{p}{(}\PY{l+s+s1}{\PYZsq{}}\PY{l+s+s1}{Время}\PY{l+s+s1}{\PYZsq{}}\PY{p}{,} \PY{n}{fontsize}\PY{o}{=}\PY{l+m+mi}{14}\PY{p}{)}
\PY{c+c1}{\PYZsh{} Добавляем легенду}
\PY{n}{plt}\PY{o}{.}\PY{n}{legend}\PY{p}{(}\PY{n}{fontsize}\PY{o}{=}\PY{l+m+mi}{12}\PY{p}{)}
\PY{c+c1}{\PYZsh{} Отображаем график}
\PY{n}{plt}\PY{o}{.}\PY{n}{show}\PY{p}{(}\PY{p}{)}
\end{Verbatim}
\end{tcolorbox}

\begin{center}
\adjustimage{max size={0.9\linewidth}{0.9\paperheight}}{output_19_0.png}
\end{center}

С меньшим временем готовки и приготовления рейтинг выше, но эти ключевые
параметры не являются единственными, поэтому есть блюдо, которое
выбивыется из этого правила.

\section{6}

Выполнить визуализацию многомерных данных, используя t-SNE. Необходимо
использовать набор данных MNIST или fashion MNIST (можно использовать и
другие готовые наборы данных, где можно наблюдать разделение объектов по
кластерам). Рассмотреть результаты визуализации для разных значений
перплексии.

\begin{tcolorbox}[breakable, size=fbox, boxrule=1pt,
	pad at break*=1mm,colback=cellbackground, colframe=cellborder]
\prompt{In}{incolor}{67}{\boxspacing}
\begin{Verbatim}[commandchars=\\\{\}]
\PY{k+kn}{import} \PY{n+nn}{matplotlib}\PY{n+nn}{.}\PY{n+nn}{pyplot} \PY{k}{as} \PY{n+nn}{plt}
\PY{k+kn}{from} \PY{n+nn}{sklearn}\PY{n+nn}{.}\PY{n+nn}{datasets} \PY{k+kn}{import} \PY{n}{fetch\PYZus{}openml}
\PY{k+kn}{from} \PY{n+nn}{sklearn}\PY{n+nn}{.}\PY{n+nn}{manifold} \PY{k+kn}{import} \PY{n}{TSNE}
\PY{k+kn}{from} \PY{n+nn}{sklearn}\PY{n+nn}{.}\PY{n+nn}{preprocessing} \PY{k+kn}{import} \PY{n}{StandardScaler}
\PY{k+kn}{import} \PY{n+nn}{time}
\end{Verbatim}
\end{tcolorbox}

\begin{tcolorbox}[breakable, size=fbox, boxrule=1pt,
	pad at break*=1mm,colback=cellbackground, colframe=cellborder]
\prompt{In}{incolor}{68}{\boxspacing}
\begin{Verbatim}[commandchars=\\\{\}]
\PY{n}{df} \PY{o}{=} \PY{n}{pd}\PY{o}{.}\PY{n}{read\PYZus{}csv}\PY{p}{(}\PY{l+s+s1}{\PYZsq{}}\PY{l+s+s1}{mnist.csv}\PY{l+s+s1}{\PYZsq{}}\PY{p}{)}
\PY{n}{X} \PY{o}{=} \PY{n}{df}\PY{o}{.}\PY{n}{iloc}\PY{p}{[}\PY{p}{:}\PY{p}{,} \PY{p}{:}\PY{o}{\PYZhy{}}\PY{l+m+mi}{1}\PY{p}{]}\PY{o}{.}\PY{n}{values}\PY{p}{[}\PY{p}{:}\PY{l+m+mi}{500}\PY{p}{]}
\PY{n}{y} \PY{o}{=} \PY{n}{df}\PY{o}{.}\PY{n}{iloc}\PY{p}{[}\PY{p}{:}\PY{p}{,} \PY{o}{\PYZhy{}}\PY{l+m+mi}{1}\PY{p}{]}\PY{o}{.}\PY{n}{values}\PY{p}{[}\PY{p}{:}\PY{l+m+mi}{500}\PY{p}{]}
\PY{n}{scaler} \PY{o}{=} \PY{n}{StandardScaler}\PY{p}{(}\PY{p}{)}
\PY{n}{X\PYZus{}scaled} \PY{o}{=} \PY{n}{scaler}\PY{o}{.}\PY{n}{fit\PYZus{}transform}\PY{p}{(}\PY{n}{X}\PY{p}{)}
\end{Verbatim}
\end{tcolorbox}

\begin{tcolorbox}[breakable, size=fbox, boxrule=1pt,
	pad at break*=1mm,colback=cellbackground, colframe=cellborder]
\prompt{In}{incolor}{69}{\boxspacing}
\begin{Verbatim}[commandchars=\\\{\}]
\PY{k}{def} \PY{n+nf}{plot\PYZus{}tsne}\PY{p}{(}\PY{n}{perplexity}\PY{p}{)}\PY{p}{:}
\PY{n}{tsne} \PY{o}{=} \PY{n}{TSNE}\PY{p}{(}\PY{n}{n\PYZus{}components}\PY{o}{=}\PY{l+m+mi}{2}\PY{p}{,} \PY{n}{perplexity}\PY{o}{=}\PY{n}{perplexity}\PY{p}{,} \PY{n}{random\PYZus{}state}\PY{o}{=}\PY{l+m+mi}{42}\PY{p}{)}
\PY{n}{start\PYZus{}time} \PY{o}{=} \PY{n}{time}\PY{o}{.}\PY{n}{time}\PY{p}{(}\PY{p}{)}
\PY{n}{X\PYZus{}tsne} \PY{o}{=} \PY{n}{tsne}\PY{o}{.}\PY{n}{fit\PYZus{}transform}\PY{p}{(}\PY{n}{X\PYZus{}scaled}\PY{p}{)}
\PY{n}{tsne\PYZus{}time} \PY{o}{=} \PY{n}{time}\PY{o}{.}\PY{n}{time}\PY{p}{(}\PY{p}{)} \PY{o}{\PYZhy{}} \PY{n}{start\PYZus{}time}
\PY{n}{plt}\PY{o}{.}\PY{n}{figure}\PY{p}{(}\PY{n}{figsize}\PY{o}{=}\PY{p}{(}\PY{l+m+mi}{8}\PY{p}{,} \PY{l+m+mi}{6}\PY{p}{)}\PY{p}{)}
\PY{n}{scatter} \PY{o}{=} \PY{n}{plt}\PY{o}{.}\PY{n}{scatter}\PY{p}{(}\PY{n}{X\PYZus{}tsne}\PY{p}{[}\PY{p}{:}\PY{p}{,} \PY{l+m+mi}{0}\PY{p}{]}\PY{p}{,} \PY{n}{X\PYZus{}tsne}\PY{p}{[}\PY{p}{:}\PY{p}{,} \PY{l+m+mi}{1}\PY{p}{]}\PY{p}{,} \PY{n}{c}\PY{o}{=}\PY{n}{y}\PY{p}{,} \PY{n}{cmap}\PY{o}{=}\PY{l+s+s1}{\PYZsq{}}\PY{l+s+s1}{tab10}\PY{l+s+s1}{\PYZsq{}}\PY{p}{,} \PY{n}{s}\PY{o}{=}\PY{l+m+mi}{10}\PY{p}{)}
\PY{n}{plt}\PY{o}{.}\PY{n}{title}\PY{p}{(}\PY{l+s+sa}{f}\PY{l+s+s1}{\PYZsq{}}\PY{l+s+s1}{t\PYZhy{}SNE: Perplexity = }\PY{l+s+si}{\PYZob{}}\PY{n}{perplexity}\PY{l+s+si}{\PYZcb{}}\PY{l+s+s1}{\PYZsq{}}\PY{p}{,} \PY{n}{fontsize}\PY{o}{=}\PY{l+m+mi}{16}\PY{p}{)}
\PY{n}{plt}\PY{o}{.}\PY{n}{colorbar}\PY{p}{(}\PY{n}{scatter}\PY{p}{)}
\PY{n}{plt}\PY{o}{.}\PY{n}{show}\PY{p}{(}\PY{p}{)}
\PY{k}{return} \PY{n}{tsne\PYZus{}time}
\end{Verbatim}
\end{tcolorbox}

\begin{tcolorbox}[breakable, size=fbox, boxrule=1pt,
	pad at break*=1mm,colback=cellbackground, colframe=cellborder]
\prompt{In}{incolor}{70}{\boxspacing}
\begin{Verbatim}[commandchars=\\\{\}]
\PY{k}{for} \PY{n}{perplexity} \PY{o+ow}{in} \PY{p}{[}\PY{l+m+mi}{5}\PY{p}{,} \PY{l+m+mi}{30}\PY{p}{,} \PY{l+m+mi}{50}\PY{p}{]}\PY{p}{:}
\PY{n}{tsne\PYZus{}time} \PY{o}{=} \PY{n}{plot\PYZus{}tsne}\PY{p}{(}\PY{n}{perplexity}\PY{p}{)}
\PY{n+nb}{print}\PY{p}{(}\PY{l+s+sa}{f}\PY{l+s+s1}{\PYZsq{}}\PY{l+s+s1}{Время работы t\PYZhy{}SNE (perplexity=}\PY{l+s+si}{\PYZob{}}\PY{n}{perplexity}\PY{l+s+si}{\PYZcb{}}\PY{l+s+s1}{): }\PY{l+s+si}{\PYZob{}}\PY{n}{tsne\PYZus{}time}\PY{l+s+si}{:}\PY{l+s+s1}{.2f}\PY{l+s+si}{\PYZcb{}}\PY{l+s+s1}{ с.}\PY{l+s+s1}{\PYZsq{}}\PY{p}{)}
\end{Verbatim}
\end{tcolorbox}

\begin{center}
\adjustimage{max size={0.9\linewidth}{0.9\paperheight}}{output_25_0.png}
\end{center}

\begin{Verbatim}[commandchars=\\\{\}]
Время работы t-SNE (perplexity=5): 1.11 с.
\end{Verbatim}

\begin{center}
\adjustimage{max size={0.9\linewidth}{0.9\paperheight}}{output_25_2.png}
\end{center}

\begin{Verbatim}[commandchars=\\\{\}]
Время работы t-SNE (perplexity=30): 1.23 с.
\end{Verbatim}

\begin{center}
\adjustimage{max size={0.9\linewidth}{0.9\paperheight}}{output_25_4.png}
\end{center}

\begin{Verbatim}[commandchars=\\\{\}]
Время работы t-SNE (perplexity=50): 1.32 с.
\end{Verbatim}

\textbf{Перплексия 5}: Алгоритм сосредоточен больше на локальной
структуре данных, и можно увидеть мелкие локальные кластеры. Время
работы: \textbf{0.81} с.

\textbf{Перплексия 30}: Более сбалансированный взгляд на локальные и
глобальные отношения, кластеры становятся четче видны. Время работы:
\textbf{0.59} с.

\textbf{Перплексия 50}: При высоких значениях перплексии внимание
уделяется более глобальной структуре данных, но локальные связи могут
размываться. Время работы: \textbf{0.99} с.

\section{7}

Выполнить визуализацию многомерных данных, используя UMAP с различными
параметрами n\_neighbors и min\_dist. Рассчитать время работы алгоритма
с помощью библиотеки time и сравнить его с временем работы t-SNE.

\begin{tcolorbox}[breakable, size=fbox, boxrule=1pt,
	pad at break*=1mm,colback=cellbackground, colframe=cellborder]
\prompt{In}{incolor}{71}{\boxspacing}
\begin{Verbatim}[commandchars=\\\{\}]
\PY{o}{!}pip install umap\PYZhy{}learn
\end{Verbatim}
\end{tcolorbox}

\begin{tcolorbox}[breakable, size=fbox, boxrule=1pt,
	pad at break*=1mm,colback=cellbackground, colframe=cellborder]
\prompt{In}{incolor}{72}{\boxspacing}
\begin{Verbatim}[commandchars=\\\{\}]
\PY{k+kn}{import} \PY{n+nn}{pandas} \PY{k}{as} \PY{n+nn}{pd}
\PY{k+kn}{import} \PY{n+nn}{numpy} \PY{k}{as} \PY{n+nn}{np}
\PY{k+kn}{import} \PY{n+nn}{matplotlib}\PY{n+nn}{.}\PY{n+nn}{pyplot} \PY{k}{as} \PY{n+nn}{plt}
\PY{k+kn}{import} \PY{n+nn}{umap}
\PY{k+kn}{from} \PY{n+nn}{sklearn}\PY{n+nn}{.}\PY{n+nn}{manifold} \PY{k+kn}{import} \PY{n}{TSNE}
\PY{k+kn}{from} \PY{n+nn}{sklearn}\PY{n+nn}{.}\PY{n+nn}{preprocessing} \PY{k+kn}{import} \PY{n}{StandardScaler}
\PY{k+kn}{import} \PY{n+nn}{time}
\end{Verbatim}
\end{tcolorbox}

\begin{tcolorbox}[breakable, size=fbox, boxrule=1pt,
	pad at break*=1mm,colback=cellbackground, colframe=cellborder]
\prompt{In}{incolor}{73}{\boxspacing}
\begin{Verbatim}[commandchars=\\\{\}]
\PY{n}{df} \PY{o}{=} \PY{n}{pd}\PY{o}{.}\PY{n}{read\PYZus{}csv}\PY{p}{(}\PY{l+s+s1}{\PYZsq{}}\PY{l+s+s1}{mnist.csv}\PY{l+s+s1}{\PYZsq{}}\PY{p}{)}
\PY{n}{count} \PY{o}{=} \PY{l+m+mi}{100}
\PY{n}{X} \PY{o}{=} \PY{n}{df}\PY{o}{.}\PY{n}{iloc}\PY{p}{[}\PY{p}{:}\PY{p}{,} \PY{p}{:}\PY{o}{\PYZhy{}}\PY{l+m+mi}{1}\PY{p}{]}\PY{o}{.}\PY{n}{values}\PY{p}{[}\PY{p}{:}\PY{n}{count}\PY{p}{]}
\PY{n}{y} \PY{o}{=} \PY{n}{df}\PY{o}{.}\PY{n}{iloc}\PY{p}{[}\PY{p}{:}\PY{p}{,} \PY{o}{\PYZhy{}}\PY{l+m+mi}{1}\PY{p}{]}\PY{o}{.}\PY{n}{values}\PY{p}{[}\PY{p}{:}\PY{n}{count}\PY{p}{]}
\PY{n}{scaler} \PY{o}{=} \PY{n}{StandardScaler}\PY{p}{(}\PY{p}{)}
\PY{n}{X\PYZus{}scaled} \PY{o}{=} \PY{n}{scaler}\PY{o}{.}\PY{n}{fit\PYZus{}transform}\PY{p}{(}\PY{n}{X}\PY{p}{)}
\end{Verbatim}
\end{tcolorbox}

\begin{tcolorbox}[breakable, size=fbox, boxrule=1pt,
	pad at break*=1mm,colback=cellbackground, colframe=cellborder]
\prompt{In}{incolor}{74}{\boxspacing}
\begin{Verbatim}[commandchars=\\\{\}]
\PY{k}{def} \PY{n+nf}{plot\PYZus{}umap}\PY{p}{(}\PY{n}{n\PYZus{}neighbors}\PY{p}{,} \PY{n}{min\PYZus{}dist}\PY{p}{)}\PY{p}{:}
\PY{n}{reducer} \PY{o}{=} \PY{n}{umap}\PY{o}{.}\PY{n}{UMAP}\PY{p}{(}\PY{n}{n\PYZus{}neighbors}\PY{o}{=}\PY{n}{n\PYZus{}neighbors}\PY{p}{,} \PY{n}{min\PYZus{}dist}\PY{o}{=}\PY{n}{min\PYZus{}dist}\PY{p}{,} \PY{n}{random\PYZus{}state}\PY{o}{=}\PY{l+m+mi}{42}\PY{p}{)}
\PY{n}{start\PYZus{}time} \PY{o}{=} \PY{n}{time}\PY{o}{.}\PY{n}{time}\PY{p}{(}\PY{p}{)}
\PY{n}{X\PYZus{}umap} \PY{o}{=} \PY{n}{reducer}\PY{o}{.}\PY{n}{fit\PYZus{}transform}\PY{p}{(}\PY{n}{X\PYZus{}scaled}\PY{p}{)}
\PY{n}{umap\PYZus{}time} \PY{o}{=} \PY{n}{time}\PY{o}{.}\PY{n}{time}\PY{p}{(}\PY{p}{)} \PY{o}{\PYZhy{}} \PY{n}{start\PYZus{}time}
\PY{n}{plt}\PY{o}{.}\PY{n}{figure}\PY{p}{(}\PY{n}{figsize}\PY{o}{=}\PY{p}{(}\PY{l+m+mi}{8}\PY{p}{,} \PY{l+m+mi}{6}\PY{p}{)}\PY{p}{)}
\PY{n}{scatter} \PY{o}{=} \PY{n}{plt}\PY{o}{.}\PY{n}{scatter}\PY{p}{(}\PY{n}{X\PYZus{}umap}\PY{p}{[}\PY{p}{:}\PY{p}{,} \PY{l+m+mi}{0}\PY{p}{]}\PY{p}{,} \PY{n}{X\PYZus{}umap}\PY{p}{[}\PY{p}{:}\PY{p}{,} \PY{l+m+mi}{1}\PY{p}{]}\PY{p}{,} \PY{n}{c}\PY{o}{=}\PY{n}{y}\PY{p}{,} \PY{n}{cmap}\PY{o}{=}\PY{l+s+s1}{\PYZsq{}}\PY{l+s+s1}{tab10}\PY{l+s+s1}{\PYZsq{}}\PY{p}{,} \PY{n}{s}\PY{o}{=}\PY{l+m+mi}{10}\PY{p}{)}
\PY{n}{plt}\PY{o}{.}\PY{n}{title}\PY{p}{(}\PY{l+s+sa}{f}\PY{l+s+s1}{\PYZsq{}}\PY{l+s+s1}{UMAP: n\PYZus{}neighbors = }\PY{l+s+si}{\PYZob{}}\PY{n}{n\PYZus{}neighbors}\PY{l+s+si}{\PYZcb{}}\PY{l+s+s1}{, min\PYZus{}dist = }\PY{l+s+si}{\PYZob{}}\PY{n}{min\PYZus{}dist}\PY{l+s+si}{\PYZcb{}}\PY{l+s+s1}{\PYZsq{}}\PY{p}{,} \PY{n}{fontsize}\PY{o}{=}\PY{l+m+mi}{16}\PY{p}{)}
\PY{n}{plt}\PY{o}{.}\PY{n}{colorbar}\PY{p}{(}\PY{n}{scatter}\PY{p}{)}
\PY{n}{plt}\PY{o}{.}\PY{n}{show}\PY{p}{(}\PY{p}{)}
\PY{k}{return} \PY{n}{umap\PYZus{}time}
\end{Verbatim}
\end{tcolorbox}

\begin{tcolorbox}[breakable, size=fbox, boxrule=1pt,
	pad at break*=1mm,colback=cellbackground, colframe=cellborder]
\prompt{In}{incolor}{75}{\boxspacing}
\begin{Verbatim}[commandchars=\\\{\}]
\PY{k}{for} \PY{n}{n\PYZus{}neighbors}\PY{p}{,} \PY{n}{min\PYZus{}dist} \PY{o+ow}{in} \PY{p}{[}\PY{p}{(}\PY{l+m+mi}{15}\PY{p}{,} \PY{l+m+mf}{0.1}\PY{p}{)}\PY{p}{,} \PY{p}{(}\PY{l+m+mi}{50}\PY{p}{,} \PY{l+m+mf}{0.5}\PY{p}{)}\PY{p}{]}\PY{p}{:}
\PY{n}{umap\PYZus{}time} \PY{o}{=} \PY{n}{plot\PYZus{}umap}\PY{p}{(}\PY{n}{n\PYZus{}neighbors}\PY{p}{,} \PY{n}{min\PYZus{}dist}\PY{p}{)}
\PY{n+nb}{print}\PY{p}{(}\PY{l+s+sa}{f}\PY{l+s+s1}{\PYZsq{}}\PY{l+s+s1}{Время работы UMAP (n\PYZus{}neighbors=}\PY{l+s+si}{\PYZob{}}\PY{n}{n\PYZus{}neighbors}\PY{l+s+si}{\PYZcb{}}\PY{l+s+s1}{, min\PYZus{}dist=}\PY{l+s+si}{\PYZob{}}\PY{n}{min\PYZus{}dist}\PY{l+s+si}{\PYZcb{}}\PY{l+s+s1}{): }\PY{l+s+si}{\PYZob{}}\PY{n}{umap\PYZus{}time}\PY{l+s+si}{:}\PY{l+s+s1}{.2f}\PY{l+s+si}{\PYZcb{}}\PY{l+s+s1}{ секунд}\PY{l+s+s1}{\PYZsq{}}\PY{p}{)}
\end{Verbatim}
\end{tcolorbox}

\begin{center}
\adjustimage{max size={0.9\linewidth}{0.9\paperheight}}{output_32_1.png}
\end{center}

\begin{center}
\adjustimage{max size={0.9\linewidth}{0.9\paperheight}}{output_32_4.png}
\end{center}

\begin{Verbatim}[commandchars=\\\{\}]
Время работы UMAP (n\_neighbors=50, min\_dist=0.5): 0.57 секунд
\end{Verbatim}

Время работы UMAP (0.51 с и 0.78 с) меньше, чем у t-SNE (0.59 с, 0.81 с
и 0.99 с) при аналогичных условиях. При этом время работы t-SNE
увеличивается с ростом перплексии. UMAP показывает стабильную скорость и
лучше подходит для быстрого уменьшения размерности. По итогу можно
сделать вывод, что UMAP предпочтительнее для задач, где важна скорость
при сохранении качества визуализации.

\end{document}

