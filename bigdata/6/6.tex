\documentclass[14pt,a4paper,oneside]{extarticle}

	\usepackage{cmap} % for pdfLaTeX
	\usepackage[T1,T2A]{fontenc} % correct encoding for pdfLaTeX
	\usepackage[utf8]{inputenc} % correct encoding source file
	\usepackage[english,russian]{babel} % correct language

	% -------------------
	% TEXT SETTINGS
	% -------------------

	\usepackage{fontspec} % use standart fonts (only for xelatex!!!)
	\setmainfont{Times New Roman}
	\setsansfont{Arial}
	\setmonofont[Scale=0.6]{Courier New}

	\usepackage[none]{hyphenat} % no word breaks

	\usepackage{setspace}
	%\singlespacing % 1.0
	\onehalfspacing % 1.5

	% -------------------
	% PAGE SETTINGS
	% -------------------

	\usepackage[left=3cm, right=1.5cm, vmargin=2cm, headheight=2cm]{geometry}
	\linespread{1.5} % line spacing 1.5
	\usepackage{indentfirst} % indent first paragraph
	\setlength{\parindent}{1.25cm}
	\sloppy

	\frenchspacing % ???
	\pagestyle{plain}
	\usepackage{fancyhdr} % for headers and footers

	\clubpenalty=10000
	\widowpenalty=10000

	% ------------------
	% ATTACHMENTS SETTINGS
	% ------------------

	\usepackage[labelsep=endash]{caption}
	%\setlength{\abovecaptionskip}{3pt}
	%\setlength{\belowcaptionskip}{3pt}

	% for non-end-to-end numbering
	%\usepackage{chngcntr}
	%\counterwithin{figure}{section}
	%\counterwithin{table}{section}

	% for long table
	\usepackage{longtable}

	% for graphics
	\usepackage{graphicx}
	\newcommand{\rref}[1]{(Рисунок~\ref{#1})}
	\newcommand{\rdref}[2]{(Рисунки~\ref{#1}\,-\,\ref{#2})}
	\newcommand{\tref}[1]{(Таблица~\ref{#1})}
	\newcommand{\tdref}[2]{(Таблици~\ref{#1}\,-\,\ref{#2})}
	\newenvironment{image}{
		\begin{figure}[h!tp]
		\centering
	}{
		\end{figure}
	}
	\newcommand\includegrph[2][width=0.8\textwidth]{\includegraphics[#1]{#2}}

	\usepackage{amsmath} % more flexibility equations (use in title!!!)
	\usepackage{pdfpages} % include pdf


	% ------------------
	% SECTION SETTINGS
	% ------------------

	\usepackage{titlesec}

	% section size
	\titleformat{\chapter}[block]
	  {\fontsize{18pt}{22pt}\bfseries}
	  {\chaptertitlename\ \thechapter.}{0.5em}{}
	\titleformat{\section}
		{\fontsize{16pt}{20pt}\bfseries}{\thesection.}{1em}{}
	\titleformat{\subsection}
		{\fontsize{14pt}{18pt}\bfseries}{\thesubsection.}{1em}{}
	\titleformat{\subsubsection}
		{\fontsize{14pt}{18pt}\bfseries}{\thesubsubsection.}{1em}{}
	\titlespacing{\chapter}{1.25cm}{1pt}{1pt}
	\titlespacing{\section}{1.25cm}{1pt}{1pt}
	\titlespacing{\subsection}{1.25cm}{1pt}{1pt}
	\titlespacing{\subsubsection}{1.25cm}{1pt}{1pt}
	\titlespacing{\paragraph}{1.25cm}{0pt}{0pt}

	% new command for sections
	\newcommand\Chapter[1]{
		\refstepcounter{chapter}
		\chapter*{\textbf{ГЛАВА\;\arabic{chapter}.}
			\raggedright #1
		}
		\addcontentsline{toc}{chapter}{ГЛАВА\;\arabic{chapter}.\ #1}
	}
	\newcommand\Section[1]{
		\refstepcounter{section}
		\section*{\textbf{\arabic{chapter}.\arabic{section}.}
			\raggedright #1
		}
		\addcontentsline{toc}{section}{\arabic{chapter}.\arabic{section}.\ #1}
	}
	\newcommand\Subsection[1]{
		\refstepcounter{subsection}
		\subsection*{\textbf{\arabic{chapter}.\arabic{section}.\arabic{subsection}.}
			\raggedright #1
		}
		\addcontentsline{toc}{subsection}
			{\arabic{chapter}.\arabic{section}.\arabic{subsection}.\ #1}
	}
	\newcommand\Subsubsection[1]{
		\refstepcounter{subsubsection}
		\subsubsection*{\textbf{\arabic{chapter}.\arabic{section}.\arabic{subsection}.\arabic{subsubsection}.}
			\raggedright #1
		}
		\addcontentsline{toc}{subsubsection}
			{\arabic{chapter}.\arabic{section}.\arabic{subsection}.\arabic{subsubsection}.\ #1}
	}

	% new command for appendix
	\newcommand\AppendixChapter[1]{
		\refstepcounter{chapter}
		\chapter*{\textbf{\appendixname\;\thechapter.}
			\raggedright #1
		}
		\addcontentsline{toc}{chapter}{\appendixname~\thechapter.\ #1}
	}
	\newcommand\AppendixSection[1]{
		\refstepcounter{section}
		\section*{\textbf{\appendixname\;\thesection.}
			\raggedright #1
		}
	}
	\newcommand\AppendixSubsection[1]{
		\refstepcounter{subsection}
		\section*{\textbf{\appendixname\;\thesubsection.}
			\raggedright #1
		}
	}

	% ------------------
	% NEW COMMAND
	% ------------------

	\providecommand{\No}{\textnumero}

	% ------------------
	% ------------------

    \usepackage[breakable]{tcolorbox}
    

    \usepackage{graphicx}
    \usepackage{caption}

    \usepackage{xcolor} % Allow colors to be defined
    \usepackage{amsmath} % Equations
    \usepackage{amssymb} % Equations
    \usepackage{geometry} % Used to adjust the document margins

    \usepackage{fancyvrb} % verbatim replacement that allows latex

    \makeatletter % fix for old versions of grffile with XeLaTeX
    \@ifpackagelater{grffile}{2019/11/01}
    {
      % Do nothing on new versions
    }
    {
      \def\Gread@@xetex#1{%
        \IfFileExists{"\Gin@base".bb}%
        {\Gread@eps{\Gin@base.bb}}%
        {\Gread@@xetex@aux#1}%
      }
    }
    \makeatother
    \usepackage[Export]{adjustbox} % Used to constrain images to a maximum size
    \adjustboxset{max size={0.9\linewidth}{0.9\paperheight}}

    % The hyperref package gives us a pdf with properly built
    % internal navigation ('pdf bookmarks' for the table of contents,
    % internal cross-reference links, web links for URLs, etc.)
    \usepackage{hyperref}
    % The default LaTeX title has an obnoxious amount of whitespace. By default,
    % titling removes some of it. It also provides customization options.
    \usepackage{titling}
    \usepackage{longtable} % longtable support required by pandoc >1.10
    \usepackage{booktabs}  % table support for pandoc > 1.12.2
    \usepackage{array}     % table support for pandoc >= 2.11.3
    \usepackage{calc}      % table minipage width calculation for pandoc >= 2.11.1
    \usepackage[inline]{enumitem} % IRkernel/repr support (it uses the enumerate* environment)
    \usepackage[normalem]{ulem} % ulem is needed to support strikethroughs (\sout)
                                % normalem makes italics be italics, not underlines
    \usepackage{mathrsfs}
    

    
    % Colors for the hyperref package
    \definecolor{urlcolor}{rgb}{0,.145,.698}
    \definecolor{linkcolor}{rgb}{.71,0.21,0.01}
    \definecolor{citecolor}{rgb}{.12,.54,.11}

    % ANSI colors
    \definecolor{ansi-black}{HTML}{3E424D}
    \definecolor{ansi-black-intense}{HTML}{282C36}
    \definecolor{ansi-red}{HTML}{E75C58}
    \definecolor{ansi-red-intense}{HTML}{B22B31}
    \definecolor{ansi-green}{HTML}{00A250}
    \definecolor{ansi-green-intense}{HTML}{007427}
    \definecolor{ansi-yellow}{HTML}{DDB62B}
    \definecolor{ansi-yellow-intense}{HTML}{B27D12}
    \definecolor{ansi-blue}{HTML}{208FFB}
    \definecolor{ansi-blue-intense}{HTML}{0065CA}
    \definecolor{ansi-magenta}{HTML}{D160C4}
    \definecolor{ansi-magenta-intense}{HTML}{A03196}
    \definecolor{ansi-cyan}{HTML}{60C6C8}
    \definecolor{ansi-cyan-intense}{HTML}{258F8F}
    \definecolor{ansi-white}{HTML}{C5C1B4}
    \definecolor{ansi-white-intense}{HTML}{A1A6B2}
    \definecolor{ansi-default-inverse-fg}{HTML}{FFFFFF}
    \definecolor{ansi-default-inverse-bg}{HTML}{000000}

    % common color for the border for error outputs.
    \definecolor{outerrorbackground}{HTML}{FFDFDF}

    % commands and environments needed by pandoc snippets
    % extracted from the output of `pandoc -s`
    \providecommand{\tightlist}{%
      \setlength{\itemsep}{0pt}\setlength{\parskip}{0pt}}
    \DefineVerbatimEnvironment{Highlighting}{Verbatim}{commandchars=\\\{\}}
    % Add ',fontsize=\small' for more characters per line
    \newenvironment{Shaded}{}{}
    \newcommand{\KeywordTok}[1]{\textcolor[rgb]{0.00,0.44,0.13}{\textbf{{#1}}}}
    \newcommand{\DataTypeTok}[1]{\textcolor[rgb]{0.56,0.13,0.00}{{#1}}}
    \newcommand{\DecValTok}[1]{\textcolor[rgb]{0.25,0.63,0.44}{{#1}}}
    \newcommand{\BaseNTok}[1]{\textcolor[rgb]{0.25,0.63,0.44}{{#1}}}
    \newcommand{\FloatTok}[1]{\textcolor[rgb]{0.25,0.63,0.44}{{#1}}}
    \newcommand{\CharTok}[1]{\textcolor[rgb]{0.25,0.44,0.63}{{#1}}}
    \newcommand{\StringTok}[1]{\textcolor[rgb]{0.25,0.44,0.63}{{#1}}}
    \newcommand{\CommentTok}[1]{\textcolor[rgb]{0.38,0.63,0.69}{\textit{{#1}}}}
    \newcommand{\OtherTok}[1]{\textcolor[rgb]{0.00,0.44,0.13}{{#1}}}
    \newcommand{\AlertTok}[1]{\textcolor[rgb]{1.00,0.00,0.00}{\textbf{{#1}}}}
    \newcommand{\FunctionTok}[1]{\textcolor[rgb]{0.02,0.16,0.49}{{#1}}}
    \newcommand{\RegionMarkerTok}[1]{{#1}}
    \newcommand{\ErrorTok}[1]{\textcolor[rgb]{1.00,0.00,0.00}{\textbf{{#1}}}}
    \newcommand{\NormalTok}[1]{{#1}}

    % Additional commands for more recent versions of Pandoc
    \newcommand{\ConstantTok}[1]{\textcolor[rgb]{0.53,0.00,0.00}{{#1}}}
    \newcommand{\SpecialCharTok}[1]{\textcolor[rgb]{0.25,0.44,0.63}{{#1}}}
    \newcommand{\VerbatimStringTok}[1]{\textcolor[rgb]{0.25,0.44,0.63}{{#1}}}
    \newcommand{\SpecialStringTok}[1]{\textcolor[rgb]{0.73,0.40,0.53}{{#1}}}
    \newcommand{\ImportTok}[1]{{#1}}
    \newcommand{\DocumentationTok}[1]{\textcolor[rgb]{0.73,0.13,0.13}{\textit{{#1}}}}
    \newcommand{\AnnotationTok}[1]{\textcolor[rgb]{0.38,0.63,0.69}{\textbf{\textit{{#1}}}}}
    \newcommand{\CommentVarTok}[1]{\textcolor[rgb]{0.38,0.63,0.69}{\textbf{\textit{{#1}}}}}
    \newcommand{\VariableTok}[1]{\textcolor[rgb]{0.10,0.09,0.49}{{#1}}}
    \newcommand{\ControlFlowTok}[1]{\textcolor[rgb]{0.00,0.44,0.13}{\textbf{{#1}}}}
    \newcommand{\OperatorTok}[1]{\textcolor[rgb]{0.40,0.40,0.40}{{#1}}}
    \newcommand{\BuiltInTok}[1]{{#1}}
    \newcommand{\ExtensionTok}[1]{{#1}}
    \newcommand{\PreprocessorTok}[1]{\textcolor[rgb]{0.74,0.48,0.00}{{#1}}}
    \newcommand{\AttributeTok}[1]{\textcolor[rgb]{0.49,0.56,0.16}{{#1}}}
    \newcommand{\InformationTok}[1]{\textcolor[rgb]{0.38,0.63,0.69}{\textbf{\textit{{#1}}}}}
    \newcommand{\WarningTok}[1]{\textcolor[rgb]{0.38,0.63,0.69}{\textbf{\textit{{#1}}}}}

    
    
    
    
	% Pygments definitions
	\makeatletter
	\def\PY@reset{\let\PY@it=\relax \let\PY@bf=\relax%
		\let\PY@ul=\relax \let\PY@tc=\relax%
		\let\PY@bc=\relax \let\PY@ff=\relax}
	\def\PY@tok#1{\csname PY@tok@#1\endcsname}
	\def\PY@toks#1+{\ifx\relax#1\empty\else%
		\PY@tok{#1}\expandafter\PY@toks\fi}
	\def\PY@do#1{\PY@bc{\PY@tc{\PY@ul{%
		\PY@it{\PY@bf{\PY@ff{#1}}}}}}}
	\def\PY#1#2{\PY@reset\PY@toks#1+\relax+\PY@do{#2}}

	\@namedef{PY@tok@w}{\def\PY@tc##1{\textcolor[rgb]{0.73,0.73,0.73}{##1}}}
	\@namedef{PY@tok@c}{\let\PY@it=\textit\def\PY@tc##1{\textcolor[rgb]{0.24,0.48,0.48}{##1}}}
	\@namedef{PY@tok@cp}{\def\PY@tc##1{\textcolor[rgb]{0.61,0.40,0.00}{##1}}}
	\@namedef{PY@tok@k}{\let\PY@bf=\textbf\def\PY@tc##1{\textcolor[rgb]{0.00,0.50,0.00}{##1}}}
	\@namedef{PY@tok@kp}{\def\PY@tc##1{\textcolor[rgb]{0.00,0.50,0.00}{##1}}}
	\@namedef{PY@tok@kt}{\def\PY@tc##1{\textcolor[rgb]{0.69,0.00,0.25}{##1}}}
	\@namedef{PY@tok@o}{\def\PY@tc##1{\textcolor[rgb]{0.40,0.40,0.40}{##1}}}
	\@namedef{PY@tok@ow}{\let\PY@bf=\textbf\def\PY@tc##1{\textcolor[rgb]{0.67,0.13,1.00}{##1}}}
	\@namedef{PY@tok@nb}{\def\PY@tc##1{\textcolor[rgb]{0.00,0.50,0.00}{##1}}}
	\@namedef{PY@tok@nf}{\def\PY@tc##1{\textcolor[rgb]{0.00,0.00,1.00}{##1}}}
	\@namedef{PY@tok@nc}{\let\PY@bf=\textbf\def\PY@tc##1{\textcolor[rgb]{0.00,0.00,1.00}{##1}}}
	\@namedef{PY@tok@nn}{\let\PY@bf=\textbf\def\PY@tc##1{\textcolor[rgb]{0.00,0.00,1.00}{##1}}}
	\@namedef{PY@tok@ne}{\let\PY@bf=\textbf\def\PY@tc##1{\textcolor[rgb]{0.80,0.25,0.22}{##1}}}
	\@namedef{PY@tok@nv}{\def\PY@tc##1{\textcolor[rgb]{0.10,0.09,0.49}{##1}}}
	\@namedef{PY@tok@no}{\def\PY@tc##1{\textcolor[rgb]{0.53,0.00,0.00}{##1}}}
	\@namedef{PY@tok@nl}{\def\PY@tc##1{\textcolor[rgb]{0.46,0.46,0.00}{##1}}}
	\@namedef{PY@tok@ni}{\let\PY@bf=\textbf\def\PY@tc##1{\textcolor[rgb]{0.44,0.44,0.44}{##1}}}
	\@namedef{PY@tok@na}{\def\PY@tc##1{\textcolor[rgb]{0.41,0.47,0.13}{##1}}}
	\@namedef{PY@tok@nt}{\let\PY@bf=\textbf\def\PY@tc##1{\textcolor[rgb]{0.00,0.50,0.00}{##1}}}
	\@namedef{PY@tok@nd}{\def\PY@tc##1{\textcolor[rgb]{0.67,0.13,1.00}{##1}}}
	\@namedef{PY@tok@s}{\def\PY@tc##1{\textcolor[rgb]{0.73,0.13,0.13}{##1}}}
	\@namedef{PY@tok@sd}{\let\PY@it=\textit\def\PY@tc##1{\textcolor[rgb]{0.73,0.13,0.13}{##1}}}
	\@namedef{PY@tok@si}{\let\PY@bf=\textbf\def\PY@tc##1{\textcolor[rgb]{0.64,0.35,0.47}{##1}}}
	\@namedef{PY@tok@se}{\let\PY@bf=\textbf\def\PY@tc##1{\textcolor[rgb]{0.67,0.36,0.12}{##1}}}
	\@namedef{PY@tok@sr}{\def\PY@tc##1{\textcolor[rgb]{0.64,0.35,0.47}{##1}}}
	\@namedef{PY@tok@ss}{\def\PY@tc##1{\textcolor[rgb]{0.10,0.09,0.49}{##1}}}
	\@namedef{PY@tok@sx}{\def\PY@tc##1{\textcolor[rgb]{0.00,0.50,0.00}{##1}}}
	\@namedef{PY@tok@m}{\def\PY@tc##1{\textcolor[rgb]{0.40,0.40,0.40}{##1}}}
	\@namedef{PY@tok@gh}{\let\PY@bf=\textbf\def\PY@tc##1{\textcolor[rgb]{0.00,0.00,0.50}{##1}}}
	\@namedef{PY@tok@gu}{\let\PY@bf=\textbf\def\PY@tc##1{\textcolor[rgb]{0.50,0.00,0.50}{##1}}}
	\@namedef{PY@tok@gd}{\def\PY@tc##1{\textcolor[rgb]{0.63,0.00,0.00}{##1}}}
	\@namedef{PY@tok@gi}{\def\PY@tc##1{\textcolor[rgb]{0.00,0.52,0.00}{##1}}}
	\@namedef{PY@tok@gr}{\def\PY@tc##1{\textcolor[rgb]{0.89,0.00,0.00}{##1}}}
	\@namedef{PY@tok@ge}{\let\PY@it=\textit}
	\@namedef{PY@tok@gs}{\let\PY@bf=\textbf}
	\@namedef{PY@tok@gp}{\let\PY@bf=\textbf\def\PY@tc##1{\textcolor[rgb]{0.00,0.00,0.50}{##1}}}
	\@namedef{PY@tok@go}{\def\PY@tc##1{\textcolor[rgb]{0.44,0.44,0.44}{##1}}}
	\@namedef{PY@tok@gt}{\def\PY@tc##1{\textcolor[rgb]{0.00,0.27,0.87}{##1}}}
	\@namedef{PY@tok@err}{\def\PY@bc##1{{\setlength{\fboxsep}{\string -\fboxrule}\fcolorbox[rgb]{1.00,0.00,0.00}{1,1,1}{\strut ##1}}}}
	\@namedef{PY@tok@kc}{\let\PY@bf=\textbf\def\PY@tc##1{\textcolor[rgb]{0.00,0.50,0.00}{##1}}}
	\@namedef{PY@tok@kd}{\let\PY@bf=\textbf\def\PY@tc##1{\textcolor[rgb]{0.00,0.50,0.00}{##1}}}
	\@namedef{PY@tok@kn}{\let\PY@bf=\textbf\def\PY@tc##1{\textcolor[rgb]{0.00,0.50,0.00}{##1}}}
	\@namedef{PY@tok@kr}{\let\PY@bf=\textbf\def\PY@tc##1{\textcolor[rgb]{0.00,0.50,0.00}{##1}}}
	\@namedef{PY@tok@bp}{\def\PY@tc##1{\textcolor[rgb]{0.00,0.50,0.00}{##1}}}
	\@namedef{PY@tok@fm}{\def\PY@tc##1{\textcolor[rgb]{0.00,0.00,1.00}{##1}}}
	\@namedef{PY@tok@vc}{\def\PY@tc##1{\textcolor[rgb]{0.10,0.09,0.49}{##1}}}
	\@namedef{PY@tok@vg}{\def\PY@tc##1{\textcolor[rgb]{0.10,0.09,0.49}{##1}}}
	\@namedef{PY@tok@vi}{\def\PY@tc##1{\textcolor[rgb]{0.10,0.09,0.49}{##1}}}
	\@namedef{PY@tok@vm}{\def\PY@tc##1{\textcolor[rgb]{0.10,0.09,0.49}{##1}}}
	\@namedef{PY@tok@sa}{\def\PY@tc##1{\textcolor[rgb]{0.73,0.13,0.13}{##1}}}
	\@namedef{PY@tok@sb}{\def\PY@tc##1{\textcolor[rgb]{0.73,0.13,0.13}{##1}}}
	\@namedef{PY@tok@sc}{\def\PY@tc##1{\textcolor[rgb]{0.73,0.13,0.13}{##1}}}
	\@namedef{PY@tok@dl}{\def\PY@tc##1{\textcolor[rgb]{0.73,0.13,0.13}{##1}}}
	\@namedef{PY@tok@s2}{\def\PY@tc##1{\textcolor[rgb]{0.73,0.13,0.13}{##1}}}
	\@namedef{PY@tok@sh}{\def\PY@tc##1{\textcolor[rgb]{0.73,0.13,0.13}{##1}}}
	\@namedef{PY@tok@s1}{\def\PY@tc##1{\textcolor[rgb]{0.73,0.13,0.13}{##1}}}
	\@namedef{PY@tok@mb}{\def\PY@tc##1{\textcolor[rgb]{0.40,0.40,0.40}{##1}}}
	\@namedef{PY@tok@mf}{\def\PY@tc##1{\textcolor[rgb]{0.40,0.40,0.40}{##1}}}
	\@namedef{PY@tok@mh}{\def\PY@tc##1{\textcolor[rgb]{0.40,0.40,0.40}{##1}}}
	\@namedef{PY@tok@mi}{\def\PY@tc##1{\textcolor[rgb]{0.40,0.40,0.40}{##1}}}
	\@namedef{PY@tok@il}{\def\PY@tc##1{\textcolor[rgb]{0.40,0.40,0.40}{##1}}}
	\@namedef{PY@tok@mo}{\def\PY@tc##1{\textcolor[rgb]{0.40,0.40,0.40}{##1}}}
	\@namedef{PY@tok@ch}{\let\PY@it=\textit\def\PY@tc##1{\textcolor[rgb]{0.24,0.48,0.48}{##1}}}
	\@namedef{PY@tok@cm}{\let\PY@it=\textit\def\PY@tc##1{\textcolor[rgb]{0.24,0.48,0.48}{##1}}}
	\@namedef{PY@tok@cpf}{\let\PY@it=\textit\def\PY@tc##1{\textcolor[rgb]{0.24,0.48,0.48}{##1}}}
	\@namedef{PY@tok@c1}{\let\PY@it=\textit\def\PY@tc##1{\textcolor[rgb]{0.24,0.48,0.48}{##1}}}
	\@namedef{PY@tok@cs}{\let\PY@it=\textit\def\PY@tc##1{\textcolor[rgb]{0.24,0.48,0.48}{##1}}}

	\def\PYZbs{\char`\\}
	\def\PYZus{\char`\_}
	\def\PYZob{\char`\{}
	\def\PYZcb{\char`\}}
	\def\PYZca{\char`\^}
	\def\PYZam{\char`\&}
	\def\PYZlt{\char`\<}
	\def\PYZgt{\char`\>}
	\def\PYZsh{\char`\#}
	\def\PYZpc{\char`\%}
	\def\PYZdl{\char`\$}
	\def\PYZhy{\char`\-}
	\def\PYZsq{\char`\'}
	\def\PYZdq{\char`\"}
	\def\PYZti{\char`\~}
	% for compatibility with earlier versions
	\def\PYZat{@}
	\def\PYZlb{[}
	\def\PYZrb{]}
	\makeatother


    % For linebreaks inside Verbatim environment from package fancyvrb.
    \makeatletter
        \newbox\Wrappedcontinuationbox
        \newbox\Wrappedvisiblespacebox
        \newcommand*\Wrappedvisiblespace {\textcolor{red}{\textvisiblespace}}
        \newcommand*\Wrappedcontinuationsymbol {\textcolor{red}{\llap{\tiny$\m@th\hookrightarrow$}}}
        \newcommand*\Wrappedcontinuationindent {3ex }
        \newcommand*\Wrappedafterbreak {\kern\Wrappedcontinuationindent\copy\Wrappedcontinuationbox}
        % Take advantage of the already applied Pygments mark-up to insert
        % potential linebreaks for TeX processing.
        %        {, <, #, %, $, ' and ": go to next line.
        %        _, }, ^, &, >, - and ~: stay at end of broken line.
        % Use of \textquotesingle for straight quote.
        \newcommand*\Wrappedbreaksatspecials {%
            \def\PYGZus{\discretionary{\char`\_}{\Wrappedafterbreak}{\char`\_}}%
            \def\PYGZob{\discretionary{}{\Wrappedafterbreak\char`\{}{\char`\{}}%
            \def\PYGZcb{\discretionary{\char`\}}{\Wrappedafterbreak}{\char`\}}}%
            \def\PYGZca{\discretionary{\char`\^}{\Wrappedafterbreak}{\char`\^}}%
            \def\PYGZam{\discretionary{\char`\&}{\Wrappedafterbreak}{\char`\&}}%
            \def\PYGZlt{\discretionary{}{\Wrappedafterbreak\char`\<}{\char`\<}}%
            \def\PYGZgt{\discretionary{\char`\>}{\Wrappedafterbreak}{\char`\>}}%
            \def\PYGZsh{\discretionary{}{\Wrappedafterbreak\char`\#}{\char`\#}}%
            \def\PYGZpc{\discretionary{}{\Wrappedafterbreak\char`\%}{\char`\%}}%
            \def\PYGZdl{\discretionary{}{\Wrappedafterbreak\char`\$}{\char`\$}}%
            \def\PYGZhy{\discretionary{\char`\-}{\Wrappedafterbreak}{\char`\-}}%
            \def\PYGZsq{\discretionary{}{\Wrappedafterbreak\textquotesingle}{\textquotesingle}}%
            \def\PYGZdq{\discretionary{}{\Wrappedafterbreak\char`\"}{\char`\"}}%
            \def\PYGZti{\discretionary{\char`\~}{\Wrappedafterbreak}{\char`\~}}%
        }
        % Some characters . , ; ? ! / are not pygmentized.
        % This macro makes them "active" and they will insert potential linebreaks
        \newcommand*\Wrappedbreaksatpunct {%
            \lccode`\~`\.\lowercase{\def~}{\discretionary{\hbox{\char`\.}}{\Wrappedafterbreak}{\hbox{\char`\.}}}%
            \lccode`\~`\,\lowercase{\def~}{\discretionary{\hbox{\char`\,}}{\Wrappedafterbreak}{\hbox{\char`\,}}}%
            \lccode`\~`\;\lowercase{\def~}{\discretionary{\hbox{\char`\;}}{\Wrappedafterbreak}{\hbox{\char`\;}}}%
            \lccode`\~`\:\lowercase{\def~}{\discretionary{\hbox{\char`\:}}{\Wrappedafterbreak}{\hbox{\char`\:}}}%
            \lccode`\~`\?\lowercase{\def~}{\discretionary{\hbox{\char`\?}}{\Wrappedafterbreak}{\hbox{\char`\?}}}%
            \lccode`\~`\!\lowercase{\def~}{\discretionary{\hbox{\char`\!}}{\Wrappedafterbreak}{\hbox{\char`\!}}}%
            \lccode`\~`\/\lowercase{\def~}{\discretionary{\hbox{\char`\/}}{\Wrappedafterbreak}{\hbox{\char`\/}}}%
            \catcode`\.\active
            \catcode`\,\active
            \catcode`\;\active
            \catcode`\:\active
            \catcode`\?\active
            \catcode`\!\active
            \catcode`\/\active
            \lccode`\~`\~
        }
    \makeatother

    \let\OriginalVerbatim=\Verbatim
    \makeatletter
    \renewcommand{\Verbatim}[1][1]{%
        %\parskip\z@skip
        \sbox\Wrappedcontinuationbox {\Wrappedcontinuationsymbol}%
        \sbox\Wrappedvisiblespacebox {\FV@SetupFont\Wrappedvisiblespace}%
        \def\FancyVerbFormatLine ##1{\hsize\linewidth
            \vtop{\raggedright\hyphenpenalty\z@\exhyphenpenalty\z@
                \doublehyphendemerits\z@\finalhyphendemerits\z@
                \strut ##1\strut}%
        }%
        % If the linebreak is at a space, the latter will be displayed as visible
        % space at end of first line, and a continuation symbol starts next line.
        % Stretch/shrink are however usually zero for typewriter font.
        \def\FV@Space {%
            \nobreak\hskip\z@ plus\fontdimen3\font minus\fontdimen4\font
            \discretionary{\copy\Wrappedvisiblespacebox}{\Wrappedafterbreak}
            {\kern\fontdimen2\font}%
        }%

        % Allow breaks at special characters using \PYG... macros.
        \Wrappedbreaksatspecials
        % Breaks at punctuation characters . , ; ? ! and / need catcode=\active
        \OriginalVerbatim[#1,codes*=\Wrappedbreaksatpunct]%
    }
    \makeatother

    % Exact colors from NB
    \definecolor{incolor}{HTML}{303F9F}
    \definecolor{outcolor}{HTML}{D84315}
    \definecolor{cellborder}{HTML}{CFCFCF}
    \definecolor{cellbackground}{HTML}{F7F7F7}

    % prompt
    \makeatletter
    \newcommand{\boxspacing}{\kern\kvtcb@left@rule\kern\kvtcb@boxsep}
    \makeatother
    \newcommand{\prompt}[4]{
        {\ttfamily\llap{{\color{#2}[#3]:\hspace{3pt}#4}}\vspace{-\baselineskip}}
    }
    

    
    % Prevent overflowing lines due to hard-to-break entities
    \sloppy
    % Setup hyperref package
    \hypersetup{
      breaklinks=true,  % so long urls are correctly broken across lines
      colorlinks=true,
      urlcolor=urlcolor,
      linkcolor=linkcolor,
      citecolor=citecolor,
      }
    

\begin{document}
    
\singlespacing % 1.0
\begin{titlepage}
	\thispagestyle{fancy}
	\renewcommand{\headrulewidth}{0pt}
	\cfoot{Москва 2024}

	\centering
	\includegraphics[scale=0.75]{./res/logo2} \break % вставка логотипа
	{\footnotesize МИНИСТЕРСТВО НАУКИ
		И ВЫСШЕГО ОБРАЗОВАНИЯ РОССИЙСКОЙ ФЕДЕРАЦИИ}\\
	Федеральное государственное бюджетное образовательное учреждение 
		высшего образования\\
	\textbf{<<МИРЭА --- Российский технологический университет>>}\\
	\vfill
	\textbf{\large РТУ МИРЭА}\\
	\bigskip \hrule \smallskip \hrule \smallskip
	\vfill
	Институт информационных технологий (ИТ)\\
	Математического обеспечения
		и стандартизации информационных технологий (МОСИТ)\\
	\vfill
	\textbf{ОТЧЕТ ПО ПРАКТИЧЕСКОЙ РАБОТЕ \No\,1}\\
	\textbf{по дисциплине}\\
	\textbf{<<Разработка кроссплатформенных мобильных приложений>>}\\
	\vfill
	\vfill
	\vfill
	\vfill
	\begin{tabular}{p{0.7\textwidth}p{0.2\textwidth}}
		Выполнил студент группы ИКБО-06-21 & \rightline{Бондарь А.Р.} \\
		Принял старший преподаватель & \rightline{Шешуков Л.С.} \\
	\end{tabular}
	\vfill
	\vfill
	\vfill
	\vfill
\end{titlepage}
\onehalfspacing % 1.5
\setcounter{page}{2}
\clearpage


    
    
    \hypertarget{ux434ux430ux43dux43dux44bux435-ux434ux43bux44f-ux43aux43bux430ux441ux442ux435ux440ux438ux437ux430ux446ux438ux438}{%
\section{Данные для
кластеризации}\label{ux434ux430ux43dux43dux44bux435-ux434ux43bux44f-ux43aux43bux430ux441ux442ux435ux440ux438ux437ux430ux446ux438ux438}}

    Найти данные для кластеризации. Данные в группе не должны повторяться.
Если признаки в данных имеют очень сильно разные масштабы, то необходимо
данные предварительно нормализовать.

    \begin{tcolorbox}[breakable, size=fbox, boxrule=1pt, pad at break*=1mm,colback=cellbackground, colframe=cellborder]
\prompt{In}{incolor}{1}{\boxspacing}
\begin{Verbatim}[commandchars=\\\{\}]
\PY{k+kn}{import} \PY{n+nn}{pandas} \PY{k}{as} \PY{n+nn}{pd}
\PY{k+kn}{import} \PY{n+nn}{numpy} \PY{k}{as} \PY{n+nn}{np}
\PY{k+kn}{import} \PY{n+nn}{matplotlib}\PY{n+nn}{.}\PY{n+nn}{pyplot} \PY{k}{as} \PY{n+nn}{plt}
\PY{k+kn}{import} \PY{n+nn}{seaborn} \PY{k}{as} \PY{n+nn}{sns}
\end{Verbatim}
\end{tcolorbox}

    \begin{tcolorbox}[breakable, size=fbox, boxrule=1pt, pad at break*=1mm,colback=cellbackground, colframe=cellborder]
\prompt{In}{incolor}{2}{\boxspacing}
\begin{Verbatim}[commandchars=\\\{\}]
\PY{n}{df} \PY{o}{=} \PY{n}{pd}\PY{o}{.}\PY{n}{read\PYZus{}csv}\PY{p}{(}\PY{l+s+s2}{\PYZdq{}}\PY{l+s+s2}{./6/fish\PYZus{}data.csv}\PY{l+s+s2}{\PYZdq{}}\PY{p}{)}
\PY{n}{df}\PY{o}{.}\PY{n}{head}\PY{p}{(}\PY{p}{)}
\end{Verbatim}
\end{tcolorbox}

            \begin{tcolorbox}[breakable, size=fbox, boxrule=.5pt, pad at break*=1mm, opacityfill=0]
\prompt{Out}{outcolor}{2}{\boxspacing}
\begin{Verbatim}[commandchars=\\\{\}]
              species  length  weight  w\_l\_ratio
0  Anabas testudineus   10.66    3.45       0.32
1  Anabas testudineus    6.91    3.27       0.47
2  Anabas testudineus    8.38    3.46       0.41
3  Anabas testudineus    7.57    3.36       0.44
4  Anabas testudineus   10.83    3.38       0.31
\end{Verbatim}
\end{tcolorbox}
        
    \hypertarget{ux443ux434ux430ux43bux438ux43c-ux434ux443ux431ux43bux438ux43aux430ux442ux44b-ux438ux437-ux434ux430ux442ux430ux444ux440ux435ux439ux43cux430}{%
\subsection{Удалим дубликаты из
датафрейма}\label{ux443ux434ux430ux43bux438ux43c-ux434ux443ux431ux43bux438ux43aux430ux442ux44b-ux438ux437-ux434ux430ux442ux430ux444ux440ux435ux439ux43cux430}}

    \begin{tcolorbox}[breakable, size=fbox, boxrule=1pt, pad at break*=1mm,colback=cellbackground, colframe=cellborder]
\prompt{In}{incolor}{3}{\boxspacing}
\begin{Verbatim}[commandchars=\\\{\}]
\PY{n}{df}\PY{o}{.}\PY{n}{duplicated}\PY{p}{(}\PY{p}{)}\PY{o}{.}\PY{n}{sum}\PY{p}{(}\PY{p}{)}
\end{Verbatim}
\end{tcolorbox}

            \begin{tcolorbox}[breakable, size=fbox, boxrule=.5pt, pad at break*=1mm, opacityfill=0]
\prompt{Out}{outcolor}{3}{\boxspacing}
\begin{Verbatim}[commandchars=\\\{\}]
109
\end{Verbatim}
\end{tcolorbox}
        
    \begin{tcolorbox}[breakable, size=fbox, boxrule=1pt, pad at break*=1mm,colback=cellbackground, colframe=cellborder]
\prompt{In}{incolor}{4}{\boxspacing}
\begin{Verbatim}[commandchars=\\\{\}]
\PY{n}{df}\PY{o}{.}\PY{n}{drop\PYZus{}duplicates}\PY{p}{(}\PY{n}{inplace}\PY{o}{=}\PY{k+kc}{True}\PY{p}{)}
\end{Verbatim}
\end{tcolorbox}

    В коде df.drop\_duplicates(inplace=True) аргумент inplace=True означает,
что операция будет выполняться непосредственно над исходным DataFrame
(df) без создания новой копии.

С inplace=True: DataFrame модифицируется на месте, то есть дублирующиеся
строки удаляются из самого df, а новый объект не возвращается. Без
inplace=True: Создается и возвращается новый DataFrame с удаленными
дубликатами, при этом исходный DataFrame (df) остается неизменным, если
только он явно не переназначен (например, df = df.drop\_duplicates()).

    \hypertarget{ux443ux434ux430ux43bux438ux43c-ux441ux442ux43eux43bux431ux435ux446-ux432ux438ux434}{%
\subsection{Удалим столбец
Вид}\label{ux443ux434ux430ux43bux438ux43c-ux441ux442ux43eux43bux431ux435ux446-ux432ux438ux434}}

    \begin{tcolorbox}[breakable, size=fbox, boxrule=1pt, pad at break*=1mm,colback=cellbackground, colframe=cellborder]
\prompt{In}{incolor}{5}{\boxspacing}
\begin{Verbatim}[commandchars=\\\{\}]
\PY{n}{X} \PY{o}{=} \PY{n}{df}\PY{o}{.}\PY{n}{drop}\PY{p}{(}\PY{l+s+s1}{\PYZsq{}}\PY{l+s+s1}{species}\PY{l+s+s1}{\PYZsq{}}\PY{p}{,} \PY{n}{axis}\PY{o}{=}\PY{l+m+mi}{1}\PY{p}{)}
\PY{n}{y} \PY{o}{=} \PY{n}{df}\PY{o}{.}\PY{n}{species}
\end{Verbatim}
\end{tcolorbox}

    \hypertarget{ux43dux43eux440ux43cux430ux43bux438ux437ux430ux446ux438ux44f}{%
\subsection{Нормализация}\label{ux43dux43eux440ux43cux430ux43bux438ux437ux430ux446ux438ux44f}}

    \begin{tcolorbox}[breakable, size=fbox, boxrule=1pt, pad at break*=1mm,colback=cellbackground, colframe=cellborder]
\prompt{In}{incolor}{6}{\boxspacing}
\begin{Verbatim}[commandchars=\\\{\}]
\PY{k+kn}{from} \PY{n+nn}{sklearn}\PY{n+nn}{.}\PY{n+nn}{preprocessing} \PY{k+kn}{import} \PY{n}{MinMaxScaler}
\end{Verbatim}
\end{tcolorbox}

    \begin{tcolorbox}[breakable, size=fbox, boxrule=1pt, pad at break*=1mm,colback=cellbackground, colframe=cellborder]
\prompt{In}{incolor}{7}{\boxspacing}
\begin{Verbatim}[commandchars=\\\{\}]
\PY{n}{scaler} \PY{o}{=} \PY{n}{MinMaxScaler}\PY{p}{(}\PY{p}{)}
\PY{n}{X\PYZus{}scaled} \PY{o}{=} \PY{n}{scaler}\PY{o}{.}\PY{n}{fit\PYZus{}transform}\PY{p}{(}\PY{n}{X}\PY{p}{)}
\end{Verbatim}
\end{tcolorbox}

    MinMaxScaler - применяет нормализацию: масштабирует данные так, чтобы
значения каждого признака находились в заданном диапазоне (по умолчанию
от 0 до 1).

    \hypertarget{ux434ux430ux43dux43dux44bux435-ux434ux43bux44f-ux432ux438ux437ux443ux430ux43bux438ux437ux430ux446ux438ux438}{%
\subsection{Данные для
визуализации}\label{ux434ux430ux43dux43dux44bux435-ux434ux43bux44f-ux432ux438ux437ux443ux430ux43bux438ux437ux430ux446ux438ux438}}

    Преобразуем масштабированные данные в исходный масштаб с помощью
обратного преобразования

    \begin{tcolorbox}[breakable, size=fbox, boxrule=1pt, pad at break*=1mm,colback=cellbackground, colframe=cellborder]
\prompt{In}{incolor}{8}{\boxspacing}
\begin{Verbatim}[commandchars=\\\{\}]
\PY{n}{X\PYZus{}transformed} \PY{o}{=} \PY{n}{scaler}\PY{o}{.}\PY{n}{inverse\PYZus{}transform}\PY{p}{(}\PY{n}{X\PYZus{}scaled}\PY{p}{)}
\end{Verbatim}
\end{tcolorbox}

\clearpage
    \hypertarget{ux43fux43eux438ux441ux43a-ux43eux43fux442ux438ux43cux430ux43bux44cux433ux43e-ux43aux43eux43bux438ux447ux435ux441ux442ux432ux430-ux43aux43bux430ux441ux442ux435ux440ux43eux432}{%
\section{Поиск оптимальго количества
кластеров}\label{ux43fux43eux438ux441ux43a-ux43eux43fux442ux438ux43cux430ux43bux44cux433ux43e-ux43aux43eux43bux438ux447ux435ux441ux442ux432ux430-ux43aux43bux430ux441ux442ux435ux440ux43eux432}}

    \begin{tcolorbox}[breakable, size=fbox, boxrule=1pt, pad at break*=1mm,colback=cellbackground, colframe=cellborder]
\prompt{In}{incolor}{9}{\boxspacing}
\begin{Verbatim}[commandchars=\\\{\}]
\PY{k+kn}{from} \PY{n+nn}{sklearn}\PY{n+nn}{.}\PY{n+nn}{cluster} \PY{k+kn}{import} \PY{n}{KMeans}
\PY{k+kn}{from} \PY{n+nn}{sklearn}\PY{n+nn}{.}\PY{n+nn}{metrics} \PY{k+kn}{import} \PY{n}{silhouette\PYZus{}score}
\end{Verbatim}
\end{tcolorbox}

    \begin{tcolorbox}[breakable, size=fbox, boxrule=1pt, pad at break*=1mm,colback=cellbackground, colframe=cellborder]
\prompt{In}{incolor}{10}{\boxspacing}
\begin{Verbatim}[commandchars=\\\{\}]
\PY{n}{wcss} \PY{o}{=} \PY{p}{[}\PY{p}{]}
\PY{n}{silhouette\PYZus{}scores} \PY{o}{=} \PY{p}{[}\PY{p}{]}
\PY{n}{cluster\PYZus{}range} \PY{o}{=} \PY{n+nb}{range}\PY{p}{(}\PY{l+m+mi}{2}\PY{p}{,} \PY{l+m+mi}{11}\PY{p}{)}

\PY{k}{for} \PY{n}{k} \PY{o+ow}{in} \PY{n}{cluster\PYZus{}range}\PY{p}{:}
    \PY{n}{kmeans} \PY{o}{=} \PY{n}{KMeans}\PY{p}{(}\PY{n}{n\PYZus{}clusters}\PY{o}{=}\PY{n}{k}\PY{p}{,} \PY{n}{random\PYZus{}state}\PY{o}{=}\PY{l+m+mi}{42}\PY{p}{)}
    \PY{n}{kmeans}\PY{o}{.}\PY{n}{fit}\PY{p}{(}\PY{n}{X\PYZus{}scaled}\PY{p}{)}
    \PY{n}{wcss}\PY{o}{.}\PY{n}{append}\PY{p}{(}\PY{n}{kmeans}\PY{o}{.}\PY{n}{inertia\PYZus{}}\PY{p}{)}
    
    \PY{n}{silhouette\PYZus{}avg} \PY{o}{=} \PY{n}{silhouette\PYZus{}score}\PY{p}{(}\PY{n}{X\PYZus{}scaled}\PY{p}{,} \PY{n}{kmeans}\PY{o}{.}\PY{n}{labels\PYZus{}}\PY{p}{)}
    \PY{n}{silhouette\PYZus{}scores}\PY{o}{.}\PY{n}{append}\PY{p}{(}\PY{n}{silhouette\PYZus{}avg}\PY{p}{)}
\end{Verbatim}
\end{tcolorbox}

    \begin{tcolorbox}[breakable, size=fbox, boxrule=1pt, pad at break*=1mm,colback=cellbackground, colframe=cellborder]
\prompt{In}{incolor}{11}{\boxspacing}
\begin{Verbatim}[commandchars=\\\{\}]
\PY{n}{fig}\PY{p}{,} \PY{n}{ax} \PY{o}{=} \PY{n}{plt}\PY{o}{.}\PY{n}{subplots}\PY{p}{(}\PY{l+m+mi}{1}\PY{p}{,} \PY{l+m+mi}{2}\PY{p}{,} \PY{n}{figsize}\PY{o}{=}\PY{p}{(}\PY{l+m+mi}{15}\PY{p}{,} \PY{l+m+mi}{5}\PY{p}{)}\PY{p}{)}

\PY{n}{ax}\PY{p}{[}\PY{l+m+mi}{0}\PY{p}{]}\PY{o}{.}\PY{n}{grid}\PY{p}{(}\PY{p}{)}
\PY{n}{ax}\PY{p}{[}\PY{l+m+mi}{0}\PY{p}{]}\PY{o}{.}\PY{n}{plot}\PY{p}{(}\PY{n}{cluster\PYZus{}range}\PY{p}{,} \PY{n}{wcss}\PY{p}{,} \PY{n}{marker}\PY{o}{=}\PY{l+s+s1}{\PYZsq{}}\PY{l+s+s1}{o}\PY{l+s+s1}{\PYZsq{}}\PY{p}{,} \PY{n}{linestyle}\PY{o}{=}\PY{l+s+s1}{\PYZsq{}}\PY{l+s+s1}{\PYZhy{}}\PY{l+s+s1}{\PYZsq{}}\PY{p}{,} \PY{n}{color}\PY{o}{=}\PY{l+s+s1}{\PYZsq{}}\PY{l+s+s1}{b}\PY{l+s+s1}{\PYZsq{}}\PY{p}{)}
\PY{n}{ax}\PY{p}{[}\PY{l+m+mi}{0}\PY{p}{]}\PY{o}{.}\PY{n}{set\PYZus{}title}\PY{p}{(}\PY{l+s+s1}{\PYZsq{}}\PY{l+s+s1}{Метод локтя}\PY{l+s+s1}{\PYZsq{}}\PY{p}{)}
\PY{n}{ax}\PY{p}{[}\PY{l+m+mi}{0}\PY{p}{]}\PY{o}{.}\PY{n}{set\PYZus{}xlabel}\PY{p}{(}\PY{l+s+s1}{\PYZsq{}}\PY{l+s+s1}{Количество кластеров}\PY{l+s+s1}{\PYZsq{}}\PY{p}{)}
\PY{n}{ax}\PY{p}{[}\PY{l+m+mi}{0}\PY{p}{]}\PY{o}{.}\PY{n}{set\PYZus{}ylabel}\PY{p}{(}\PY{l+s+s1}{\PYZsq{}}\PY{l+s+s1}{WCSS (Сумма внутрикластерных расстояний)}\PY{l+s+s1}{\PYZsq{}}\PY{p}{)}

\PY{n}{ax}\PY{p}{[}\PY{l+m+mi}{1}\PY{p}{]}\PY{o}{.}\PY{n}{grid}\PY{p}{(}\PY{p}{)}
\PY{n}{ax}\PY{p}{[}\PY{l+m+mi}{1}\PY{p}{]}\PY{o}{.}\PY{n}{plot}\PY{p}{(}\PY{n}{cluster\PYZus{}range}\PY{p}{,} \PY{n}{silhouette\PYZus{}scores}\PY{p}{,} \PY{n}{marker}\PY{o}{=}\PY{l+s+s1}{\PYZsq{}}\PY{l+s+s1}{o}\PY{l+s+s1}{\PYZsq{}}\PY{p}{,} \PY{n}{linestyle}\PY{o}{=}\PY{l+s+s1}{\PYZsq{}}\PY{l+s+s1}{\PYZhy{}}\PY{l+s+s1}{\PYZsq{}}\PY{p}{,} \PY{n}{color}\PY{o}{=}\PY{l+s+s1}{\PYZsq{}}\PY{l+s+s1}{r}\PY{l+s+s1}{\PYZsq{}}\PY{p}{)}
\PY{n}{ax}\PY{p}{[}\PY{l+m+mi}{1}\PY{p}{]}\PY{o}{.}\PY{n}{set\PYZus{}title}\PY{p}{(}\PY{l+s+s1}{\PYZsq{}}\PY{l+s+s1}{Коэффициент силуэта}\PY{l+s+s1}{\PYZsq{}}\PY{p}{)}
\PY{n}{ax}\PY{p}{[}\PY{l+m+mi}{1}\PY{p}{]}\PY{o}{.}\PY{n}{set\PYZus{}xlabel}\PY{p}{(}\PY{l+s+s1}{\PYZsq{}}\PY{l+s+s1}{Количество кластеров}\PY{l+s+s1}{\PYZsq{}}\PY{p}{)}
\PY{n}{ax}\PY{p}{[}\PY{l+m+mi}{1}\PY{p}{]}\PY{o}{.}\PY{n}{set\PYZus{}ylabel}\PY{p}{(}\PY{l+s+s1}{\PYZsq{}}\PY{l+s+s1}{Средний коэффициент силуэта}\PY{l+s+s1}{\PYZsq{}}\PY{p}{)}

\PY{n}{plt}\PY{o}{.}\PY{n}{show}\PY{p}{(}\PY{p}{)}
\end{Verbatim}
\end{tcolorbox}

    \begin{center}
    \adjustimage{max size={0.9\linewidth}{0.9\paperheight}}{output_20_0.png}
    \end{center}
    { \hspace*{\fill} \\}
    
    Коэффициента силуэта достигает максимума при k = 3 или 8.

    \begin{tcolorbox}[breakable, size=fbox, boxrule=1pt, pad at break*=1mm,colback=cellbackground, colframe=cellborder]
\prompt{In}{incolor}{12}{\boxspacing}
\begin{Verbatim}[commandchars=\\\{\}]
\PY{n}{n\PYZus{}clusters} \PY{o}{=} \PY{l+m+mi}{8}
\end{Verbatim}
\end{tcolorbox}

\clearpage
    \hypertarget{k-means}{%
\section{k-means}\label{k-means}}

Провести кластеризацию данных с помощью алгоритма k-means. Использовать
«правило локтя» и коэффициент силуэта для поиска оптимального количества
кластеров.

    \begin{tcolorbox}[breakable, size=fbox, boxrule=1pt, pad at break*=1mm,colback=cellbackground, colframe=cellborder]
\prompt{In}{incolor}{13}{\boxspacing}
\begin{Verbatim}[commandchars=\\\{\}]
\PY{k+kn}{from} \PY{n+nn}{sklearn}\PY{n+nn}{.}\PY{n+nn}{cluster} \PY{k+kn}{import} \PY{n}{KMeans}
\end{Verbatim}
\end{tcolorbox}

    \begin{tcolorbox}[breakable, size=fbox, boxrule=1pt, pad at break*=1mm,colback=cellbackground, colframe=cellborder]
\prompt{In}{incolor}{14}{\boxspacing}
\begin{Verbatim}[commandchars=\\\{\}]
\PY{n}{kmeans} \PY{o}{=} \PY{n}{KMeans}\PY{p}{(}\PY{n}{n\PYZus{}clusters}\PY{o}{=}\PY{n}{n\PYZus{}clusters}\PY{p}{,} \PY{n}{max\PYZus{}iter}\PY{o}{=}\PY{l+m+mi}{100}\PY{p}{)}

\PY{n}{y\PYZus{}kmeans} \PY{o}{=} \PY{n}{kmeans}\PY{o}{.}\PY{n}{fit\PYZus{}predict}\PY{p}{(}\PY{n}{X\PYZus{}scaled}\PY{p}{)}
\end{Verbatim}
\end{tcolorbox}

    \begin{tcolorbox}[breakable, size=fbox, boxrule=1pt, pad at break*=1mm,colback=cellbackground, colframe=cellborder]
\prompt{In}{incolor}{15}{\boxspacing}
\begin{Verbatim}[commandchars=\\\{\}]
\PY{n}{sns}\PY{o}{.}\PY{n}{scatterplot}\PY{p}{(}\PY{n}{x}\PY{o}{=}\PY{n}{X\PYZus{}transformed}\PY{p}{[}\PY{p}{:}\PY{p}{,} \PY{l+m+mi}{0}\PY{p}{]}\PY{p}{,} \PY{n}{y}\PY{o}{=}\PY{n}{X\PYZus{}transformed}\PY{p}{[}\PY{p}{:}\PY{p}{,} \PY{l+m+mi}{1}\PY{p}{]}\PY{p}{,} \PY{n}{hue}\PY{o}{=}\PY{n}{y\PYZus{}kmeans}\PY{p}{,} \PY{n}{palette}\PY{o}{=}\PY{l+s+s1}{\PYZsq{}}\PY{l+s+s1}{Paired}\PY{l+s+s1}{\PYZsq{}}\PY{p}{)}
\PY{n}{plt}\PY{o}{.}\PY{n}{title}\PY{p}{(}\PY{l+s+s2}{\PYZdq{}}\PY{l+s+s2}{K\PYZhy{}means кластеризация}\PY{l+s+s2}{\PYZdq{}}\PY{p}{)}
\PY{n}{plt}\PY{o}{.}\PY{n}{xlabel}\PY{p}{(}\PY{l+s+s1}{\PYZsq{}}\PY{l+s+s1}{Длина}\PY{l+s+s1}{\PYZsq{}}\PY{p}{)}
\PY{n}{plt}\PY{o}{.}\PY{n}{ylabel}\PY{p}{(}\PY{l+s+s1}{\PYZsq{}}\PY{l+s+s1}{Вес}\PY{l+s+s1}{\PYZsq{}}\PY{p}{)}
\PY{n}{plt}\PY{o}{.}\PY{n}{show}\PY{p}{(}\PY{p}{)}
\end{Verbatim}
\end{tcolorbox}

    \begin{center}
    \adjustimage{max size={0.9\linewidth}{0.9\paperheight}}{output_26_0.png}
    \end{center}
    

\clearpage
    \hypertarget{ux438ux435ux440ux430ux440ux445ux438ux447ux435ux441ux43aux430ux44f-ux43aux43bux430ux441ux442ux435ux440ux438ux437ux430ux446ux438ux44f}{%
\section{Иерархическая
кластеризация}\label{ux438ux435ux440ux430ux440ux445ux438ux447ux435ux441ux43aux430ux44f-ux43aux43bux430ux441ux442ux435ux440ux438ux437ux430ux446ux438ux44f}}

Провести кластеризацию данных с помощью алгоритма иерархической
кластеризации.

    \begin{tcolorbox}[breakable, size=fbox, boxrule=1pt, pad at break*=1mm,colback=cellbackground, colframe=cellborder]
\prompt{In}{incolor}{16}{\boxspacing}
\begin{Verbatim}[commandchars=\\\{\}]
\PY{k+kn}{from} \PY{n+nn}{sklearn}\PY{n+nn}{.}\PY{n+nn}{cluster} \PY{k+kn}{import} \PY{n}{AgglomerativeClustering}
\end{Verbatim}
\end{tcolorbox}

    \begin{tcolorbox}[breakable, size=fbox, boxrule=1pt, pad at break*=1mm,colback=cellbackground, colframe=cellborder]
\prompt{In}{incolor}{17}{\boxspacing}
\begin{Verbatim}[commandchars=\\\{\}]
\PY{n}{agg\PYZus{}clustering} \PY{o}{=} \PY{n}{AgglomerativeClustering}\PY{p}{(}\PY{n}{n\PYZus{}clusters}\PY{o}{=}\PY{n}{n\PYZus{}clusters}\PY{p}{)}

\PY{n}{y\PYZus{}agg} \PY{o}{=} \PY{n}{agg\PYZus{}clustering}\PY{o}{.}\PY{n}{fit\PYZus{}predict}\PY{p}{(}\PY{n}{X\PYZus{}scaled}\PY{p}{)}
\end{Verbatim}
\end{tcolorbox}

    \begin{tcolorbox}[breakable, size=fbox, boxrule=1pt, pad at break*=1mm,colback=cellbackground, colframe=cellborder]
\prompt{In}{incolor}{18}{\boxspacing}
\begin{Verbatim}[commandchars=\\\{\}]
\PY{n}{sns}\PY{o}{.}\PY{n}{scatterplot}\PY{p}{(}\PY{n}{x}\PY{o}{=}\PY{n}{X\PYZus{}transformed}\PY{p}{[}\PY{p}{:}\PY{p}{,} \PY{l+m+mi}{0}\PY{p}{]}\PY{p}{,} \PY{n}{y}\PY{o}{=}\PY{n}{X\PYZus{}transformed}\PY{p}{[}\PY{p}{:}\PY{p}{,} \PY{l+m+mi}{1}\PY{p}{]}\PY{p}{,} \PY{n}{hue}\PY{o}{=}\PY{n}{y\PYZus{}agg}\PY{p}{,} \PY{n}{palette}\PY{o}{=}\PY{l+s+s1}{\PYZsq{}}\PY{l+s+s1}{Paired}\PY{l+s+s1}{\PYZsq{}}\PY{p}{)}
\PY{n}{plt}\PY{o}{.}\PY{n}{title}\PY{p}{(}\PY{l+s+s2}{\PYZdq{}}\PY{l+s+s2}{Иерархическая кластеризация}\PY{l+s+s2}{\PYZdq{}}\PY{p}{)}
\PY{n}{plt}\PY{o}{.}\PY{n}{xlabel}\PY{p}{(}\PY{l+s+s1}{\PYZsq{}}\PY{l+s+s1}{Длина}\PY{l+s+s1}{\PYZsq{}}\PY{p}{)}
\PY{n}{plt}\PY{o}{.}\PY{n}{ylabel}\PY{p}{(}\PY{l+s+s1}{\PYZsq{}}\PY{l+s+s1}{Вес}\PY{l+s+s1}{\PYZsq{}}\PY{p}{)}
\PY{n}{plt}\PY{o}{.}\PY{n}{show}\PY{p}{(}\PY{p}{)}
\end{Verbatim}
\end{tcolorbox}

    \begin{center}
    \adjustimage{max size={0.9\linewidth}{0.9\paperheight}}{output_30_0.png}
    \end{center}
    { \hspace*{\fill} \\}
    
	\clearpage
    \hypertarget{dbscan}{%
\section{DBSCAN}\label{dbscan}}

Провести кластеризацию данных с помощью алгоритма DBSCAN.

    \begin{tcolorbox}[breakable, size=fbox, boxrule=1pt, pad at break*=1mm,colback=cellbackground, colframe=cellborder]
\prompt{In}{incolor}{19}{\boxspacing}
\begin{Verbatim}[commandchars=\\\{\}]
\PY{k+kn}{from} \PY{n+nn}{sklearn}\PY{n+nn}{.}\PY{n+nn}{cluster} \PY{k+kn}{import} \PY{n}{DBSCAN}
\end{Verbatim}
\end{tcolorbox}

    \begin{tcolorbox}[breakable, size=fbox, boxrule=1pt, pad at break*=1mm,colback=cellbackground, colframe=cellborder]
\prompt{In}{incolor}{20}{\boxspacing}
\begin{Verbatim}[commandchars=\\\{\}]
\PY{n}{dbscan} \PY{o}{=} \PY{n}{DBSCAN}\PY{p}{(}\PY{n}{eps}\PY{o}{=}\PY{l+m+mf}{0.03}\PY{p}{,} \PY{n}{min\PYZus{}samples}\PY{o}{=}\PY{l+m+mi}{5}\PY{p}{)}
\PY{n}{y\PYZus{}dbscan} \PY{o}{=} \PY{n}{dbscan}\PY{o}{.}\PY{n}{fit\PYZus{}predict}\PY{p}{(}\PY{n}{X\PYZus{}scaled}\PY{p}{)}
\end{Verbatim}
\end{tcolorbox}

    \begin{tcolorbox}[breakable, size=fbox, boxrule=1pt, pad at break*=1mm,colback=cellbackground, colframe=cellborder]
\prompt{In}{incolor}{21}{\boxspacing}
\begin{Verbatim}[commandchars=\\\{\}]
\PY{n}{sns}\PY{o}{.}\PY{n}{scatterplot}\PY{p}{(}\PY{n}{x}\PY{o}{=}\PY{n}{X\PYZus{}transformed}\PY{p}{[}\PY{p}{:}\PY{p}{,} \PY{l+m+mi}{0}\PY{p}{]}\PY{p}{,} \PY{n}{y}\PY{o}{=}\PY{n}{X\PYZus{}transformed}\PY{p}{[}\PY{p}{:}\PY{p}{,} \PY{l+m+mi}{1}\PY{p}{]}\PY{p}{,} \PY{n}{hue}\PY{o}{=}\PY{n}{y\PYZus{}dbscan}\PY{p}{,} \PY{n}{palette}\PY{o}{=}\PY{l+s+s1}{\PYZsq{}}\PY{l+s+s1}{Paired}\PY{l+s+s1}{\PYZsq{}}\PY{p}{)}
\PY{n}{plt}\PY{o}{.}\PY{n}{title}\PY{p}{(}\PY{l+s+s2}{\PYZdq{}}\PY{l+s+s2}{DBSCAN кластеризация}\PY{l+s+s2}{\PYZdq{}}\PY{p}{)}
\PY{n}{plt}\PY{o}{.}\PY{n}{xlabel}\PY{p}{(}\PY{l+s+s1}{\PYZsq{}}\PY{l+s+s1}{Длина}\PY{l+s+s1}{\PYZsq{}}\PY{p}{)}
\PY{n}{plt}\PY{o}{.}\PY{n}{ylabel}\PY{p}{(}\PY{l+s+s1}{\PYZsq{}}\PY{l+s+s1}{Вес}\PY{l+s+s1}{\PYZsq{}}\PY{p}{)}
\PY{n}{plt}\PY{o}{.}\PY{n}{show}\PY{p}{(}\PY{p}{)}
\end{Verbatim}
\end{tcolorbox}

    \begin{center}
    \adjustimage{max size={0.9\linewidth}{0.9\paperheight}}{output_34_0.png}
    \end{center}
    { \hspace*{\fill} \\}
    
	\clearpage
    \hypertarget{ux438ux441ux445ux43eux434ux43dux44bux435-ux43aux43bux430ux441ux442ux435ux440ux44b}{%
\section{Исходные
кластеры}\label{ux438ux441ux445ux43eux434ux43dux44bux435-ux43aux43bux430ux441ux442ux435ux440ux44b}}

    \begin{tcolorbox}[breakable, size=fbox, boxrule=1pt, pad at break*=1mm,colback=cellbackground, colframe=cellborder]
\prompt{In}{incolor}{22}{\boxspacing}
\begin{Verbatim}[commandchars=\\\{\}]
\PY{k+kn}{import} \PY{n+nn}{matplotlib}\PY{n+nn}{.}\PY{n+nn}{pyplot} \PY{k}{as} \PY{n+nn}{plt}
\PY{k+kn}{import} \PY{n+nn}{seaborn} \PY{k}{as} \PY{n+nn}{sns}
\PY{k+kn}{from} \PY{n+nn}{sklearn}\PY{n+nn}{.}\PY{n+nn}{preprocessing} \PY{k+kn}{import} \PY{n}{LabelEncoder}
\PY{k+kn}{from} \PY{n+nn}{matplotlib}\PY{n+nn}{.}\PY{n+nn}{patches} \PY{k+kn}{import} \PY{n}{Patch}
\end{Verbatim}
\end{tcolorbox}

    \begin{tcolorbox}[breakable, size=fbox, boxrule=1pt, pad at break*=1mm,colback=cellbackground, colframe=cellborder]
\prompt{In}{incolor}{23}{\boxspacing}
\begin{Verbatim}[commandchars=\\\{\}]
\PY{n}{label\PYZus{}encoder} \PY{o}{=} \PY{n}{LabelEncoder}\PY{p}{(}\PY{p}{)}
\PY{n}{y\PYZus{}encoded} \PY{o}{=} \PY{n}{label\PYZus{}encoder}\PY{o}{.}\PY{n}{fit\PYZus{}transform}\PY{p}{(}\PY{n}{y}\PY{p}{)}

\PY{n}{palette} \PY{o}{=} \PY{n}{sns}\PY{o}{.}\PY{n}{color\PYZus{}palette}\PY{p}{(}\PY{l+s+s2}{\PYZdq{}}\PY{l+s+s2}{Paired}\PY{l+s+s2}{\PYZdq{}}\PY{p}{,} \PY{n+nb}{len}\PY{p}{(}\PY{n}{label\PYZus{}encoder}\PY{o}{.}\PY{n}{classes\PYZus{}}\PY{p}{)}\PY{p}{)}
\PY{n}{colors} \PY{o}{=} \PY{p}{[}\PY{n}{palette}\PY{p}{[}\PY{n}{i}\PY{p}{]} \PY{k}{for} \PY{n}{i} \PY{o+ow}{in} \PY{n}{y\PYZus{}encoded}\PY{p}{]}

\PY{n}{plt}\PY{o}{.}\PY{n}{figure}\PY{p}{(}\PY{n}{figsize}\PY{o}{=}\PY{p}{(}\PY{l+m+mi}{15}\PY{p}{,} \PY{l+m+mi}{8}\PY{p}{)}\PY{p}{)}
\PY{n}{scatter} \PY{o}{=} \PY{n}{plt}\PY{o}{.}\PY{n}{scatter}\PY{p}{(}\PY{n}{X}\PY{o}{.}\PY{n}{iloc}\PY{p}{[}\PY{p}{:}\PY{p}{,} \PY{l+m+mi}{0}\PY{p}{]}\PY{p}{,} \PY{n}{X}\PY{o}{.}\PY{n}{iloc}\PY{p}{[}\PY{p}{:}\PY{p}{,} \PY{l+m+mi}{1}\PY{p}{]}\PY{p}{,} \PY{n}{c}\PY{o}{=}\PY{n}{colors}\PY{p}{,} \PY{n}{s}\PY{o}{=}\PY{l+m+mi}{25}\PY{p}{)}

\PY{n}{legend\PYZus{}labels} \PY{o}{=} \PY{n}{label\PYZus{}encoder}\PY{o}{.}\PY{n}{classes\PYZus{}}
\PY{n}{legend\PYZus{}elements} \PY{o}{=} \PY{p}{[}\PY{n}{Patch}\PY{p}{(}\PY{n}{facecolor}\PY{o}{=}\PY{n}{palette}\PY{p}{[}\PY{n}{i}\PY{p}{]}\PY{p}{,} \PY{n}{edgecolor}\PY{o}{=}\PY{l+s+s1}{\PYZsq{}}\PY{l+s+s1}{k}\PY{l+s+s1}{\PYZsq{}}\PY{p}{,} \PY{n}{label}\PY{o}{=}\PY{n}{legend\PYZus{}labels}\PY{p}{[}\PY{n}{i}\PY{p}{]}\PY{p}{)} \PY{k}{for} \PY{n}{i} \PY{o+ow}{in} \PY{n+nb}{range}\PY{p}{(}\PY{n+nb}{len}\PY{p}{(}\PY{n}{legend\PYZus{}labels}\PY{p}{)}\PY{p}{)}\PY{p}{]}
\PY{n}{plt}\PY{o}{.}\PY{n}{legend}\PY{p}{(}\PY{n}{handles}\PY{o}{=}\PY{n}{legend\PYZus{}elements}\PY{p}{,} \PY{n}{title}\PY{o}{=}\PY{l+s+s2}{\PYZdq{}}\PY{l+s+s2}{Species}\PY{l+s+s2}{\PYZdq{}}\PY{p}{)}

\PY{n}{plt}\PY{o}{.}\PY{n}{title}\PY{p}{(}\PY{l+s+s1}{\PYZsq{}}\PY{l+s+s1}{Кластеры по столбцу Вид}\PY{l+s+s1}{\PYZsq{}}\PY{p}{,} \PY{n}{fontsize}\PY{o}{=}\PY{l+m+mi}{20}\PY{p}{)}
\PY{n}{plt}\PY{o}{.}\PY{n}{xlabel}\PY{p}{(}\PY{l+s+s1}{\PYZsq{}}\PY{l+s+s1}{Длина}\PY{l+s+s1}{\PYZsq{}}\PY{p}{)}
\PY{n}{plt}\PY{o}{.}\PY{n}{ylabel}\PY{p}{(}\PY{l+s+s1}{\PYZsq{}}\PY{l+s+s1}{Вес}\PY{l+s+s1}{\PYZsq{}}\PY{p}{)}

\PY{n}{plt}\PY{o}{.}\PY{n}{show}\PY{p}{(}\PY{p}{)}
\end{Verbatim}
\end{tcolorbox}

    \begin{center}
    \adjustimage{max size={0.9\linewidth}{0.9\paperheight}}{output_37_0.png}
    \end{center}
    { \hspace*{\fill} \\}
    
\clearpage

    \hypertarget{ux432ux44bux432ux43eux434}{%
\section*{Вывод}\label{ux432ux44bux432ux43eux434}}

  Алгоритмы иерархической кластеризации и DBSCAN показал хорошые
  результаты, создав хорошо разделенные кластеры.

  Истинные метки показывают, что \textbf{Setipinna taty} и
  \textbf{Otolithoides biauritus} тесно расположены в пространстве
  признаков, что затрудняет их разделение.

  Ни одна методика не смогла успешно разделить \textbf{Setipinna taty} и
  \textbf{Otolithoides biauritus}.

  Метод k-means неправильно классифицировали \textbf{Pethia conchonius},
  разделив его на два отдельных кластера.


\end{document}
