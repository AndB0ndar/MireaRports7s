\section*{ЦЕЛЬ ПРАКТИЧЕСКОЙ РАБОТЫ}
\addcontentsline{toc}{section}{ЦЕЛЬ ПРАКТИЧЕСКОЙ РАБОТЫ}

План практической работы:

\begin{itemize}
	\item Страничная навигация в приложениях;
	\item Основные методы страничной навигации;
	\item Навигация на маршрутах;
	\item Выполнение практической работы \No6.
\end{itemize}

\clearpage

\section*{ВЫПОЛНЕНИЕ ПРАКТИЧЕСКОЙ РАБОТЫ}
\addcontentsline{toc}{section}{ВЫПОЛНЕНИЕ ПРАКТИЧЕСКОЙ РАБОТЫ}

\section{Страничная навигация в приложении}

В Flutter есть два основных способа навигации в приложении:
страничная и маршрутизированная навигация.
Страничная навигация работает на \texttt{Navigator} и \texttt{Route}.
\texttt{Navigator} --- это виджет, который управляет набором дочерних виджетов
с помощью структуры \texttt{Stack}.
Экраны в Flutter называются \texttt{Route}.
Многие приложения имеют Navigator в верхней части иерархии виджетов,
чтобы отображать их логическую историю с помощью наложения
с последними посещенными страницами, визуально поверх старых страниц.
Использование этого шаблона позволяет навигатору визуально переходить
с одной страницы на другую, перемещая виджеты в наложении.
Аналогично, навигатор можно использовать для отображения диалога,
разместив виджет диалога над текущей страницей.

\section{Основные методы страничной навигации}

Для исполнения основных действий навигации \texttt{Navigator} 
предоставляет ряд методов для работы.

\subsection{Метод push}

В навигации часто требуется переходить между страницами
с возможностью к ним в последующем вернуться.
Данный способ навигации называется --- вертикальной навигацией.
Смысл вертикальной навигации заключается в сохранении состояний экранов,
с которой был выполнен навигационный переход.
По сути, навигационная структура реализуется
на основе структуры \texttt{Stack} 
и вертикальная навигация --- это добавление новой страницы в него.
Для реализации добавления новой страницы в навигационный стек
или реализации вертикального навигационного перехода класс
\texttt{Navigator} предоставляет метод \texttt{push}.
Метод принимает в себя навигационную страницу,
на которую требуется выполнить вертикальный навигационный переход.

\subsection{Метод pop}

В вертикальной навигации помимо самого перехода так же есть
и обратное действие --- вертикальный навигационный возврат.
Когда мы добавляем страницу в навигационный \texttt{Stack} 
у пользователя должна быть возможность вернуться на страницу назад
для просмотра предыдущей страницы.
Для реализации возврата на предыдущую страницу в навигационном стеке
или реализации вертикального навигационного возврата класс \texttt{Navigator} 
предоставляет метод \texttt{pop}.
Метод может принимать в себя какие-либо данные,
которые могут считаться в качестве результата навигационного перехода.
Подробнее о возврате результата навигационного перехода будет рассказано
в последующих практических работах.

\subsection{Метод pushReplacement}

Иногда в приложениях реализуется логика,
когда пользователь должен перейти на страницу
и при этом ему не нужна предыдущая страница в истории навигации.
Такая логика часто реализуется в нижних и боковых меню,
когда переход на самих реализован с сохранением навигационной истории,
то есть вертикальной навигацией,
а вот переходы между самими позициями в меню реализуются
без сохранения предыдущего выбранного элемента меню.
Данный способ навигации называется --- горизонтальная навигация.
Ее отличительным признаком является не наложение новой страницы
на навигационный \texttt{Stack} системы,
а снятием верхней страницы из него и заменой ее на новую,
которую предоставляет пользователь горизонтальным переходом.
Для реализации горизонтального навигационного перехода
класс \texttt{Navigator} предоставляет метод \texttt{pushReplacement}.
Метод принимает в себя навигационную страницу,
на которую требуется выполнить горизонтальный навигационный переход.

\subsection{Приложение на Navigator}

\subsubsection{Экран навигации}

Экран \texttt{HomePage} \rref{fig:nav:home} необходим
как главный навигационный центр приложения.
Он служит стартовой точкой, с которой пользователь может перейти
к различным ключевым функциям приложения, таким как управление задачами,
фильтрация задач, просмотр статистики или изменение пользовательских настроек.
В нем осуществляется переход на страницы с помощью метода \texttt{push}.

\begin{image}
	\includegrph[scale=0.34]{Screenshot from 2024-10-05 21-25-10}
	\caption{Класс HomePage}
	\label{fig:nav:home}
\end{image}

Экран проиллюстрирован на рисунке~\ref{fig:home:show}.

\begin{image}
	\includegrph[scale=0.5]{Screenshot from 2024-10-05 21-39-07}
	\caption{Класс HomePage}
	\label{fig:home:show}
\end{image}

\subsubsection{Экран статистики}

Для того, чтобы продемонстрировать использование метода \texttt{pop}
рассмотрим одни из реализованных экранов.
Экран \texttt{TaskStatsScreen} предназначен
для отображения статистики выполнения задач.
Он предоставляет пользователю визуальную информацию
о завершенных и незавершенных задачах,
а также позволяет легко вернуться на главный экран.

Код экрана \texttt{TaskStatsScreen} проиллюстрирован
на рисунке~\ref{fig:nav:stats}.

\begin{image}
	\includegrph[scale=0.45]{Screenshot from 2024-10-05 21-24-55}
	\caption{Класс TaskStatsScreen}
	\label{fig:nav:stats}
\end{image}

Отображение экрана показано на рисунке~\ref{fig:nav:stats:show}.

\begin{image}
	\includegrph[scale=0.6]{Screenshot from 2024-10-05 22-48-06}
	\caption{Класс HomePage}
	\label{fig:nav:stats:show}
\end{image}

\subsubsection{Экраны списка задач и информации о задаче}

Чтобы продемонстрировать использование метода \texttt{pushReplacement}
и реализации горизонтальной навигации,
рассмотрим одни из реализованных экранов.\par
Экран \texttt{TaskListScreen} предназначен
для отображения списка задач и позволяет пользователю добавлять новые задачи.
Он также обеспечивает навигацию к экрану деталей задачи.

Код экрана проиллюстрирован на рисунке~\ref{fig:nav:list}.

\begin{image}
	\includegrph[scale=0.37]{Screenshot from 2024-10-05 21-26-18}
	\caption{Класс TaskListScreen}
	\label{fig:nav:list}
\end{image}

В нем метод \texttt{pushReplacement} используется для перехода
на экран информации о задаче \texttt{TaskDetailScreen}.
Так как этот метод снимает верхнюю страницы из стека и заменяет ее на новую,
то для возвращения назад, на экран со списком задач,
также будет использоваться этот метод.

Код экрана \texttt{TaskDetailScreen} показан на рисунке~\ref{fig:nav:detail}.

\begin{image}
	\includegrph[scale=0.45]{Screenshot from 2024-10-05 21-26-58}
	\caption{Класс TaskDetailScreen}
	\label{fig:nav:detail}
\end{image}

\section{Навигация на маршрутах}

в фреймворке Flutter реализована абстрактным слоем функциональности,
которую разработчик должен реализовать под приложение самостоятельно.
Однако, на нынешний момент на общедоступных пространствах
уже присутствует большое количество подготовленных систем маршрутизированной
навигации для фреймворка Flutter.
К примеру, для быстрой реализации маршрутизированной навигации
в приложении можно использовать пакет \texttt{GoRouter}.
Данный пакет предоставляет основной набор функциональности,
которая необходима для маршрутизированной навигации.
А именно: маршрутная карта и делегат навигации.

\subsection{Установка зависимости}

Для использования пакета \texttt{GoRouter} сначало его нужно
прописать в зависимостях к проекту \rref{fig:gorouter:dep}.

\begin{image}
	\includegrph{Screenshot from 2024-10-05 20-49-22}
	\caption{Установка зависимости GoRouter}
	\label{fig:gorouter:dep}
\end{image}

\subsection{Маршрутная карта}

Маршрутная карта в приложении используется
для составления навигационных маршрутов,
которые в дальнейшем разработчик может использовать в приложении.
Данная карта должна составляться в начале приложения
и не может быть изменена в ходе его работы.
Маршрутная карта может включать маршруты разных уровней и вложенностей.\par
После того как маршрутная карта создана,
ее необходимо подключить в наше приложение.
Для этого используется специализированный конструктор
Widget-а приложения --- \texttt{MaterialApp.router}.
В нем необходимо или по отдельности задать делегат навигации в приложение,
или в целом задать конфигурацию навигации.

Пример маршрутной карты и использования \texttt{MaterialApp.router}
предоставлен на рисунке~\ref{fig:gorouter:router}.

\begin{image}
	\includegrph[scale=0.55]{Screenshot from 2024-10-05 21-34-26}
	\caption{Маршрутная карта и MaterialApp.router}
	\label{fig:gorouter:router}
\end{image}

\subsection{Экран навигации}

Изменим экран \texttt{HomePage} \rref{fig:gorouter:home},
который представляет главный навигационный центр приложения.
В нем, по сравнению с использованием \texttt{Navigator},
переход на страницы осуществляется с помощью \texttt{context.go}.
Для его использования в файле необходимо импортировать
пакет \verb|go_router/go_router.dart|.

\begin{image}
	\includegrph[scale=0.5]{Screenshot from 2024-10-05 21-34-50}
	\caption{Класс HomePage}
	\label{fig:gorouter:home}
\end{image}

Этот экран уже был проиллюстрирован на рисунке~\ref{fig:home:show}.
Так как изменился только способ навигации визуально он остался таким же.

\subsection{Экран статистики}

Экран статистики \texttt{TaskStatsScreen} также изметися
для использования GoRouter.
Необходимо удалить метод \texttt{pop} для возврата на экран навигации,
а заместо него использовать \texttt{context.go}.
Новый код продемонстрирован на рисунке~\ref{fig:gorouter:stats}.

\begin{image}
	\includegrph[scale=0.6]{Screenshot from 2024-10-05 21-41-05}
	\caption{Класс TaskStatsScreen}
	\label{fig:gorouter:stats}
\end{image}

Этот экран также уже был проиллюстрирован на рисунке~\ref{fig:home:show}.
И также визуально он остался таким же.

\subsection{Экраны списка задач и информации о задаче}

Чтобы продемонстрировать реализацию горизонтальной навигации в GoRouter,
рассмотрим экран \texttt{TaskListScreen}.

Новый код продемонстрирован на рисунке~\ref{fig:gorouter:list}.

\begin{image}
	\includegrph[scale=0.5]{Screenshot from 2024-10-05 21-40-19}
	\caption{Класс TaskListScreen}
	\label{fig:gorouter:list}
\end{image}

В нем метод \texttt{context.go} используется для перехода
на экран информации о задачеол. Особенность этого вызова в том,
что в путь также передается информация о задаче для дальнейшего использования
в экране с информацией о задаче.\par
Для реализации возврата в TaskListScreen
также вызывается метод \texttt{context.go} \rref{fig:gorouter:detail}.

\begin{image}
	\includegrph[scale=0.6]{Screenshot from 2024-10-05 21-40-41}
	\caption{Класс TaskDetailScreen}
	\label{fig:gorouter:detail}
\end{image}

\clearpage

\section*{ВЫВОД}
\addcontentsline{toc}{section}{ВЫВОД}

В данной практической работе была рассмотрена страничная навигация
в приложениях с использованием фреймворка Flutter,
что является важным аспектом разработки современных мобильных приложений.
Основные методы страничной навигации,
такие как использование \texttt{Navigator} и \texttt{GoRouter},
позволяют разработчикам эффективно управлять переходами между экранами,
обеспечивая плавный и интуитивно понятный интерфейс для пользователей.
При этом, навигация на маршрутах предоставляет возможность организовать
структуру приложения в виде отдельных страниц или экранов,
что значительно упрощает поддержку и расширяемость кода.
В результате применения указанных методов,
разработанное приложение демонстрирует высокую степень удобства
и функциональности, позволяя пользователю легко перемещаться
между различными разделами и взаимодействовать с контентом.

