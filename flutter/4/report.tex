\section*{ЦЕЛЬ ПРАКТИЧЕСКОЙ РАБОТЫ}
\addcontentsline{toc}{section}{ЦЕЛЬ ПРАКТИЧЕСКОЙ РАБОТЫ}

План практической работы:

\begin{itemize}
	\item Создание списков
	\item Знакомство со SingleChildScrollView
	\item Знакомство с ListView и его конструкторами
	\item Удаление элемента из списка
	\item Знакомство с Key и корректное удаление из списка
	\item Выполнение практической работы \No\,4;
\end{itemize}

\clearpage

\section*{ВЫПОЛНЕНИЕ ПРАКТИЧЕСКОЙ РАБОТЫ}
\addcontentsline{toc}{section}{ВЫПОЛНЕНИЕ ПРАКТИЧЕСКОЙ РАБОТЫ}

\section{Создание списков}

Widget вертикальной верстки \texttt{Column} позволяет располагать Widget-ы
в нем по вертикали друг за другом.
При взятии нескольких элементов в Widget-е\texttt{Column} 
у нас получается элемент верстки --- список.
Для корректной работы со списком всегда необходимы 2 составляющие:
контейнер для списка и элемент списка.
В качестве контейнера списка будет взят Widget \texttt{Column},
но для элемента списка требуется создать или один из библиотечных Widget-ов,
или собственный Widget.
Так же любой элемент состоит из визуального представления и данных,
которые должны отображаться.
Поэтому часто для не примитивных данных требуется создание класса-структуры,
описывающего данные элемента списка.\par
Итого простейший список может представлять из себя контейнер
в виде Widget-а \texttt{Column}, а также элемент списка
с отображением в виде Widget-a Text и данными в виде String.
Код с простейшим списком представлен на рисунке~\ref{fig:column:simple},
а его визуализация на рисунке~\ref{fig:column:simple:show}.

\begin{image}
	\includegrph{Screenshot from 2024-09-28 21-27-24}
	\caption{Код простого списка на Column}
	\label{fig:column:simple}
\end{image}

\begin{image}
	\includegrph[scale=0.45]{Screenshot from 2024-09-28 21-04-41}
	\caption{Отображение простого списка на Column}
	\label{fig:column:simple:show}
\end{image}

Однако при увеличении количества элементов можно столкнуться с проблемой,
что не все элементы списка умещаются на экране
и фреймвокр выдает ошибку отрисовки, продемонстрированную
на рисунке~\ref{fig:column:simple:show:error}.
Это связано с тем, что наш контейнер в виде Widget-а \texttt{Column}
имеет высоту равную высоте всего его содержимого.
Для решения этой проблемы необходимо
Widget \texttt{Column} научиться прокручиваться.

\begin{image}
	\includegrph[scale=0.45]{Screenshot from 2024-09-28 21-05-09}
	\caption{Ошибка отрисовки}
	\label{fig:column:simple:show:error}
\end{image}

\section{Знакомство со SingleChildScrollView}

В фреймворке Flutter имеется Widget,
позволяющий уместить любой элемент больший, чем размер экрана, в контейнер.
Widget \texttt{SingleChildScrollView} позволяет разместить внутри себя
любой Widget большего размера, чем сам он имеет,
с возможностью скроллить его содержимое в одном из направлений.
Это позволяет решить проблему, которая была получена в предыдущем пункте.
При оборачивании контейнера списка в виде Widget-а Column
в контейнер прокрутки \texttt{SingleChildScrollView}
полученный список получает возможность скролла и проблема
с высотой списка более чем размер экрана решается.
Код прокручиваемого списка изображен на рисунке~\ref{fig:column:scroll},
а его визуализация на рисунке~\ref{fig:column:scroll:show}.

\begin{image}
	\includegrph[scale=0.5]{Screenshot from 2024-09-28 21-27-48}
	\caption{Код обернутого списка на Column}
	\label{fig:column:scroll}
\end{image}

\begin{image}
	\includegrph[scale=0.5]{Screenshot from 2024-09-28 21-06-35}
	\caption{Отрисовка обернутого списка на Column}
	\label{fig:column:scroll:show}
\end{image}

\section{Знакомство с ListView и его конструкторами}

Проблема с прокручиваемыми списками настолько часта,
что в фреймвоке Flutter создали специальный Widget,
позволяющий создавать списки сразу с функционалом скролла.
Widget \texttt{ListView} в качестве основного аргумента принимает
массив элементов списка, которые будут отображены в нем.
Помимо аргумента, устанавливающего содержимое списка,
у Widget-а \texttt{ListView} есть ряд интересных параметров:
\texttt{reversed} --- позволяет развернуть очередность отображение элементов
в списке, \texttt{scrollDirection} --- задается объектом класса Axis
и указывает значение как направления списка, так и направления скролла.\par
Так же интерес к Widget-у \texttt{ListView} заключается не только
в его основном конструкторе, но и к его вспомогательным конструкторам.
\texttt{ListView} имеет ряд вспомогательных конструкторов,
меняющих его основную механику работы.

\subsection{Builder}

Вспомогательный конструктор\texttt{ListView.builder} 
позволяет списку создавать и хранить в памяти отображение
не всех элементов списка, а только тех, которые сейчас находятся на экране.
Сигнатура конструктора \texttt{ListView.builder} отличительна
от его основного конструктора.
Основной особенностью заключается факт того,
что элементы списка задаются контейнеру не массивом,
а двумя отдельными параметрами:
\texttt{itemBuilder} --- отвечающим за отрисовку конкретного элемента
на экране, \texttt{itemCount} --- отвечающим
за количество элементов списка в контейнере.
Смысл заключается в том, что данные списка не лежат в самом контейнере
и при вызове метода отрисовки элемента списка передается индекс элемента
и задача разработчика указать, какие данные будут взяты для отрисовки данного
элемента. Пример кода работы с \texttt{ListView.builder}
продемонстрирован на рисунке~\ref{fig:list},
а его визуализация на рисунке~\ref{fig:list:show}.

\begin{image}
	\includegrph[scale=0.5]{Screenshot from 2024-09-28 21-14-22}
	\caption{Код списка на ListView.builder}
	\label{fig:list}
\end{image}

\begin{image}
	\includegrph[scale=0.5]{Screenshot from 2024-09-28 21-14-40}
	\caption{Отображение списка на ListView.builder}
	\label{fig:list:show}
\end{image}

\subsection{Separated}

Вспомогательный конструктор \texttt{ListView.separated}
имеет схожую логику с \texttt{ListView.builder},
однако он позволяет создавать разделитель между элементами списка.
Сигнатура конструктора \texttt{ListView.separated}
не сильно отличительна от его ранее рассмотренного конструктора.
Единственным отличием является параметр \texttt{separatorBuilder},
отвечающий за отрисовку разделителя элементов на экране.
Пример кода работы с \texttt{ListView.separated}
продемонстрирован на рисунке~\ref{fig:list:sep},
а его визуализация на рисунке~\ref{fig:list:sep:show}.

\begin{image}
	\includegrph[scale=0.4]{Screenshot from 2024-09-28 21-32-56}
	\caption{Код списка на ListView.separated}
	\label{fig:list:sep}
\end{image}

\begin{image}
	\includegrph[scale=0.4]{Screenshot from 2024-09-28 21-20-50}
	\caption{Отображение списка на ListView.separated}
	\label{fig:list:sep:show}
\end{image}

\section{Удаление элемента списка}

Иногда в разработке требуется возможность произвести
корректировку перечня данных в списке, к примеру удалить один
или несколько элементов на экране.
Если контейнер нашего списка расположен в \texttt{Stateless} Widget-е,
сделать это будет невозможно, однако,
если перенести наш контейнер в \texttt{Stateful} Widget,
то при обновлении \texttt{State} нашего Widget-а
становиться возможно изменить и данные, лежащие в нем.\par
Для удаления можно обернуть элемент списка в \texttt{GestureDetector} Widget
для обработки нажатия на него и имея данные элемента списка можно удалить его
из массива данных через метод \texttt{setState}.
Реализацию можно рассмотреть на рисунке~\ref{fig:remove}.

\begin{image}
	\includegrph{Screenshot from 2024-09-28 21-36-07}
	\caption{Код удаления с помощью GestureDetector}
	\label{fig:remove}
\end{image}

Однако, если при удалении последнего элемента списка все выполняется корректно,
то при удаления любого другого элемента списка удалится не желаемый элемент,
а все равно последний.
Связано это с тем, что при удалении не последнего элемента списка,
при перерисовке экрана, фреймвокр проверяет связь между Widget-ом
и Element-ом проверяется через тип используемого Widget-а.
То есть при удалении 3-его элемента списка в виде Widget-а \texttt{Text}
система удалит его из структуры Widget-ов и перестроит отображение.
При перестройке оно пройдется по списку
и сопоставит все Widget-ы с их Element-ами.
Так как последнему Element-у не достанется Widget-а,
так как один мы удалили, то он и будет отброшен.
Эта проблема описывается как стандартная проблема
<<Удаления не последнего элемента списка>>.\par
Для решения этой проблемы используются ключи.
Каждый Widget в качестве одного из параметров своего конструктора принимает
уникальный ключ, в качестве объекта класса Key.
Данный ключ позволяет фреймвоку проверять связь между Widget-ом
и Element-ом не по типу Widget-а, а как раз по данному ключу.
В качестве объекта значения ключа виджета чаще всего используют класс ValueKey,
так как он позволяет сделать ключ на основе каких-нибудь данных,
к примеру данных элемента списка.
Реализацию списка с ключами можно изучить на рисунке~\ref{fig:remove:key}.

\begin{image}
	\includegrph{Screenshot from 2024-09-28 21-36-17}
	\caption{Код удаления с использованиес key}
	\label{fig:remove:key}
\end{image}

\clearpage

\section{Реализованное приложение}

Создадим приложение, демонстрирующее на одном экране список
построенный на Widget-е \texttt{Column},
на втором на Widget-е \texttt{ListView}
и на третьем Widget-е \texttt{ListView.separated}.
На каждом экране должна быть возможность добавить
и удалить элемент в список из приложения.

\subsection{main.dart}

Файл \texttt{main.dart} нужен для того,
чтобы предоставить пользователю интерфейс с тремя различными экранами,
демонстрирующими различные способы отображения списков в Flutter:
с помощью \texttt{Column}, \texttt{ListView} и \texttt{ListView.separated}.
Он использует \texttt{BottomNavigationBar},
позволяя пользователю легко переключаться между этими экранами
\rdref{fig:lst:1}{fig:lst:2}.

\begin{image}
	\includegrph{Screenshot from 2024-09-28 21-49-03}
	\caption{Код классов HomePage и TaskListsApp}
	\label{fig:lst:1}
\end{image}

\begin{image}
	\includegrph[scale=0.37]{Screenshot from 2024-09-28 21-48-33}
	\caption{Код класса \_HomePageState}
	\label{fig:lst:2}
\end{image}

Отображение списка проиллюстрировано на рисунке~\ref{fig:column:show}

\begin{image}
	\includegrph[scale=0.37]{Screenshot from 2024-09-28 20-54-42}
	\caption{Экран со списком на Column}
	\label{fig:column:show}
\end{image}

\subsection{column.dart}

Эта страница предназначена для демонстрации функциональности управления
списком задач с использованием виджета \texttt{Column} в Flutter.
Она предоставляет пользователю возможность добавлять
и удалять элементы из списка, используя текстовое поле для ввода
и кнопки для взаимодействия \rdref{fig:lst:column}{fig:lst:column:state}.

\begin{image}
	\includegrph{Screenshot from 2024-09-28 21-45-01}
	\caption{Код классов ColumnScreen}
	\label{fig:lst:column}
\end{image}

\begin{image}
	\includegrph[scale=0.42]{Screenshot from 2024-09-28 21-44-33}
	\caption{Код класса \_ColumnScreenState}
	\label{fig:lst:column:state}
\end{image}

Отображение списка проиллюстрировано на рисунке~\ref{fig:column:show}

\begin{image}
	\includegrph[scale=0.57]{Screenshot from 2024-09-28 20-54-42}
	\caption{Экран со списком на Column}
	\label{fig:column:show}
\end{image}

\subsection{listview.dart}

Эта страница предназначена для демонстрации функциональности управления
списком задач с использованием виджета \texttt{Column} в Flutter.
Она предоставляет пользователю возможность добавлять
и удалять элементы из списка, используя текстовое поле для ввода
и кнопки для взаимодействия \rdref{fig:lst:listview}{fig:lst:listview:state}.

\begin{image}
	\includegrph[scale=0.65]{Screenshot from 2024-09-28 21-46-05}
	\caption{Код классов ListViewScreen}
	\label{fig:lst:listview}
\end{image}

Отображение списка проиллюстрировано на рисунке~\ref{fig:listview:show}

\begin{image}
	\includegrph[scale=0.4]{Screenshot from 2024-09-28 21-45-57}
	\caption{Код класса \_ListViewScreenState}
	\label{fig:lst:listview:state}
\end{image}

\begin{image}
	\includegrph[scale=0.39]{Screenshot from 2024-09-28 20-54-54}
	\caption{Экран со списком на ListView}
	\label{fig:listview:show}
\end{image}

\clearpage

\subsection{separated.dart}

В файле \texttt{separated.dart} реализовывается управление списком задач
с использованием виджета \texttt{ListView.separated} в Flutter.
Эта страница позволяет пользователям добавлять и удалять задачи,
а также демонстрирует,
как организовать список с разделителями между элементами
\rdref{fig:lst:listview:sep}{fig:lst:listview:sep:state}.

\begin{image}
	\includegrph[scale=0.5]{Screenshot from 2024-09-28 21-47-51}
	\caption{Код классов ListViewSeparatedScreen}
	\label{fig:lst:listview:sep}
\end{image}

\begin{image}
	\includegrph[scale=0.43]{Screenshot from 2024-09-28 21-47-33}
	\caption{Код класса \_ListViewSeparatedScreenState}
	\label{fig:lst:listview:sep:state}
\end{image}

Отображение списка проиллюстрировано на рисунке~\ref{fig:listview:sep:show}

\begin{image}
	\includegrph{Screenshot from 2024-09-28 20-56-35}
	\caption{Экран со списком на ListView.separated}
	\label{fig:listview:sep:show}
\end{image}


\clearpage

\section*{ВЫВОД}
\addcontentsline{toc}{section}{ВЫВОД}

В ходе выполнения практической работы было разработано приложение
для управления списком задач,
что позволило изучить основы работы со списками в Flutter.
Мы научились создавать и отображать динамические списки
с использованием различных подходов:
от простого \texttt{Column} до более гибких виджетов \texttt{ListView} 
и \texttt{ListView.separated}. 

Реализована функциональность добавления и удаления задач,
что дало практический опыт работы с состоянием приложения
через \texttt{setState()}.
Также было рассмотрено использование \texttt{SingleChildScrollView} 
для прокрутки контента и важность применения ключей (\texttt{Key})
для корректного удаления элементов списка.

Эта работа позволила ознакомиться с базовыми инструментами
для создания динамических интерфейсов
и эффективного управления данными в приложениях Flutter.

