\section*{ЦЕЛЬ ПРАКТИЧЕСКОЙ РАБОТЫ}
\addcontentsline{toc}{section}{ЦЕЛЬ ПРАКТИЧЕСКОЙ РАБОТЫ}

План практической работы:
\begin{itemize}
	\item Установка фреймворка Flutter
		и окружения для языка программирования Dart;
	\item Установка интегрированной среды разработки Android Studio;
	\item Установка браузера Google Chrome;
	\item Установка и настройка требуемых расширений для Android Studio;
	\item Проверка корректности настройки системы;
	\item Создание проекта
	\item Запуск его на трех платформах:
		Android (эмулятор или реальное устройство), Linux и Web.
\end{itemize}

\clearpage

\section*{ВЫПОЛНЕНИЕ ПРАКТИЧЕСКОЙ РАБОТЫ}
\addcontentsline{toc}{section}{ВЫПОЛНЕНИЕ ПРАКТИЧЕСКОЙ РАБОТЫ}
\section{Flutter и Dart}

Для корректной установки SDK Flutter требуется перейти
на официальный сайт Flutter и перейти по ссылке Get Started.
После требуется выбрать операционную систему рабочей машины
и перейти к инструкции по установке на нее (Рисунок~\ref{fig:install:os}).

\begin{image}
	\includegrph{Screenshot from 2024-09-13 19-55-21}
	\caption{Выбор ОС}
	\label{fig:install:os}
\end{image}

После выбора операционной системы требуется выбрать платформу для старта.
Выбрать нужно мобильную платформу (Рисунок~\ref{fig:install:platform}).

\begin{image}
	\includegrph{Screenshot from 2024-09-13 19-55-28}
	\caption{Выбор платформы}
	\label{fig:install:platform}
\end{image}

После перехода на страницу с инструкциями по установке
Flutter SDK требуется произвести указанные действия
и установить Flutter SDK на рабочую машину
и переменную окружения с путем до установленной Flutter SDK.\par
Для установки Flutter нужно загрузить установочный архив,
чтобы получить последнюю стабильную версию Flutter SDK (версия 3.24.3).\par
После установки нужно извлечь файл в каталог,
в котором мы хотим сохранить Flutter SDK (Рисунок~\ref{fig:install:unpack}).

\begin{image}
	\includegrph{Screenshot from 2024-09-14 14-29-02}
	\caption{Установка Flutter SDK в каталог}
	\label{fig:install:unpack}
\end{image}

Чтобы запускать команды Flutter в терминале,
нужно добавить Flutter в переменную среды PATH
(Рисунок~\ref{fig:install:path}).

\begin{image}
	\includegrph{Screenshot from 2024-09-14 14-34-08}
	\caption{Добавление путь до Flatter SDK в переменную среды}
	\label{fig:install:path}
\end{image}

\section{Установка интегрированной среды разработки Android Studio}

В начале инструкций по установке Flutter SDK находится перечень предлагаемых
для взаимодействия интегрированных сред разработки.\par
Для выполнения практических работ предлагается использовать Android Studio,
так как она позваляет проще взаимодействовать с Android устройствами.\par

Установим новейшую версию Android Studio (версии 2024.1.2.12 Koala)
с официального сайта (Рисунок~\ref{fig:android:site}).

\begin{image}
	\includegrph{Screenshot from 2024-09-14 15-49-53}
	\caption{Официальный сайт Android Studio}
	\label{fig:android:site}
\end{image}

Для установки скачиваем архив и распаковываем его в нужную директории
\rref{fig:android:opt}.
В данном случае \texttt{/opt/} для общего пользования.
После этого Android Studio можно будет запустить командой:
\texttt{./opt/android-studio/bin/studio.sh}.

\begin{image}
	\includegrph{Screenshot from 2024-09-14 15-00-02}
	\caption{Установка Android Studio в системный каталог}
	\label{fig:android:opt}
\end{image}


После скачивания скрипта начнем процесс установки
(Рисунок~\ref{fig:android:sdk}).

\begin{image}
	\includegrph{Screenshot from 2024-09-14 15-00-32}
	\caption{Установка Android SDK}
	\label{fig:android:sdk}
\end{image}

Далее нам будет предложено выбрать нужные компоненты для установки
(Рисунок~\ref{fig:android:sdk:components}),
выберем их все и перейдем на следующий экран.

\begin{image}
	\includegrph{Screenshot from 2024-09-14 15-00-39}
	\caption{Установка SDK Components}
	\label{fig:android:sdk:components}
\end{image}

Затем нам предложат проверить установку (Рисунок~\ref{fig:android:verify}),
выберем их все и перейдем на следующий экран.

\begin{image}
	\includegrph{Screenshot from 2024-09-14 15-00-54}
	\caption{Проверка установки}
	\label{fig:android:verify}
\end{image}

Следующим экраном является лицензионное соглашение
(Рисунок~\ref{fig:android:license}), принимаем все условия.

\begin{image}
	\includegrph{Screenshot from 2024-09-14 15-01-01}
	\caption{Лицензионное соглашение}
	\label{fig:android:license}
\end{image}

В результате установки Android Studio Setup Wizard
нас встречает окно приветствия Android Studio
(Рисунок~\ref{fig:android:start}).

\begin{image}
	\includegrph{Screenshot from 2024-09-14 15-03-26}
	\caption{Окно Android Studio}
	\label{fig:android:start}
\end{image}

Теперь сконфигурируем Android эмулятор для работы.
Откроем Device Manager (Рисунок~\ref{fig:android:device:manager})
и создадим новый эмулятор.

\begin{image}
	\includegrph{Screenshot from 2024-09-14 15-07-43}
	\caption{Device Manager}
	\label{fig:android:device:manager}
\end{image}

В качестве устройства выберем Medium Phone
(Рисунок~\ref{fig:android:device:manager:phone}),
указав это название в поисковой строке.

\begin{image}
	\includegrph{Screenshot from 2024-09-14 15-07-57}
	\caption{Выбор устройства}
	\label{fig:android:device:manager:phone}
\end{image}

В качестве образа выберем API 35
(Рисунок~\ref{fig:android:device:manager:api}).

\begin{image}
	\includegrph{Screenshot from 2024-09-14 15-09-50}
	\caption{Выбор образа}
	\label{fig:android:device:manager:api}
\end{image}

После выбора 35 API мы перейдем на конечный экран
(Рисунок~\ref{fig:android:device:manager:res}),
с просмотром всех характеристик эмулятора.
Здесь также можно изменить некоторые параметры в виде названия
или внутри Advanced Settings.

\begin{image}
	\includegrph{Screenshot from 2024-09-14 15-09-55}
	\caption{Результирующий экран создания эмулятора}
	\label{fig:android:device:manager:res}
\end{image}

\section{Установка браузера Google Chrome}

Для корректной работы с Web платформой потребуется установленный
на рабочей машине один из общедоступных браузеров.
В связи с тем, что Flutter был разработан компанией Google,
то SDK предлагает установить Google Chrome (версия 128.0.6613.119-1)
для корректной работы с Web платформой.
Для этого скачиваем deb-файл с официального сайта
и устанавливаем в систему (Рисунок~\ref{fig:install:chrome}).

\begin{image}
	\includegrph{Screenshot from 2024-09-07 10-05-40}
	\caption{Установка Google Chrome}
	\label{fig:install:chrome}
\end{image}

\section{Установка и настройка требуемых расширений для Android Studio}

Чтобы создавать приложения для Android с помощью Flutter,
нужно установить следующие компоненты Android.

\begin{itemize}
	\item Android SDK Build-Tools, API 35;
	\item Android SDK Command-line Tools;
	\item Android SDK Platform-Tools (версии 35.0.2);
	\item Android Emulator (версии 35.1.20);
\end{itemize}

Чтобы установить эти компоненты нужно перейти в SDK Manager
(Рисунок~\ref{fig:android:sdk:manager}). 

\begin{image}
	\includegrph[scale=0.4]{Screenshot from 2024-09-14 15-14-43}
	\caption{SDK Manager}
	\label{fig:android:sdk:manager}
\end{image}

Далее нужно установить плагин Flutter (версия 81.0.2)
и Dart (версия 241.18808).
Сделаем это через среду Android Studio через вкладку Plugins
(Рисунок~\ref{fig:android:plugin}).

\begin{image}
	\includegrph{Screenshot from 2024-09-07 10-26-44}
	\caption{Flutter плагин для Android}
	\label{fig:android:plugin}
\end{image}

Так же после завершения всех настроек Android Studio,
требуется пройти процедуру, связанную с соглашением на использование
Android SDK командой \verb|flutter dockor --android-licenses|
(Рисунок~\ref{fig:dockor:android:licenses}).

\begin{image}
	\includegrph{Screenshot from 2024-09-07 10-07-55}
	\caption{Лицензионное соглашение Android}
	\label{fig:dockor:android:licenses}
\end{image}

\section{Проверка корректности настройки системы}

По окончанию всех установок и настроек рабочего окружения требуется
провести проверку готовности системы к работе.
Для этого в командной строке вызовем \texttt{flutter dockor -v}
(Рисунок~\ref{fig:flutter:dockor}).

\begin{image}
	\includegrph[scale=0.3]{Screenshot from 2024-09-20 20-01-19}
	\caption{Вызов flutter dockor}
	\label{fig:flutter:dockor}
\end{image}

Если система говорит, что не знает команды flutter,
то это означает, что либо Flutter SDK установлен не верно,
либо Flutter не внесен в переменные окружения.
Корректная работа команды должна вывести инструментарий
для работы с Flutter SDK,
а также статус настройки этого инструмента в виде галочки
или восклицательного знака. Настройка может быть завершена,
если галочки стоят у пунктов: \textit{Flutter}, \textit{Linux Toolchain},
\textit{Android Toolchain}, \textit{Android Studio}
(или другой среду разработки, если устанавливали другую)
и \textit{Chrome} (если производили установку).
Если настройка прошла корректно, то можно приступать к созданию проекта.

\section{Создание проекта}

Для создания проекта в Android Studio требуется запустить среду разработки
и в открывшемся окне выбрать <<New Flutter Project>>.
Если же до этого уже был открыт другой проект,
то требуется открыть пункт меню <<File>>, в нем <<New>>
и в нем <<New Flutter project>>.\par
После чего среда разработки может попросить указать маршрут
до Flutter SDK (Рисунок~\ref{fig:project:flutter})
и после попросит заполнить данные о проекте
(Рисунок~\ref{fig:project:settings}).

\begin{image}
	\includegrph[scale=0.4]{Screenshot from 2024-09-14 15-03-40}
	\caption{Выбор маршрута до SDK}
	\label{fig:project:flutter}
\end{image}

\begin{image}
	\includegrph[scale=0.4]{Screenshot from 2024-09-14 15-04-15}
	\caption{Настройка проекта}
	\label{fig:project:settings}
\end{image}

В данных о проекте находится следующая информация:

\begin{itemize}
	\item Название проекта;
	\item Месторасположение будущего проекта;
	\item Описание проекта;
	\item Тип проекта --- должно быть выбрано Application;
	\item Организация;
	\item Нативный язык Android платформы;
	\item Нативный язык IOS платформы;
	\item Создаваемые платформы --- обязательно должны быть выбраны:
		Android, Linux, Web
\end{itemize}

Когда все значения установлены, можно нажимать клавишу <<Create>>
и ожидать создание и индексации нового тестового проекта
(Рисунок~\ref{fig:project}).

\begin{image}
	\includegrph{Screenshot from 2024-09-14 20-19-00}
	\caption{Созданный проект}
	\label{fig:project}
\end{image}

\section{Запуск проекта}

Когда проект создастся и проиндексируется,
среда разработки позволит запустить проект на выбранной платформе.
Для регулирования платформы в верхнем тулбаре можно нажать на левый дропдаун,
который продемонстрирует все подключенные устройства различных платформ.\par
Для запуска приложения необходимо нажать на <<Запуск>>, в виде зеленой клавши
<<Воспроизвести>>. При запуске проекта в консоли отладки появится информация
по запуску и проект запуститься на выбранном устройстве платформы.

Запущенное тестовое приложение на платформе Linux
продемонстрировано на рисунке \ref{fig:run:linux}.

\begin{image}
	\includegrph[scale=0.23]{Screenshot from 2024-09-07 10-41-35}
	\caption{Запуск приложения на Linux}
	\label{fig:run:linux}
\end{image}

Запущенное тестовое Web приложение
продемонстрировано на рисунке \ref{fig:run:web}.

\begin{image}
	\includegrph[scale=0.23]{Screenshot from 2024-09-07 10-31-29}
	\caption{Запуск web приложения}
	\label{fig:run:web}
\end{image}
\clearpage

Запущенное тестовое android приложение
продемонстрировано на рисунке \ref{fig:run:android}.

\begin{image}
	\includegrph{Screenshot from 2024-09-13 19-56-09}
	\caption{Запуск android приложения}
	\label{fig:run:android}
\end{image}
\clearpage

\clearpage

\section*{ВЫВОД}
\addcontentsline{toc}{section}{ВЫВОД}

В ходе выполнения практической работы была проведена установка
и настройка необходимых компонентов для разработки приложений
с использованием фреймворка Flutter и языка программирования Dart.
Была произведена установка интегрированной среды разработки Android Studio,
а также браузера Google Chrome,
необходимого для тестирования веб-версии приложений.\par
В Android Studio были установлены и настроены требуемые расширения
для работы с Flutter и Dart.
После этого была проведена проверка корректности настройки системы
с использованием команды flutter doctor,
которая подтвердила успешную настройку всех компонентов.\par
Далее был создан новый проект на Flutter,
который был успешно запущен на трех различных платформах:

\begin{itemize}
	\item Android --- проект был запущен на эмуляторе Android,
		а также протестирован на реальном устройстве.
	\item Linux --- была скомпилирована
		и запущена десктопная версия приложения.
	\item Web --- приложение было развернуто
		и протестировано в браузере Google Chrome.
\end{itemize}

Таким образом, все этапы практической работы были выполнены, и проект успешно заработал на всех указанных платформах.


