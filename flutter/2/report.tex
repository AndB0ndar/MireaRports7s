\section*{ЦЕЛЬ ПРАКТИЧЕСКОЙ РАБОТЫ}
\addcontentsline{toc}{section}{ЦЕЛЬ ПРАКТИЧЕСКОЙ РАБОТЫ}

План практической работы:
\begin{itemize}
	\item Знакомство со структурой стартового проекта;
	\item Изучение конфигурационного файла pubspec.yaml
		и его слепка pubspec.lock;
	\item Изучение lib директории и знакомство
		с основными принципами работы с файлами в проекте;
	\item Знакомство с точкой запуска приложения main.dart,
		а так же с виджетом приложения MaterialApp
		и виджетом страницы приложения Scaffold;
	\item Внести корректировки в страницу проекта таким образом,
		чтобы вместо информации о количестве нажатий на клавишу
		и самой клавиши на экране было написано ФИО студента,
		номер его группы, а также его номер студенческого.
\end{itemize}

\clearpage

\section*{ВЫПОЛНЕНИЕ ПРАКТИЧЕСКОЙ РАБОТЫ}
\addcontentsline{toc}{section}{ВЫПОЛНЕНИЕ ПРАКТИЧЕСКОЙ РАБОТЫ}

\section{Структура стартового проекта}

При создании стартового проекта сразу создается перечень дирректорий
и файлов, которые используются в процессе разработки приложений
на фреймворке Flutter \rref{fig:tree:project}.

\begin{image}
	\includegrph{Screenshot from 2024-09-13 20-48-30}
	\caption{Дерево проекта}
	\label{fig:tree:project}
\end{image}

Каждая директория проекта имеет свое прямое предназначение
и в целом все их можно разбить на основные 3 группы:

\begin{itemize}
	\item Сервисные директории --- \texttt{.dart\_tool}, \texttt{.idea},
		и все другие, начинающиеся с <<.>>, \texttt{build};
	\item Директории нативного блока --- \texttt{android}, \texttt{ios},
		\texttt{linux}, \texttt{macos}, \texttt{web}, \texttt{windows};
	\item Директории разработки --- \texttt{lib}, \texttt{test},
		\texttt{assets} и другие создаваемые разработчиком директории.
\end{itemize}

Помимо директорий, стартовый проект содержит ряд файлов,
используемых для настройки проекта. Рассмотрим ряд из них:

\begin{itemize}
	\item \texttt{pubspec.yaml} и \texttt{pubspec.lock} --- основной файл
		конфигурации проекта;
	\item \texttt{analysis\_option.yaml} --- файл конфигурации
		статического анализатора;
	\item \texttt{README.md} --- файл с публичным описанием задач
		и смысла проекта.
\end{itemize}

\section{pubspec.yaml и pubspec.lock}

Основными файлами настройки конфигурации проекта являются ---
\texttt{pubspec.yaml} и \texttt{pubspec.lock}.
Эти два файла имеют определенную структуру
и в них заложен определенный смысл взаимодействия.\par
Изучений конфигурационных файлов легче всего начать
с файла \texttt{pubspec.lock} \rref{fig:pubspec:lock},
так как взаимодействия разработчика с ним сведено к минимуму.
Данный файл является слепком основного файла конфигурации
и несет в себе сервисную информацию о проекте.
В преобладающем большинстве процессов взаимодействия разработчика
с проектом нет необходимости открывать
и как либо корректировать файл \texttt{pubspec.lock}.

\begin{image}
	\includegrph[scale=0.33]{Screenshot from 2024-09-15 15-25-33}
	\caption{Файл pubspec.lock}
	\label{fig:pubspec:lock}
\end{image}

Основным же файлом конфигурации,
с которым разработчик взаимодействует на постоянной основе,
является \texttt{pubspec.yaml}. Расширение файла yaml говорит о том,
что структура файла и способ взаимодействия с ним стандартны,
как и для других конфигурационных yaml файлов. Файл имеет достаточно простую,
но при этом комплексную структуру \rref{fig:pubspec:yaml}.

\begin{image}
	\includegrph[scale=0.53]{Screenshot from 2024-09-13 21-14-50}
	\caption{Файл pubspec.yaml}
	\label{fig:pubspec:yaml}
\end{image}

На верхнем уровне файла можно увидеть основные параметры:

\begin{itemize}
	\item \textbf{Name} --- параметр устанавливающий название проекта;
	\item \textbf{Description} --- параметр устанавливающий
		приватное описание проекта;
	\item \textbf{Publish\_to} --- параметр задающий возможность открытого
		распространения данного проекта. Значение «none» означает,
		что данный проект не планируется к открытому распространению
		как зависимость;
	\item \textbf{Version} --- устанавливает нынешнюю версию приложения.
		Чаще всего используется трехуровневая версионирование
		с использованием номера сборки;
	\item \textbf{Dependencies} --- зависимости проекта от других пакетов,
		проектов или зависимостей, используемых в проекте
		и добавляемых в итоговую сборку приложения;
	\item \textbf{Dev\_dependencies} --- зависимости проекта от других пакетов,
		проектов или зависимостей, используемых в проекте,
		но не добавляемых в итоговую сборку приложения;
	\item \textbf{Flutter} --- установка зависимостей проекта
		от других источников.
\end{itemize}

\section{lib директории и основные принципы работы с файлами}

В директории lib \rref{fig:lib} располагаются файлы, создаваемые разработчиком.

\begin{image}
	\includegrph{Screenshot from 2024-09-15 15-09-24}
	\caption{Директория lib}
	\label{fig:lib}
\end{image}

Создание файлов и поддиректорий в данной директории подлежат ряду правил,
для соблюдения консистентности и удобного, в дальнейшем, взаимодействия.
В основном есть 2 основных подхода: Feature First и Screen First.\par
\textbf{Feature first} подход описывает создание директорий
и файлов в них на основе реализуемых фичей в проекте.
То есть в директории создается директория \texttt{feature},
в котором в дальнейшем создается поддиректория с названием той фичи,
которая планируется к разработке.\par
\textbf{Screen First} подход описывает создание директорий
в проекте на основе тех экранов, которые создаются в приложении.\par
Какого принципа придерживаться в ходе разработки приложения,
принимает решение сам разработчик.
Однако в сообществе Flutter разработчиков чаще
встречается использование Feature First подхода.\par
Так же в Flutter проектах требуется соблюдение принципа инкапсуляции
и разграничения доступа к файлам.
Для этого в директории \texttt{lib} часто создается
на корневом уровне поддиректория \texttt{src},
которая не дает внешним пакетам получать доступ к файлам внутри нее.
Однако, часто бывает случаи, требующие предоставлять доступ
к отдельным файлам в директории \texttt{src}.
Для этого требуется создать на уровне директории \texttt{export} файл,
с названием проекта и вынести в него файлы,
с указанием ключевого слова \texttt{export}, для предоставления к ним доступа.

\section{Точка запуска приложения и виджеты MaterialApp Scaffold}

Приложение в фреймворке Flutter основываются на работе разработчика
с Widget-ами фреймворка. Flutter имеет большой перечень стандартных Widget-ов,
но помимо этого, разработчик в праве создавать собственные,
и даже, в отдельных случаях, даже задавать им нестандартные способы отрисовки.
Однако в основном, разработка ведется на основе
стандартных Widget-ов фреймворка.\par
Приложение на Flutter не может быть создано,
без использования Widget-ов приложения.
В фреймворке присутствуют два стандартных Widget-а,
позволяющие создавать приложение: \texttt{MaterialApp} и \texttt{CupertinoApp}.
В основном, если не требуется затачивать внешний вид
и основные паттерны взаимодействия приложения под ОС компании Apple,
в разработке используется Widget --- \texttt{MaterialApp}
\rref{fig:widget:materialapp}.

\begin{image}
	\includegrph{Screenshot from 2024-09-15 15-28-25}
	\caption{Виджет MaterialApp}
	\label{fig:widget:materialapp}
\end{image}

В атрибут \texttt{Title} закладывается название приложения,
которое требуется для работы с ним.
В атрибут \texttt{Theme} закладывается тема приложения.
Чаще всего используется стандартная тема приложения,
однако при необходимости, разработчик может изменить ее.
В атрибут home задается значение основной страницы приложения,
если в нем не планируется навигационная система.\par
После того, как приложение добавлено
и в него требуется добавить страницу приложения.
Для отрисовки страницы используется
стандартный Widget \texttt{Scaffold} \rref{fig:widget:scaffold}.
Данный виджет отвечает за отрисовку страницы и имеет сигнатуру
для удобного ее конфигурации.

\begin{image}
	\includegrph[scale=0.5]{Screenshot from 2024-09-15 15-28-48}
	\caption{Виджет Scaffold}
	\label{fig:widget:scaffold}
\end{image}

В атрибуте \texttt{AppBar} задатеся верхняя шапка приложения.
Атрибут \texttt{FloatingActionButton} используется для конфигурации клавиши,
отображаемой поверх всего экрана.
По умолчанию она находится в нижнем правом углу.
А вот атрибут \texttt{body} используется для отображения содержимого страницы.
В нем как-раз устанавливается все те Widget-ы,
которые разработчик хочет использовать на странице.

\section{Исправление страницы проекта}

Изменим код в странице созданной по умолчанию \rref{fig:page:code:new}.

\begin{image}
	\includegrph[scale=0.4]{Screenshot from 2024-09-20 21-42-00}
	\caption{Изменный код домашней страницы}
	\label{fig:page:code:new}
\end{image}

Удалены переменные и методы,
связанные с подсчётом нажатий на кнопку
(такие как \texttt{\_counter} и метод \texttt{\_incrementCounter}).
Удалён виджет FloatingActionButton, который использовался
для увеличения счётчика.\par
Теперь вместо информации о количестве нажатий на экран выводятся
три строки текста:

\begin{itemize}
	\item <<Бондарь Андрей Ренатович>> --- с жирным шрифтом
		и размером текста 24.
	\item <<ИКБО-06-21>> --- размер текста 20.
	\item <<21И1874>> --- также размер текста 20.
\end{itemize}

Надписи созданы с помощью виджетов \texttt{Text},
каждый из которых отображает строку с заданным стилем.
Для первой строки указан размер шрифта 24 и жирное начертание.
Все надписи помещены в виджет \texttt{Column} для вертикального расположения,
а для отступов между ними использован \texttt{SizedBox}.\par
Новая страница отображена на рисунке~\ref{fig:page:new}.

\begin{image}
	\includegrph[scale=0.37]{Screenshot from 2024-09-15 16-38-08}
	\caption{Новая страница}
	\label{fig:page:new}
\end{image}

\clearpage

\section*{ВЫВОД}
\addcontentsline{toc}{section}{ВЫВОД}

В ходе практической работы мы познакомились
со структурой стартового проекта Flutter,
что позволило понять организацию файлов и директорий.
Был изучен конфигурационный файл \texttt{pubspec.yaml},
где задаются зависимости проекта, а также его слепок \texttt{pubspec.lock},
который фиксирует версии этих зависимостей
для обеспечения стабильной сборки.\par
Мы рассмотрели директорию \texttt{lib},
в которой находятся основные файлы исходного кода,
и разобрались с основными принципами работы с файлами в проекте.
Особое внимание было уделено файлу \texttt{main.dart}
как точке запуска приложения.
Мы изучили виджеты \texttt{MaterialApp} для создания приложения
и \texttt{Scaffold} для построения структуры пользовательского интерфейса,
что позволило получить базовое понимание организации
и функциональности Flutter-приложений.\par
Также в рамках практической работы было также выполнено изменение
в классе \texttt{\_MyHomePageState}.
Изменения включали удаление счётчика и кнопки для увеличения числа,
а также замену отображаемого текста на статичную информацию.
Эти изменения позволили на практике применить полученные знания о виджетах
и их стилизации в Flutter.

