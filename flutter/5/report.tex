\section*{ЦЕЛЬ ПРАКТИЧЕСКОЙ РАБОТЫ}
\addcontentsline{toc}{section}{ЦЕЛЬ ПРАКТИЧЕСКОЙ РАБОТЫ}

План практической работы:

\begin{itemize}
	\item Определение нынешней платформы
	\item Платформенно-зависимый код
	\item Выполнение практической работы \No\,5;
\end{itemize}

\clearpage

\section*{ВЫПОЛНЕНИЕ ПРАКТИЧЕСКОЙ РАБОТЫ}
\addcontentsline{toc}{section}{ВЫПОЛНЕНИЕ ПРАКТИЧЕСКОЙ РАБОТЫ}

\section{Определение платформы}

\subsection{Определение веб-платформы}

Единственной платформой,
которую нельзя отличить стандартным способом является веб платформа.
Для определения веб платформы в фреймворке Flutter используется
глобальная переменная \textbf{kIsWeb}.
Если ее значение является истиной,
то запускаемая платформа --- веб, если же нет,
то запускаемая платформа мобильная или десктопная.

Код с использования глобальное переменной проиллюстрирован
на рисунке~\ref{fig:kisweb}.

\begin{image}
	\includegrph{Screenshot from 2024-10-04 18-55-58}
	\caption{Код использования kIsWeb}
	\label{fig:kisweb}
\end{image}

Запущенные приложения проиллюстрированы
на рисунках~\ref{fig:kisweb:web}\,-\,\ref{fig:kisweb:linux}.

\begin{image}
	\includegrph[scale=0.5]{Screenshot from 2024-10-04 18-56-42}
	\caption{Web приложение}
	\label{fig:kisweb:web}
\end{image}

\begin{image}
	\includegrph[scale=0.5]{Screenshot from 2024-10-04 19-25-35}
	\caption{Десктопное приложение}
	\label{fig:kisweb:linux}
\end{image}

\subsection{Определение мобильных и десктопных платформ}

При работе с мобильными или десктопными платформами
для определения платформы и операционной системы, запускающей приложение,
используется класс \texttt{Platform}.
В данном классе есть статические логические поля,
возвращающие операционную систему, на которой запускается приложение.
Для мобильных платформ это:
\texttt{isAndroid}, \texttt{isIOS} и \texttt{isFuchsia},
а для десктопных платформ это:
\texttt{isWindows}, \texttt{isMacOS}, \texttt{isLinux}.
Определив платформу, на которой запущено приложение можно
далее писать платформенно-зависимый код.

Код с использования глобальное переменной проиллюстрирован
на рисунке~\ref{fig:platform}.

\begin{image}
	\includegrph[scale=0.6]{Screenshot from 2024-10-04 19-29-04}
	\caption{Код использования Platform}
	\label{fig:platform}
\end{image}

Запущенные приложения проиллюстрированы
на рисунках~\ref{fig:platform:linux}\,-\,\ref{fig:platform:android}.

\begin{image}
	\includegrph[scale=0.4]{Screenshot from 2024-10-04 19-29-29}
	\caption{Linux приложение}
	\label{fig:platform:linux}
\end{image}

\begin{image}
	\includegrph[scale=0.4]{Screenshot from 2024-10-04 19-34-54}
	\caption{Android приложение}
	\label{fig:platform:android}
\end{image}

\section{Написание платформенно-зависимого кода}

После определения платформы можно использовать различные реализации
логики для каждой из них.
Например, для работы с файловой системой:

\begin{itemize}
	\item На мобильных устройствах (iOS, Android)
		можно использовать пакеты вроде
		\texttt{path\_provider} для доступа к файлам.
	\item На десктопных платформах (Windows, macOS, Linux)
		доступ к файловой системе может быть реализован
		с использованием стандартных средств работы с файлами.
	\item Для веб-платформы прямого доступа к файловой системе нет.
\end{itemize}

\subsection{Выбор платформы}

Создадим класс \texttt{PlatformSpecificHomePage},
который определяет, какой виджет загружать в зависимости от платформы
(веб, мобильные или десктопные устройства)
\rref{fig:lst:todo:taskapp}.

\begin{image}
	\includegrph[scale=0.6]{Screenshot from 2024-10-04 20-46-52}
	\caption{Код класса PlatformSpecificHomePage}
	\label{fig:path:home}
\end{image}

\subsection{Установка зависимости}

Для работы с файловой системой на мобильных устройствах (iOS, Android)
можно использовать пакет \texttt{path\_provider} для доступа к файлам.
Для этого необходимо прописать этот пакет
в зависимостях проекта \rref{fig:path:dep}, в файле \texttt{pubspec.yaml}.

\begin{image}
	\includegrph[scale=0.8]{Screenshot from 2024-10-04 20-33-46}
	\caption{Установка зависимости path\_provider}
	\label{fig:path:dep}
\end{image}

\subsection{Web приложение}

Так как в web приложения нет работы с файлами, данное приложение не должно
ничего выводит.

Код продемонстрирован на рисунке~\ref{fig:path:web}.

\begin{image}
	\includegrph{Screenshot from 2024-10-04 20-46-59}
	\caption{Код для Web приложение}
	\label{fig:path:web}
\end{image}

Страница проиллюстрирована на рисунке~\ref{fig:path:web:show}.

\begin{image}
	\includegrph{Screenshot from 2024-10-04 20-48-32}
	\caption{Web приложение}
	\label{fig:path:web:show}
\end{image}

\subsection{Мобильное приложение}

Для получения пути в мобильном приложении нужно импортировать пакет
\verb|package:path_provider/path_provider.dart|.

Код реализации мобильного приложения продемонстрирован
на рисунке~\ref{fig:path:android}.

\begin{image}
	\includegrph[scale=0.4]{Screenshot from 2024-10-04 20-47-13}
	\caption{Код для Android приложение}
	\label{fig:path:android}
\end{image}

Страница проиллюстрирована на рисунке~\ref{fig:path:android:show}.

\begin{image}
	\includegrph[scale=0.4]{Screenshot from 2024-10-04 20-52-53}
	\caption{Android приложение}
	\label{fig:path:android:show}
\end{image}

\subsection{Десктопное приложение}

Код реализации десктопного приложения продемонстрирован
на рисунке~\ref{fig:path:linux}.

\begin{image}
	\includegrph{Screenshot from 2024-10-04 20-47-24}
	\caption{Код для десктопного приложения}
	\label{fig:path:linux}
\end{image}

Страница проиллюстрирована на рисунке~\ref{fig:path:linux:show}.

\begin{image}
	\includegrph{Screenshot from 2024-10-04 20-49-10}
	\caption{Десктопное приложение}
	\label{fig:path:linux:show}
\end{image}

\section{Приложение}

\subsection{main.dart}

Это главный файл приложения, который запускает приложение.
Он отвечает за инициализацию и настройку корневого виджета.
Также этот файл содержит класс \texttt{PlatformSpecificHomePage},
который определяет, какой виджет загружать в зависимости от платформы
(веб, мобильные или десктопные устройства).

Код продемонстрирован
на рисунках~\ref{fig:lst:todo:taskapp}\,-\,\ref{fig:lst:todo:home}.

\begin{image}
	\includegrph{Screenshot from 2024-10-04 19-58-27}
	\caption{Код класса TaskApp}
	\label{fig:lst:todo:taskapp}
\end{image}

\begin{image}
	\includegrph{Screenshot from 2024-10-04 19-58-36}
	\caption{Код класса PlatformSpecificHomePage}
	\label{fig:lst:todo:home}
\end{image}

\section{web.dart}

Этот файл определяет виджет \texttt{WebTodoPage},
который отображается на веб-платформах \rref{fig:lst:todo:web}.
Он показывает заголовок и сообщение.

\begin{image}
	\includegrph{Screenshot from 2024-10-04 19-58-51}
	\caption{Код класса WebTodoPage}
	\label{fig:lst:todo:web}
\end{image}

Отображение этого приложения проиллюстрировано
на рисунке~\ref{fig:todo:web}.

\begin{image}
	\includegrph{Screenshot from 2024-10-04 20-04-27}
	\caption{Web приложение}
	\label{fig:todo:web}
\end{image}

\subsection{mobile.dart}

Класс \texttt{MobileTodoPage} \rref{fig:lst:todo:mobile} реализует
страницу приложения для мобильных устройств,
которая позволяет пользователю управлять списком задач (todo list).
Это состояние \texttt{StatefulWidget}, что означает,
что он может хранить изменяемые данные и реагировать на изменения.

\begin{image}
	\includegrph[scale=0.5]{Screenshot from 2024-10-04 20-01-43}
	\caption{Код класса MobileTodoPage}
	\label{fig:lst:todo:mobile}
\end{image}

Когда пользователь вводит текст в поле и нажимает кнопку <<Добавить>>,
задача добавляется в список, и интерфейс обновляется,
отображая новую задачу в виде элемента списка.
Пользователь может добавлять сколько угодно задач,
и все они будут динамически отображаться в списке.

Отображение этого приложения проиллюстрировано
на рисунке~\ref{fig:todo:mobile}.

\begin{image}
	\includegrph[scale=0.9]{Screenshot from 2024-10-04 19-57-15}
	\caption{Android приложение}
	\label{fig:todo:mobile}
\end{image}

\subsection{desctop.dart}

Класс \texttt{DesktopTodoPage} \rref{fig:lst:todo:desctop}
реализует страницу приложения для настольных устройств (десктопов),
которая позволяет пользователю управлять списком задач (todo list).
Как и в случае с \texttt{MobileTodoPage}, этот класс является состоянием
(\texttt{StatefulWidget}), что позволяет ему хранить изменяемые данные
и реагировать на изменения.

\begin{image}
	\includegrph[scale=0.47]{Screenshot from 2024-10-04 20-02-05}
	\caption{Код класса DesctopTodoPage}
	\label{fig:lst:todo:desctop}
\end{image}

Пользователь может вводить текст в поле и нажимать кнопку <<Добавить>>,
чтобы добавлять задачи в список.
Каждая добавленная задача отображается в списке,
который обновляется при добавлении новой задачи.

Отображение этого приложения проиллюстрировано
на рисунке~\ref{fig:todo:linux}.

\begin{image}
	\includegrph[scale=0.5]{Screenshot from 2024-10-04 19-57-51}
	\caption{Linux приложение}
	\label{fig:todo:linux}
\end{image}

\clearpage

\section*{ВЫВОД}
\addcontentsline{toc}{section}{ВЫВОД}

В ходе прктической работы была изучена работа
с платформенно-зависимым кодом в разработке приложений на Flutter.
Были рассмотрены ключевые аспекты определения платформы,
написания платформенно-зависимого кода
и создания приложений для различных типов платформ,
таких как веб, мобильные устройства и десктопы.

