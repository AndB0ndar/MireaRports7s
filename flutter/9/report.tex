\section*{ЦЕЛЬ ПРАКТИЧЕСКОЙ РАБОТЫ}
\addcontentsline{toc}{section}{ЦЕЛЬ ПРАКТИЧЕСКОЙ РАБОТЫ}

План практической работы:

\begin{itemize}
	\item Изучить проблему зависимостей в проектах;
	\item Произвести работу с Inherited Widget;
	\item Произвести подключение и работу с DI контейнером GetIt;
	\item Выполнение практической работы \No\,9.
\end{itemize}

\clearpage

\section*{ВЫПОЛНЕНИЕ ПРАКТИЧЕСКОЙ РАБОТЫ}
\addcontentsline{toc}{section}{ВЫПОЛНЕНИЕ ПРАКТИЧЕСКОЙ РАБОТЫ}

\section{Проблема зависимостей в проекте}

Ранее в практических работах мы уже сталкивались
с необходимостью передавать один
или более параметр в конструктор при создании объекта того или иного класса.
Бывают ситуации, когда аргументов не много
или же все их значения заранее известны,
либо вычисляются в данном месте.
В таких ситуациях создать объект такого класса
не составляет никакого труда.\par
Однако бывают ситуации, когда перечень аргументов становится многочисленным,
а также значения аргументов конструктора уже невозможно рассчитать на месте
и их приходится брать из других мест.
В таких ситуациях приходится делать место,
в котором будет создаваться объект такого сложного класса зависимым
от тех данных, которые необходимы в конструкторе.
При достижении определенного уровня такой вложенности становится все
тяжелее взаимодействовать с программным кодом
и требуется решать такого рода проблемы.\par
Самым простым способом решения проблемы поочередной зависимости классов
от какоголибо набора данных является вынос этих данных во внешнее пространство,
к которому у этих классов будет доступ.
Это позволяет не передавать зависимость из класса в класс,
а получать ее напрямую из хранилища.
Такой принцип называется --- внедрение зависимостей или,
как в дальнейшем будет называться, \texttt{DI}.
В фреймворке Flutter есть разные способы реализации \texttt{DI},
однако рассмотрено будет 2 из них:
\texttt{Inherited} Widget и \texttt{DI} контейнер \texttt{GetIt}.

\section{Inherited Widget}

\texttt{Inherited} Widget является одним
из основных типов Widget-ов в фреймворке Flutter.
Он в основном выступает в качестве внешнего хранилища данных,
располагаемых в дереве виджетов, как любой другой Widget.
В отличии от \texttt{Stateless} и \texttt{Stateful} Widget,
\texttt{Inherited} Widget не имеет
и не предполагает какого-либо отображения на экране,
поэтому имеет отличные как внутреннее устройство,
так и принцип взаимодействия.\par
Все Widget-ы, зависящие от данных в \texttt{Inherited} Widget
подписываются на его состояние и получают данные по структуре Widget-ов.
Если Widget имеет собственный State,
то его жизненный цикл подразумевает автоматическое обновление
при обновлении данных в \texttt{Inherited} Widget
при помощи метода \texttt{didChangedDepencies}.
Однако получить данные без автоматической актуализации данных
может любой виджет, у которого есть собственный \texttt{context}.
По нему Widget ищет объект класса наследника
\texttt{Inherited} Widget выше по дереву и возвращает объект этого класса,
если такой был найден.

\subsection{Статический метод of}

Основным требованием к хранилищу зависимостей является доступ
к нему из разных мест приложения.
Данное требование в \texttt{Inherited} Widget
и его наследниках реализуется посредством реализации
статического метода \texttt{of}.
Данный метод позволяет найти объект класса \texttt{Inherited} Widget
или его наследника в виджете дерева выше Widget-а,
в котором необходима зависимость.
В связи с тем, что объект класса \texttt{Inherited} Widget
или его наследника будет искаться в структуре Widget-ов,
то для его корректной логики требуется context Widget-а,
в котором необходима зависимость.
Пример вызова статического метода \texttt{of}
проиллюстрирован на рисунке~\ref{fig:inherited:of}.

\begin{image}
	\includegrph{Screenshot from 2024-10-25 13-44-11}
	\caption{Статический метод of}
	\label{fig:inherited:of}
\end{image}

\subsection{Метод dependOnInheritedWidgetOfExactType}

Для поиска объекта класса \texttt{Inherited} Widget
или его наследников фреймвокр предоставляет специальный метод
\texttt{dependOnInheritedWidgetOfExactType},
который возвращает объект класса, указанный в качестве дженерик типа в случае,
если объект был найден выше по дереву, или пустоту,
если объекта данного класса обнаружено не было.
Важно заметить, что этот метод используется для поиска объекта по дереву,
а значит его нужно вызывать у context, объекта класса \texttt{BuildContext}.
Именно для этого метод of обязательно должен иметь в качестве аргумента
\texttt{context} Widget-а, в котором нужна зависимость.
Полную реализацию метода \texttt{of}
можно увидеть на рисунке~\ref{fig:inherited:depend}.

\begin{image}
	\includegrph{Screenshot from 2024-10-25 13-51-14}
	\caption{Метод dependOnInheritedWidgetOfExactType}
	\label{fig:inherited:depend}
\end{image}

Важно заметить,
что метод \texttt{dependOnInheritedWidgetOfExactType}
вернет первый найденный объект класса, указанный в качестве дженерик класса.
То есть если в дереве Widget-ов находятся
два Widget-а одного \texttt{Inherited} Widget-а или его наследника,
то метод \texttt{dependOnInheritedWidgetOfExactType}
вернет именно ближайший Widget,
расположенный верх по дереву от зависимого от данных Widget-а.

\subsection{Метод updateShouldNotify}

Так как \texttt{Inherited} Widget
и его наследники используются для передачи набора данных в Widget-ы,
где эти данные требуются,
то они также должны иметь реализацию автоматического оповещения
этих виджетов при обновлении хранимых данных.
Этот механизм реализован при помощи метода \texttt{updateShouldNotify},
возвращающего логическое значение,
которое описывает необходимость слушателям этих данных обновить свои состояния,
так как данные обновились.
Разработчик вправе сам регулировать,
когда и при каких условиях слушатели обновят свои состояния,
именно регулируя логику возвращения данного логического значения,
сравнивая экземпляр класса до изменения и после изменения.
Фреймвокр самостоятельно сохраняет состояние объекта до изменения
и предоставляет его в данный метод через аргумент \texttt{oldWidget}.
Пример реализации логики метода \texttt{updateShouldNotify}
можно увидеть на рисунке~\ref{fig:inherited:update}.

\begin{image}
	\includegrph{Screenshot from 2024-10-25 14-00-40}
	\caption{Рализация метода updateShouldNotify}
	\label{fig:inherited:update}
\end{image}

\subsection{Использование Inherited Widget в приложении}

\subsubsection{Inherited Widget}

Для передачи данных через \texttt{InheritedWidget} \rref{fig:inherited:class}
в Flutter необходимо создать класс,
который будет предоставлять доступ к параметрам приложения.
Этот класс будет использоваться для передачи данных
(например, списка задач) между различными экранами,
такими как \texttt{TaskListScreen} и \texttt{FilterTasksScreen}.

\begin{image}
	\includegrph[scale=0.7]{Screenshot from 2024-10-25 12-25-06}
	\caption{Код InheritedWidget}
	\label{fig:inherited:class}
\end{image}

\subsubsection{TaskApp}

В классе \texttt{TaskApp} \rref{fig:inherited:app} были внесены изменения,
чтобы интегрировать использование \texttt{InheritedWidget}
для управления состоянием задач в приложении.

В методе \texttt{build} класса \texttt{\_TaskAppState}
используется \texttt{TaskInheritedWidget},
чтобы передать список задач и функцию
для добавления задач через его конструктор.
Это обеспечивает доступ к данным во всем приложении.

\begin{image}
	\includegrph[scale=0.65]{Screenshot from 2024-10-25 12-24-46}
	\caption{Код TaskApp}
	\label{fig:inherited:app}
\end{image}

\subsubsection{TaskListScreen}

Изменения в \texttt{TaskListScreen} --- использует
\texttt{TaskInheritedWidget.of(context)?.tasks}
для получения задач и функцию \texttt{addTask} для добавления новой задачи
\rref{fig:inherited:list}.

\begin{image}
	\includegrph[scale=0.49]{Screenshot from 2024-10-25 12-26-26}
	\caption{Код TaskListScreen}
	\label{fig:inherited:list}
\end{image}

На рисунке~\ref{fig:inherited:list:show} показана страница со списком задач.

\begin{image}
	\includegrph[scale=0.3]{Screenshot from 2024-10-25 12-40-26}
	\caption{Страница со списком задач}
	\label{fig:inherited:list:show}
\end{image}

\subsubsection{FilterTasksScreen}

Чтобы добавить фильтрацию задач
на экране FilterTasksScreen \rref{fig:inherited:filter},
нужно расширить функционал, позволяя пользователю выбирать,
какие задачи отображать (например, по строке поиска или другим критериям).
Для этого \texttt{FilterTasksScreen} будет работать с доступом к списку задач
через \texttt{TaskInheritedWidget}
и добавит возможность фильтровать их в реальном времени.

\begin{image}
	\includegrph{Screenshot from 2024-10-25 13-43-23}
	\caption{Код FilterTasksScreen}
	\label{fig:inherited:filter}
\end{image}

На экране \texttt{FilterTasksScreen} добавлено поле
для поиска задач с помощью \texttt{TextField}.
Пользователь может вводить текст, чтобы фильтровать список задач.\par
Список задач фильтруется на основе \texttt{\_searchText},
который обновляется при каждом изменении текста в поле поиска.\par
Фильтр применяется в методе \texttt{build} к списку \texttt{tasks},
и отображается только отфильтрованный список \texttt{filteredTasks}.

На рисунке~\ref{fig:inherited:filter:show} показана страница фильтрации.

\begin{image}
	\includegrph{Screenshot from 2024-10-25 12-40-18}
	\caption{Страница фильтрации}
	\label{fig:inherited:filter:show}
\end{image}

\section{DI контейнер GetIt}

\texttt{GetIt} --- это \texttt{DI} контейнер,
который распространяется в одноименном пакете.
Это простой \texttt{DI} контейнер для проектов Dart
и Flutter с некоторыми дополнительными преимуществами.
Его можно использовать вместо \texttt{Inherited} Widget для доступа к объектам,
например, из вашего пользовательского интерфейса.
Главное отличие \texttt{GetIt} от \texttt{Inherited} Widget заключается в том,
что для доступа к данным не требуется context Widget-
, от которого происходит обращение.
Очень часто в ходе разработки требуется получить данные в месте,
где нет доступа к context
и в таких моментах \texttt{Inherited} Widget становится очень неудобен.
\texttt{GetIt} отлично закрывает эту проблему,
так как он не является частью фреймворка Flutter
и не зависит от его внутренней реализации.
\texttt{GetIt} строится на объектном паттерне Singleton,
который позволяет получить доступ к единому экземпляру классу
в любой точке приложения без необходимости
как-то добавлять его в дерево Widget-ов.

\subsection{Подключение в проект}

Для подключения пакета \texttt{get\_it} в проект требуется
в файл pubspec.yaml добавить пакет в dependency.
Пример добавления изображен на рисунке~\ref{fig:getit:dev}.

\begin{image}
	\includegrph{Screenshot from 2024-10-25 12-44-38}
	\caption{Добавления пакета get\_it в приложение}
	\label{fig:getit:dev}
\end{image}

\subsection{Доступ к DI контейнеру}

Для того, чтобы получить доступ к \texttt{DI} контейнеру \texttt{GetIt},
необходимо обратиться к его статичному полю \texttt{instance}
или полю \texttt{I}.
Данное поле хранит в себе единый объект \texttt{DI} контейнера,
используемого в приложении. Получив доступ к объекту \texttt{GetIt},
разработчику открывается полный доступ к контейнеру и лежащих в нем образах.
Способ обращения к контейнеру
можно рассмотреть на рисунке~\ref{fig:getit:instance}.

\begin{image}
	\includegrph{Screenshot from 2024-10-25 14-08-17}
	\caption{Добавления пакета get\_it в приложение}
	\label{fig:getit:instance}
\end{image}

\subsection{Регистрация объектов}

Для того, чтобы некий образ класса появился в контейнере
его необходимо сначала там зарегистрировать.
В контейнере можно зарегистрировать образ как уже готовый объект
или как алгоритм его создания.
Для регистрации объектов в контейнере \texttt{GetIt}
есть специализированный метод \texttt{registerSingleton<T>}.
Указывая в качестве значения его аргумента объект дженерик класса,
он регистрируется в контейнере и будет возвращаться каждый раз,
когда из контейнера будут запрашивать
образ данного класса \rref{fig:getit:registration}.

\begin{image}
	\includegrph{Screenshot from 2024-10-25 14-10-27}
	\caption{Регистрация объектов в контейнер GetIt}
	\label{fig:getit:registration}
\end{image}

Иногда требуется хранить более одного объекта одного
и того же класса в контейнере,
тогда при регистрации объекта помимо самого объекта передают
его идентификационное название,
по которому в дальнейшем можно будет получить именно
этот образ запрашиваемого класса \rref{fig:getit:registration:name}.

\begin{image}
	\includegrph{Screenshot from 2024-10-25 14-11-36}
	\caption{Указание идентификационное имени объекта в контейнере GetIt}
	\label{fig:getit:registration:name}
\end{image}

При стандартной настройке контейнера если попытаться зарегистрировать
объект уже имеющегося в контейнере класса, то будет получена runtime ошибка.
Для решения этой проблемы можно либо задать
каждому экземпляру собственное идентификационное название
или включить настройку переписи объектов
внутри контейнера \rref{fig:getit:registration:reassing}.
Она позволяет заменять в контейнере объект одного класса
без ручной очистки образа класса.

\begin{image}
	\includegrph{Screenshot from 2024-10-25 14-13-32}
	\caption{Включение настройки переписи объектов в контейнере GetIt}
	\label{fig:getit:registration:reassing}
\end{image}

Так как регистрация объектов в контейнере --- это очень затратный процесс
по памяти, \texttt{GetIt} имеет возможность
произвести <<\textit{отложенную}>> регистрацию.
Для этого необходимо вызвать метод \texttt{registerLazySingleton<T>},
в аргументы которого передается функция,
возвращающая требуемых объект дженерик класса.
Это позволяет заложить в контейнер не сам объект, а функцию его получения,
что уменьшает потребляемую память
и позволяет заложить в контейнер объект на момент первого обращения
к образу класса.

\subsection{Регистрация фабрик создания образов}

Не во всех случаях требуется хранить объект класса в контейнере.
Существуют подходы, что классы описывают логику работы алгоритмов,
однако не хранят в своих объектах никаких данных.
Так как хранение объектов в контейнере очень затратный процесс по памяти,
то хранить такого рода классы нет никакой необходимости.
Вместо этого, контейнер позволяет сохранять в качестве образа не сам объект,
а логику получение образа класса.
Такие функции с логикой создания объектов
требуемых классов называются фабриками.
Для регистрации фабрики в качестве образа класса
в контейнере используется метод \texttt{registerFactory<T>}
\rref{fig:getit:registration:factory}.
При обращении в контейнер за образом класса будет каждый раз получено
новый объект данного класса, построенный по логике, заложенной в фабрике.
Так же при регистрации фабрики ей можно задать идентификационное название,
что бы отличать различные фабрики, возвращающие образ одного и того же класса.

\begin{image}
	\includegrph{Screenshot from 2024-10-25 14-21-19}
	\caption{Регистрация фабрики в контейнер GetIt}
	\label{fig:getit:registration:factory}
\end{image}

\subsection{Получение образа из контейнера}

В ситуации, когда требуется получить образ из контейнера,
требуется использовать специализированный метод \texttt{get<T>},
который вернет образ запрашиваемого дженерик класса
\rref{fig:getit:instance:get}.
Если образ не был ранее зарегистрирован в контейнере,
то данный метод вернет \texttt{runtime} ошибку,
что такого образа в контейнере нет.

\begin{image}
	\includegrph{Screenshot from 2024-10-25 14-24-39}
	\caption{Получение образа из контейнера GetIt}
	\label{fig:getit:instance:get}
\end{image}

\subsection{Проверка на наличие образа в контейнере}

Для того, чтобы быть уверенным в получении образа искомого класса
из контейнера, \texttt{GetIt} позволяет проверять,
зарегистрирован ли образ класса до того, как его получать.
Для проверки регистрации образа используется метод\texttt{isRegistered<T>} 
\rref{fig:getit:isregistered}.
Данный метод возвращает логическое значение,
обозначающее регистрацию образа дженерик класса в контейнере.
Так же, как и в методах регистрации,
при помощи дополнительного аргумента можно указать идентификационное название
для проверки регистрации конкретного образа дженерик класса.

\begin{image}
	\includegrph{Screenshot from 2024-10-25 14-30-19}
	\caption{Проверка на наличие образа в контейнере GetIt}
	\label{fig:getit:isregistered}
\end{image}

\subsection{Использование DI контейнер GetIt в приложении}

\subsubsection{TaskService}

Сначала создадим сервис \texttt{TaskService},
который будет хранить и управлять списком задач \rref{fig:getit:class}.

\begin{image}
	\includegrph{Screenshot from 2024-10-25 13-23-57}
	\caption{Код TaskService}
	\label{fig:getit:class}
\end{image}

\texttt{TaskService} содержит методы для добавления и фильтрации задач.
Метод \texttt{filterTasks} используется для получения отфильтрованных задач.

\subsubsection{Инициализация контейнера GetIt}

Теперь модифицируем главный файл приложения,
чтобы инициализировать GetIt \rref{fig:getit:app}.

\begin{image}
	\includegrph{Screenshot from 2024-10-25 13-23-43}
	\caption{Инициализация контейнера GetIt}
	\label{fig:getit:app}
\end{image}

\subsubsection{TaskListScreen}

Изменим класс \texttt{TaskListScreen} \rref{fig:getit:list}
для использования контейнера \texttt{GetIt}.
Добавлен импорт для \texttt{GetIt} и \texttt{TaskService},
чтобы использовать контейнер зависимостей.
Вместо управления задачами внутри состояния экрана,
теперь получаем доступ к \texttt{TaskService}
через \texttt{GetIt.I<TaskService>()}.
Метод \texttt{\_addTask} использует \texttt{TaskService}
для добавления задач в список.
В \texttt{build} методе получаем актуальный список задач
из \texttt{TaskService}, чтобы отобразить их в интерфейсе.

\begin{image}
	\includegrph{Screenshot from 2024-10-25 13-24-27}
	\caption{Код TaskListScreen}
	\label{fig:getit:list}
\end{image}

Отображение страницы не изменилось и было показано
на рисунке~\ref{fig:inherited:list:show}.

\subsubsection{FilterTasksScreen}

Изменим класс \texttt{TaskListScreen} \rref{fig:getit:filter}
для использования контейнера \texttt{GetIt}.
Так же, как и в \texttt{TaskListScreen},
добавлены импорты для работы с сервисом.
Получение экземпляра \texttt{TaskService}
для работы с задачами через контейнер зависимостей.
Используется метод \texttt{filterTasks} из \texttt{TaskService}
для получения отфильтрованных задач на основе введенного текста.
В методе \texttt{onChanged} теперь обновляется состояние
\texttt{\_searchText}, что позволяет фильтровать задачи в реальном времени.

\begin{image}
	\includegrph{Screenshot from 2024-10-25 13-24-48}
	\caption{Код FilterTasksScreen}
	\label{fig:getit:filter}
\end{image}

Отображение страницы не изменилось и было показано
на рисунке~\ref{fig:inherited:filter:show}.

\clearpage

\section*{ВЫВОД}
\addcontentsline{toc}{section}{ВЫВОД}

В результате выполнения практической работы получили практический опыт
в управлении зависимостями в приложениях Flutter.
Изучение и реализация двух методов передачи параметров
(Inherited Widget и DI контейнер GetIt)
позволили глубже понять архитектурные паттерны и подходы,
используемые в современных приложениях.
Это знание будет полезно в дальнейшей практике разработки
и поддержке сложных приложений, обеспечивая более легкое
и эффективное управление состоянием.

