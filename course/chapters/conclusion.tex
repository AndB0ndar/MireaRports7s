\chapter*{Заключение}
\addcontentsline{toc}{chapter}{Заключение}

В ходе выполнения данной выпускной квалификационной работы
была разработана программа-конвертер DRC правил,
предназначенная для автоматизации процесса перевода проектных правил
между различными CAD-системами.
Исследование актуальности темы подтвердило,
что современная электронная промышленность требует
эффективных инструментов для оптимизации проектирования,
что и стало основной целью работы.

Реализация программы позволила решить поставленные задачи,
включая анализ существующих методов проверки проектных правил,
изучение форматов DRC правил и разработку архитектуры программного обеспечения.
Программа была протестирована на различных сценариях,
что продемонстрировало её высокую точность
и эффективность в преобразовании правил.
Результаты тестирования подтвердили,
что использование конвертера позволяет значительно сократить время,
необходимое для подготовки проектной документации,
а также уменьшить риск возникновения ошибок,
связанных с ручным вводом и интерпретацией правил.

Таким образом, разработанное программное решение представляет
собой значительный вклад в автоматизацию проектирования электронных устройств,
что может быть полезно как для научных исследований,
так и для практического применения в промышленности.
В дальнейшем возможна доработка программы,
включая расширение функциональности и поддержку новых форматов DRC правил,
что обеспечит ещё большую универсальность
и применение инструмента в различных областях проектирования.

В заключение, работа подтверждает важность интеграции современных технологий
и подходов в процесс разработки,
что в конечном итоге способствует повышению качества
и надежности проектируемых электронных компонентов.

