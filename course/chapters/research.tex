\chapter{Исследовательский раздел}

\section{Моделирование бизнес-процессов}

\subsection{Входные и выходные данные}

Входные данные:

\begin{itemize}
	\item Путь до файла правил DRC,
		созданный для коммерческих инструментов в формате SVRF Calibre.
	\item Путь до файла конфигурации в форматах: INI, JSON и YAML.
\end{itemize}

Выходные данные:

\begin{itemize}
	\item Правила DRC, преобразованный для использования
		в открытых инструментах в формате Klayout.
	\item Отчет о результатах трансляции.
\end{itemize}

\subsection{Модель БП в вариации TO-BE}

Модель бизнес-процессов системы конвертации DRC
в формате TO-BE демонстрирует целевую архитектуру процесса.
На верхнем уровне на вход поступают пути до DRC-файла и файла конфигурации.
На выходе формируются конвертированные правила
и отчет об трансляции \rref{fig:idef0}.

\begin{image}
	\includegrph[scale=0.2]{2024-09-29_16-46-34}
	\caption{Контекстная диаграмма}
	\label{fig:idef0}
\end{image}

Процесс конвертирования DRC декомпозируется в 5 подпроцессов:
чтение конфигурационного файла, парсинг, предобработка, конвертация
и генерация отчета \rref{fig:idef0:a0}.

\begin{image}
	\includegrph[scale=0.25]{2024-09-29_16-46-47}
	\caption{Декомпозиция контекстной диаграммы}
	\label{fig:idef0:a0}
\end{image}

Процесс предобработки декомпозируется в 3 подпроцесса:
конкатенация многострочных команд, развертывание макросов
и упорядочивание \rref{fig:idef0:a3}.

\begin{image}
	\includegrph[scale=0.25]{2024-09-29_16-16-02}
	\caption{Декомпозиция процесса предобработки}
	\label{fig:idef0:a3}
\end{image}

Процесс конвертации декомпозируется в 5 подпроцессов:
конвертация декларативных команд, конвертация выражений,
конвертация геометрических проверок, конвертация правил
и объединение кода \rref{fig:idef0:a4}.

\begin{image}
	\includegrph[scale=0.25]{2024-09-29_16-15-22}
	\caption{Декомпозиция процесса конвертации}
	\label{fig:idef0:a4}
\end{image}

Целевая модель TO-BE определяет четкую последовательность обработки данных,
что повышает точность конвертации
и упрощает дальнейшую модификацию или масштабирование системы.

\section{Границы проекта}

В контексте программы-конвертора DRC,
автоматизация включает следующие процессы:

\begin{itemize}
	\item Конвертация правил из проприетарного формата в формат,
		поддерживаемый open-source инструментами
		--- автоматическое преобразование данных
		и настроек для использования в открытых системах;
	\item Генерация отчета о трансляции --- автоматическое создание отчетов
		о процессе преобразования и возникающих ошибках;
	\item Развертывание макросов --- автоматизация внедрения макросов
		для выполнения повторяющихся операций;
	\item Упорядочивание кода --- автоматическое форматирование
		и стандартизация кода для улучшения его структуры и читаемости.
\end{itemize}

Однако проверка корректности конвертированных файлов
в open-source инструментах
и валидация данных на уровне конечных инструментов проектирования остаются
за пределами автоматизации, требуя пользовательской проверки и анализа.

\section{Анализ систем-аналогов}

Существующие коммерческие решения, такие как Calibre и Assura,
не подходят для использования в открытых проектах ввиду их высокой стоимости
и закрытых форматов данных.
С другой стороны есть открытое решение, предоставленное приложением Klayout,
но он не поддерживат коммерческие форматы файлов проверок,
используемые большинством фабрик-производителей.
Это создаёт необходимость разработки собственной системы,
которая обеспечит автоматическую конвертацию правил DRC
из коммерческих форматов в форматы, совместимые с открытыми инструментами,
такими как Klayout,
поскольку прямых аналогов в открытом доступе не существует.

\clearpage  % XXX: Fine-tuning

\section{Техническое задание}

\subsection{Функциональные, нефункциональные и пользовательские требования}

\subsubsection{Функциональные требования}

Это требования, которые описывают, что система должна делать.
Они определяют конкретные функции,
которые система должна выполнять для решения задач пользователя.

Перечислим их:

\begin{itemize}
    \item Программа должна принимать на вход файлы
		формата SVRF (Standard Verification Rule Format)
		для DRC (Design Rule Check) правил,
		используемый коммерческими инструментами физической проверки
		Calibre;
	\item Программа должна читать файл конфигурации;
    \item Программа должна поддерживать преобразование правил из формата
		Calibre в формат поддерживаемый открытым инструментом Klayout;
    \item Программа должна обрабатывать макросы перед конвертацией;
    \item Программа должна обрабатывать последовательность 
		инициализации переменных перед конвертацией.
	\item Программа должна иметь встроенную систему логирования,
		которая будет записывать все действия в процессе конвертации;
	\item Программа должна создавать файл с результом конвертации;
	\item Программа должна создавать отчет о результатах конвертации;
\end{itemize}

\subsubsection{Пользовательские требования}

Эти требования формируются с точки зрения пользователя и описывают задачи,
которые пользователь должен выполнить с помощью системы.
Они часто менее технические и более ориентированы на взаимодействие.

Перечислим их:

\begin{itemize}
    \item Пользователь должен иметь возможность выбрать исходный файл правил;
    \item Пользователь должен иметь возможность
		выбрать исходный файл конфигурации;
    \item Пользователь должен иметь возможность настроить ход конвертации
		через конфигурационный файл;
    \item Интерфейс программы должен быть реализован
		в виде консольного приложения;
    \item Программа должна предоставлять отчет о результатах конвертации;
    \item Пользователь должен иметь возможность выбрать
		целевой форматы для отчета о результатах конвертации.
\end{itemize}

\subsubsection{Нефункциональные требования}

Эти требования касаются характеристик системы, которые определяют,
как она выполняет свои функции,
но не связаны непосредственно с поведением системы.
Они включают такие аспекты, как производительность, надежность, безопасность,
удобство использования и масштабируемость.

Перечислим их:

\begin{itemize}
	\item Конкретное время зависит от объема файла,
		приблизительно должно составлять не больше 5 минут на 10Мб-ый файл.
    \item Программа должна быть реализован на языке Python 3.9
		для совместимости с основными файловыми системами,
		такими как Windows и Linux;
    \item Программа не должна изменять исходные файлы правил;
    \item Полученные конвертацией правила должены выдавать
		такой же результат, как исходные в коммерческих приложениях.
\end{itemize}

\subsection{Требования к программному обеспечению}

\subsubsection{Состав ПО}

Система автоматической конвертации правил DRC
включает несколько ключевых модулей:

\begin{itemize}
	\item Пользовательский интерфейс (UI):
		Консольное приложение для взаимодействия с пользователем.
		Позволяет вводить параметры, запускать конвертацию
		и отображать логи работы программы;
	\item Модуль чтения конфигурации:
		Загружает и интерпретирует конфигурационные файлы,
		содержащие настройки системы,
		такие как поддерживаемые форматы и пути к данным;
	\item Модуль парсинга:
		Читает и анализирует исходные данные,
		извлекая необходимые правила DRC из коммерческих форматов
		и преобразуя их в промежуточный формат;
	\item Модуль предобработки:
		Выполняет очистку и подготовку данных для конвертации,
		включая фильтрацию, упорядочивание и развертывание макросов;
	\item Модуль конвертации:
		Преобразует правила DRC из коммерческих форматов в открытые,
		обеспечивая совместимость с инструментами, такими как Klayout;
	\item Модуль генерации отчетов:
		Создает отчеты о процессе конвертации,
		включая информацию об ошибках и успешности,
		и экспортирует их в удобные форматы.
\end{itemize}

\subsubsection{Выбор ПО}

Для разработки системы автоматической конвертации правил DRC
выбран язык программирования Python (версии 3.9 и выше).
Для реализации различных функциональных частей системы
будут использованы следующие библиотеки Python.

Библиотеки, входящие в стандартную библиотеку Python:

\begin{itemize}
	\item os --- для взаимодействия с операционной системой,
		работы с файлами и каталогами.
	\item sys --- для работы с системными параметрами
		и аргументами командной строки.
	\item configparse --- для чтения
		и обработки конфигурационных файлов в формате INI.
	\item json --- для работы с данными в формате JSON,
		которые будут использоваться для хранения конфигураций.
\end{itemize}
  
Библиотеки для работы с конфигурациями также дополняются
PyYAML (версии 6.0.2 и выше) для чтения с YAML-файлами.

Библиотеки для парсинга и обработки данных:

\begin{itemize}
	\item Lark (версии 1.2.2 и выше) --- для создания парсеров,
		обработки и трансформации данных в нужные форматы.
	\item NetworkX (версии 3.3 и выше) --- для работы с графами
		и моделями данных,
		что понадобиться при анализе и предобработке правил DRC.
\end{itemize}

Для разработки интерфейса будет использована библиотека
click (версии 8.1.7 и выше), позволяющая создавать удобный
консольный интерфейс и обрабатывать командную строку.

\subsubsection{Разрабатываемое ПО}

В рамках разработки системы автоматической конвертации правил DRC
будут созданы уникальные алгоритмы для обработки и преобразования DRC правил,
а также для многопоточного парсинга исходного файла правил
и упорядочивания последовательности операций.
Они направлены направленные на эффективную обработку
и преобразование правил, а также на оптимизацию обработки данных.

\subsubsection{Покупные средства}

Для реализации ряда задач, требующих уже существующих решений,
будут использованы сторонние библиотеки с открытым исходным кодом.
Основные библиотеки, которые будут использоваться в проекте:

\begin{itemize}
	\item Lark --- это библиотека для построения парсеров,
		которая поможет в разработке компонентов системы,
		отвечающих за обработку исходных файлов правил DRC.
		Она позволит легко и эффективно создать грамматики для парсинга данных
		и преобразования их в структуры,
		подходящие для дальнейшей обработки и конвертации.
	\item NetworkX будет использоваться для работы
		с графовыми структурами данных,
		которые возникнут при предобработке правил DRC.
		Например, NetworkX предоставит удобные методы для моделирования,
		анализа и манипуляции с DAG.
\end{itemize}

Использование этих библиотек позволит ускорить процесс разработки
и обеспечит стабильность и надежность системы,
избавляя от необходимости писать все компоненты с нуля.

\subsection{Требования к техническому обеспечению}

Необходимые технические средства включают
наличие персонального компьютеров или сервера,
на котором будет работать программа,
под управлением одной из операционных систем Linux, Windows или MacOS.

Для запуска системы потребуется стандартное оборудование:

\begin{itemize}
	\item Процессор, желательно с поддержкой многопоточной обработки данных.
	\item Необходим объём оперативной памяти,
		достаточный для обработки больших объемов данных
		(рекомендуется от 18 Гб и выше).
	\item Наличие жесткого диска или SSD объема, достаточного
		для хранения исходных файлов, временных данных и отчетов.
		(рекомендуется от 64 Гб и выше).
\end{itemize}

%Функциональные характеристики.
%Обеспечение устойчивой работы при больших объемах данных.

\subsection{Требования к информационному обеспечению}

\subsubsection{Состав и структура данных}

ПК DRC должна работать с несколькими типами данных.
Это включают исходные файлы DRC правил,
которые содержат правила для проверки дизайна,
конфигурационные файлы, содержащие настройки
для корректного выполнения конвертации, и логи работы системы,
которые фиксируют выполненные операции и возникающие ошибки.

\subsubsection{Организация данных}

Информация будет храниться в виде текстовых файлов в файловой системе
на локальном устройстве пользователя.
Конфигурационные файлы будут содержать
все необходимые настройки для корректного выполнения конвертации,
включая параметры форматов, пути к исходным и выходным данным.

\subsubsection{Информационный обмен}

Взаимодействие между модулями системы (парсинг, предобработка, конвертация)
должно быть организовано так, чтобы обеспечить непрерывность
и правильную последовательность передачи данных,
что необходимо для корректной конвертации исходных правил.

Взаимодействие между модулями системы (парсинг, предобработка, конвертация)
должно быть организовано таким образом, чтобы обеспечить непрерывность
и правильную последовательность передачи данных.
Это необходимо
для корректного выполнения процесса конвертации исходных правил.
Каждый модуль должен получать
и передавать данные в стандартизированном формате,
что минимизирует возможность ошибок
и улучшает общую производительность системы.

\subsubsection{Информационная совместимость}

Система должна работать со стандартными форматами DRC правил,
используемыми в таких системах, как Calibre для входных данных
и KLayout для выходных данных.
Это обеспечит совместимость между различными инструментами и системами.

\subsubsection{Классификаторы и справочники}

Для правильной интерпретации
и преобразования правил будут использоваться справочники команд
и макросов DRC правил, а также классификаторы для различных типов проверок.
Это позволит систематизировать данные
и обеспечить точную интерпретацию конвертируемых правил,
что является необходимым для точности выполнения конвертации.

\subsubsection{Системы управления базами данных}

В данной версии системы использование
системы управления базами данных (СУБД) не требуется,
поскольку данные будут храниться в файловой системе
и не нуждаются в сложной организации или реляционных связях.
Простое хранение и доступ к данным через файловую систему
будет достаточно для выполнения всех операций.

\subsubsection{Представление данных}

Данные должны быть представлены в текстовом формате,
поддерживаемом основными операционными системами,
такими как Linux и Windows.
Это обеспечит доступность данных для просмотра
и редактирования в случае необходимости,
а также совместимость с основными инструментами
для работы с текстовыми файлами.

\subsubsection{Контроль и восстановление данных}

Система должна вести логи работы,
в которых фиксируются все выполненные операции,
а также возникающие ошибки и предупреждения.
Логи помогут провести анализ и устранение ошибок.
Автоматическое восстановление данных не требуется,
так как в случае сбоя достаточно будет перезапустить процесс
с исходными данными для повторного выполнения конвертации.

\subsection{Требования к математическому обеспечению}

Для реализации математического обеспечения
системы автоматической конвертации правил DRC требуется использование
различных алгоритмов и методов, которые обеспечат эффективную обработку,
анализ и преобразование данных.

\subsubsection{Алгоритмы для построения DAG}

Ориентированные ациклические графы (DAG) используются
для моделирования зависимостей между различными элементами DRC скрипта.
В контексте DRC скрипта, каждый элемент может зависеть от других,
что можно выразить через вершины и рёбра графа.

Математическое представление DAG выглядит следующим образом:

\begin{equation}
G = (V, E)
\end{equation}

где: \( V \) --- множество вершин, представляющих элементы DRC скрипта,
\( E \) --- множество рёбер,
где каждое ребро \( (u, v) \) из вершины \( u \) в вершину \( v \)
обозначает зависимость между двумя элементами.

Для каждого элемента \( v \in V \)
определяется его зависимость от других элементов.
Если \( u \) зависит от \( v \), то существует ребро \( (v, u) \) в графе.
Этот граф должен быть ориентированным и ациклическим,
что означает отсутствие циклов.

\subsubsection{Алгоритмы для топологической сортировки DAG}

Топологическая сортировка ориентированного ациклического графа (DAG)
позволяет упорядочить вершины таким образом, что для каждого ребра \((u, v)\)
вершина \( u \) появляется перед вершиной \( v \).
Важно, чтобы топологическая сортировка соблюдала
все зависимости между элементами.

Топологическая сортировка графа
\( G = (V, E) \) --- это отображение \( \tau: V \to \{1, 2, \dots, |V|\} \),
где для всех рёбер \( (u, v) \in E \) выполняется условие:

\begin{equation}
\tau(u) < \tau(v)
\end{equation}

Это условие гарантирует,
что вершина \( u \) будет обработана до вершины \( v \),
что важно для соблюдения всех зависимостей между элементами в DRC скрипте.

\subsubsection{Алгоритмы для парсинга}

Для парсинга исходных данных DRC правил необходимо применить несколько методов,
включая \textbf{Earley Parer}, \textbf{абстрактное синтаксическое дерево (AST)}
и \textbf{токенизацию}.

\textbf{Earley Parser} является эффективным алгоритмом
для парсинга контекстно--свободных грамматик (CFG).
Он позволяет строить абстрактные синтаксические деревья для входных данных,
даже если грамматика амфиболическая (неоднозначная).
Алгоритм работает с состояниями, которые можно формализовать следующим образом:

\textbf{Абстрактное синтаксическое дерево (AST)}
строится на основе грамматики входных данных.
Для каждого правила синтаксической грамматики генерируется узел в дереве.
Строительство дерева происходит с использованием рекурсии,
где каждый узел дерева соответствует определённой грамматической конструкции.

\textbf{Токенизация} используется
для разбиения исходных данных на минимальные единицы --- токены,
которые будут обработаны на следующем этапе парсинга.
Токены могут быть представлены как последовательность символов,
которая соответствует лексемам.

\subsubsection{Алгоритмы преобразования DRC правил}

Алгоритмы преобразования отвечают за трансформацию DRC правил
из одного формата в другой, сохраняя логику и структуру правил.
Для этого используется несколько техник, включая парсинг,
проверку на корректность и генерацию выходных данных в целевом формате.

Преобразование можно описать как функцию \( T \),
которая преобразует исходные правила \( R_{\text{in}} \)
в целевые правила \( R_{\text{out}} \):

\begin{equation}
T: R_{\text{in}} \to R_{\text{out}}
\end{equation}

где:
\( R_{\text{in}} \) --- исходный набор правил в формате,
поддерживаемом системой Calibre,
\( R_{\text{out}} \) --- преобразованный набор правил в формате,
совместимом с системой KLayout.

Алгоритмы преобразования должны учитывать синтаксические
и семантические особенности каждого формата,
чтобы гарантировать правильную трансформацию данных без потери информации.

\subsection{Требования к документации на программное решение}

\subsubsection{Перечень подлежащих разработке документов}

Для создания программы-конвертера DRC
правил необходимо подготовить следующие документы:

\begin{itemize}
	\item Техническое задание (ТЗ):
		Описание целей, задач, функциональных
		и нефункциональных требований к программе,
		этапов разработки и сроков, а также ресурсов и бюджета;
	\item Проектная документация:
		Включает описание архитектуры программы, спецификаций,
		схем и диаграмм, определяющих логику работы
		и взаимодействие компонентов;
	\item Программа и методика испытаний:
		План и методика тестирования программы,
		включая функциональные и производительные испытания,
		критерии приемки и результативности;
	\item Руководства пользователя и администраторов:
		Инструкции для конечных пользователей по использованию программы,
		а также для администраторов по установке, настройке и поддержке;
	\item Отчеты о внедрении и эксплуатации системы:
		Документация по процессу внедрения программы,
		возникшим проблемам и рекомендациям,
		а также отчет об ее работе в реальных условиях.
\end{itemize}

Эти документы обеспечат полный цикл разработки,
внедрения и поддержки программы.

\subsubsection{Вид представления и количество документов}

\textbf{Формат}: Все документы предоставляются в электронном формате (PDF),
удобном для хранения и печати, а также в текстовом формате (Word или аналог),
чтобы обеспечить возможность внесения правок.

\textbf{Количество экземпляров}:
по одному электронному экземпляру каждой документации предоставляется заказчику
и хранится у разработчика для внутреннего использования.

При необходимости, предоставляются печатные версии:

\begin{itemize}
	\item один экземпляр технического задания для заказчика;
	\item один экземпляр руководства пользователя для заказчика;
	\item один экземпляр акта приемки для заказчика.
\end{itemize}

\subsubsection{Требования по использованию ЕСКД и ЕСПД}

Документация должна соответствовать требованиям
Единой системы конструкторской документации (ЕСКД)
и Единой системы программной документации (ЕСПД):

\begin{itemize}
	\item ЕСКД: Применяется для проектной документации,
		где должны быть соблюдены стандарты по оформлению, обозначениям
		и структуре документа.
		Это включает требования к оформлению чертежей,
		блок-схем и структурных схем модулей системы.
	\item ЕСПД: Используется для программной и эксплуатационной документации.
		Все описания, инструкции и руководства должны быть оформлены
		в соответствии с требованиями ЕСПД, включая единообразие терминов,
		нумерацию разделов и описание функциональных возможностей.
\end{itemize}

Соблюдение стандартов ЕСКД и ЕСПД необходимо
для обеспечения удобства использования и единообразия документации,
что облегчает эксплуатацию, обслуживание и сопровождение АС.

\subsection{Требования к надежности программного решения}

Программа-конвертер DRC должна обеспечивать стабильную
и корректную работу при каждом запуске по запросу пользователя.
Вероятность аварийного завершения работы системы должна составлять
не более одного случая на 1000 запусков.

В случае возникновения ошибок или сбоев система
должна завершить выполнение с выводом соответствующего сообщения об ошибке.
Все ошибки должны быть логированы для дальнейшего анализа и устранения.

\subsection{Требования к безопасности программного решения}

Разрабатываемая система должна обеспечивать безопасность данных,
при этом исходные файлы правил DRC не должны сохраняться
в системе после выполнения конвертации.
Это необходимо для защиты DRC правил,
связанных с PDK, и предотвращения их несанкционированного хранения или утечек.

\section{Программа и методика испытаний}

\subsection{Объект испытаний}

\subsubsection{Наименование системы}

Программа-конвертер DRC правил для использования
в открытых инструментах проектирования цифровых микросхем.

\subsubsection{Область применения системы}

Программный продукт, представляющий собой конвертер DRC правил,
предназначен для преобразования правил проектирования
из проприетарных форматов (SVRF Calibre) в форматы, совместимые
с открытыми инструментами проектирования цифровых микросхем (KLayout).
Основная цель --- обеспечить удобство
и доступность работы с DRC правилами для разработчиков и инженеров,
работающих в области проектирования интегральных схем.\par
Основная функция заключается в преобразования DRC правил
из проприетарных форматах в форматы,
поддерживаемые открытыми инструментами.

Области применения:

\begin{itemize}
	\item Использование в процессе проектирования интегральных схем,
		где необходимо соблюдать DRC правила
		для корректного функционирования и производства.
	\item Интеграция в существующие инструменты проектирования
		для автоматизации работы с DRC правилами и улучшения совместимости.
	\item Применение в учебных заведениях
		и исследовательских лабораториях для обучения студентов
		и специалистов работе с правилами проектирования.
	\item Помощь компаниям, желающим перейти
		на открытые инструменты проектирования,
		в конвертации существующих данных для обеспечения совместимости.
\end{itemize}

\subsubsection{Условное обозначение системы}

Условное обозначение Системы --- ПК DRC.

\subsection{Цель испытаний}

Целью проводимых по настоящей программе и методике испытаний ПК DRC
является определение функциональной работоспособности системы
на этапе проведения испытаний. 

Программа испытаний должна удостоверить работоспособность ПК DRC
в соответствии с функциональным предназначением.

\subsection{Общие положения}

\subsubsection{Перечень руководящих документов,
	на основании которых проводятся испытания}

Приёмочные испытания ПК DRC проводятся на основании следующих документов:

\begin{itemize}
	\item Утверждённое Техническое задание на разработку ПК DRC;
	\item Настоящая Программа и методика приёмочных испытаний;
\end{itemize}

\subsubsection{Место и продолжительность испытаний}

Место проведения испытаний --- площадка Заказчика.
Продолжительность испытаний устанавливается Приказом Заказчика
о составе приёмочной комиссии и проведении приёмочных испытаний.

\subsubsection{Организации, участвующие в испытаниях}

В приёмочных испытаниях участвуют представители следующих организаций:

\begin{itemize}
	\item ООО "Мальт Систем" (Заказчик);
	\item Бондарь Андрей Ренатович (Исполнитель).
\end{itemize}

Конкретный перечень лиц, ответственных за проведение испытаний системы,
определяется Заказчиком.

\subsubsection{Перечень предъявляемых на испытания документов}

Для проведения испытаний Исполнителем предъявляются следующие документы:

\begin{itemize}
	%\item Государственный контракт №2124-05-08 от 12.11.2008;
	\item Техническое задание на создание ПК DRC;
	\item Технический проект ПК DRC.
	\item Программа и методика испытаний на ПК DRC;
\end{itemize}

\subsection{Объём испытаний}

\subsubsection{Перечень этапов испытаний и проверок}

В процессе проведения приёмочных испытаний должны
быть протестированы следующие подсистемы ПК DRC:

\begin{itemize}
	\item Подсистема обработки пользовательского ввода;
	\item Подсистема парсинга исходного DRC файла;
	\item Подсистема предобработки;
	\item Подсистема конвертации;
	\item Подсистема генерации отчета.
\end{itemize}

Все подсистемы испытываются одновременно
на корректность взаимодействия подсистем,
влияние подсистем друг на друга, т.е. испытания проводятся комплексно.

Приемочные испытания включают проверку:

\begin{itemize}
	\item полноты и качества реализации функций, указанных в ТЗ;
	\item выполнения каждого требования, относящегося к интерфейсу Системы;
	\item работы пользователей в диалоговом режиме;
	\item полноты действий, доступных пользователю,
		и их достаточность для функционирования Системы;
	\item сложности процедур диалога,
		возможности работы пользователей без специальной подготовки;
	\item реакции системы на ошибки пользователя;
	\item практической выполнимости рекомендованных процедур.
\end{itemize}

\subsubsection{Испытания подсистемы обработки пользовательского ввода}

Испытания подсистемы обработки пользовательского ввода
направлены на проверку доступности всех функций программы,
удобства использования, корректного ввода данных и обработки ошибок.
Также проверяется устойчивость интерфейса при сбоях.

\subsubsection{Испытания подсистемы парсинга исходного DRC файла}

Тестируется корректность извлечения данных из различных DRC форматов.
Проверяется реакция на некорректные файлы,
а также производительность при обработке больших объемов данных.
Важно убедиться в корректной работе с форматом SVRF Calibre.

\subsubsection{Испытания подсистемы предобработки}

Проверяется правильность разворачивания макросов и упорядочивания кода.
Также тестируется производительность предобработки
и корректная обработка некорректных макросов.

\subsubsection{Испытания подсистемы конвертации}

Основная цель --- убедиться, что конвертация выполняется правильно
и соответствует эталонным данным для формата KLayout.
Проверяется работоспособность выходного файла,
производительность, а также обработка ошибок.

\subsubsection{Испытания подсистемы генерации отчета}

Тестируется полнота и корректность создаваемых отчетов,
поддержка различных форматов (текст, LaTeX, Markdown),
а также корректное отражение ошибок.
Проверяется производительность генерации отчетов при больших объемах данных.

\clearpage  % XXX: Fine-tuning

\subsection{Методика проведения испытаний}

Ниже представлена таблица~\ref{table:test:method}
с методикой проведения испытаний.

\begin{longtable}{|c|p{7.5cm}|p{7.5cm}|}
	\caption{Методика проведения испытаний} \label{table:test:method} \\
	\hline
	\textbf{\No} & \textbf{Действие} & \textbf{Результат} \\
	\hline
	\endfirsthead
	\conttable{table:test:method} \\
	\hline
	\textbf{\No} & \textbf{Действие} & \textbf{Результат} \\
	\hline
	\endhead

	\textbf{1}
	& \multicolumn{2}{|l|}{\textbf{
		Сценарий <<Тестирование обработки пользовательского ввода>>}} \\ \hline
	1.1
	& Ввести некорректные данные
	(например, пустой путь к несуществующему файлу),
	попытаться запустить конвертацию.
	& Отображается сообщение об ошибке,
	программа не завершает работу аварийно. \\ \hline

	1.2
	& Ввести путь к корректному файлу проверок и файлу конфигурации.
	& Конвертация успешно запущена, отображается статус выполнения. \\ \hline

	1.3
	& Открыть лог выполнения, проверить, что отображаются все этапы обработки.
	& Лог корректно отображает все этапы работы программы. \\ \hline

	\textbf{2}
	& \multicolumn{2}{|p{15cm}|}{\textbf{Сценарий
		<<Тестирование подсистемы парсинга исходного DRC файла>>}} \\ \hline
	2.1
	& Загрузить файл DRC правил формата Calibre,
	проверить, что все элементы файла корректно извлечены.
	& Элементы извлечены, парсинг завершен без ошибок. \\ \hline

	2.2
	& Загрузить файл с синтаксическими ошибками, запустить парсинг.
	& Программа сообщает об ошибке, парсинг не завершен,
	информация о проблеме отображается. \\ \hline

	2.3
	& Проверить время парсинга для большого файла DRC правил (5000+ строк).
	& Парсинг завершен, время выполнения измерено
	и соответствует заявленным параметрам. \\ \hline

	\textbf{3}
	& \multicolumn{2}{|l|}{\textbf{
		Сценарий <<Тестирование подсистемы предобработки>>}} \\ \hline
	3.1
	& Загрузить файл с макросами, запустить предобработчик.
	& Макросы успешно развернуты,
	файл подготовлен к дальнейшей обработке. \\ \hline

	3.2
	& Загрузить файл с некорректным макросом и запустить предобработку.
	& Программа сообщает об ошибке, требует корректировки данных. \\ \hline

	3.3
	& Загрузить файл с некорректной последовательностью операций,
	запустить предобработчик.
	& Последовательность восстановлена,
	файл подготовлен к дальнейшей обработке. \\ \hline

	\textbf{4}
	& \multicolumn{2}{|l|}{\textbf{
		Сценарий <<Тестирование подсистемы конвертации>>}} \\ \hline
	4.1
	& Запустить конвертацию файла DRC правил
	из формата Calibre в формат Klayout.
	& Конвертация успешно выполнена,
	целевой файл сгенерирован без ошибок. \\ \hline

	4.2
	& Загрузить файл большого объема (5000+ строк) и запустить конвертацию.
	& Конвертация завершена,
	производительность системы соответствует заявленным требованиям. \\ \hline

	4.3
	& Передать файл проверок с явной ошибкой в параметрах операции
	и запустить конвертацию.
	& Программа сообщает об ошибке и завершает конвертацию с ошибкой. \\ \hline

	\textbf{5}
	& \multicolumn{2}{|l|}{\textbf{
		Сценарий <<Тестирование подсистемы генерации отчета>>}}  \\ \hline
	5.1
	& По завершении конвертации проверить наличие
	отчета о выполненных операциях.
	& Отчет сгенерирован,
	содержит все ключевые данные (успешные шаги, ошибки). \\ \hline

	5.2
	& Проверить отчет на наличие ошибок
	и предупреждений при использовании файла правил с ошибкой.
	& Отчет включает информацию об ошибках,
	предупреждения отображены корректно. \\ \hline

	5.3
	& Сохранить отчет в разных форматах (текстовый файл, LaTeX, Markdown)
	и проверить их корректность.
	& Отчеты корректно сохранены в выбранных форматах
	и доступны для анализа. \\ \hline
\end{longtable}

\subsection{Требования по испытаниям программных средств}

Испытания программных средств ПК DRC проводятся
в процессе функционального тестирования Системы
и её нагрузочного тестирования.
Других требований по испытаниям программных средств ПК DRC не предъявляется.

\subsection{Перечень работ, проводимых после завершения испытаний}

По результатам испытаний делается заключение
о соответствии ПК DRC требованиям ТЗ на Систему
и возможности оформления акта сдачи ПК DRC в опытную эксплуатацию.
При этом производится (при необходимости) доработка программных средств
и документации.

\subsection{Условия и порядок проведения испытаний}

Испытания ПК DRC должны проводиться на целевом оборудовании Заказчика.
Оборудование должно быть предоставлено в той конфигурации,
которая запланирована для начального развёртывания системы,
и указанна в Техническом задании.

Во время испытаний проводится полное функциональное тестирование,
согласно требованиям, указанным в Техническом задании.

В ходе проведения опытной эксплуатации
для каждого участника испытаний
Системы администратор выдает доступ к исполняемому файлу Системы
для проведения полнофункционального тестирования.

Данные пользователи работают с Системой,
выполняя свои служебные обязанности,
то есть создают файлы правил для конкретных техпроцессов и выполняются
их в Klayout на дизайнах цифровых микросхем,
подвергая тем самым ПК DRC полнофункциональному тестированию
в течение установленного срока.

\subsection{Материально-техническое обеспечение испытаний}

Приёмочные испытания проводятся
на программно-аппаратном комплексе Заказчика
в следующей минимальной конфигурации:

\begin{itemize}
	\item ПК в составе АРМ пользователя
		или серверная площадка, выделенная Заказчиком на территории
		для проведения приемочных испытаний;
	\item Операционная система: Linux;
	\item Программы: Python версии 3.9 или выше
		и KLayout версии 0.29.6 или выше.
\end{itemize}

\subsection{Метрологическое обеспечение испытаний}

Программа испытаний не требует
использования специализированного измерительного оборудования.

\subsection{Отчётность}

Результаты испытаний ПК DRC, предусмотренные настоящей программой,
фиксируются в протоколах, содержащих следующие разделы:

\begin{itemize}
	\item Назначение испытаний и номер раздела требований ТЗ на ПК DRC,
		по которому проводят испытание;
	\item Состав технических и программных средств,
		используемых при испытаниях;
	\item Указание методик, в соответствии с которыми проводились испытания,
		обработка и оценка результатов;
	\item Условия проведения испытаний и характеристики исходных данных;
	\item Средства хранения и условия доступа к тестирующей программе;
	\item Обобщённые результаты испытаний;
	\item Выводы о результатах испытаний
		и соответствии созданной Системы
		определённому разделу требований ТЗ на ПК DRC.
\end{itemize}

В протоколах могут быть занесены замечания персонала
по удобству эксплуатации Системы.
Этап проведения предварительных испытаний завершается
оформлением <<Акта предварительных и приемочных испытаний ПК DRC>>.

