\chapter{Исследовательский раздел}

\section{Моделирование бизнес-процессов}

\subsection{Входные и выходные данные}

Входные данные:

\begin{itemize}
	\item Путь до файла правил DRC,
		созданный для коммерческих инструментов (формат SVRF Calibre).
	\item Путь до файла конфигурации
\end{itemize}

Выходные данные:

\begin{itemize}
	\item Правила DRC, преобразованный для использования
		в открытых инструментах (в формате Klayout).
	\item Отчет о результатах трансляции.
\end{itemize}

\subsection{Модель БП в вариации TO-BE}

Использование нотации IDEF0 позволяет описать процессы на верхнем уровне
и их взаимодействие \rdref{fig:idef0}{fig:idef0:a4}.

\begin{image}
	\includegrph{2024-09-29_16-46-34}
	\caption{Контекстная диаграмма}
	\label{fig:idef0}
\end{image}

\begin{image}
	\includegrph{2024-09-29_16-46-47}
	\caption{Декомпозиция контекстной диаграммы}
	\label{fig:idef0:a0}
\end{image}

\begin{image}
	\includegrph{2024-09-29_16-16-02}
	\caption{Декомпозиция процесса предобработки}
	\label{fig:idef0:a3}
\end{image}

\begin{image}
	\includegrph{2024-09-29_16-15-22}
	\caption{Декомпозиция процесса конвертации}
	\label{fig:idef0:a4}
\end{image}

\section{Границы проекта}

Границы автоматизируемых бизнес-процессов включают:

\begin{itemize}
	\item Автоматизация процесса конвертации правил
		из проприетарного формата в формат,
		поддерживаемый open-source инструментами;
	\item Генерация отчета о трансляции;
	\item Автоматизация развертывания макросов;
	\item Автоматизация упорядочивания кода.
\end{itemize}

Границы автоматизации не включают:

\begin{itemize}
	\item Проверку корректности конвертированных файлов
		в открытых инструментах;
	\item Валидацию данных на уровне конечных инструментов проектирования;
\end{itemize}

\section{Анализ систем-аналогов}

Существующие коммерческие решения, такие как Calibre и Assura,
не подходят для использования в открытых проектах ввиду их высокой стоимости
и закрытых форматов данных.
С другой стороны есть открытое решение, предоставленное приложением Klayout,
но он не поддерживат коммерческие форматы файлов проверок,
используемые большинством фабрик-производителей.
Это создает потребность в создании собственной системы,
которая позволит автоматически конвертировать правила DRC
из коммерческих форматов в открытые инструменты, такие как Klayout.

\section{Техническое задание}

\subsection{Функциональные, нефункциональные и пользовательские требования}

\subsubsection{Функциональные требования}

Это требования, которые описывают, что система должна делать.
Они определяют конкретные функции,
которые система должна выполнять для решения задач пользователя.

Перечислим их:

\begin{itemize}
    \item Программа должна принимать на вход файлы
		формата SVRF (Standard Verification Rule Format)
		для DRC (Design Rule Check) правил,
		используемый коммерческими инструментами физической проверки
		Calibre;
	\item Программа должна читать файл конфигурации;
    \item Программа должна поддерживать преобразование правил из формата
		Calibre в формат поддерживаемый открытым инструментом Klayout;
    \item Программа должна обрабатывать макросы перед конвертацией;
    \item Программа должна обрабатывать последовательность 
		инициализации переменных перед конвертацией.
	\item Программа должна проверять и верифицировать полученый скрипт;
	\item Программа должна создавать файл с результом конвертации;
	\item Программа должна создавать отчет о результатах конвертации;
\end{itemize}

\subsubsection{Пользовательские требования}

Эти требования формируются с точки зрения пользователя и описывают задачи,
которые пользователь должен выполнить с помощью системы.
Они часто менее технические и более ориентированы на взаимодействие.

Перечислим их:

\begin{itemize}
    \item Пользователь должен иметь возможность выбрать исходный;
    \item Пользователь должен иметь возможность выбрать
		целевой форматы для конвертации правил;
    \item Пользователь должен иметь возможность настроить ход конвертации
		через конфигурационный файл;
    \item Интерфейс программы должен представлять собой консольное приложение;
    \item Программа должна предоставлять отчет о результатах конвертации;
\end{itemize}

\subsubsection{Нефункциональные требования}

Эти требования касаются характеристик системы, которые определяют,
как она выполняет свои функции,
но не связаны непосредственно с поведением системы.
Они включают такие аспекты, как производительность, надежность, безопасность,
удобство использования и масштабируемость.

Перечислим их:

\begin{itemize}
	\item Конкретное время зависит от объема файла,
		приблизительно должно составлять не больше 5 минут на 10Мб-ый файл.
    \item Программа должна быть реализован на языке Python 3.9
		для совместимости с основными файловыми системами Windows и Linux;
    \item Программа не должна изменять исходные файлы правил;
    \item Полученные конвертацией правила должены выдавать
		такой же результат, как исходные в коммерческих приложениях.
\end{itemize}

\subsection{Требования к программному обеспечению}

\subsubsection{Состав ПО}

\begin{itemize}
	\item система включает пользовательский интерфейс;
	\item модуль парсинга;
	\item модуль предобработки;
	\item модуль конвертации;
	\item модуль генерации отчетов.
\end{itemize}

\subsubsection{Выбор ПО}

Язык программирования: Python (версии 3.9 и выше).

Библиотеки Python:

\begin{itemize}
	\item configparse;
	\item PyYAML (версии 6.0.2 и выше);
	\item click (версии 8.1.7 и выше);
	\item os;
	\item sys;
	\item Lark (версии 1.2.2 и выше);
	\item NetworkX (версии 3.3 и выше).
\end{itemize}

\subsubsection{Разрабатываемое ПО}

Написание уникальных алгоритмов для обработки и преобразования DRC правил,
а также для многопоточного парсинга исходного файла праил
и упорядочивания последовательности операций.

\subsubsection{Покупные средства}

Использование библиотек с открытым исходным кодом,
таких как Lark для парсинга и NetworkX для работы с графом.

\subsection{Требования к техническому обеспечению}

Технические средства.
Персональные компьютер или сервера на ОС Linux или Windows.

Функциональные характеристики.
Обеспечение устойчивой работы при больших объемах данных.

\subsection{Требования к информационному обеспечению}

\subsubsection{Состав и структура данных}

ПК DRC должна работать с данными, включающими исходные файлы DRC правил,
конфигурационные файлы и логи работы системы.
Данные структурированы по логическим модулям,
чтобы обеспечить удобный доступ и обработку.

\subsubsection{Организация данных}

Информация хранится в виде текстовых файлов на локальном устройстве
или в файловой системе организации.
Конфигурационные файлы содержат настройки,
необходимые для корректного выполнения конвертации.

\subsubsection{Информационный обмен}

Взаимодействие между модулями системы (парсинг, предобработка, конвертация)
должно быть организовано так, чтобы обеспечить непрерывность
и правильную последовательность передачи данных,
что необходимо для корректной конвертации исходных правил.

\subsubsection{Информационная совместимость}

АС должна быть совместима с форматами DRC правил,
используемыми в системах Calibre для входных данных,
а также KLayout для выходных данных.

\subsubsection{Классификаторы и справочники}

Необходимо использовать справочник команд и макросов DRC правил,
а также классификаторы для разных типов проверок,
что позволит систематизировать данные
и обеспечить правильную интерпретацию конвертируемых правил.

\subsubsection{Системы управления базами данных}

Для данной версии системы использование СУБД не требуется,
так как данные хранятся в файловой системе
и не требуют сложной организации или реляционных связей.

\subsubsection{Представление данных}

Данные должны представляться в текстовом формате,
поддерживающем основными операционными системами (Linux и Windows),
чтобы быть доступными для просмотра и редактирования при необходимости.

\subsubsection{Контроль и восстановление данных}

Система должна сохранять логи работы, фиксирующие каждую выполненную операцию
и возникающие ошибки, для последующего анализа и устранения ошибок.
Автоматическое восстановление данных не требуется;
в случае сбоя рекомендуется повторный запуск с исходными данными.

\subsection{Требования к математическому обеспечению}

Для реализации математического обеспечения ПК DRC
требуются следующие алгоритмы и методы:

\begin{itemize}
	\item Алгоритмы для построения ориентированного ациклического графа;
	\item Алгоритмы для топологической сортировки ациклического орт-графа;
	\item Алгоритмы для парсинга
		(Earley Parser, абстрактное синтаксическое дерево (AST),
		токенизации или лексический анализ);
	\item Алгоритмы преобразования, отвечающие за трансформацию DRC
		правил из одного формата в другой с сохранением логики
		и структуры правил.
\end{itemize}

\subsection{Требования к документации на программное решение}

\subsubsection{Перечень подлежащих разработке документов}

\begin{itemize}
	\item Техническое задание (ТЗ);
	\item Проектная документация (проект, спецификации);
	\item Программа и методика испытаний;
	\item Руководства пользователя и администраторов;
	\item Отчеты о внедрении и эксплуатации системы.
\end{itemize}

\subsubsection{Вид представления и количество документов}

\paragraph{Формат}

Все документы предоставляются в электронном формате (PDF),
удобном для хранения и печати,
а также в текстовом формате (Word или аналог) для возможных правок.

\paragraph{Количество экземпляров}

По одному электронному экземпляру каждой документации
для заказчика и для внутреннего хранения у разработчика.

При необходимости печатные версии, по одному экземпляру Технического задания,
Руководства пользователя и Акта приемки для заказчика.

\subsubsection{Требования по использованию ЕСКД и ЕСПД}

Документация должна соответствовать требованиям
Единой системы конструкторской документации (ЕСКД)
и Единой системы программной документации (ЕСПД):

\begin{itemize}
	\item ЕСКД: Применяется для проектной документации,
		где должны быть соблюдены стандарты по оформлению, обозначениям
		и структуре документа.
		Это включает требования к оформлению чертежей,
		блок-схем и структурных схем модулей системы.
	\item ЕСПД: Используется для программной и эксплуатационной документации.
		Все описания, инструкции и руководства должны быть оформлены
		в соответствии с требованиями ЕСПД, включая единообразие терминов,
		нумерацию разделов и описание функциональных возможностей.
\end{itemize}

Соблюдение стандартов ЕСКД и ЕСПД необходимо
для обеспечения удобства использования и единообразия документации,
что облегчает эксплуатацию, обслуживание и сопровождение АС.

\subsection{Требования к надежности программного решения}

ПК DRC должна обеспечивать корректное выполнение конвертации
при каждом запуске по запросу пользователя.
Вероятность аварийного завершения работы системы должна быть
не более одного случая на 1000 запусков.

При возникновении ошибок или сбоев система должна завершать выполнение
с выдачей соответствующего сообщения об ошибке.
Все ошибки логируются для последующего анализа и устранения.

\subsection{Требования к безопасности программного решения}

Разрабатываемая система не должна куда-либо сохранять исходные файлы правил
для обеспечения сохранности DRC правил под PDK.

\section{Программа и методика испытаний}

\subsection{Объект испытаний}

\subsubsection{Наименование системы}

Программа-конвертер DRC правил для использования
в открытых инструментах проектирования цифровых микросхем.

\subsubsection{Область применения системы}

Программный продукт, представляющий собой конвертер DRC правил,
предназначен для преобразования правил проектирования
из проприетарных форматов (SVRF Calibre) в форматы, совместимые
с открытыми инструментами проектирования цифровых микросхем (KLayout).
Основная цель --- обеспечить удобство
и доступность работы с DRC правилами для разработчиков и инженеров,
работающих в области проектирования интегральных схем.\par
Основная функция заключается в преобразования DRC правил
из проприетарных форматах в форматы,
поддерживаемые открытыми инструментами.

Области применения:

\begin{itemize}
	\item Использование в процессе проектирования интегральных схем,
		где необходимо соблюдать DRC правила
		для корректного функционирования и производства.
	\item Интеграция в существующие инструменты проектирования
		для автоматизации работы с DRC правилами и улучшения совместимости.
	\item Применение в учебных заведениях
		и исследовательских лабораториях для обучения студентов
		и специалистов работе с правилами проектирования.
	\item Помощь компаниям, желающим перейти
		на открытые инструменты проектирования,
		в конвертации существующих данных для обеспечения совместимости.
\end{itemize}

\subsubsection{Условное обозначение системы}

Условное обозначение Системы --- ПК DRC.

\subsection{Цель испытаний}

Целью проводимых по настоящей программе и методике испытаний ПК DRC
является определение функциональной работоспособности системы
на этапе проведения испытаний. 

Программа испытаний должна удостоверить работоспособность ПК DRC
в соответствии с функциональным предназначением.

\subsection{Общие положения}

\subsubsection{Перечень руководящих документов,
	на основании которых проводятся испытания}

Приёмочные испытания ПК DRC проводятся на основании следующих документов:

\begin{itemize}
	\item Утверждённое Техническое задание на разработку ПК DRC;
	\item Настоящая Программа и методика приёмочных испытаний;
\end{itemize}

\subsubsection{Место и продолжительность испытаний}

Место проведения испытаний --- площадка Заказчика.
Продолжительность испытаний устанавливается Приказом Заказчика
о составе приёмочной комиссии и проведении приёмочных испытаний.

\subsubsection{Организации, участвующие в испытаниях}

В приёмочных испытаниях участвуют представители следующих организаций:

\begin{itemize}
	\item ООО "Мальт Систем" (Заказчик);
	\item Бондарь Андрей Ренатович (Исполнитель).
\end{itemize}

Конкретный перечень лиц, ответственных за проведение испытаний системы,
определяется Заказчиком.

\subsubsection{Перечень предъявляемых на испытания документов}

Для проведения испытаний Исполнителем предъявляются следующие документы:

\begin{itemize}
	%\item Государственный контракт №2124-05-08 от 12.11.2008;
	\item Техническое задание на создание ПК DRC;
	\item Технический проект ПК DRC.
	\item Программа и методика испытаний на ПК DRC;
\end{itemize}

\subsection{Объём испытаний}

\subsubsection{Перечень этапов испытаний и проверок}

В процессе проведения приёмочных испытаний должны
быть протестированы следующие подсистемы ПК DRC:

\begin{itemize}
	\item Подсистема обработки пользовательского ввода;
	\item Подсистема парсинга исходного DRC файла;
	\item Подсистема предобработки;
	\item Подсистема конвертации;
	\item Подсистема генерации отчета.
\end{itemize}

Все подсистемы испытываются одновременно
на корректность взаимодействия подсистем,
влияние подсистем друг на друга, т.е. испытания проводятся комплексно.

Приемочные испытания включают проверку:

\begin{itemize}
	\item полноты и качества реализации функций, указанных в ТЗ;
	\item выполнения каждого требования, относящегося к интерфейсу Системы;
	\item работы пользователей в диалоговом режиме;
	\item полноты действий, доступных пользователю,
		и их достаточность для функционирования Системы;
	\item сложности процедур диалога,
		возможности работы пользователей без специальной подготовки;
	\item реакции системы на ошибки пользователя;
	\item практической выполнимости рекомендованных процедур.
\end{itemize}

\subsubsection{Испытания подсистемы обработки пользовательского ввода}

Испытания подсистемы обработки пользовательского ввода
направлены на проверку доступности всех функций программы,
удобства использования, корректного ввода данных и обработки ошибок.
Также проверяется устойчивость интерфейса при сбоях.

\subsubsection{Испытания подсистемы парсинга исходного DRC файла}

Тестируется корректность извлечения данных из различных DRC форматов.
Проверяется реакция на некорректные файлы,
а также производительность при обработке больших объемов данных.
Важно убедиться в корректной работе с форматом SVRF Calibre.

\subsubsection{Испытания подсистемы предобработки}

Проверяется правильность разворачивания макросов и упорядочивания кода.
Также тестируется производительность предобработки
и корректная обработка некорректных макросов.

\subsubsection{Испытания подсистемы конвертации}

Основная цель --- убедиться, что конвертация выполняется правильно
и соответствует эталонным данным для формата KLayout.
Проверяется работоспособность выходного файла,
производительность, а также обработка ошибок.

\subsubsection{Испытания подсистемы генерации отчета}

Тестируется полнота и корректность создаваемых отчетов,
поддержка различных форматов (текст, LaTeX, Markdown),
а также корректное отражение ошибок.
Проверяется производительность генерации отчетов при больших объемах данных.

\subsection{Методика проведения испытаний}

\begin{longtable}{|c|p{7.5cm}|p{7.5cm}|}
	\caption{Методика проведения испытаний} \label{table:test:method} \\
	\hline
	\textbf{\No} & \textbf{Действие} & \textbf{Результат} \\
	\hline
	\endfirsthead
	\conttable{table:test:method} \\
	\hline
	\textbf{\No} & \textbf{Действие} & \textbf{Результат} \\
	\hline
	\endhead

	\textbf{1}
	& \multicolumn{2}{|l|}{\textbf{
		Сценарий <<Тестирование обработки пользовательского ввода>>}} \\ \hline
	1.1
	& Ввести некорректные данные
	(например, пустой путь к несуществующему файлу),
	попытаться запустить конвертацию.
	& Отображается сообщение об ошибке,
	программа не завершает работу аварийно. \\ \hline

	1.2
	& Ввести путь к корректному файлу проверок и файлу конфигурации.
	& Конвертация успешно запущена, отображается статус выполнения. \\ \hline

	1.3
	& Открыть лог выполнения, проверить, что отображаются все этапы обработки.
	& Лог корректно отображает все этапы работы программы. \\ \hline

	\textbf{2}
	& \multicolumn{2}{|p{15cm}|}{\textbf{Сценарий
		<<Тестирование подсистемы парсинга исходного DRC файла>>}} \\ \hline
	2.1
	& Загрузить файл DRC правил формата Calibre,
	проверить, что все элементы файла корректно извлечены.
	& Элементы извлечены, парсинг завершен без ошибок. \\ \hline

	2.2
	& Загрузить файл с синтаксическими ошибками, запустить парсинг.
	& Программа сообщает об ошибке, парсинг не завершен,
	информация о проблеме отображается. \\ \hline

	2.3
	& Проверить время парсинга для большого файла DRC правил (5000+ строк).
	& Парсинг завершен, время выполнения измерено
	и соответствует заявленным параметрам. \\ \hline

	\textbf{3}
	& \multicolumn{2}{|l|}{\textbf{
		Сценарий <<Тестирование подсистемы предобработки>>}} \\ \hline
	3.1
	& Загрузить файл с макросами, запустить предобработчик.
	& Макросы успешно развернуты,
	файл подготовлен к дальнейшей обработке. \\ \hline

	3.2
	& Загрузить файл с некорректным макросом и запустить предобработку.
	& Программа сообщает об ошибке, требует корректировки данных. \\ \hline

	3.3
	& Загрузить файл с некорректной последовательностью операций,
	запустить предобработчик.
	& Последовательность восстановлена,
	файл подготовлен к дальнейшей обработке. \\ \hline

	\textbf{4}
	& \multicolumn{2}{|l|}{\textbf{
		Сценарий <<Тестирование подсистемы конвертации>>}} \\ \hline
	4.1
	& Запустить конвертацию файла DRC правил
	из формата Calibre в формат Klayout.
	& Конвертация успешно выполнена,
	целевой файл сгенерирован без ошибок. \\ \hline

	4.2
	& Загрузить файл большого объема (5000+ строк) и запустить конвертацию.
	& Конвертация завершена,
	производительность системы соответствует заявленным требованиям. \\ \hline

	4.3
	& Передать файл проверок с явной ошибкой в параметрах операции
	и запустить конвертацию.
	& Программа сообщает об ошибке и завершает конвертацию с ошибкой. \\ \hline

	\textbf{5}
	& \multicolumn{2}{|l|}{\textbf{
		Сценарий <<Тестирование подсистемы генерации отчета>>}}  \\ \hline
	5.1
	& По завершении конвертации проверить наличие
	отчета о выполненных операциях.
	& Отчет сгенерирован,
	содержит все ключевые данные (успешные шаги, ошибки). \\ \hline

	5.2
	& Проверить отчет на наличие ошибок
	и предупреждений при использовании файла правил с ошибкой.
	& Отчет включает информацию об ошибках,
	предупреждения отображены корректно. \\ \hline

	5.3
	& Сохранить отчет в разных форматах (текстовый файл, LaTeX, Markdown)
	и проверить их корректность.
	& Отчеты корректно сохранены в выбранных форматах
	и доступны для анализа. \\ \hline
\end{longtable}

\subsection{Требования по испытаниям программных средств}

Испытания программных средств ПК DRC проводятся
в процессе функционального тестирования Системы
и её нагрузочного тестирования.
Других требований по испытаниям программных средств ПК DRC не предъявляется.

\subsection{Перечень работ, проводимых после завершения испытаний}

По результатам испытаний делается заключение
о соответствии ПК DRC требованиям ТЗ на Систему
и возможности оформления акта сдачи ПК DRC в опытную эксплуатацию.
При этом производится (при необходимости) доработка программных средств
и документации.

\subsection{Условия и порядок проведения испытаний}

Испытания ПК DRC должны проводиться на целевом оборудовании Заказчика.
Оборудование должно быть предоставлено в той конфигурации,
которая запланирована для начального развёртывания системы,
и указанна в Техническом задании.

Во время испытаний проводится полное функциональное тестирование,
согласно требованиям, указанным в Техническом задании.

В ходе проведения опытной эксплуатации
для каждого участника испытаний
Системы администратор выдает доступ к исполняемому файлу Системы
для проведения полнофункционального тестирования.

Данные пользователи работают с Системой,
выполняя свои служебные обязанности,
то есть создают файлы правил для конкретных техпроцессов и выполняются
их в Klayout на дизайнах цифровых микросхем,
подвергая тем самым ПК DRC полнофункциональному тестированию
в течение установленного срока.

\subsection{Материально-техническое обеспечение испытаний}

Приёмочные испытания проводятся
на программно-аппаратном комплексе Заказчика
в следующей минимальной конфигурации:

\begin{itemize}
	\item ПК в составе АРМ пользователя
		или серверная площадка, выделенная Заказчиком на территории
		для проведения приемочных испытаний;
	\item Операционная система: Linux;
	\item Программы: Python версии 3.9 или выше
		и KLayout версии 0.29.6 или выше.
\end{itemize}

\subsection{Метрологическое обеспечение испытаний}

Программа испытаний не требует
использования специализированного измерительного оборудования.

\subsection{Отчётность}

Результаты испытаний ПК DRC, предусмотренные настоящей программой,
фиксируются в протоколах, содержащих следующие разделы:

\begin{itemize}
	\item Назначение испытаний и номер раздела требований ТЗ на ПК DRC,
		по которому проводят испытание;
	\item Состав технических и программных средств,
		используемых при испытаниях;
	\item Указание методик, в соответствии с которыми проводились испытания,
		обработка и оценка результатов;
	\item Условия проведения испытаний и характеристики исходных данных;
	\item Средства хранения и условия доступа к тестирующей программе;
	\item Обобщённые результаты испытаний;
	\item Выводы о результатах испытаний
		и соответствии созданной Системы
		определённому разделу требований ТЗ на ПК DRC.
\end{itemize}

В протоколах могут быть занесены замечания персонала
по удобству эксплуатации Системы.
Этап проведения предварительных испытаний завершается
оформлением <<Акта предварительных и приемочных испытаний ПК DRC>>.

