\chapter{Проектный раздел}

\section{Дерево функций}

Дерево зависимостей функций можно построить на основе взаимосвязанных
шагов процесса конвертации DRC правил \rref{fig:tree}.

\begin{image}
	\includegrph{functree.drawio.png}
	\caption{Дерево зависимостей функций}
	\label{fig:tree}
\end{image}

Это дерево функций показывает зависимость между этапами работы программы
и связанными с ними подфункциями.

\section{Логика работы программного решения}

\subsection{Диаграмма прецедентов}

Диаграмма прецедентов (Use case diagram) --- это диаграмма поведения,
на которой показано множество прецедентов и актёров,
а также отношения между ними. Она применяется для моделирования вида системы
с точки зрения внешнего наблюдателя.

Диаграмма прецедентов представляет собой визуальное представление
взаимодействий между пользователем и системой, описывающее,
какие действия (прецеденты) может выполнить пользователь
в контексте работы с программой-конвертером DRC правил \rref{fig:use:case}.

\begin{image}
	\includegrph{Screenshot from 2024-10-09 21-08-03}
	\caption{Диаграмма прецедентов}
	\label{fig:use:case}
\end{image}

Пользователь инициирует процесс конвертации, запуская его через систему.
В ответ на это система выполняет несколько подпрецедентов:
генерирует отчет, выполняет конвертацию,
а также предоставляет возможность выбрать исходный файл DRC в формате Calibre
и конфигурационный файл для настройки параметров конвертации.

В процессе конвертации система обрабатывает ввод пользователя,
выполняет парсинг исходных данных и предобработку,
а затем приступает к самой конвертации.
После завершения процесса пользователь получает файл DRC правил
в целевом формате, а также отчет о ходе и результатах конвертации.

\subsection{Диаграмма последовательности}

Диаграмма последовательности (sequence diagram) --- это наглядное
представление совокупности разных элементов модели системы,
изображение того, как и в каком порядке они взаимодействуют.

Диаграмма последовательности в контексте программы-конвертера DRC правил
отображает последовательность взаимодействий между пользователем,
системой и компонентами программы для выполнения процесса конвертации.
Она описывает порядок выполнения операций и сообщений,
которые передаются между объектами
в процессе работы программы \rref{fig:sequence}.

\begin{image}
	\includegrph{Screenshot from 2024-10-09 21-06-24}
	\caption{Диаграмма последовательности}
	\label{fig:sequence}
\end{image}

Эта диаграмма отражает пошаговую последовательность операций,
которые происходят при запуске и выполнении процесса конвертации,
начиная с ввода пользователя и заканчивая предоставлением результатов.

\subsection{Выбор и обоснование архитектурного
	паттерна проектирования кода приложения}

Для программы-конвертера DRC правил выбран паттерн
\textbf{<<Цепочка обязанностей>>} (Chain of Responsibility).\par
Паттерн <<Цепочка обязанностей>> позволяет передавать запросы
по цепочке обработчиков,
что идеально подходит для модульной структуры системы.
Каждый модуль (обработка пользовательского ввода, парсинг,
предобработка, конвертация, генерация отчета)
может обрабатывать запросы последовательно,
что упрощает реализацию и позволяет легко добавлять новые модули
или изменять порядок обработки.\par
Этот паттерн предоставляет гибкость в обработке различных этапов конвертации,
позволяя легко управлять процессом и встраивать дополнительные шаги,
если это необходимо.\par
Использование <<Цепочки обязанностей>> также упрощает управление ошибками
и статусами, так как каждый компонент может завершить выполнение
или передать управление следующему,
в зависимости от результата своей работы.\par
Таким образом, паттерн <<Цепочка обязанностей>> подходит
для разработки системы конвертации DRC правил, обеспечивая гибкость,
масштабируемость и простоту модификации.

\clearpage  % XXX: Fine-tuning

\section{Путь пользователя}

Путь пользователя --- это общий алгоритм работы с продуктом. Так
называемый User Flow или путь пользователя, это последовательный
список действий или экранов, по которым может переходить
пользователь в процессе взаимодействия с продуктом.
Как пользователь будет взаимодействовать с продуктом
продемонстрированно на рисунке~\ref{fig:user:flow}.

\begin{image}
	\includegrph{user_flow.drawio.png}
	\caption{Путь пользователя в разработке}
	\label{fig:user:flow}
\end{image}

В результате построения такой диаграммы можно четко увидеть,
как взаимодействуют разные компоненты системы
и как пользователь пошагово проходит через процесс конвертации,
начиная от запуска программы до получения итогового результата.
Этот путь помогает лучше понять логику работы программы
и предоставляет ясное представление о том,
как пользователь будет взаимодействовать с системой на каждом этапе.

\section{Проектирование интерфейсов}

В данной работе графический пользовательский интерфейс (GUI)
не был реализован, однако, если бы он был необходим,
его проектирование могло бы быть выполнено в соответствии с подходами,
описанными в данной главе.

\subsection{Описание используемых технологий и их обоснование}

\subsubsection{Язык программирования}

Python был выбран за свою простоту и читаемость,
что позволяет быстро разрабатывать приложения.
Он поддерживает множество библиотек и инструментов,
а также объектно-ориентированное программирование,
что облегчает организацию и поддержку кода.

\subsubsection{Библиотека Tkinter}

Tkinter является стандартной библиотекой
для создания графических интерфейсов в Python.
Ее популярность обусловлена рядом ключевых преимуществ,
которые делают ее удобным инструментом для разработки GUI:

\begin{itemize}
	\item Удобство использования:
		Tkinter предоставляет простой и интуитивно понятный интерфейс
		для создания окон, кнопок, текстовых полей,
		меню и других элементов управления.
		Это позволяет разработчикам быстро разрабатывать интерфейсы
		без необходимости глубокого погружения в сложные детали.
	\item Кросс-платформенность:
		Программы, созданные с использованием Tkinter,
		могут быть запущены на Windows, macOS и Linux
		без необходимости внесения значительных изменений в код.
		Это обеспечивает высокую степень универсальности приложений.
	\item Поддержка: Tkinter сопровождается обширной документацией,
		которая упрощает изучение библиотеки.
		Также существует активное сообщество разработчиков,
		готовое помочь с возникающими вопросами или проблемами.
\end{itemize}

Благодаря этим качествам,
библиотека Tkinter является удобным и доступным выбором
для создания простых и эффективных графических интерфейсов в Python.

\subsection{Описание интерфейса}

Интерфейс приложения состоит из нескольких вкладок,
каждая из которых предназначена для выполнения определенных задач.

\subsubsection{Вкладка Input (Ввод)}

Данная вкладка содержит поля для ввода путей до исходного файла
и файла конфигурации.
Для удобства пользователя реализованы кнопки <<Browse>>,
позволяющие выбрать файлы через стандартный диалог.

Данная вкладка продемонстрированна на рисунке~\ref{fig:input}.

\begin{image}
	\includegrph{Screenshot from 2024-10-12 15-57-42}
	\caption{Вкладка Input (Ввод)}
	\label{fig:input}
\end{image}

\subsubsection{Вкладка Output (Вывод)}

В этой вкладке находятся поля для ввода путей до файлов отчёта и результата,
также с кнопками <<Browse>> для выбора файлов.
Это позволяет пользователю быстро
и удобно указывать места для сохранения результатов работы.

Данная вкладка продемонстрированна на рисунке~\ref{fig:output}.

\begin{image}
	\includegrph{Screenshot from 2024-10-12 15-57-52}
	\caption{Вкладка Output (Вывод)}
	\label{fig:output}
\end{image}

\subsubsection{Вкладка Settings (Настройки)}

Эта вкладка предоставляет текстовое поле
для ввода детальных настроек,
которые могут быть применены к различным модулям конвертора.
Кнопка <<Apply Settings>> позволяет применить введенные настройки,
что обеспечивает гибкость в конфигурации приложения.

Данная вкладка продемонстрированна на рисунке~\ref{fig:settings}.

\begin{image}
	\includegrph{Screenshot from 2024-10-12 15-57-54}
	\caption{Вкладка Settings (Настройки)}
	\label{fig:settings}
\end{image}

\subsubsection{Вкладка Logs (Логи)}

В этой вкладке реализовано текстовое поле для вывода логов работы приложения.
Пользователь может видеть сообщения об ошибках,
предупреждения и другую информацию,
необходимую для мониторинга работы программы.

Кнопка <<Clear Logs>> позволяет пользователю очищать текстовое
поле логов для упрощения восприятия информации.

Данная вкладка продемонстрированна на рисунке~\ref{fig:logs}.

\begin{image}
	\includegrph{Screenshot from 2024-10-12 15-57-57}
	\caption{Вкладка Logs (Логи)}
	\label{fig:logs}
\end{image}

