\chapter{Проектный раздел}

\section{Дерево функций}

Дерево зависимостей функций можно построить на основе взаимосвязанных
шагов процесса конвертации DRC правил.

\begin{image}
	\includegrph{functree.drawio.png}
	\caption{Дерево зависимостей функций}
	\label{fig:tree}
\end{image}

Это дерево функций показывает зависимость между этапами работы программы
и связанными с ними подфункциями.

\section{Логика работы программного решения}

\subsection{Диаграмма прецедентов}

Диаграмма прецедентов (Use case diagram) --- это диаграмма поведения,
на которой показано множество прецедентов и актёров,
а также отношения между ними.\par
Диаграммы прецедентов применяются для моделирования вида системы
с точки зрения внешнего наблюдателя.\par
Основные элементы диаграммы прецедентов:

\begin{itemize}
	\item Субъект (actor) --- любая сущность,
		взаимодействующая с системой извне.
	\item Прецеденты (use case) --- описание множества последовательностей
		действий (включая их варианты),
		которые выполняются системой для того,
		чтобы актёр получил результат,
		имеющий для него определённое значение.
\end{itemize}

Между субъектами и прецедентами могут существовать различные отношения,
которые описывают взаимодействие экземпляров одних субъектов
и прецедентов с экземплярами других субъектов и прецедентов.

\begin{image}
	\includegrph{Screenshot from 2024-10-09 21-08-03}
	\caption{Диаграмма прецедентов}
	\label{fig:use:case}
\end{image}

\subsection{Диаграмма последовательности}

Диаграмма последовательности (sequence diagram) --- это наглядное
представление совокупности разных элементов модели системы,
изображение того, как и в каком порядке они взаимодействуют.\par
Такие диаграммы подробно описывают, как выполняются разные операции.
При этом они показывают временной порядок или хронологию:
то, когда, как и в какой очереди передаются сообщения.\par
Диаграммы удобно использовать при проектировании или проверке архитектуры,
логики системы или интерфейса.

\begin{image}
	\includegrph[scale=0.27]{Screenshot from 2024-10-09 21-06-24}
	\caption{Диаграмма последовательности}
	\label{fig:sequence}
\end{image}

\subsection{Выбор и обоснование архитектурного
	паттерна проектирования кода приложения}

Для программы-конвертера DRC правил выбран паттерн
\textbf{<<Цепочка обязанностей>>} (Chain of Responsibility).\par
Паттерн <<Цепочка обязанностей>> позволяет передавать запросы
по цепочке обработчиков,
что идеально подходит для модульной структуры системы.
Каждый модуль (обработка пользовательского ввода, парсинг,
предобработка, конвертация, генерация отчета)
может обрабатывать запросы последовательно,
что упрощает реализацию и позволяет легко добавлять новые модули
или изменять порядок обработки.\par
Этот паттерн предоставляет гибкость в обработке различных этапов конвертации,
позволяя легко управлять процессом и встраивать дополнительные шаги,
если это необходимо.\par
Использование <<Цепочки обязанностей>> также упрощает управление ошибками
и статусами, так как каждый компонент может завершить выполнение
или передать управление следующему,
в зависимости от результата своей работы.\par
Таким образом, паттерн <<Цепочка обязанностей>> подходит
для разработки системы конвертации DRC правил, обеспечивая гибкость,
масштабируемость и простоту модификации.

\section{Путь пользователя}

Путь пользователя --- это общий алгоритм работы с продуктом. Так
называемый User Flow или путь пользователя, это последовательный
список действий или экранов, по которым может переходить
пользователь в процессе взаимодействия с продуктом.
Как пользователь будет взаимодействовать с продуктом
продемонстрированно на рисунке~\ref{fig:user:flow}.

\begin{image}
	\includegrph{user_flow.drawio.png}
	\caption{Путь пользователя в разработке}
	\label{fig:user:flow}
\end{image}

\section{Проектирование интерфейсов}

\subsection{Описание используемых технологий и их обоснование}

\subsubsection{Язык программирования}

Python был выбран за свою простоту и читаемость,
что позволяет быстро разрабатывать приложения.
Он поддерживает множество библиотек и инструментов,
а также объектно-ориентированное программирование,
что облегчает организацию и поддержку кода.

\subsubsection{Библиотека Tkinter}

Tkinter является стандартной библиотекой
для создания графических интерфейсов в Python.
Она была выбрана благодаря:

\begin{itemize}
	\item Удобству использования:
		Простой интерфейс для создания окон и элементов управления.
	\item Кросс-платформенности:
		Приложения работают на Windows, macOS и Linux.
	\item Поддержке: Обширная документация и активное сообщество.
\end{itemize}

\subsection{Описание интерфейса}

Интерфейс приложения состоит из нескольких вкладок,
каждая из которых предназначена для выполнения определенных задач.

\subsubsection{Вкладка Input (Ввод)}

Данная вкладка содержит поля для ввода путей до исходного файла
и файла конфигурации.
Для удобства пользователя реализованы кнопки <<Browse>>,
позволяющие выбрать файлы через стандартный диалог.

Данная вкладка продемонстрированна на рисунке~\ref{fig:input}.

\begin{image}
	\includegrph{Screenshot from 2024-10-12 15-57-42}
	\caption{Вкладка Input (Ввод)}
	\label{fig:input}
\end{image}

\subsubsection{Вкладка Output (Вывод)}

В этой вкладке находятся поля для ввода путей до файлов отчёта и результата,
также с кнопками <<Browse>> для выбора файлов.
Это позволяет пользователю быстро
и удобно указывать места для сохранения результатов работы.

Данная вкладка продемонстрированна на рисунке~\ref{fig:output}.

\begin{image}
	\includegrph{Screenshot from 2024-10-12 15-57-52}
	\caption{Вкладка Output (Вывод)}
	\label{fig:output}
\end{image}

\subsubsection{Вкладка Settings (Настройки)}

Эта вкладка предоставляет текстовое поле
для ввода детальных настроек,
которые могут быть применены к различным модулям конвертора.
Кнопка <<Apply Settings>> позволяет применить введенные настройки,
что обеспечивает гибкость в конфигурации приложения.

Данная вкладка продемонстрированна на рисунке~\ref{fig:settings}.

\begin{image}
	\includegrph{Screenshot from 2024-10-12 15-57-54}
	\caption{Вкладка Settings (Настройки)}
	\label{fig:settings}
\end{image}

\subsubsection{Вкладка Logs (Логи)}

В этой вкладке реализовано текстовое поле для вывода логов работы приложения.
Пользователь может видеть сообщения об ошибках,
предупреждения и другую информацию,
необходимую для мониторинга работы программы.

Кнопка <<Clear Logs>> позволяет пользователю очищать текстовое
поле логов для упрощения восприятия информации.

Данная вкладка продемонстрированна на рисунке~\ref{fig:logs}.

\begin{image}
	\includegrph{Screenshot from 2024-10-12 15-57-57}
	\caption{Вкладка Logs (Логи)}
	\label{fig:logs}
\end{image}

