\chapter*{Приложения}
\addcontentsline{toc}{chapter}{Приложения}

\appendixitem{Техническое задание}

% ----------------------------------------------------------------------------
% ----------------------------------------------------------------------------


\achapter{Техническое задание}

\asection{Общие сведения}

\asubsection{Полное наименование автоматизированной системы (АС)
	и её условное обозначение}

\textbf{Полное наименование}:
Программа-конвертер DRC правил для использования
в открытых инструментах проектирования цифровых микросхем

\textbf{Условное обозначение}: ПК DRC.

\asubsection{Наименование организаций}

В в разработке участвуют следующие организаций:

\begin{itemize}
	\item ООО "Мальт Систем" (Заказчик);
	\item Бондарь Андрей Ренатович (Исполнитель).
\end{itemize}

\asubsection{Перечень документов, на основании которых создается АС}

Для создания АС Исполнителем предъявляются следующие документы:

\begin{itemize}
	%\item Государственный контракт №2124-05-08 от 12.11.2008;
	\item Техническое задание на создание ПК DRC;
	\item Технический проект ПК DRC.
	\item Программа и методика испытаний на ПК DRC;
\end{itemize}

\asubsection{Плановые сроки начала и окончания работ по созданию АС}

Дата начала работ: 2 сентября 2024 года

Дата окончания работ: 30 мая 2025 года

\asubsection{Общие сведения об источниках и порядке финансирования работ}

Работы по созданию ПК DRC финансируются из средств заказчика,
ООО <<Мальт Систем>>.

Основные этапы оплаты:
аванс в размере 30\% от общей стоимости договора на этапе подписания,
40\% по завершении основной части разработки
и 30\% после сдачи работ и приемки системы заказчиом.
Финансирование осуществляется в соответствии с утвержденным сметным планом.

\asection{Цели и назначение создания автоматизированной системы}

\asubsection{Цели создания АС}

Цель создания автоматизированной системы конвертации правил DRC заключается
в обеспечении совместимости правил проектирования
между различными EDA (Electronic Design Automation) системами,
используемыми для проектирования и верификации интегральных схем.
В результате создания системы должны быть достигнуты следующие показатели:

\asubsubsection{Технические показатели}

Время выполнения конвертации.
Не более 5 минут для набора правил объемом до 5000 строк.

Совместимость.
Поддержка исходных форматов DRC правил для системы Calibre,
а также целевых форматов для системы KLayout.

Процент автоматизированной конвертации.
Преобразование 95\% правил исходного формата должны корректно
конвертироваться без необходимости доработки.

\asubsubsection{Технологические показатели}

Интероперабельность.
Корректная работа на различных операционных системах (Linux, Windows).

Удобство использования.
Предоставление интуитивно понятного интерфейса для настройки
и запуска процесса конвертации через командную строку.

\asubsubsection{Производственно-экономические показатели}

Снижение временных затрат на настройку правил для различных EDA-систем,
то есть уменьшение времени на ручную адаптацию правил в 10 раза
по сравнению с текущим процессом.

\asubsubsection{Критерии оценки достижения целей}

Показатель времени выполнения
и доля конвертируемых правил проверяются на этапе тестирования
и должны соответствовать указанным значениям.

Совместимость системы оценивается
по результатам конвертации файлов в тестовых средах EDA.

Удобство интерфейса проверяется на стадии тестирования
и доработки с привлечением инженеров верификации
в качестве тестовой группы пользователей.

\asubsection{Назначение АС}

АС предназначена для автоматизации процесса конвертации DRC правил,
используемых в системах проектирования интегральных схем,
с целью обеспечения их совместимости между разными инструментами.
Основной вид автоматизируемой деятельности --- упрощение процесса
проектирования и верификации при переходе между различными EDA-системами.

АС применяется в следующих объектах автоматизации:

\begin{itemize}
	\item Проектирование интегральных схем --- система предназначена
		для использования в отделах проектирования и верификации,
		где инженеры проводят настройку и проверку DRC правил;
	\item Верификация интегральных схем --- система будет использоваться
		для проверки DRC правил в различных EDA-системах,
		что необходимо для соблюдения технологических требований
		на каждом этапе проектирования.
\end{itemize}

\asection{Характеристика объекта автоматизации}

\asubsection{Основные сведения об объекте автоматизации}

Объект автоматизации --- это процесс проектирования
и верификации интегральных схем (ИС),
который включает настройку и проверку DRC (Design Rule Check) правил
для обеспечения соответствия схем технологическим требованиям.
В современных компаниях, занимающихся проектированием ИС,
используются различные EDA (Electronic Design Automation) системы,
такие как Calibre и KLayout.
Каждая из этих систем применяет свой уникальный формат DRC правил,
что затрудняет их переносимость
и совместимость при переходе между инструментами.

ПК DRC предназначена для обеспечения совместимости DRC
правил путем автоматизированной конвертации между этими форматами,
минимизируя необходимость ручной адаптации и упрощая процесс верификации. 

\asubsection{Сведения об условиях эксплуатации
	и характеристиках окружающей среды}

ПК DRC предназначена для использования в офисных условиях
на рабочих станциях инженеров по верификации и проектированию.

Операционные системы:

\begin{itemize}
	\item Linux;
	\item Windows;
	\item macOS.
\end{itemize}

\textbf{Минимальные системные требования:}

\begin{itemize}
	\item Процессор: 2 ГГц, 2 ядра;
	\item ОЗУ: 4 ГБ;
	\item Дисковое пространство: 500 МБ для установки, 2 ГБ для данных.
\end{itemize}

\textbf{Рекомендуемые системные требования:}

\begin{itemize}
	\item Процессор: 3 ГГц, 4 ядра;
	\item ОЗУ: 8 ГБ;
	\item Дисковое пространство: 5-10 ГБ для работы с большими проектами.
\end{itemize}
  
ПК DRC разработана для эксплуатации в стандартных офисных условиях
без специфических требований к оборудованию или помещениям,
что делает её удобной для внедрения
в типовых корпоративных ИТ-инфраструктурах. 

\asection{Требования к видам обеспечения АС}

\asubsection{Требования к математическому обеспечению АС}

Для реализации математического обеспечения ПК DRC
требуются следующие алгоритмы и методы:

\begin{itemize}
	\item Алгоритмы для построения ориентированного ациклического графа;
	\item Алгоритмы для топологической сортировки ациклического орт-графа;
	\item Алгоритмы для парсинга
		(Earley Parser, абстрактное синтаксическое дерево (AST),
		токенизации или лексический анализ);
	\item Алгоритмы преобразования, отвечающие за трансформацию DRC
		правил из одного формата в другой с сохранением логики
		и структуры правил.
\end{itemize}

\asubsection{Требования к информационному обеспечению АС}

\asubsubsection{Состав и структура данных}

ПК DRC должна работать с данными, включающими исходные файлы DRC правил,
конфигурационные файлы и логи работы системы.
Данные структурированы по логическим модулям,
чтобы обеспечить удобный доступ и обработку.

\asubsubsection{Организация данных}

Информация хранится в виде текстовых файлов на локальном устройстве
или в файловой системе организации.
Конфигурационные файлы содержат настройки,
необходимые для корректного выполнения конвертации.

\asubsubsection{Информационный обмен}

Взаимодействие между модулями системы (парсинг, предобработка, конвертация)
должно быть организовано так, чтобы обеспечить непрерывность
и правильную последовательность передачи данных,
что необходимо для корректной конвертации исходных правил.

\asubsubsection{Информационная совместимость}

АС должна быть совместима с форматами DRC правил,
используемыми в системах Calibre для входных данных,
а также KLayout для выходных данных.

\asubsubsection{Классификаторы и справочники}

Необходимо использовать справочник команд и макросов DRC правил,
а также классификаторы для разных типов проверок,
что позволит систематизировать данные
и обеспечить правильную интерпретацию конвертируемых правил.

\asubsubsection{Системы управления базами данных}

Для данной версии системы использование СУБД не требуется,
так как данные хранятся в файловой системе
и не требуют сложной организации или реляционных связей.

\asubsubsection{Представление данных}

Данные должны представляться в текстовом формате,
поддерживающем основными операционными системами (Linux и Windows),
чтобы быть доступными для просмотра и редактирования при необходимости.

\asubsubsection{Контроль и восстановление данных}

Система должна сохранять логи работы, фиксирующие каждую выполненную операцию
и возникающие ошибки, для последующего анализа и устранения ошибок.
Автоматическое восстановление данных не требуется;
в случае сбоя рекомендуется повторный запуск с исходными данными.

\asubsection{Требования к лингвистическому обеспечению АС}

Поддерживаемые языки.
АС Конвертер DRC должна поддерживать английский язык
для всех интерфейсов и исходных данных.

Организация диалога.
Взаимодействие с пользователем должно быть логичным,
обеспечивая четкую навигацию и сообщения об ошибках.
Предусмотрен интерфейс командной строки (CLI).

Синтаксис формализованного языка.
При необходимости система должна включать описание синтаксиса команд DRC
и правил формализованного языка для упрощения работы с исходными данными.

\asubsection{Требования к программному обеспечению АС}

\asubsubsection{Состав ПО}

\begin{itemize}
	\item система включает пользовательский интерфейс;
	\item модуль парсинга;
	\item модуль предобработки;
	\item модуль конвертации;
	\item модуль генерации отчетов.
\end{itemize}

\asubsubsection{Выбор ПО}

Язык программирования: Python (версии 3.9 и выше).

Библиотеки Python:

\begin{itemize}
	\item configparse;
	\item PyYAML (версии 6.0.2 и выше);
	\item click (версии 8.1.7 и выше);
	\item os;
	\item sys;
	\item Lark (версии 1.2.2 и выше);
	\item NetworkX (версии 3.3 и выше).
\end{itemize}

\asubsubsection{Разрабатываемое ПО}

Написание уникальных алгоритмов для обработки и преобразования DRC правил,
а также для многопоточного парсинга исходного файла праил
и упорядочивания последовательности операций.

\asubsubsection{Покупные средства}

Использование библиотек с открытым исходным кодом,
таких как Lark для парсинга и NetworkX для работы с графом.

\asubsubsection{Архитектура ПО}

Модель C4 описывает архитектуру системы на четырёх уровнях:
контекста, контейнеров, компонентов и кода.
На основе этой модели можно создать следующие диаграммы
для программы-конвертера:

\paragraph{Диаграмма контекста системы}

В модели C4 диаграмма контекста системы (System Context Diagram)
представляет собой самый высокий уровень абстракции системы,
показывая взаимодействие системы с внешними пользователями (акторами)
и системами. Она дает общее представление о том,
кто и как взаимодействует с системой,
не углубляясь в детали внутренней реализации.
   
\begin{image}
	\includegrph[scale=0.4]{c40.drawio}
	\caption{Диаграмма контекста системы}
	\label{fig:c4:system:context}
\end{image}

Пользователь взаимодействует с системой через интерфейс командной строки (CLI)
для загрузки конфигурационных файлов и исходных DRC файлов,
а также для запуска процесса конвертации.\par
Файловая система ОС используются для чтения и записи исходных
и сконвертированных DRC правил, а также для создания отчета о конвертации.

\paragraph{Диаграмма контейнеров}

Диаграмма контейнеров представляет следующий уровень детализации модели C4.
Она показывает внутреннюю структуру системы,
отображая основные контейнеры (программные компоненты) системы,
такие как приложения, базы данных, внешние системы, и их взаимодействия.
   
\begin{image}
	\includegrph[scale=0.4]{c41.drawio}
	\caption{Диаграмма компонентов}
	\label{fig:c4:container}
\end{image}

Так как приложение не разделяется на API и базу данный,
данная диаграмма отличается от предыдущей добавлением внешней системы
консоли для взаимодействия с приложением
через интерфейс командной строки (CLI).

\paragraph{Диаграмма компонентов}

Диаграмма компонентов модели C4 детализирует
каждый контейнер на более глубоком уровне,
отображая его внутренние программные компоненты и взаимодействия между ними.
Это позволяет увидеть, как функционирует каждая часть контейнера
и какие компоненты обеспечивают выполнение основных функций.

\begin{image}
	\includegrph[scale=0.25]{c42.drawio}
	\caption{Диаграмма компонентов}
	\label{fig:c4:components}
\end{image}

\asubsection{Требования к техническому обеспечению АС}

Технические средства.
Персональные компьютер или сервера на ОС Linux или Windows.

Функциональные характеристики.
Обеспечение устойчивой работы при больших объемах данных.

\asubsection{Требования к метрологическому обеспечению АС}

Показатели метрологического обеспечения.
Точность преобразования должна соответствовать 95\%.

Методы измерений.
Контроль точности и производительности выполнения преобразования DRC правил.

Средства измерений и контроля.
Использование тестовых сценариев для измерения точности
и времени выполнения конвертации.

Программа метрологического обеспечения.
Проверка работы системы на тестовых наборах правил.

\asubsection{Требования к организационному обеспечению АС}

Структура и функции подразделений.
Задействованы подразделения по разработке и сопровождению ПО.

Организация функционирования.
АС запускается и обслуживается инженерами
без необходимости постоянного сопровождения.

Организация при сбоях, отказах и авариях.
Автоматическое логирование ошибок.

Нормативные документы.
Обеспечение разработчиков инструкциями по использованию инструментов
для разработки и тестирования.

\asubsection{Требования к методическому обеспечению АС}

Нормативные документы должны включать
ГОСТы и внутренние стандарты компании для разработки.

Для обеспечение документацией разработчики
должны быть обеспечены необходимыми нормативными
и методическими документами для правильного выполнения работ.

\asection{Общие технические требования к АС}

\asubsection{Требования к численности
	и квалификации персонала и пользователей АС}

Для эксплуатации системы требуется один инженер-пользователь
для запуска процесса конвертации и контроля результатов.

Инженер по верификации должен обладать базовыми знаниями DRC правил
и работы с EDA-системами.
Пользователю требуется минимальное обучение для работы с интерфейсом.

\asubsection{Требования к показателям назначения АС}

АС должна обеспечивать:

\begin{itemize}
	\item Точность конвертации не менее 95\%
		преобразуемых правил должны корректно
		соответствовать целевому формату.
	\item Время выполнения для файла объемом
		до 5000 строк не должно превышать 5 минут.
\end{itemize}

\asubsection{Требования к надежности}

ПК DRC должна обеспечивать корректное выполнение конвертации
при каждом запуске по запросу пользователя.
Вероятность аварийного завершения работы системы должна быть
не более одного случая на 1000 запусков.

При возникновении ошибок или сбоев система должна завершать выполнение
с выдачей соответствующего сообщения об ошибке.
Все ошибки логируются для последующего анализа и устранения.

\asubsection{Требования по безопасности}

Разрабатываемая система не должна куда-либо сохранять исходные файлы правил
для обеспечения сохранности DRC правил под PDK.

\asubsection{Требования к эргономике и технической эстетике}

Интерфейс системы должен быть простым,
обеспечивать доступ ко всем необходимым функциям
с минимальным количеством действий.

\asubsection{Требования к транспортабельности для подвижных АС}

Требования к транспортабельности для подвижных АС
не предъявляется.

\asubsection{Требования к эксплуатации, техническому обслуживанию,
	ремонту и хранению компонентов АС}

Требования к эксплуатации, техническому обслуживанию,
ремонту и хранению компонентов АС не предъявляется.

\asubsection{Требования к защите информации от несанкционированного доступа}

Требования к защите информации от несанкционированного доступа
не предъявляется.

\asubsection{Требования по сохранности информации при авариях}

Требования по сохранности информации при авариях не предъявляется.

\asubsection{Требования к защите от внешних воздействий}

Требования к защите от внешних воздействий не предъявляется.

\asubsection{Требования к патентной чистоте и патентоспособности}

Система должна соответствовать требованиям патентной чистоты.
Перед разработкой необходимо провести патентные исследования
на наличие схожих решений и патентов
для избежания нарушений прав на интеллектуальную собственность.

\asubsection{Требования по стандартизации и унификации}

Использование типовых алгоритмов
для топологической сортировки ациклического орт-графа и парсинга
(движок Earley Parser, абстрактное синтаксическое дерево (AST),
токенизации и лексический анализ).

Работа со стандартизированными форматами DRC правил.

\asubsection{Дополнительные требования}

Необходимость разработки и предоставления учебного руководства
для пользователей, включающего описание стандартных операций
и список операций не подлежащих конвертации.

\asection{Состав и содержание работ по созданию автоматизированной системы}

Разработка АС Конвертер DRC включает следующие этапы:

\begin{longtable}{|p{3.5cm}|p{4cm}|p{3cm}|p{4cm}|}
	\caption{\leftline{Этапы работ}} \label{table:stages} \\
	\hline
	\textbf{Задача}
	& \textbf{Подзадача}
	& \textbf{Время выполнения}
	& \textbf{Исполнитель} \\
	\hline
	\endfirsthead
	\conttable{table:stages} \\
	\hline
	\textbf{Задача}
	& \textbf{Подзадача}
	& \textbf{Время выполнения}
	& \textbf{Исполнитель} \\
	\hline
	\endhead
	1. Проектирование системы
	& Определение требований
	& 7 дней
	& Бондарь А.Р. \\ \hline

	& Проектирование архитектуры
	& 7 дней
	& Бондарь А.Р. \\ \hline

	2. Разработка компонентов
	& Реализация парсинга
	& 21 дней
	& Бондарь А.Р. \\ \hline

	& Реализация конвертации
	& 60 дней
	& Бондарь А.Р. \\ \hline

	& Реализация предобработки
	& 14 дней
	& Бондарь А.Р. \\ \hline

	& Реализация работы с пользователем
	& 7 дней
	& Бондарь А.Р. \\ \hline

	& Реализация генерации отчетов
	& 7 дней
	& Бондарь А.Р. \\ \hline

	3. Тестирование
	& Разработка тестов
	& 14 дней
	& Бондарь А.Р. \\ \hline

	& Проведение тестирования
	& 3 дня
	& Бондарь А.Р. \\ \hline

	4. Документация
	& Написание пользовательской документации
	& 7 дней
	& Отдел разработки документации \\ \hline

	& Подготовка технической документации
	& 7 дней
	& Отдел разработки документации \\ \hline

	& Разработка инструкции по внедрению
	& 3 дня
	& Отдел разработки документации \\ \hline

	5. Внедрение и запуск
	& Внедрение и запуск системы
	& 3 дня
	& Бондарь А.Р. \\ \hline
\end{longtable}

\asection{Порядок разработки автоматизированной системы}

\asubsection{Порядок организации разработки АС}

Разработка ПК DRC осуществляется поэтапно,
начиная с подготовки и анализа требований,
проектирования архитектуры и заканчивая внедрением
и сдачей готовой системы заказчику.
Каждый этап включает плановые проверки и согласования с заказчиком.
Проектная команда включает аналитиков, разработчиков,
тестировщиков и специалистов по документации.
Заказчик участвует в согласовании технического задания,
промежуточных отчетов, демонстраций функционала и финальной приемке системы.

\asubsection{Перечень документов и исходных данных для разработки АС}

\begin{itemize}
	\item Техническое задание на разработку АС, согласованное с заказчиком;
	\item Документация по требованиям и спецификациям EDA-систем,
		для которых будет выполняться конвертация (Calibre и KLayout);
	\item Данные об условиях эксплуатации и технические требования к системе;
	\item Методические рекомендации по применению DRC правил
		в проектировании и верификации ИС;
	\item Нормативные документы,
		регламентирующие стандарты разработки ПО и интерфейсов.
\end{itemize}

\asubsection{Перечень документов, предъявляемых по окончании этапов работ}

На каждом этапе разработки будут предоставлены следующие документы:

\begin{enumerate}
	\item Подготовительный этап:
		Аналитический отчет по требованиям, утвержденное техническое задание.
	\item Проектирование системы:
		Документация по архитектуре системы,
		прототип интерфейса, документ проектирования модулей.
	\item Разработка модулей:
		Отчет о готовности модулей, описание их функционала,
		результаты промежуточных тестов.
	\item Тестирование системы:
		Отчет о тестировании, включая результаты модульного,
		интеграционного и системного тестирования.
	\item Документация:
		Руководство пользователя, техническая документация,
		отчет о проведенных испытаниях.
	\item Внедрение и обучение пользователей:
		Отчет о внедрении, программа обучения пользователей.
	\item Сдача системы заказчику:
		Итоговый отчет о готовности системы, акт приема-передачи.
\end{enumerate}

\asubsection{Порядок проведения экспертизы технической документации}

Техническая документация, включая отчет о тестировании
и руководство пользователя, проходит экспертизу у специалистов заказчика
и специалистов компании-разработчика.
Экспертиза проводится после завершения этапа разработки документации,
результаты утверждаются обеими сторонами. 

\asubsection{Перечень макетов, порядок их разработки, испытаний и документации}

Перечень макетов, порядок их разработки, испытаний
и документации не предъявляется.

\asubsection{Порядок разработки, согласования
	и утверждения плана совместных работ по разработке АС}

План совместных работ согласуется на этапе подготовки проекта
и утверждается обеими сторонами.
Включает сроки и ответственность за выполнение каждого этапа разработки,
ключевые контрольные точки и порядок взаимодействия разработчиков и заказчика. 

\asubsection{Порядок разработки, согласования
	и утверждения программы работ по стандартизации}

Программа работ по стандартизации разрабатывается на основе требований ГОСТ
и внутренних стандартов заказчика.
Программа включает меры по обеспечению совместимости форматов,
унификации интерфейсов и стандартов качества ПО.
Программа согласуется с заказчиком
и утверждается на начальном этапе разработки.

\asubsection{Требования к гарантийным обязательствам разработчика}

Разработчик предоставляет гарантию на исправление ошибок
и поддержку системы в течение 12 месяцев после сдачи и приемки АС.
В гарантийные обязательства входит исправление критических ошибок
в конвертации, обновление документации и консультирование заказчика.

\asubsection{Порядок проведения технико-экономической оценки разработки АС}

Технико-экономическая оценка включает анализ затрат на разработку,
внедрение и поддержку АС.
Оценка проводится по завершении проектирования и тестирования системы,
чтобы определить соотношение затрат и пользы,
получаемой заказчиком от внедрения системы.

\asubsection{Порядок разработки, согласования
	и утверждения программы метрологического обеспечения,
	программы обеспечения надежности, программы эргономического обеспечения}

Программа метрологического обеспечения
разрабатывается с целью контроля точности
и корректности конвертации DRC правил.
Включает методики измерений и верификации точности преобразования.
  
Программа обеспечения надежности содержит мероприятия
по тестированию на устойчивость к отказам,
механизмы восстановления данных и обеспечение сохранности данных при сбоях.

Программа эргономического обеспечения разрабатывается
для проверки удобства и интуитивности интерфейса АС.
Включает тестирование интерфейса на восприятие
и анализ действий пользователя для оптимизации взаимодействия.

Каждая из программ согласуется с заказчиком
и утверждается на этапе проектирования системы.

\asection{Порядок контроля и приемки автоматизированной системы}

\asubsection{Виды, состав и методы испытаний АС и её составных частей}

Испытания автоматизированной системы ПК DRC проводятся в несколько этапов,
чтобы подтвердить работоспособность
и соответствие системы заявленным требованиям:

\asubsubsection{Модульные испытания}

Испытания происходит над отдельными модулями:
пользовательский ввод, парсинг, предобработка,
конвертация и генерация отчетов.

Тестирование включает проверку функциональности
и корректности каждого модуля по заранее подготовленным тест-кейсам.
Выполняется проверка корректности работы на отдельных наборах DRC правил.
  
\asubsubsection{Интеграционные испытания}

Проверка взаимодействия всех модулей системы.

Проводится последовательное тестирование связи между модулями.
Проверяется корректность передачи данных, синхронизация
и совместимость модулей при конвертации различных форматов.

\asubsubsection{Системные испытания}

Проверка всей АС как единого целого.

Системное тестирование проводится на реальных данных,
предоставленных заказчиком.
Проверяется полная функциональность системы,
включая конвертацию из различных исходных форматов,
скорость выполнения, качество конвертации, надежность системы.

\asubsubsection{Приемочные испытания}

Комплексное тестирование системы заказчиком.

Испытания проводятся на реальных задачах
с использованием DRC правил заказчика для оценки корректности
и производительности системы в рабочих условиях.
Проверяется соответствие системы всем заявленным в ТЗ характеристикам.

\asubsection{Общие требования к приемке работ,
	порядок согласования и утверждения приемочной документации}

Для приемки работ проводятся следующие мероприятия:

\begin{itemize}
	\item После успешного завершения каждого этапа испытаний создается отчет,
		содержащий результаты тестов,
		выявленные ошибки и выполненные корректировки;
	\item По завершении всех этапов испытаний и доработок,
		проведенных на основе выявленных замечаний,
		составляется итоговый акт о готовности системы к приемке;
	\item Итоговая приемка включает подтверждение соответствия системы
		всем функциональным, производственным и техническим требованиям,
		указанным в техническом задании;
	\item Документация по итогам приемки, включая акты и протоколы испытаний,
		согласуется с заказчиком и утверждается обеими сторонами.
\end{itemize}

\asubsection{Статус приемочной комиссии}

Приемочная комиссия имеет \textbf{ведомственный статус}
и включает представителей заказчика,
представителей организации-разработчика, а также, при необходимости,
независимых экспертов в области проектирования и верификации ИС.

\asection{Требования к составу и содержанию работ по подготовке
	объекта автоматизации к вводу автоматизированной системы в действие}

\asubsection{Создание условий функционирования объекта автоматизации}

Осуществить проверку системного окружения
и установку требуемого ПО для корректной работы АС,
включая библиотеки Python и необходимые пакеты.

Предоставить тестовый набор данных (DRC правила),
а также создать среду для тестирования системы в условиях,
приближенных к рабочим, для дополнительной отладки и проверки.

\asubsection{Проведение необходимых организационно-штатных мероприятий}

Выделить ответственного за ввод в эксплуатацию
и обеспечение непрерывного функционирования системы.

Ответственные лица от организации-заказчика
и организации-разработчика координируют ввод системы
и решение возможных проблем.

Организовать базовые службы технической поддержки на период эксплуатации,
включая помощь с обновлением ПО и устранением ошибок.

\asubsection{Порядок обучения персонала и пользователей АС}

Провести обучение для инженеров по верификации и проектированию,
которые будут использовать систему.
Обучение должно включать базовые и расширенные функции системы,
настройку конфигураций, обработку исходных данных и анализ отчетов.

Подготовить учебные пособия, включающие руководство пользователя,
пошаговые инструкции по запуску и настройке системы,
описание возможных ошибок и инструкцию по их устранению.

\asection{Требования к документированию}

\asubsection{Перечень подлежащих разработке документов}

\begin{itemize}
	\item Техническое задание (ТЗ);
	\item Проектная документация (проект, спецификации);
	\item Программа и методика испытаний;
	\item Руководства пользователя и администраторов;
	\item Отчеты о внедрении и эксплуатации системы.
\end{itemize}

\asubsection{Вид представления и количество документов}

\asubsubsection{Формат}

Все документы предоставляются в электронном формате (PDF),
удобном для хранения и печати,
а также в текстовом формате (Word или аналог) для возможных правок.

\asubsubsection{Количество экземпляров}

По одному электронному экземпляру каждой документации
для заказчика и для внутреннего хранения у разработчика.

При необходимости печатные версии, по одному экземпляру Технического задания,
Руководства пользователя и Акта приемки для заказчика.

\asubsection{Требования по использованию ЕСКД и ЕСПД}

Документация должна соответствовать требованиям
Единой системы конструкторской документации (ЕСКД)
и Единой системы программной документации (ЕСПД):

\begin{itemize}
	\item ЕСКД: Применяется для проектной документации,
		где должны быть соблюдены стандарты по оформлению, обозначениям
		и структуре документа.
		Это включает требования к оформлению чертежей,
		блок-схем и структурных схем модулей системы.
	\item ЕСПД: Используется для программной и эксплуатационной документации.
		Все описания, инструкции и руководства должны быть оформлены
		в соответствии с требованиями ЕСПД, включая единообразие терминов,
		нумерацию разделов и описание функциональных возможностей.
\end{itemize}

Соблюдение стандартов ЕСКД и ЕСПД необходимо
для обеспечения удобства использования и единообразия документации,
что облегчает эксплуатацию, обслуживание и сопровождение АС.

\asection{Источники разработки}

\asubsection{Перечень документов и информационных материалов}

ГОСТ 34.601-90 (Автоматизированные системы. Стадии создания)
--- стандарт регламентирует этапы создания автоматизированных систем
и общий порядок их разработки.

ГОСТ Р ИСО/МЭК 25010-2015 (Системная и программная инженерия.
Модели качества систем и программных продуктов) --- стандарт содержит модели
и критерии оценки качества программного обеспечения,
используемые для разработки и тестирования АС.

Техническая документация по системам EDA (Calibre, KLayout)
--- описание форматов DRC правил для каждого инструмента,
спецификации, синтаксис команд и поддерживаемые макросы.

Внутренняя документация заказчика --- содержит требования к формату отчетов
и стандартам оформления документации,
используемые при подготовке выходных данных и отчетов АС.

\asubsection{Использование источников}

ГОСТ 34.601-90 обеспечивает основу для организации этапов разработки
и последовательности работ.

ГОСТ Р ИСО/МЭК 25010-2015 применяется
для определения показателей качества АС, включая надежность,
производительность, совместимость и эргономику.

Методические рекомендации по DRC правилам используются
при разработке алгоритмов парсинга и конвертации,
что позволяет обеспечить корректное преобразование правил
между различными форматами.

Техническая документация по системам EDA используется
для настройки конвертации правил в целевые форматы,
а также для проверки совместимости с Calibre, Assura, KLayout и Magic.

Внутренняя документация заказчика направлена
на стандартизацию выходной документации,
формирование отчетов и описание форматов файлов,
что позволяет удовлетворить требования пользователя
и поддерживать единообразие данных.

