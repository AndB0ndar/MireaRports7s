\chapter{Аналитический раздел}

\section{Диаграмма Ганта разработки ПС}

Диаграмма Ганта представляет собой визуальное отображение задач
и временных рамок, необходимых для разработки системы конвертации DRC правил.
Ниже представлена таблица
с определением задач, подзадач, времени выполнения и исполнителей
\tref{table:ganta}.

\begin{longtable}{|p{4.5cm}|p{4cm}|p{3cm}|p{4cm}|}
	\caption{Диаграмма Ганта} \label{table:ganta} \\
	\hline
	\textbf{Задача}
	& \textbf{Подзадача}
	& \textbf{Время выполнения}
	& \textbf{Исполнитель} \\
	\hline
	\endfirsthead
	\conttable{table:ganta} \\
	\hline
	\textbf{Задача}
	& \textbf{Подзадача}
	& \textbf{Время выполнения}
	& \textbf{Исполнитель} \\
	\hline
	\endhead
	1. Проектирование системы
	& Определение требований
	& 7 дней
	& Бондарь А.Р. \\ \hline

	& Проектирование архитектуры
	& 7 дней
	& Бондарь А.Р. \\ \hline

	2. Разработка компонентов
	& Реализация парсинга
	& 21 дней
	& Бондарь А.Р. \\ \hline

	& Реализация конвертации
	& 60 дней
	& Бондарь А.Р. \\ \hline

	& Реализация предобработки
	& 14 дней
	& Бондарь А.Р. \\ \hline

	& Реализация работы с пользователем
	& 7 дней
	& Бондарь А.Р. \\ \hline

	& Реализация генерации отчетов
	& 7 дней
	& Бондарь А.Р. \\ \hline

	3. Тестирование
	& Разработка тестов
	& 14 дней
	& Бондарь А.Р. \\ \hline

	& Проведение тестирования
	& 3 дня
	& Бондарь А.Р. \\ \hline

	4. Документация
	& Написание пользовательской документации
	& 7 дней
	& Отдел разработки документации \\ \hline

	& Подготовка технической документации
	& 7 дней
	& Отдел разработки документации \\ \hline

	& Разработка инструкции по внедрению
	& 3 дня
	& Отдел разработки документации \\ \hline

	5. Внедрение и запуск
	& Внедрение и запуск системы
	& 3 дня
	& Бондарь А.Р. \\ \hline
\end{longtable}

Диаграмма ганта показана на рисунке~\ref{fig:gantt}.

\begin{image}
	\includegrph[scale=0.25]{Screenshot from 2024-10-12 14-02-22}
	\caption{Диаграмма Ганта}
	\label{fig:gantt}
\end{image}

\section{Обоснование средств разработки программного решения}

\subsection{Язык программирования}
\textbf{Python} --- это гибкий язык,
широко применяемый в области электронной разработки (EDA)
благодаря множеству библиотек
и простоте интеграции с различными форматами данных.
Поддерживает библиотеки для парсинга, работы с файлами и их преобразования.
  
\subsection{Библиотека для парсинга}
\textbf{Lark} --- это мощная и гибкая библиотека
для работы с грамматиками.
Она поддерживает несколько алгоритмов парсинга (Earley, LALR, CYK)
и легко интегрируется с Python.
Она позволяет парсить сложные грамматики,
что необходимо для работы с DRC правилами.

\subsection{Библиотека для работы с графами}
\textbf{NetworkX} --- это библиотека для работы с графами,
которая может быть полезна для построения графов зависимостей между правилами
и их дальнейшей обработки.
Использование этой библиотеки позволяет эффективно анализировать взаимосвязи
между элементами, такими как макросы или правила,
и оптимизировать порядок их применения в процессе конвертации.

\subsection{Фреймворк для тестирования}
\textbf{Pytest} предоставляет простой
и удобный способ написания модульных
и интеграционных тестов для Python-программ.
Он поддерживает автоматизацию тестирования,
что важно для обеспечения стабильности программы.

\subsection{Система управления версиями}
\textbf{Git} обеспечивает контроль версий кода,
что необходимо для отслеживания изменений и работы в команде.
GitHub или GitLab могут использоваться для хранения кода и организации CI/CD.

\subsection{Инструменты автоматизации}
\textbf{Make} позволяют автоматизировать процесс сборки программы,
обеспечивая удобство работы с зависимостями и процессом тестирования.

\subsection{Среда разработки (текстовые редакторы)}
\textbf{Vim и Neovim} --- это лёгкие
и высокопроизводительные текстовые редакторы,
которые идеально подходят для работы с кодом в командной строке.
Они поддерживают плагины для работы с Python, автодополнение, отладку,
интеграцию с Git, а также позволяют быстро
и эффективно редактировать код.
Neovim предоставляет дополнительные возможности
для кастомизации и работы с современными плагинами.

\section{Матрица рисков}

Таблица рисков помогает заранее определить
и подготовиться к потенциальным проблемам,
которые могут возникнуть в процессе разработки \tref{table:risk:matrix}.

\begin{longtable}{|p{3cm}|p{3cm}|p{2cm}|p{3cm}|p{4cm}|}
	\caption{Матрица рисков} \label{table:risk:matrix} \\
	\hline
	\textbf{Название риска}
	& \textbf{Последствия}
	& \textbf{Кач. оценка риска}
	& \textbf{Стратегия реагирования}
	& \textbf{Мероприятия} \\ \hline
	\endfirsthead
	\conttable{table:risk:matrix} \\
	\hline
	\textbf{Название риска}
	& \textbf{Последствия}
	& \textbf{Кач. оценка риска}
	& \textbf{Стратегия реагирования}
	& \textbf{Мероприятия} \\ \hline
	\endhead
	Задержка в сроках разработки
	& Пропуск сроков сдачи проекта
	& Высокая
	& Смягчение 
	& Регулярные встречи команды для мониторинга статуса задач. \\ \hline
	Нехватка ресурсов
	& Увеличение времени разработки, снижение качества
	& Низкая
	& Устранение
	& Определить критические ресурсы заранее и обеспечить их наличие. \\ \hline
	Ошибки в коде
	& Необходимость доработки, потенциальные сбои системы
	& Высокая
	& Превентивные меры
	& Регулярное рецензирование кода и тестирование. \\ \hline
	Неопределен- ность требований
	& Сложности с реализацией функций
	& Низкая
	& Гибкость
	& Обсуждение требований с заказчиком на всех этапах. \\ \hline
	Технические проблемы
	& Остановка работы, снижение производитель- ности
	& Средняя
	& Резервное копирование
	& Настройка резервных копий и тестов на выявление проблем. \\ \hline
	Отсутствие аналогов для конвертируемой команды
	& Задержка в реализации и отладке из-за отсутствия готовых решений
	& Высокая
	& Исследование и разработка
	& Проводить исследование команды DRC правил
	и реализовать необходимый функционал через последовательность
	других команд. \\ \hline
\end{longtable}

\section{Архитектура программного решения}

Основной принцип архитектуры программы-конвертера DRC правил
--- \textbf{модульность}.
Программа состоит из отдельных компонентов (модулей),
которые отвечают за конкретные функции:
парсинг, конвертация, предобработка и взаимодействие с пользователем.
Это позволяет легко изменять или добавлять новые функции
без необходимости переписывать всю систему.\par
Также в программном продукте используется принцип
\textbf{разделения ответственности} 
(Separation of Concerns или SoC), где каждому модулю назначена
своя ответственность, что снижает взаимозависимость между компонентами.
Это делает систему более поддерживаемой и упрощает тестирование и отладку.

\subsection{Архитектурная диаграмма}

Модель C4 описывает архитектуру системы на четырёх уровнях:
контекста, контейнеров, компонентов и кода.
На основе этой модели можно создать следующие диаграммы
для программы-конвертера:

\subsubsection{Диаграмма контекста системы}

В модели C4 диаграмма контекста системы (System Context Diagram)
представляет собой самый высокий уровень абстракции системы,
показывая взаимодействие системы с внешними пользователями (акторами)
и системами. Она дает общее представление о том,
кто и как взаимодействует с системой,
не углубляясь в детали внутренней реализации
\rref{fig:c4:system:context}.
   
\begin{image}
	\includegrph[scale=0.4]{c40.drawio}
	\caption{Диаграмма контекста системы}
	\label{fig:c4:system:context}
\end{image}

Пользователь взаимодействует с системой через интерфейс командной строки (CLI)
для загрузки конфигурационных файлов и исходных DRC файлов,
а также для запуска процесса конвертации.\par
Файловая система ОС используются для чтения и записи исходных
и сконвертированных DRC правил, а также для создания отчета о конвертации.

\subsubsection{Диаграмма контейнеров}

Диаграмма контейнеров представляет следующий уровень детализации модели C4.
Она показывает внутреннюю структуру системы,
отображая основные контейнеры (программные компоненты) системы,
такие как приложения, базы данных, внешние системы, и их взаимодействия
\rref{fig:c4:container}.
   
\begin{image}
	\includegrph[scale=0.4]{c41.drawio}
	\caption{Диаграмма компонентов}
	\label{fig:c4:container}
\end{image}

Так как приложение не разделяется на API и базу данный,
данная диаграмма отличается от предыдущей добавлением внешней системы
консоли для взаимодействия с приложением
через интерфейс командной строки (CLI).

\subsubsection{Диаграмма компонентов}

Диаграмма компонентов модели C4 детализирует
каждый контейнер на более глубоком уровне,
отображая его внутренние программные компоненты и взаимодействия между ними.
Это позволяет увидеть, как функционирует каждая часть контейнера
и какие компоненты обеспечивают выполнение основных функций.

Диаграмма C4 уровня 3 для программы-конвертера DRC правил
отображает взаимодействие между программой,
файловой системой и консолью \rref{fig:c4:components}.

\begin{image}
	\includegrph[scale=0.25]{c42.drawio}
	\caption{Диаграмма компонентов}
	\label{fig:c4:components}
\end{image}

Пользователь через консоль вводит команду с параметрами,
такими как путь к исходным файлам и настройки конвертации.
Программа-конвертер DRC выполняет обработку ввода, парсинг,
предобработку и конвертацию правил, а затем генерирует отчет.
Все данные, включая исходные файлы и результаты,
сохраняются в файловой системе ОС.
Программа выводит сообщения об ошибках или успехе в консоль,
а также сохраняет отчет для дальнейшего анализа.

\subsection{Масштабируемость системы}

Масштабируемость системы заключается в её способности обрабатывать
всё более сложные файлы DRC правил,
увеличивать количество поддерживаемых форматов,
а также адаптироваться к различным требованиям пользователей.

\subsubsection{Вертикальная масштабируемость}

При увеличении мощности оборудования
(например, процессора и оперативной памяти)
программа сможет обрабатывать более сложные
и объёмные файлы DRC за меньшее время.
  
\subsubsection{Масштабируемость за счёт модульности}

Каждый компонент программы (например, парсинг, предобработка, конвертация)
может быть расширен или заменён,
что позволит легко добавлять новые возможности
без изменения существующего кода.
  
\subsubsection{Интеграция с другими EDA инструментами}

Возможность добавления поддержки новых систем
(например, других DRC форматов или API для работы с EDA инструментами)
делает программу масштабируемой
в контексте взаимодействия с внешними инструментами.

\subsection{Описание инструментов для каждого компонента архитектуры}

\subsubsection{Обработка пользовательского ввода}

Для обработки пользовательского ввода через командную строку (CLI)
в программе-конвертере DRC правил используются следующие библиотеки:

\begin{itemize}
	\item click: Библиотека для создания интерфейсов командной строки,
		которая обрабатывает команды и аргументы,
		позволяя пользователю настраивать параметры конвертации через CLI;
	\item os: Библиотека для работы с файловой системой,
		позволяющий считывать исходные файлы
		и выполнять операции с ними;
	\item configparser: Библиотека для работы
		с конфигурационными файлами в формате \texttt{.ini},
		позволяющий сохранять настройки программы;
	\item PyYAML:
		Библиотека для работы с конфигурационными файлами в формате YAML,
		обеспечивающая гибкость в настройках программы
		и упрощая работу с более сложными структурами данных.
\end{itemize}

Эти библиотеки обеспечивают удобную настройку
и управление параметрами программы через командную строку.

\subsubsection{Парсинг}

Парсинг DRC правил является одной из ключевых функций системы,
и для реализации этого компонента применяется библиотека Lark-parser.
Это мощная библиотека для парсинга грамматик.
Она поддерживает контекстно-свободные грамматики
и позволяет создавать синтаксические деревья для последующей обработки.
\textbf{Lark} идеально подходит для разбора
сложных синтаксических конструкций DRC правил.

\subsubsection{Предобработка}

Предобработка включает в себя разворачивание макросов,
упорядочивание и оптимизацию данных перед конвертацией.\par
Реализация алгоритмов предобработки выполняется на Python
с использованием собственных логических структур.
Это включает работу с деревьями, построенными на этапе парсинга,
разворачивание макросов с использованием рекурсивных алгоритмов
и восстановление последовательности операций,
используя библиотеку \textbf{NetworkX}.\par
Библиотека \textbf{NetworkX} используется для работы с графами,
что полезно при работе с зависимостями между операциями.
Она помогает оптимизировать и упорядочить правила
для более эффективной конвертации.

\subsubsection{Компонент конвертации}

Основной процесс преобразования данных из одного формата в другой
требует специализированных инструментов
для работы с текстом и логической трансляции правил.\par
Основная часть конвертации выполняется
через специально разработанные алгоритмы.
Эти алгоритмы должны трансформировать геометрические проверки
и другие DRC правила из одного формата в другой.

\subsubsection{Компонент генерации отчёта}

Отчёты о результатах работы системы генерируются
с использованием пользовательского кода.
Для создания текстовых, LaTeX и Markdown видов отчетов используется
вручную написанный код.
Это обеспечивает гибкость и контроль над содержимым и форматированием отчетов.

\section{Модель базы данных}

\subsection{Логическая модель базы данных}

В данной системе логическая модель данных включает в себя только объекты,
которые используются для парсинга в библиотеке \textbf{Lark}:

\textbf{Tree} --- это основная модель,
представляющая собой абстрактное синтаксическое дерево (AST),
которое строится в результате парсинга DRC правил.
Каждый узел дерева может представлять различные конструкции правил.

Атрибуты класса:

\begin{itemize}
	\item \textbf{data:} имя правила или псевдонима;
	\item \textbf{children:} список совпадающих подправил и терминалов.
\end{itemize}

\textbf{Token} --- это модель, представляющая отдельные токены,
полученные в процессе лексического анализа.
Токены используются для формирования синтаксического дерева
и содержат информацию о типе и значении каждого элемента.

Атрибуты класса:

\begin{itemize}
	\item \textbf{type:} имя токена;
	\item \textbf{value:} значение токена.
\end{itemize}

Логическая модель данный представлена ER-диаграммой на рисунке \ref{fig:er}.

\begin{image}
	\includegrph{er.drawio}
	\caption{ER-диаграмма}
	\label{fig:er}
\end{image}

Каждый узел синтаксического дерева может быть связан
с несколькими токенами, которые соответствуют его значениям.
Связь между \textbf{Tree} и \textbf{Token} будет один ко многим
(один узел дерева может содержать несколько токенов).
Например, если в синтаксическом дереве есть правило для переменной,
то несколько токенов могут представлять идентификаторы
или числовые значения в этом правиле.

\subsection{Словарь данных}

На первом этапе проектирования базы данных необходимо собрать сведения о
предметной области, в том числе о назначении, способах использования и охраны
структуре данных, а по мере развития проекта осуществлять централизованное
накопление информации о концептуальной, логической, внутренней и внешних
моделях данных. Словарь данных является как раз тем средством, которое
позволяет при проектировании, эксплуатации и развитии базы данных
поддерживать и контролировать информацию о данных.

\begin{longtable}{|p{4cm}|p{8cm}|p{4cm}|}
	\caption{Словарь данных Tree} \\
	\hline
	\textbf{Наименование элемента}
	& \textbf{Определение (предназначение)}
	& \textbf{Тип} \\
	\hline
	\endhead
	\textbf{data} & Имя правила или псевдонима & Строка \\ \hline
	\textbf{children}
	& Список совпадающих подправил и терминалов
	& Список \\ \hline
\end{longtable}

\begin{longtable}{|p{4cm}|p{8cm}|p{4cm}|}
	\caption{Словарь данных Token} \\
	\hline
	\textbf{Наименование элемента}
	& \textbf{Определение (предназначение)}
	& \textbf{Тип} \\
	\hline
	\endhead
	\textbf{type} & Имя токена & Строка \\ \hline
	\textbf{value} & Значение токена & Строка \\ \hline
\end{longtable}

\subsection{Матрица доступа и роли}

\subsubsection{Определение ролей в системе}

В системе предусмотрена одна роль: \textbf{пользователь}.
Он запускает программу, передаёт исходный файл DRC правил и файл конфигурацию,
после чего получает результат в виде преобразованного файла
и отчёта о конвертации.
Пользователь не взаимодействует напрямую
с моделями \textbf{Tree} и \textbf{Token}.

\subsubsection{Матрица доступа}

Матрица доступа для роли разработчика по отношению к объектам,
использующимся в логической модели:

\begin{longtable}{|p{5.3cm}|p{5.3cm}|p{5.3cm}|}
	\caption{Матрица доступа} \\
	\hline
	\textbf{Роль} & \textbf{Tree} & \textbf{Token} \\
	\hline
	\endhead
	Пользователь & Нет доступа & Нет доступа \\ \hline
\end{longtable}

