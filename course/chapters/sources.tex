%\chapter*{Список используемых источников}
\addcontentsline{toc}{chapter}{Список используемых источников}

% Различают правила оформления источников:
%	Для книг указывается: Aвтор. Название / Aвторы. – Город:
%		Издательство, год выпуска. – Количество страниц.;
%	Для журналов указывается: Aвтор. Статья / Aвторы // Журнал. –
%		Год. – Номер. – Страницы размещения статьи.;
%	Для Web-ссылок указывается: Название работы / Автор. URL.

\begin{thebibliography}{2}
	\bibitem{1} Разработка приложений баз данных: метод. указ.
		по выполнению лабораторных работ:
		для спец. 07300. — М.: МИРЭА, 2002. — 32 с.
	\bibitem{2} Зубкова Т.М. Проектирование программных систем по обработке
		и анализу информации: метод.
		указания к курсовому проектированию по дисциплине
		"Технология разработки программного обеспечения".
		— Оренбург: ГОУ ОГУ, 2011. — 53 с.
	\bibitem{3} Алексанкин Я.Я., Бржозовский А.Э., Жданов В.А. и др. Автоматизированное проектирование систем автоматического управления. — М.: Машиностроение, 1990.
	\bibitem{4} Заботина Н.Н. Проектирование информационных систем: учебное пособие. — М.: ИНФРА-М, 2020. — 331 с. URL: \url{https://znanium.com/catalog/product/1036508} (дата обращения: 14.08.2022).
	\bibitem{5} Грекул В.И. Проектное управление в сфере информационных технологий. — М.: БИНОМ. Лаборатория знаний, 2013. — 336 с. URL: \url{http://znanium.com/catalog.php?bookinfo=485348} (дата обращения: 14.08.2022).
	\bibitem{6} Вдовенко Л.А. Информационная система предприятия: учебное пособие. — М.: Вузовский учебник, НИЦ ИНФРА-М, 2015. — 304 с. URL: \url{http://znanium.com/catalog.php?bookinfo=501089}.
	\bibitem{7} Чистов Д.В. Информационные системы в экономике: учеб. пособие. — М.: НИЦ Инфра-М, 2013. — 234 с. URL: \url{http://znanium.com/catalog.php?bookinfo=489996}.
	\bibitem{8} Варфоломеева А.О., Коряковский А.В., Романов В.П. Информационные системы предприятия: учебное пособие. — М.: НИЦ ИНФРА-М, 2016. — 283 с. URL: \url{http://znanium.com/catalog.php?bookinfo=536732}.
	\bibitem{9} Скрипкин К.Г. Экономическая эффективность информационных систем в России. — М.: МАКС Пресс, 2014. — 156 с. URL: \url{http://znanium.com/catalog.php?bookinfo=533938}.
	\bibitem{10} Авдеев В.А. Организация ЭВМ и периферия с демонстрацией имитационных моделей. — М.: ДМК, 2014. — 708 с.
	\bibitem{11} Антамошкин О.А. Программная инженерия. Теория и практика. Учебник. — М.: НИЦ Инфра-М, 2012. — 368 с.
	\bibitem{12} Дейтел Х.М. Операционные системы. Основы и принципы. Т. 1 — М.: Бином, 2016. — 1024 с.
	\bibitem{13} Дейтел Х.М. Операционные системы. Т. 2. Распределенные системы, сети, безопасность. — М.: Бином, 2016. — 704 с.
	\bibitem{14} Таненбаум Э. Современные операционные системы. — СПб.: Питер, 2013. — 1120 с.
	\bibitem{15} Орлов С.А. Программная инженерия.
		Учебник для вузов. 5-е издание обновленное и дополненное. — М.:
		Издательский дом «Питер», 2017. — 812 с.
	\bibitem{16} Тейлор Д. Сценарии командной оболочки.
		Linux, OS X и Unix. 2-е издание.
		— Издательский дом «Питер», 2017. — 624 с.
	\bibitem{17} SWEBOK V2, 2004 г.
	\bibitem{18} SWEBOK V3, 2013 г.
\end{thebibliography}


