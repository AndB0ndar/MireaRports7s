\chapter*{Введение}
\addcontentsline{toc}{chapter}{Введение}

Проектирование цифровых микросхем требует соблюдения строгих правил
проектирования для корректного функционирования и производства чипов.
Одним из таких стандартов являются DRC правила --- наборы ограничений,
касающихся топологии полупроводниковых схем.
Эти правила включают требования к расстояниям между элементами,
ширине дорожек и другим параметрам, важным для производственных процессов.
Актуальность выбранной темы обусловлена необходимостью повышения
производительности проектирования
и снижения ошибок на ранних стадиях разработки,
что напрямую влияет на конечную стоимость и качество продукции.

Целью данной выпускной квалификационной работы является разработка
программы-конвертера DRC правил, которая позволит автоматизировать
и оптимизировать процесс перевода проектных правил между различными
CAD-системами и стандартами.
Это позволит ускорить время разработки,
минимизировать риск человеческой ошибки
и упростить интеграцию различных инструментов проектирования.

Для достижения поставленной цели необходимо решить следующие задачи:

\begin{enumerate}
	\item Проанализировать существующие методы
		и инструменты для проверки проектных правил.
	\item Исследовать форматы представления DRC правил
		в различных CAD-системах.
	\item Разработать архитектуру программы-конвертера
		и описать алгоритмы преобразования правил.
	\item Реализовать программное обеспечение
		и провести его тестирование на предмет эффективности и точности.
	\item Оценить результаты и сформулировать рекомендации
		по использованию разработанного решения.
\end{enumerate}

Объектом исследования является процесс разработки программного обеспечения
для автоматизации конвертации DRC правил.
В предметную область исследования входят методы
и инструменты проектирования электронных устройств, а также стандарты,
регулирующие проверку проектных правил в различных CAD-системах.

Таким образом, данная работа направлена
на решение актуальной задачи повышения эффективности проектирования
через автоматизацию процессов, что имеет большое значение
для современных тенденций в области разработки электронных компонентов.

