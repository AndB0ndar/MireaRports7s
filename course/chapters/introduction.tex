\chapter*{Введение}
\addcontentsline{toc}{chapter}{Введение}

\textit{Актуальность.}
Проектирование цифровых микросхем требует соблюдения строгих правил
проектирования для корректного функционирования и производства чипов.
Одним из таких стандартов являются DRC правила --- наборы ограничений,
касающихся топологии полупроводниковых схем.
Эти правила включают требования к расстояниям между элементами,
ширине дорожек и другим параметрам, важным для производственных процессов.
Значимость выбранной темы обусловлена необходимостью повышения
производительности проектирования
и снижения ошибок на ранних стадиях разработки,
что напрямую влияет на конечную стоимость и качество продукции.

\underline{Цель работы:}
Разработка программы-конвертера DRC правил, которая автоматизирует
и оптимизирует процесс перевода проектных правил между различными
CAD-системами и стандартами.
Такое решение позволит сократить время разработки,
уменьшить риск человеческих ошибок
и упростить интеграцию различных инструментов проектирования.

\underline{Объект исследования:}
Процесс разработки программного обеспечения
для автоматизации преобразования DRC правил.

\underline{Предмет исследования:}
Методы и инструменты проектирования электронных устройств,
а также стандарты, регулирующие проверку проектных правил
в различных CAD-системах.

\underline{Значимость работы:}
Реализация данного проекта направлена на решение важной
задачи повышения эффективности процессов проектирования за счет автоматизации.
Это соответствует современным тенденциям в разработке электронных компонентов
и обеспечивает конкурентные преимущества в отрасли.

Для достижения поставленной цели
необходимо решить следующие \underline{задачи}:

\begin{enumerate}
	\item Проанализировать существующие методы
		и инструменты для проверки проектных правил.
	\item Исследовать форматы представления DRC правил
		в различных CAD-системах.
	\item Разработать архитектуру программы-конвертера
		и описать алгоритмы преобразования правил.
	\item Реализовать программное обеспечение
		и провести его тестирование на предмет эффективности и точности.
	\item Оценить результаты и сформулировать рекомендации
		по использованию разработанного решения.
\end{enumerate}

Данная курсовая работа включает: введение, три главы, заключение,
список использованной литературы и приложение.

Введение предоставляет общий обзор выбранной темы, формулирует цель
и задачи исследования, а также обосновывает актуальность проблемы.
Этот глава объясняет,
почему данная проблема важна для современной инженерной практики
и какие преимущества дает ее решение.

Первая глава посвящена исследовательскому разделу,
в рамках которого проводится моделирование бизнес-процессов.
Описываются входные и выходные данные,
разрабатываются модели бизнес-процессов в вариации TO-BE,
определяется граница проекта.
Также проводится анализ существующих систем-аналогов
и формулируется техническое задание, включая функциональные, нефункциональные
и пользовательские требования.
Здесь же разрабатываются требования к программному, техническому,
информационному и математическому обеспечению.
Завершается глава описанием программы и методики испытаний,
включая объем, условия проведения и метрологическое обеспечение.

Вторая глава посвящена аналитическому разделу.
В этой части представлена диаграмма Ганта для планирования разработки,
обоснован выбор инструментов, таких как языки программирования,
библиотеки, фреймворки и системы управления версиями.
Также в главе разрабатывается архитектура программного решения,
включая описание компонентов, их масштабируемости и интеграции.
Здесь же представлена модель базы данных с логической моделью,
словарем данных и матрицей доступа.

Третья глава раскрывает проектная глава,
где рассматриваются детальные аспекты разработки программного решения.
Разрабатывается дерево функций,
описывается логика работы программного обеспечения
с использованием диаграмм прецедентов и последовательности.
Особое внимание уделяется проектированию интерфейсов,
описанию пути пользователя и выбору технологий для реализации интерфейса.
Также в этой главе проводится обоснование выбранного
архитектурного паттерна проектирования.

Заключение подводит итоги исследования,
формулирует основные выводы и предлагает перспективы дальнейших исследований.
Здесь акцентируется внимание на значимости полученных результатов
для повышения качества проектирования электронных устройств.

Список использованной литературы включает все источники,
использованные при выполнении курсовой работы, включая научные статьи,
книги, публикации и интернет-ресурсы.

Приложения содержат дополнительные материалы, такие как техническое задание,
разработанное в соответствии с ГОСТ 34.602-89,
и другие вспомогательные документы,
необходимые для более глубокого понимания результатов исследования.

